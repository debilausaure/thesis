\documentclass
[
    twoside,                 % The thesis is formatted like a book. That is, odd and even pages are handled differently.
    openright,               % Starts a new chapter on an odd page number (right side).
    cleardoublepage = empty, % Clear pages inserted in order to have new chapters appear on odd pages are formatted with an empty style.
    fontsize = 11 pt,        % The size of the font.
    french,                % Support for American English.
    captions = tableheading, % Places the correct amount of space when the caption of a table is above the table.
    numbers = noenddot,      % Does not use a period at the end of numbered titles, such as sections or figures.
    footheight = 35 pt,      % Defines the height of the foot. Due to the line, it needs extra height.
%    draft,                   % Only displays boxes of figures. This option is useful if compilation takes a long time.
]
{scrbook}


% This file contains all sorts of commands that are used in order to specify certain options for the document.

\newif\ifprintVersion   % Defines a binary variable that signals whether the document is prepared for physical or digital print.
\newif\ifprofessionalPrint % Defines a binary variable that signals whether the print will be done by a professional printing service that requests extra margin for page cutting and is not bound to paper formats like A4.
\newif\iffancyTheorems  % Defines a binary variable that signals whether theorems are formatted in the classical style or in a new format that better suits the overall flavor of this thesis.
\newif\ifboldNumberSets % Defines a binary variable that signals whether the variables for number sets (like N or R) should be in bold. If not, they are in blackboard bold instead.

% Set all variables to their default values.
\printVersionfalse
\professionalPrintfalse
\fancyTheoremstrue
\boldNumberSetstrue

%%%%%%%%%%%%%%%%%%%%%%%%
% The following commands define certain strings that provide important information for the document.

% The title of the thesis.
\newcommand*{\printTitle}{}
\newcommand*{\myTitle}[1]{\renewcommand*{\printTitle}{#1}}
\newcommand*{\printTitleBold}{\textbf{\printTitle}}

% The author’s name.
\newcommand*{\printAuthor}{}
\newcommand*{\myName}[1]{\renewcommand*{\printAuthor}{#1}}

% The name of the author’s program.
\newcommand*{\printProgram}{}
\newcommand*{\myProgram}[1]{\renewcommand*{\printProgram}{#1}}

% A short description of the topic of the thesis. This string will be used for the PDF metadata.
\newcommand*{\printSubject}{}
\newcommand*{\mySubject}[1]{\renewcommand*{\printSubject}{#1}}

% A short description of the topic of the thesis. This string will be used for the PDF metadata.
\newcommand*{\printKeywords}{}
\newcommand*{\myKeywords}[1]{\renewcommand*{\printKeywords}{#1}}

% Defines the extra length added to each side for the print version.
\newlength{\extraborderlength}
\newcommand*{\extraBorder}[1]{\setlength{\extraborderlength}{#1}}

% Defines the length of the binding correction. (The class ›scrbook‹ has a binding correction but it does not work due to all the other packages that are loaded.)
\newlength{\mybindingcorrection}
\newcommand*{\bindingCorrection}[1]{\setlength{\mybindingcorrection}{#1}}
 % Contains commands that are used for certain information that is printed.


%%%%%%%%%%%%%%%%%%%%%%%%%%%%%%%%%%%%%%
%% Please adjust your options here. %%
%%%%%%%%%%%%%%%%%%%%%%%%%%%%%%%%%%%%%%

    % This section contains commands with important data for your thesis. Please adjust them in order for the document to be printed correctly.

    % Defines the length of the amount that a printed page will be cut from EACH side (including the inner side). This option only takes effect with \printVersiontrue and \professionalPrinttrue.
    \extraBorder{3 mm}

    % Shifts the inner margin outward by the amount specified. When the book is bound, part of the page will not be seen anymore. This option compensates for this loss. It only takes effect with \printVersiontrue.
    \bindingCorrection{6 mm}

    % Use the following command if this is a master thesis.

%    \printVersiontrue      % Use this value if you want to prepare your thesis for physical printing. In this case, links will not be colored. Without \professionalPrinttrue, the content will be moved outward by the binding correction, increasing the inner margin and decreasing the outer margin.
%    \professionalPrinttrue % Use this value if you want to have extra border for cutting and are not bound to paper formats (like A4). This option will increase the page size by the extra border on every side plus the binding correction once for the width. It only takes effect in combination with \printVersiontrue.
%    \fancyTheoremsfalse  % Use this value if you want to use the classical theorem style, where the text is italic. Further, with this style, the QED symbol is colorless.
%    \boldNumberSetsfalse % Use this value if you want variables for number sets (like N or R) to appear in blackboard bold rather than bold.

    % The title of the thesis.
    \myTitle{Conception, implémentation et preuve d'un service de transfert de flôt d'exécution au sein d'un noyau de système d'exploitation}

    % The author’s name.
    \myName{Florian Vanhems}

    % The author’s program.
    \myProgram{Informatique}

    % A short summary of the thesis. These information will be used for the PDF meta data.
    \mySubject{A cool bachelor/master thesis.}

    % Some keywords of the thesis. These information will be used for the PDF meta data. Please use | as a separator and try to avoid commas.
    \myKeywords{bachelor--master thesis | world-changing | very important | please like and share and subscribe}

%%%%%%%%%%%%%%%%%%%%%%%%%%%%%%%%%%%%%%
%% End of options to adjust. %%%%%%%%%
%%%%%%%%%%%%%%%%%%%%%%%%%%%%%%%%%%%%%%


% This file includes all of the code that is used to format the thesis.
% Some packages are included if they are needed. This is done in the respective part and not at the beginning of this file.
%
% This file contains the following parts:
%   • Language an Character Set
%   • Penalties
%   • Indentation
%   • Footnotes
%   • Colors
%   • Size and Position of the Text Body
%   • Position of the Head and the Foot
%   • Margin Position and Width
%   • Header and Footer Format
%   • Caption Format
%   • Part Format
%   • Chapter Format
%   • Table of Contents


%%%%%%%%%%%%%%%%%%%%%%%%%%%%%%%%
%% Language and Character Set %%
%%%%%%%%%%%%%%%%%%%%%%%%%%%%%%%%

\usepackage
[
    english,         % English is used for the English abstract.
    main = french, % This is the main language of the thesis.
]
{babel}                     % Is responsible for sensible hyphenations.

%%%%%%%%%%%%%%%
%% Penalties %%
%%%%%%%%%%%%%%%

\widowpenalties 2 10000 0


%%%%%%%%%%%%%%%%%
%% Indentation %%
%%%%%%%%%%%%%%%%%

\usepackage{calc} % Makes it easer to do math with TeX measurements.

\newlength{\myparindent}
\newlength{\myparskip}
\setlength{\myparindent}{1 em}
\setlength{\myparskip}{0 em}

\setlength{\parindent}{\myparindent}
\setlength{\parskip}{\myparskip}
\setlength{\parskip}{0 pt plus 1 pt minus 0 pt}


%%%%%%%%%%%%%%%
%% Footnotes %%
%%%%%%%%%%%%%%%

% Remove the footnote rule.
\setfootnoterule{0 cm}

% The footnote number is made bold and not in superscript.
\deffootnote[1.2 em]{1.2 em}{0 em}{\makebox[1.4 em][l]{\textbf{\thefootnotemark}}}

% The footnote number will not be reset after every chapter.
\makeatletter%
    \@removefromreset{footnote}{chapter}%
\makeatother


%%%%%%%%%%%%
%% Colors %%
%%%%%%%%%%%%

\usepackage[dvipsnames]{xcolor} % Allows it to define colors. The option says that common names can be used.

% Dark blue.
\definecolor{stroke1}{HTML}{2574A9} % This color is used as the standard color to highlight things.


% Coloring various different labels.
\colorlet{captionlabel}{black}
\colorlet{footerpagenr}{black}
\colorlet{footerchapter}{stroke1}
\colorlet{footerchaptername}{black}
\colorlet{footersection}{stroke1}
\colorlet{footersectionname}{black}
\colorlet{chapternumber}{stroke1}


%%%%%%%%%%%%%%%%%%%%%%%%%%%%%%%%%%%%%%%%
%% Size and Position of the Text Body %%
%%%%%%%%%%%%%%%%%%%%%%%%%%%%%%%%%%%%%%%%

% The new paper dimensions that are exclusively used.
\newlength{\mypaperwidth}
\setlength{\mypaperwidth}{210 mm}

\newlength{\mypaperheight}
\setlength{\mypaperheight}{297 mm}

% The text area uses aesthetically pleasing measurements in the same ratio as the page.
% These dimensions are always used, as the text area should be the same in the printed and digital version of the thesis.
\newlength{\mybodywidth}
\setlength{\mybodywidth}{140 mm}

\newlength{\mybodyheight}
\setlength{\mybodyheight}{198 mm}

\newlength{\myoutermargin}
\ifprintVersion
    \ifprofessionalPrint
        \setlength{\myoutermargin}{(\mypaperwidth - \mybodywidth) / \real{1.5} + \extraborderlength}
    \else
        \setlength{\myoutermargin}{(\mypaperwidth - \mybodywidth) / \real{1.5} - \mybindingcorrection}
    \fi
\else
    \setlength{\myoutermargin}{(\mypaperwidth - \mybodywidth) / \real{1.5}}
\fi

\newlength{\mytopmargin}
\setlength{\mytopmargin}{(\mypaperheight - \mybodyheight) / 3}
\ifprintVersion
    \ifprofessionalPrint
        \setlength{\mytopmargin}{(\mypaperheight - \mybodyheight) / 3 + \extraborderlength}
    \fi
\fi

\newlength{\myinnermargin}
\setlength{\myinnermargin}{\mypaperwidth - \mybodywidth - \myoutermargin}
\ifprintVersion
    \ifprofessionalPrint
        \setlength{\myinnermargin}{\mypaperwidth + \mybindingcorrection + 2\extraborderlength - \mybodywidth - \myoutermargin}
    \fi
\fi

\newlength{\mybottommargin}
\setlength{\mybottommargin}{\mypaperheight - \mybodyheight - \mytopmargin}
\ifprintVersion
    \ifprofessionalPrint
        \setlength{\mybottommargin}{\mypaperheight + 2\extraborderlength - \mybodyheight - \mytopmargin}
    \fi
\fi


%%%%%%%%%%%%%%%%%%%%%%%%%%%%%%%%%%%
%% Position of the Head And Foot %%
%%%%%%%%%%%%%%%%%%%%%%%%%%%%%%%%%%%

\newcommand{\goldenratio}{1.618}

\newlength{\myheadsep} % Distance from the header to the body.
\setlength{\myheadsep}{\mytopmargin / \real{\goldenratio} / \real{\goldenratio} - 1 ex}
\ifprintVersion
    \ifprofessionalPrint
        \setlength{\myheadsep}{(\mytopmargin - \extraborderlength) / \real{\goldenratio} / \real{\goldenratio} - 1 ex}
    \fi
\fi

\newlength{\myfootskip} % Distance from the body to the footer.
\setlength{\myfootskip}{\mybottommargin / \real{\goldenratio} - 1 ex}
\ifprintVersion
    \ifprofessionalPrint
        \setlength{\myfootskip}{(\mybottommargin - \extraborderlength) / \real{\goldenratio} - 1 ex}
    \fi
\fi


%%%%%%%%%%%%%%%%%%%%%%%%%%%%%%%
%% Margin Position And Width %%
%%%%%%%%%%%%%%%%%%%%%%%%%%%%%%%

\newlength{\mymargininnersep} % Distance between the margin and the body.
\setlength{\mymargininnersep}{7 mm}

\newlength{\mymarginoutersep} % Distance between the margin and the paper border.
\setlength{\mymarginoutersep}{12 mm}
\ifprintVersion
    \ifprofessionalPrint
        \setlength{\mymarginoutersep}{12 mm + \extraborderlength}
    \fi
\fi

\newlength{\mymarginwidth} % Width of the margin.
\setlength{\mymarginwidth}{\myoutermargin - \mymargininnersep - \mymarginoutersep}

\newlength{\mymarginwidthwithinnersep} % Width of the margin.
\setlength{\mymarginwidthwithinnersep}{\mymarginwidth + \mymargininnersep}

\usepackage
[
    % In the printed version, we add an extra border to each side as well as the binding correction for the width.
    \ifprintVersion
        \ifprofessionalPrint
            paperwidth = \mypaperwidth + 2\extraborderlength + \mybindingcorrection,
            paperheight = \mypaperheight + 2\extraborderlength,
        \else
            paperwidth = \mypaperwidth,
            paperheight = \mypaperheight,
        \fi
    \else
        paperwidth = \mypaperwidth,
        paperheight = \mypaperheight,
    \fi
    textwidth = \mybodywidth,
    textheight = \mybodyheight,
    outer = \myoutermargin,
    top = \mytopmargin,
    headsep = \myheadsep,
    footskip = \myfootskip,
    marginparsep = \mymargininnersep,
    marginparwidth = \mymarginwidth,
%    showframe, % Use this option for debugging purposes in order to the an outline of all of the different parts of the page layout.
]
{geometry} % Used in order to define the dimensions of the page and its layout.


%%%%%%%%%%%%%%%%%%%%%%%%%%%%%%
%% Header and Footer Format %%
%%%%%%%%%%%%%%%%%%%%%%%%%%%%%%

\usepackage
[
%    draft, % Shows a lot of rules denoting the dimensions of the head and foot. Use this option only for debugging.
]
{scrlayer-scrpage} % Allows to adjust the definitions of the head and foot of a page.

%%%%%%%%%%%%%%%%%%%%%%%%%%%%%%
% Dimensions and formats are defined.

% Define the dimensions of the head and the foot. Since we want some information to appear in the margin, we extend the head and the foot by the respective lengths.
\KOMAoptions
{%
    headwidth = \textwidth + \mymarginwidthwithinnersep,%
    footwidth = \myoutermargin : \textwidth,%
}

% Defines the formats for the chapter and section titles in the marks of the head.
\renewcommand*{\chaptermarkformat}{\normalfont\sffamily\small\color{footerchaptername}}
\renewcommand*{\sectionmarkformat}{\normalfont\sffamily\small\color{footersectionname}}

% Displays the chapter names in the head of both odd and even pages.
\automark[chapter]{chapter}
% Replaces the chapter name to the head of right pages with the section name if a section is present.
\automark*[section]{}

%%%%%%%%%%%%%%%%%%%%%%%%%%%%%%
% The head is defined.

% Head for even pages.
% Puts ›Chapter‹ followed by the current chapter number.
\lehead%
{%
    \begin{minipage}[b]{\mymarginwidth}%
        \small\raggedleft\normalfont\textsf{\textbf{\color{footerchapter}\chaptername\ \thechapter}}
    \end{minipage}
}
% Put the title of the current chapter/section into the center of the head but push it to the border.
\cehead{\hspace*{\mymarginwidthwithinnersep}\parbox{\textwidth}{\raggedright\leftmark}}

% Head for odd pages.
\rohead%
{%
    % Check whether a section has already started or not.
    \Ifstr{\rightmark}{\leftmark}%
    {%
        \begin{minipage}[b]{\mymarginwidth}%
            \small\raggedright\normalfont\textsf{\textbf{\color{footersection}Chapter\ \thechapter}}%
        \end{minipage}%
    }%
    {%
        \begin{minipage}[b]{\mymarginwidth}%
            \small\raggedright\normalfont\textsf{\textbf{\color{footersection}Section\ \thesection}}%
        \end{minipage}%
    }%
}
\cohead{\hspace*{-\mymarginwidthwithinnersep}\parbox{\textwidth}{\raggedleft\rightmark}}

%%%%%%%%%%%%%%%%%%%%%%%%%%%%%%
% The foot is defined.

% Displays the page number in bold in the margin, aligned toward the center. Further, a blue line is drawn above number.
% The starred variant is used, since we want the format of the foot to also apply to the pagestyle ›plain‹.
\lefoot*%
{%
    \vspace*{1 ex}%
    {\color{stroke1}\rule{\myoutermargin - \mymargininnersep}{0.5 mm}}\\
    \begin{minipage}[b]{\myoutermargin - \mymargininnersep}%
        \raggedleft\normalfont\color{footerpagenr}\textbf{\thepage}%
    \end{minipage}%
}
\rofoot*%
{%
    {\color{stroke1}\rule{\myoutermargin - \mymargininnersep}{0.5 mm}}\\
    \begin{minipage}[b]{\myoutermargin - \mymargininnersep}%
        \raggedright\normalfont\color{footerpagenr}\textbf{\thepage}%
    \end{minipage}%
}


%%%%%%%%%%%%%%%%%%%%
%% Caption Format %%
%%%%%%%%%%%%%%%%%%%%

\usepackage[justification=centering]{caption}
\captionsetup
{
    font = small,
    labelfont = {bf, sf, color = captionlabel},
    format = plain,
    singlelinecheck = off,
}

%%%%%%%%%%%%%%%%%
%% Part Format %%
%%%%%%%%%%%%%%%%%
\usepackage{tikz} % Used in order to draw the stylistic elements.

\usetikzlibrary{arrows,calc,positioning,patterns,shadows} 
\tikzset{
    %Define standard arrow tip
    >=stealth'
}

%define hatched pattern
\pgfdeclarepatternformonly{south west lines}{\pgfqpoint{-0pt}{-0pt}}{\pgfqpoint{3pt}{3pt}}{\pgfqpoint{3pt}{3pt}}{
        \pgfsetlinewidth{0.4pt}
        \pgfpathmoveto{\pgfqpoint{0pt}{0pt}}
        \pgfpathlineto{\pgfqpoint{3pt}{3pt}}
        \pgfpathmoveto{\pgfqpoint{2.8pt}{-.2pt}}
        \pgfpathlineto{\pgfqpoint{3.2pt}{.2pt}}
        \pgfpathmoveto{\pgfqpoint{-.2pt}{2.8pt}}
        \pgfpathlineto{\pgfqpoint{.2pt}{3.2pt}}
        \pgfusepath{stroke}}

\newlength{\mytmpa}
\setlength{\mytmpa}{1 mm}
\newlength{\mytmpb}
\newlength{\mytmpc}

%%%%%%%%%%%%%%%%%
% The following code draws the outline for a ›part‹ of the thesis.
% This command is used before the name of the part is displayed. It is void, as the part is added via \partlineswithprefixformat.
% Unfortunately this redefinition is needed every time the language switched back to french so it has its own file
\renewcommand*{\partformat}{}
% This command calls \partformat (#2) and displays the name of the part (#3).
\renewcommand*{\partlineswithprefixformat}[3]%
{%
    #2
    \thispagestyle{empty}
    \setlength{\mytmpa}{0.618\mypaperwidth}%
    \setlength{\mytmpb}{0.382\mypaperheight}%
    \ifprintVersion
        \ifprofessionalPrint
            \setlength{\mytmpa}{0.618\mypaperwidth + \mybindingcorrection + \extraborderlength}%
            \setlength{\mytmpb}{0.382\mypaperheight + \extraborderlength}%
        \fi
    \fi
    \begin{tikzpicture}[overlay, remember picture]%
        \node [inner sep = 0, outer sep = 0, anchor = north] at (current page.north west)%
        {%
            \begin{tikzpicture}[overlay, remember picture]%
            \draw[color = stroke1, line width = 0.7 mm] (\mytmpa, 0) -- (\mytmpa, -\mytmpb);%
            \end{tikzpicture}%
        };%
        \node (align) [align = right, below = \mytmpb - 2 ex, inner sep = 0, outer sep = 0, anchor = north west] at (current page.north west)%
        {%
            %\hspace{\mytmpa}\hspace{0.5 em}\partname\ \thepart\\[1 ex]
            \hspace{\mytmpa}\hspace{0.5 em}\partname\\[1 ex]
            \color{stroke1}#3%
        };%
    \end{tikzpicture}%
}


% This command defines various parameters for the ›part‹ format.
\RedeclareSectionCommand%
[%
    font = \normalfont\Huge\sffamily,
    prefixfont = \normalfont\Huge\sffamily,
]
{part}

%%%%%%%%%%%%%%%%%%%%%
%%% Chapter Format %%
%%%%%%%%%%%%%%%%%%%%%

\usepackage{etoolbox}

\newbool{chapterHasANumber}
\newbool{chapterHasAStar}
\renewcommand*{\chapterlinesformat}[3]%
{%
    % Check whether \chapter of \addchap has been used.
    \Ifnumbered{#1}{\setbool{chapterHasANumber}{true}}{\setbool{chapterHasANumber}{false}}%
    % Check whether \chapter* or \chapter has been used.
    \Ifstr{#2}{}{\setbool{chapterHasAStar}{true}}{\setbool{chapterHasAStar}{false}}%
    % Check whether a normal \chapter or something else is used.
    \ifboolexpr{bool{chapterHasANumber} and not bool{chapterHasAStar}}%
    {%
        \begin{tikzpicture}[overlay, remember picture]%
            \node [right = \myinnermargin, below = \mytopmargin, inner sep = 0, outer sep = 0, anchor = north west] (numbernode) at (current page.north west)%
            {%
                \hspace{\myinnermargin}%
                \sffamily\fontsize{60}{60}\selectfont%
                \color{chapternumber}%
                \thechapter%
            };%
            \node [inner sep = 0, outer sep = 0, anchor = north west] at (numbernode.south west)%
            {%
                \begin{tikzpicture}[overlay, remember picture]%
                    \draw[color = stroke1, line width = 0.7 mm] (\myinnermargin, -1 ex) -- (\paperwidth, -1 ex);%
                \end{tikzpicture}%
            };%
            \node (align) [text width = \textwidth - 2 cm, align = right, right = \myinnermargin + \mybodywidth, inner sep = 0, outer sep = 0, anchor = east] at (numbernode.west)%
            {%
                #3%
            };%
        \end{tikzpicture}%
    }%
    {%
        \begin{tikzpicture}[overlay, remember picture]%
            \node [right = \myinnermargin, below = \mytopmargin, inner sep = 0, outer sep = 0, anchor = north west] (numbernode) at (current page.north west)%
            {%
                \hspace{\myinnermargin}%
                \sffamily\fontsize{60}{60}\selectfont%
                \color{white}%
                \thechapter%
            };%
            \node [inner sep = 0, outer sep = 0, anchor = north west] at (numbernode.south west)%
            {%
                \begin{tikzpicture}[overlay, remember picture]%
                    \draw[color = stroke1, line width = 0.7 mm] (\myinnermargin, -1 ex) -- (\paperwidth, -1 ex);%
                \end{tikzpicture}%
            };%
            \node (align) [align = left, right = \myinnermargin, inner sep = 0, outer sep = 0, anchor = south west] at (numbernode.south west)%
            {%
                #3%
            };%
        \end{tikzpicture}%
    }%
}
\RedeclareSectionCommand%
[%
    font = \color{stroke1}\normalfont\huge\sffamily,
    afterskip = 20 pt,
]
{chapter}


%%%%%%%%%%%%%%%%%%%%%%%%
%%% Table of Contents %%
%%%%%%%%%%%%%%%%%%%%%%%%

% Format the table of contents to have a ›plain‹ page style.
\BeforeStartingTOC[toc]{\pagestyle{plain}}
\AfterStartingTOC{\thispagestyle{plain}}
                        % Contains commands that define the general format and layout of the thesis.
% This file contains all of the code that formats the bibliography. Since be package ›biblatex‹ is used, the bibliography needs to be compiled with ›biber‹.
%
% This file contains the following parts:
%   • Resources
%   • Redefined Keywords
%   • Coloring
%   • Format of the Entries
%   • Format of the Own Publications


\usepackage
[
    sortcites,              % Sort multiple references when citing them together.
    style = alphabetic,     % The style of a citation mark.
    defernumbers,           % Makes sure that references always have unique numbers. This is important if you use multiple bibliographies.
    safeinputenc,           % Allows to use UTF8 characters in the bibliography and tries to translate them into TeX automatically.
    backref = true,         % Creates back references in the bibliography.
    backrefstyle = three,   % Compresses three or more consecutive pages in the back references into a range.
    hyperref = true,        % Makes links generated by biblatex clickable. If hyperref is not used, a warning is issued.
    maxbibnames = 99,       % The maximum number of names displayed in the bibliography.
    maxcitenames = 2,       % The maximum number of names displayed when using commands like ›textcite‹. The default is 3. After that, ›et al.‹ is used.
%    useprefix,              % Prints name prefixes, such as ›von‹. The default is false. This means that prefixes are not considered to be part of the last name.
]
{biblatex} % Used in order to format the bibliography.

% The following command changes the space between the list of authors and the citation mark into a non-breaking space.
\renewcommand\namelabeldelim{\addnbspace}


%%%%%%%%%%%%%%%
%% Resources %%
%%%%%%%%%%%%%%%

\addbibresource{references/strings.bib}                     % Contains many strings for common conference names etc. These strings can then be used in the references.
\addbibresource{references/references.bib}                  % The file that contains the references that are used for the thesis.


%%%%%%%%%%%%%%%%%%%%%%%%
%% Redefined Keywords %%
%%%%%%%%%%%%%%%%%%%%%%%%

\renewbibmacro{in:}%
{%
    \ifentrytype{article}{}{\printtext{\bibstring{in}\intitlepunct}}%
}
% \renewcommand*{\volumenumberdelim}{\addcolon}

\renewbibmacro*{volume+number+eid}%
{%
    \printfield{volume}%
    \iffieldundef{number}{}{\addcolon}%
    %  \setunit*{\addnbthinspace}%
    \printfield{number}%
    \setunit*{\addcomma\space}%
    \printfield{eid}%
}

\DefineBibliographyStrings{english}%
{%
    backrefpage  = {\lowercase{s}ee page}, % For a single page number.
    backrefpages = {\lowercase{s}ee pages} % For multiple page numbers.
}


%%%%%%%%%%%%%%
%% Coloring %%
%%%%%%%%%%%%%%

\DeclareFieldFormat[article]{title}{\textbf{\color{stroke1}#1}}
\DeclareFieldFormat[inproceedings]{title}{\textbf{\color{stroke1}#1}}
\DeclareFieldFormat[thesis]{title}{\textbf{\color{stroke1}#1}}
\DeclareFieldFormat[book]{title}{\textbf{\color{stroke1}#1}}
\DeclareFieldFormat[unpublished]{title}{\textbf{\color{stroke1}#1}}
\DeclareFieldFormat[report]{title}{\textbf{\color{stroke1}#1}}
\DeclareFieldFormat[inbook]{chapter}{\textbf{\color{stroke1}#1}}
\DeclareFieldFormat[inbook]{title}{#1}
\DeclareFieldFormat{pages}{#1}


%%%%%%%%%%%%%%%%%%%%%%%%%%%
%% Format of the Entries %%
%%%%%%%%%%%%%%%%%%%%%%%%%%%

% The following toggle defines how the citation mark formats the author names. If this toggle is true, more information is used.
\newtoggle{authorend}
\togglefalse{authorend}

% Article
\DeclareBibliographyDriver{article}%
{%
  \usebibmacro{bibindex}%
  \usebibmacro{begentry}%
  \iftoggle{authorend}{}{\usebibmacro{author/translator+others}}%
  \setunit{\labelnamepunct}\newblock
  \usebibmacro{title}%
  \newunit
  \printlist{language}%
  \newunit\newblock
  \usebibmacro{byauthor}%
  \newunit\newblock
  \usebibmacro{bytranslator+others}%
  \newunit\newblock
  \printfield{version}%
  \newunit\newblock
  \usebibmacro{in:}%
  \usebibmacro{journal+issuetitle}%
  \newunit
  \usebibmacro{byeditor+others}%
  \newunit
  \usebibmacro{note+pages}%
  \newunit\newblock
  \iftoggle{bbx:isbn}
  {\printfield{issn}}
  {}%
  \newunit\newblock
  \usebibmacro{doi+eprint+url}%
  \newunit\newblock
  \usebibmacro{addendum+pubstate}%
  \setunit{\bibpagerefpunct}\newblock
  \usebibmacro{pageref}%
  \newunit\newblock
  \iftoggle{bbx:related}
  {\usebibmacro{related:init}%
    \usebibmacro{related}}
  {}%
  \usebibmacro{finentry}%
  \iftoggle{authorend}{\usebibmacro{author/translator+others}}{}%
}

% Book Chapter
\DeclareBibliographyDriver{inbook}%
{%
  \usebibmacro{bibindex}%
  \usebibmacro{begentry}%
  \iftoggle{authorend}{}{\usebibmacro{author/translator+others}}%
  \setunit{\labelnamepunct}\newblock
  % \usebibmacro{title}%
  \usebibmacro{chapter+pages}%
  % \printfield{chapter}%
  \newunit
  \printlist{language}%
  \newunit\newblock
  \usebibmacro{byauthor}%
  \newunit\newblock
  \usebibmacro{in:}%
  \usebibmacro{bybookauthor}%
  \newunit\newblock
  \usebibmacro{maintitle+booktitle}%
  \newunit\newblock
  \usebibmacro{byeditor+others}%
  \newunit\newblock
  \printfield{edition}%
  \newunit
  \iffieldundef{maintitle}
  {\printfield{volume}%
    \printfield{part}}
  {}%
  \newunit
  \printfield{volumes}%
  \newunit\newblock
  \usebibmacro{series+number}%
  \newunit\newblock
  \printfield{note}%
  \newunit\newblock
  \usebibmacro{publisher+location+date}%
  \newunit\newblock
  % \usebibmacro{chapter+pages}%
  \newunit\newblock
  \iftoggle{bbx:isbn}
  {\printfield{isbn}}
  {}%
  \newunit\newblock
  \usebibmacro{doi+eprint+url}%
  \newunit\newblock
  \usebibmacro{addendum+pubstate}%
  \setunit{\bibpagerefpunct}\newblock
  \usebibmacro{pageref}%
  \newunit\newblock
  \iftoggle{bbx:related}
  {\usebibmacro{related:init}%
    \usebibmacro{related}}
  {}%
  \usebibmacro{finentry}%
  \iftoggle{authorend}{\usebibmacro{author/translator+others}}{}%
}

% Proceedings Article
\DeclareBibliographyDriver{inproceedings}%
{%
  \usebibmacro{bibindex}%
  \usebibmacro{begentry}%
  \iftoggle{authorend}{}{\usebibmacro{author/translator+others}}%
  \setunit{\labelnamepunct}\newblock
  \usebibmacro{title}%
  \newunit
  \printlist{language}%
  \newunit\newblock
  \usebibmacro{byauthor}%
  \newunit\newblock
  \usebibmacro{in:}%
  \usebibmacro{maintitle+booktitle}%
  \newunit\newblock
  \usebibmacro{event+venue+date}%
  \newunit\newblock
  \usebibmacro{byeditor+others}%
  \newunit\newblock
  \iffieldundef{maintitle}
  {\printfield{volume}%
    \printfield{part}}
  {}%
  \newunit
  \printfield{volumes}%
  \newunit\newblock
  \usebibmacro{series+number}%
  \newunit\newblock
  \printfield{note}%
  \newunit\newblock
  \printlist{organization}%
  \newunit
  \usebibmacro{publisher+location+date}%
  \newunit\newblock
  \usebibmacro{chapter+pages}%
  \newunit\newblock
  \iftoggle{bbx:isbn}
  {\printfield{isbn}}
  {}%
  \newunit\newblock
  \usebibmacro{doi+eprint+url}%
  \newunit\newblock
  \usebibmacro{addendum+pubstate}%
  \setunit{\bibpagerefpunct}\newblock
  \usebibmacro{pageref}%
  \newunit\newblock
  \iftoggle{bbx:related}
  {\usebibmacro{related:init}%
    \usebibmacro{related}}
  {}%
  \usebibmacro{finentry}%
  \iftoggle{authorend}{\usebibmacro{author/translator+others}}{}%
}

% Thesis
\DeclareBibliographyDriver{thesis}%
{%
  \usebibmacro{bibindex}%
  \usebibmacro{begentry}%
  \iftoggle{authorend}{}{\usebibmacro{author}}%
  \setunit{\labelnamepunct}\newblock
  \usebibmacro{title}%
  \newunit
  \printlist{language}%
  \newunit\newblock
  \usebibmacro{byauthor}%
  \newunit\newblock
  \printfield{note}%
  \newunit\newblock
  \printfield{type}%
  \newunit
  \usebibmacro{institution+location+date}%
  \newunit\newblock
  \usebibmacro{chapter+pages}%
  \newunit
  \printfield{pagetotal}%
  \newunit\newblock
  \iftoggle{bbx:isbn}
  {\printfield{isbn}}
  {}%
  \newunit\newblock
  \usebibmacro{doi+eprint+url}%
  \newunit\newblock
  \usebibmacro{addendum+pubstate}%
  \setunit{\bibpagerefpunct}\newblock
  \usebibmacro{pageref}%
  \newunit\newblock
  \iftoggle{bbx:related}
  {\usebibmacro{related:init}%
    \usebibmacro{related}}
  {}%
  \usebibmacro{finentry}%
  \iftoggle{authorend}{\usebibmacro{author}}{}%
}


%%%%%%%%%%%%%%%%%%%%%%%%%%%%%%%%%%%%
%% Format of the Own Publications %%
%%%%%%%%%%%%%%%%%%%%%%%%%%%%%%%%%%%%

% The own publications are formatted using a numeric list, whereas the bibliography of the thesis uses an alphanumeric style.

% Copied from numeric.cbx in order to imitate numerical citations.
\providebool{bbx:subentry}
\newbibmacro*{citenum}%
{% Note: the original macro was called ›cite‹. I did not redefine ›cite‹ but instead defined a new macro ›citenum‹ because the author-year citations use the ›cite‹ macro too. Using ›\renewbibmacro*{cite}‹ would have caused all the author-year citations to become numeric too.
  \printtext[bibhyperref]{% If you ever want to use hyperref.
    \printfield{prefixnumber}%
    \printfield{labelnumber}%
    \ifbool{bbx:subentry}
    {\printfield{entrysetcount}}
    {}}%
}

% Copied from numeric.cbx to define a new numeric citation command for @online entries.
\DeclareCiteCommand{\conline}[\mkbibbrackets]
{\usebibmacro{prenote}}
{\usebibmacro{citeindex}%
  \usebibmacro{citenum}}% Note: this was originally "cite" but I changed it to "citenum" to avoid clashes with the author-year style.
{\multicitedelim}
{\usebibmacro{postnote}}
       % Contains commands for the layout of the bibliography.
% This file contains most of the packages used for this document. If you want to add a package, do it below the appropriate ribbon.
% Some packages are already included in other files in the ›settings‹ folder if they were already necessary. Thus, make sure to go through these files too if you want to know whether a certain package is already included.
%
% This file contains the following parts:
%   • Typography
%   • Math
%   • Fonts
%   • Graphics
%   • Tables
%   • Enumerations
%   • Algorithms
%   • Spaces and Special Characters
%   • Miscellaneous
%   • Additional Packages
%   • Hyperlinks

%%%%%%%%%%
%% Math %%
%%%%%%%%%%

% The following packages are the standard packages used in order to typeset math. They contain a lot of useful commands.
\usepackage{amsmath}
\usepackage{amssymb}
\usepackage{amsthm}
\usepackage{thmtools}
\usepackage{mathtools}
\usepackage{thm-restate}
\usepackage{dsfont}        % Yields far better blackboard-bold letters than \mathbb. Use \mathds in order to write such letters.
\usepackage{braceMnSymbol} % Adjusts overbraces and underbraces such that longer versions are put together seamlessly.


%%%%%%%%%%%
%% Fonts %%
%%%%%%%%%%%

\usepackage
[
    mono=false, % Disables the mono/typewriter font.
]
{libertinus-otf} % The main font used in this thesis.

\usepackage{fontspec}  % Package required to load custom fonts
\setmonofont{FiraCode} % Use Fira Code as default mono font
[
	Extension = .ttf,
	Path = fonts/Fira-Code/,
	UprightFont = *-Regular,
	BoldFont = *-Bold,
	ItalicFont = *-Light,
	Scale = 0.85
]

\usepackage{url} % Responsible for URL formatting.
\usepackage{bm}  % Allows to use sensible bold letters in math mode. This package has to go after the font packages. Otherwise it does not work correctly!

%%%%%%%%%%%%%%
%% Graphics %%
%%%%%%%%%%%%%%

\usepackage{graphicx} % The standard package for including graphics into your document.
\usepackage
[
    subrefformat = simple, % Formats the label of the \subref command without parentheses.
    labelformat = simple,  % Formats the mark of a subfigure without parentheses.
]
{subcaption}         % Enables it to have subfigures inside of a single figure.
\usepackage{wrapfig} % Allows to put figures next to text.

% Changing the \columnsep adds some space next to a warpfigure.
\columnsep = \mymargininnersep
% The reference label of a subfigure is redefined to have a non-breaking space and parentheses. (Thus, the subfigures show parentheses although the package options removed parentheses; otherwise, two pairs of brackets would be seen.)
\renewcommand*{\thesubfigure}{~(\alph{subfigure})}

% tikz has already been included in the file "settings/format.tex"

%%%%%%%%%%%%
%% Tables %%
%%%%%%%%%%%%

\usepackage{array}     % Improves the way that tables can be formatted.
\usepackage{booktabs}  % Adds lines (called ›rules‹) that can be used in tables and improves spacing.
\usepackage{longtable} % Allows to make tables that span multiple pages.
\usepackage{pdflscape} % Allows to change a page into landscape. This is handy if a table is very wide.


%%%%%%%%%%%%%%%%%%
%% Enumerations %%
%%%%%%%%%%%%%%%%%%

\usepackage{enumitem} % Adds tons of useful features to enumeration environments.


%%%%%%%%%%%%%%%%
%% Algorithms %%
%%%%%%%%%%%%%%%%

\usepackage[outputdir=build, newfloat=true]{minted} % WARNING : Requires shell escape
\newmintedfile[coqcode]{coq}{
	fontfamily=tt, % select the correct font family
	linenos=false,
	numberblanklines=true,
	numbersep=5pt,
	gobble=0,
	frame=lines,
	framerule=0.4pt,
	framesep=2mm,
	funcnamehighlighting=true,
	tabsize=4,
	obeytabs=false,
	mathescape=true
	samepage=false, %with this setting you can force the list to appear on the same page
	showspaces=false,
	showtabs =false,
	texcl=false,
}

%%%%%%%%%%%%%%%%%%%%%%%%%%%%%%%%%%%
%% Spaces and Special Characters %%
%%%%%%%%%%%%%%%%%%%%%%%%%%%%%%%%%%%

\usepackage{xspace}   % Adds the functionality that a space after a command will be shown as a space in the output.
\usepackage
[
    shortcuts, % Allows to use short symbols for non-breaking hyphens and dashes instead of lengthy commands.
]
{extdash}             % Adds non-breaking hyphens and dashes.
\usepackage{setspace} % Allows to easily chnage the spacing inside of the document.

%%%%%%%%%%%%%%%%
%% Typography %%
%%%%%%%%%%%%%%%%

\usepackage
[
    babel = true, % Enables language-specific tuning.
]
{microtype}           % Uses the text space more efficiently.
\usepackage{csquotes} % Uses the correct quotes according to the current language.

%%%%%%%%%%%%%%%%%%%
%% Miscellaneous %%
%%%%%%%%%%%%%%%%%%%

\usepackage{xparse}    % Is used in order to define reasonable commands.
\usepackage{footnote}  % Allows it to extend the environments footnotes can be used in. It is said that this package is in conflict with ›hyperref‹. I did not note any troubles. However, if something is fishy, it is probably best to not use this package.
\usepackage{afterpage} % Adds the \afterpage command, which specifies that the provided argument shall be processed after the current page is finished.
\usepackage
[
    textsize = scriptsize, % Determines the text size of the TODO note.
]
{todonotes}            % Adds TODO notes to the document. These are small text areas inside of the margin of a page.

%%%%%%%%%%%%%%%%%%%%%%%%%%%%%%%%%%%%%%%%%%%%%%%%%%%%%%%%%%%%%%%%%
%% If you want to add new packages, add them below this ribbon %%
%%%%%%%%%%%%%%%%%%%%%%%%%%%%%%%%%%%%%%%%%%%%%%%%%%%%%%%%%%%%%%%%%


%%%%%%%%%%%%%%%%
%% Hyperlinks %%
%%%%%%%%%%%%%%%%

\usepackage
[
    bookmarks = true,                 % Generates boodmarks for the PDF.
    bookmarksopen = false,            % The bookmarks are closed by default.
    bookmarksnumbered = true,         % The bookmarks use the numbers of the corresponding headline.
    pdfstartpage = 1,                 % The first page seen when opening the PDF.
    pdftitle = {{\printTitle}},       % The PDF’s title in the meta data.
    pdfauthor = {{\printAuthor}},     % The PDF’s author name in the meta data.
    pdfsubject = {{\printSubject}},   % The PDF’s subject in the meta data.
    pdfkeywords = {{\printKeywords}}, % The PDF’s keywords in the meta data.
    breaklinks = true,                % Allows it to break links.
    \ifprintVersion
        hidelinks,                    % In the printed version, links are not highlighted, as they are not clickable.
    \else
    colorlinks = true,            % The text of hyperlinks is colored instead of having a colored box around it.
    allcolors = stroke1,          % Every hyperlink uses the same color. If you want to change specific colors, use the commands below.
    %        linkcolor = stroke1,          % The color of an in-document hyperlink.
    %        citecolor = stroke1,          % The color of a citation.
    %        filecolor = stroke1,          % The color of a file link.
    %        pagecolor = stroke1,          % The color of a reference to a page.
    %        urlcolor = stroke1,           % The color of a weblink.
    \fi
]
{hyperref} % The standard package that is used for creating hyperlinks inside of a document.

\usepackage
[
    %    capitalise, % Capitalizes the words in front of the labels. This can also be done by simply using \Cref instead of \cref. In order to have a greater variety, this option is not used.
    noabbrev,   % The words in front of the labels are not abbreviated.
    nameinlink, % Extends the link of a reference to the word in front of it.
]
{cleveref} % This package must be included after ›hyperref‹. It creates clever references that know what they refer to.
     % Contains the packages that this template provides.
% This file contains all sorts of macros that are globally used. Further, certain options made available through packages are set here as well.
%
% This file contains the following parts:
%   • Type of Degree
%   • Miscellaneous
%   • Footnotes
%   • Theorem Environments
%   • Meta Commands
%   • Common Commands


%%%%%%%%%%%%%%%%%%%%
%% Type of Degree %%
%%%%%%%%%%%%%%%%%%%%

% The colloquial term of the degree.
\newcommand*{\colloquialDegreeName}{Philosophal Doctorate}
\newcommand*{\colloquialDegreeNameLowercase}{philosophal doctorate}

% The abbreviation of the degree.
\newcommand*{\degreeAbbreviation}{Ph. D.}


%%%%%%%%%%%%%%%%%%%
%% Miscellaneous %%
%%%%%%%%%%%%%%%%%%%

% Defines the environment used at the beginning of each chapter.
\newenvironment{jointwork}
{\itshape}
{\ignorespacesafterend\bigskip}

% Defines the IfEmptyTF command. This is useful for optional arguments provided as [].
\makeatletter
    \def\IfEmptyTF#1%
    {%
        \if\relax\detokenize{#1}\relax%
            \expandafter\@firstoftwo%
        \else%
            \expandafter\@secondoftwo%
        \fi%
    }
\makeatother

% Creates an environment that automatically uses math mode if necessary and creates a space afterward if wanted. Basically, if the command \example is defined to use this environment, you can use \example without mathe mode in normal text as if it were ordinary text.
\NewDocumentCommand{\mathOrText}{m}
{%
    \ensuremath{#1}\xspace%
}

% Reduces the space around scaling bracekts.
\let\originalleft\left
\let\originalright\right
\renewcommand{\left}{\mathopen{}\mathclose\bgroup\originalleft}
\renewcommand{\right}{\aftergroup\egroup\originalright}

% Lets math text in an environment of bold text also appear bold.
\makeatletter
    \DeclareRobustCommand{\bfseries}%
    {%
        \not@math@alphabet\bfseries\mathbf%
        \fontseries\bfdefault\selectfont%
        \boldmath%
    }
\makeatother

% Adds square and curly brackets to the exceptions for xspace such that no space is used right in front of them.
\xspaceaddexceptions{]\}}

% Formats URLs by using the normal font (not the typewriter font).
\urlstyle{rm}

% Allows large display formulas to span multiple pages.
\allowdisplaybreaks

% Defines an optional argument for labels named ›ineq‹ that signals that cleveref should name the respective reference ›inequality‹ instead of its actual name.
\crefname{ineq}{inequality}{inequalities}
\creflabelformat{ineq}{#2{\upshape(#1)}#3} 

% Defines an optional argument for labels named ›term‹ that signals that cleveref should name the respective reference ›term‹ instead of its actual name.
\crefname{term}{term}{terms}
\creflabelformat{term}{#2{\upshape(#1)}#3}


%%%%%%%%%%%%%%%
%% Footnotes %%
%%%%%%%%%%%%%%%

% In the following, the command ›footnote‹ is redefined such that the footnote mark can be more easily adjusted.
\let\oldfootnote\footnote

% The following are variables used by the command.
\newlength{\spaceBeforeFootnote} % Denotes the space before the footnote mark in em.
\newlength{\spaceAfterFootnote}  % Denotes the space after the footnote mark in em.

% The new footnote command. The first three arguments are optional, the fourth mandatory. Its arguments have the following meaning:
%   1. The amount of space before the footnote mark in em. The default is 0.
%   2. The amount of space after the footnote mark in em. The default is 0.
%   3. The number of the footnote mark.
%   4. The text of the footnote.
\RenewDocumentCommand{\footnote}{o o o m}%
{%
    \IfNoValueTF{#1}%
    {%
        \oldfootnote{#4}%
    }%
    {%
        \setlength{\spaceBeforeFootnote}{\IfEmptyTF{#1}{0}{#1} em}%
        \IfNoValueTF{#2}%
        {%
            \hspace*{\spaceBeforeFootnote}\oldfootnote{#4}%
        }%
        {%
            \setlength{\spaceAfterFootnote}{\IfEmptyTF{#2}{0}{#2} em}%
            \hspace*{\spaceBeforeFootnote}\IfNoValueTF{#3}{\oldfootnote{#4}}{\oldfootnote[#3]{#4}}\hspace*{\spaceAfterFootnote}%
        }%
    }%
}

% The following commands enable it such that footnotes can be used in various other environments other than simple text.
\makesavenoteenv{figure}
\makesavenoteenv{table}
\makesavenoteenv{tabular}


%%%%%%%%%%%%%%%%%%%%%%%%%%
%% Theorem Environments %%
%%%%%%%%%%%%%%%%%%%%%%%%%%

\iffancyTheorems
    % The following theorem style uses a bold heading for the theorem and normal (upright) text. The environment begins with a triangle of color ›stroke1‹ pointing to the right and uses a QED symbol that is a triangle of the same color pointing to the left. Thus, the environment is enclosed by triangles.
    \declaretheoremstyle
    [
        spaceabove = \topsep,
        spacebelow = \topsep,
        headfont = \bfseries,
        headformat = \textcolor{stroke1}{$\blacktriangleright$} \NAME~\NUMBER \NOTE,
        notefont = \bfseries,
        notebraces = {(}{)},
        bodyfont = \normalfont,
        postheadspace = 0.5 em,
        qed = \textcolor{stroke1}{\bfseries$\blacktriangleleft$},
    ]
    {myTheoremStyle}
    
    % The QED symbol used in proofs is a squre with color ›stroke1‹ in order to look similar to the theorem environments.
    \renewcommand*{\qedsymbol}{\textcolor{stroke1}{$\blacksquare$}}
    
    \declaretheorem
    [
        style = myTheoremStyle,
        name = Conjecture,
        numberwithin = chapter,
    ]
    {conjecture}
    \declaretheorem
    [
        style = myTheoremStyle,
        name = Proposition,
        sharenumber = conjecture,
    ]
    {proposition}
    \declaretheorem
    [
        style = myTheoremStyle,
        name = Claim,
        sharenumber = conjecture,
    ]
    {claim}
    \declaretheorem
    [
        style = myTheoremStyle,
        name = Lemma,
        sharenumber = conjecture,
    ]
    {lemma}
    \declaretheorem
    [
        style = myTheoremStyle,
        name = Corollary,
        sharenumber = conjecture,
    ]
    {corollary}
    \declaretheorem
    [
        style = myTheoremStyle,
        name = Theorem,
        sharenumber = conjecture,
    ]
    {theorem}
    \declaretheorem
    [
        style = myTheoremStyle,
        name = Definition,
        sharenumber = conjecture,
    ]
    {definition}
    \declaretheorem
    [
        style = myTheoremStyle,
        name = Example,
        sharenumber = conjecture,
    ]
    {example}
    \declaretheorem
    [
        style = myTheoremStyle,
        name = Remark,
        sharenumber = conjecture,
    ]
    {remark}
\else
    % This is the default style. That is, the head is bold, the rest is italic, and there is no symbol to denote the end of the environment.
    \theoremstyle{plain}
    
    \newtheorem{conjecture}{Conjecture}[chapter]
    \newtheorem{proposition}[conjecture]{Proposition}
    \newtheorem{claim}[conjecture]{Claim}
    \newtheorem{lemma}[conjecture]{Lemma}
    \newtheorem{corollary}[conjecture]{Corollary}
    \newtheorem{theorem}[conjecture]{Theorem}
    \newtheorem{definition}[conjecture]{Definition}
    \newtheorem{example}[conjecture]{Example}
    \newtheorem{remark}[conjecture]{Remark}
\fi


%%%%%%%%%%%%%%%%%%%
%% Meta Commands %%
%%%%%%%%%%%%%%%%%%%

% A template for a function that can use an optional variable bracket size. Its arguments have the following meaning:
%   1. The name of the function.
%   2. The type of the left bracket. This should be a bracket symbol, as it will be forwarded to the command \left.
%   3. The type of the right bracket. The same restrictions as with parameter 2 hold here.
%   4. The arguments that the function takes, that is, the things that are enclosed by the brackets.
%   5. The size of the brackets. This should be a value like \big or similar, as it will be forwarded to the command \left.
\NewDocumentCommand{\functionTemplate}{m m m m o}%
{%
    \IfNoValueTF{#5}%
    {%
        \mathOrText{#1\left#2{#4}\right#3}%
    }%
    {%
        \mathOrText{#1#5#2{#4}#5#3}%
    }%
}

% The following two commands are used as variables for the following command.
\newcommand*{\leftBracketType}{(}
\newcommand*{\rightBracketType}{)}

% This is a command that creates a command that is a function as defined by the command \functionTemplate. Its arguments have the following meaning:
%   1. The name of the function command.
%   2. The name of the function itself.
%   3. The type of the left bracket. This will be forwarded to parameter 2 of \functionTemplate. The default is (. Use \lbrack for [ and \{ for }.
%   4. The type of the right bracket. This will be forwarded to parameter 3 of \functionTemplate. The default is ). The rest is similar to parameter 3.
% The command created has two optional arguments, which are as follows:
%   1. The arguments of the function. If this is empty, only the name of the function will be used.
%   2. The size of the brackets. This will be forwarded to parameter 5 of \functionTemplate.
\NewDocumentCommand{\createFunction}{m m o o}%
{%
    \renewcommand*{\leftBracketType}{\IfNoValueTF{#3}{(}{#3}}%
    \renewcommand*{\rightBracketType}{\IfNoValueTF{#4}{)}{#4}}%
    \NewDocumentCommand{#1}{o o}%
    {%
        \IfNoValueTF{##1}%
        {%
            \mathOrText{#2}%
        }%
        {%
            \functionTemplate{#2}{\leftBracketType}{\rightBracketType}{##1}[##2]%
        }%
    }%
}

% A template for a probabilistic symbol, which can make use of a condition denoted by |. Its arguments have the following meaning:
%   1. The name of the function.
%   2. The argument of the function.
%   3. The condition of the function. The default is that there is no condition.
%   4. The size of the brackets. This will be forwarded to parameter 5 of \functionTemplate.
\DeclareDocumentCommand{\probabilisticFunctionTemplate}{m m O{} o}
{%
    \functionTemplate{#1}%
    {\lbrack}%
    {\rbrack}%
    {#2\IfEmptyTF{#3}{}{\ \IfNoValueTF{#4}{\left}{#4}\vert\ \vphantom{#2}#3\IfNoValueTF{#4}{\right.}{}}}%
    [#4]%
}


%%%%%%%%%%%%%%%%%%%%%
%% Common Commands %%
%%%%%%%%%%%%%%%%%%%%%

%%%%%%%%%%%%%%%%%%%%%
% Number Sets

% Number sets appear in bold by default. The other option is to make them appear in blackboard bold.
\ifboldNumberSets
    \newcommand*{\N}{\mathOrText{\mathbf{N}}}
    \newcommand*{\Z}{\mathOrText{\mathbf{Z}}}
    \newcommand*{\Q}{\mathOrText{\mathbf{Q}}}
    \newcommand*{\R}{\mathOrText{\mathbf{R}}}
    \newcommand*{\C}{\mathOrText{\mathbf{C}}}
    \newcommand*{\indicatorFunctionSymbol}{\mathbf{1}}
\else
    \newcommand*{\N}{\mathOrText{\mathds{N}}}
    \newcommand*{\Z}{\mathOrText{\mathds{Z}}}
    \newcommand*{\Q}{\mathOrText{\mathds{Q}}}
    \newcommand*{\R}{\mathOrText{\mathds{R}}}
    \newcommand*{\C}{\mathOrText{\mathds{C}}}
    \newcommand*{\indicatorFunctionSymbol}{\mathds{1}}
\fi

%%%%%%%%%%%%%%%%%%%%%
% Probabilistic Functions
% All of these functions follow the outline of \probabilisticFunctionTemplate. That is, the syntax is, for example, \Pr{A}[B][\big], which would be shown as Pr[A | B] with \big brackets.

% Probability measure
\RenewDocumentCommand{\Pr}{m O{} o}%
{%
    \probabilisticFunctionTemplate{\mathrm{Pr}}{#1}[#2][#3]%
}

% Expected value
\NewDocumentCommand{\E}{m O{} o}%
{%
    \probabilisticFunctionTemplate{\mathrm{E}}{#1}[#2][#3]%
}

% Variance
\NewDocumentCommand{\Var}{m O{} o}%
{%
    \probabilisticFunctionTemplate{\mathrm{Var}}{#1}[#2][#3]%
}

%%%%%%%%%%%%%%%%%%%%%
% Landau Notation
% The following commands all take a mandatory argument, which is the term of the Landau notation, as well as an optional argument, which determines the size of the brackets.

% Big O
\DeclareDocumentCommand{\bigO}{m o}%
{%
    \functionTemplate{\mathrm{O}}{(}{)}{#1}[#2]%
}

% Small O
\DeclareDocumentCommand{\smallO}{m o}%
{%
    \functionTemplate{\mathrm{o}}{(}{)}{#1}[#2]%
}

% Big Theta
\DeclareDocumentCommand{\bigTheta}{m o}%
{%
    \functionTemplate{\upTheta}{(}{)}{#1}[#2]%
}

% Big Omega
\DeclareDocumentCommand{\bigOmega}{m o}%
{%
    \functionTemplate{\upOmega}{(}{)}{#1}[#2]%
}

% Small Omega
\DeclareDocumentCommand{\smallOmega}{m o}%
{%
    \functionTemplate{\upomega}{(}{)}{#1}[#2]%
}

%%%%%%%%%%%%%%%%%%%%%
% Constants

% Pi; ratio of a circle’s circumference to its diameter
\newcommand*{\circlePi}{\mathOrText{\uppi}}

% Euler’s constant. This command takes an optional parameter, which becomes the exponent of this constant.
\DeclareDocumentCommand{\eulerE}{o}%
{%
    \mathOrText{\mathrm{e}\IfNoValueTF{#1}{}{^{#1}}}%
}

% i; the imaginary unit
\newcommand*{\imaginaryUnit}{\mathOrText{\mathrm{i}}}

%%%%%%%%%%%%%%%%%%%%%
% Other

% A polynomial function. The mandatory parameter is the argument of the function, the optional one is the size of the brackets.
\DeclareDocumentCommand{\poly}{m o}%
{%
    \functionTemplate{\mathrm{poly}}{(}{)}{#1}[#2]%
}

% The identity function
\createFunction{\id}{\mathrm{id}}

% An indicator function. The first parameter is set as an index, the second is the argument of the function, and the third is the size of the brackets.
\NewDocumentCommand{\ind}{m o o}%
{%
    \IfNoValueTF{#2}%
    {%
        \mathOrText{\indicatorFunctionSymbol_{#1}}%
    }%
    {%
        \functionTemplate{\indicatorFunctionSymbol_{#1}}{(}{)}{#2}[#3]%
    }%
}

% The domain of a function. Its parameters are the same as for \poly.
\DeclareDocumentCommand{\dom}{m o}%
{%
    \functionTemplate{\mathrm{dom}}{(}{)}{#1}[#2]%
}

% The range of a function. Its parameters are the same as for \poly.
\DeclareDocumentCommand{\rng}{m o}%
{%
    \functionTemplate{\mathrm{rng}}{(}{)}{#1}[#2]%
}

% The d for an integral. The optional parameter becomes the exponent/degree of the operator.
\DeclareDocumentCommand{\d}{o}%
{%
    \mathrm{d}\IfNoValueTF{#1}{}{^{#1}}%
}

% A command that creates sets. The first parameter is the left-hand side, the second is the right-hand side, and the third (optional) parameter is the size of the brackets.
\DeclareDocumentCommand{\set}{m m o}%
{
    \mathOrText{\IfNoValueTF{#3}{\left}{#3}\{#1\ \IfNoValueTF{#3}{\left}{#3}\vert\
    \vphantom{#1}#2\IfNoValueTF{#3}{\right.}{}\IfNoValueTF{#3}{\right}{#3}\}}
}
      % Contains newly defined commands useful for mathematics.

% This is the thesis. The front matter as well as the references should not be changed. The main matter can be edited freely.
\begin{document}


    \frontmatter
    % This file contains the layout of the title page.

% As taken from the MADIS doctoral school page : 
% https://lilliad.univ-lille.fr/doctorant/conseils-redaction-page-garde

%La page de garde (ou page de titre) de votre thèse doit comporter au moins les éléments suivants : 
%    le nom de l’université et son logo : Université de Lille ;
%    le nom de votre école doctorale ;
%    le nom de votre laboratoire ;
%    le titre de la thèse ;
%    la mention «Thèse préparée et soutenue publiquement par Votre Nom le XX/XX/20XX, pour obtenir le grade de Docteur en Votre discipline de thèse» (ou toute autre formulation équivalente) ;
%    la liste des membres du jury, avec leur fonction et leur affiliation.

%Votre discipline ou votre spécialité doit être indiquée telle qu’elle a été saisie lors de l’enregistrement du jury de thèse auprès du Service des affaires doctorales - Bureau des soutenances de l’Université. 

%Si vous avez rédigé votre thèse en anglais, n’oubliez pas de faire figurer les titres en anglais et en français sur votre page de garde. Tous deux sont nécessaires.


% This page uses a different geometry, as the content will be centered (not including the binding correction).
\ifprintVersion
    \ifprofessionalPrint
        \newgeometry
        {
            textwidth = 134 mm,
            textheight = 220 mm,
            top = 38 mm + \extraborderlength,
            inner = 38 mm + \mybindingcorrection + \extraborderlength,
        }
    \else
        \newgeometry
        {
            textwidth = 134 mm,
            textheight = 220 mm,
            top = 38 mm,
            inner = 38 mm + \mybindingcorrection,
        }
    \fi
\else
    \newgeometry
    {
        textwidth = 134 mm,
        textheight = 220 mm,
        top = 38 mm,
        inner = 38 mm,
    }
\fi

% The format of the title page.
\begin{titlepage}
    \sffamily
    \begin{center}
	{
        	\def\svgwidth{20 em}
		\input{images/ULille_black.pdf_tex}\\
		{\tiny{CRIStAL - UMR 9189}\hfil{ED MADIS - 631}}
	}
        \vfil
	{
		{
			{\Huge{\textsc{\textbf{Thèse}}}}
		}\\[1 em]
		{
			{présentée et soutenue publiquement le}\\
			{\textbf{T.B.D.}}
		}\\[1 em]
		{
			{pour l'obtention du grade de}\\
			{\LARGE{\textsc{\textbf{Docteur de l'Université de Lille}}}}\\
			{\large\textit{spécialité \printProgram}}
		}\\[1 em]
		{
			{par}\\[0.3 em]
			{\Large\textbf{\printAuthor}}
		}
	}
        \vfil
        {\LARGE
            \rule[1 ex]{\textwidth}{1.5 pt}
            \onehalfspacing\printTitleBold\\[1 ex]
            \rule[-1 ex]{\textwidth}{1.5 pt}
        }
    \end{center}
    
    \vfil
	{\small \centering \underline{Composition du jury :}}
    \begin{table}[h]
	\small
        \sffamily 
        {%\def\arraystretch{1.2}
	    \begin{tabular}{
		>{\raggedright\arraybackslash}p{0.32\textwidth}
		>{\bfseries\raggedright\arraybackslash}p{0.262\textwidth}% 0.738
		>{\itshape\raggedleft\arraybackslash}p{0.32\textwidth}
	    }
		Gilles Grimaud	& Directeur de thèse	& Professeur des universités, Université de Lille\\
		XXXXXX XXXXXXX	& Rapportrice		& Maître de conférences, Université de La Rochelle\\
		XXXXXX XXXXXXX	& Rapporteur		& Maître de conférences, Université de La Rochelle\\
		XXXXXX XXXXXXX	& Examinatrice		& Professeur des universités, Université de Lille\\
		XXXXXX XXXXXXX	& Examinateur		& Professeur des universités, Université de Lille\\
		XXXXXX XXXXXXX	& Invité		& Professeur des universités, Université de Lille\\
            \end{tabular}
        }
    \end{table}
\end{titlepage}

\restoregeometry


    \pagestyle{plain}

    \addchap{Abstract}
    % This file should contain the abstract.

\vfil

Les travaux présentés dans ce document de thèse sont liés à la vérification formelle de propriétés sur des composants de systèmes d'exploitation. Les premiers travaux piliers de ce domaine sont ceux du projet seL4 ; démontrant que la vérification de propriétés formelles sur un micro noyau est réalisable, malgré un coût élevé. Pour réduire le coût de la preuve, le projet CertikOS a proposé une méthode de preuve plus étagée et plus modulaire, en tirant à l'extrême la méthode de preuve par raffinement. L'équipe du noyau Pip a pris le contrepied de ces travaux, en prônant le minimalisme, en utilisant une méthodologie reposant sur un \emph{shallow embedding} et en prouvant les propriétés désirées directement plutôt qu'en utilisant la méthode par raffinement.

Les travaux présentés dans cette thèse sont plus spécifiquement liés au noyau Pip. Les travaux précédents sur le noyau Pip ont porté sur une preuve de préservation de l'isolation des services fournis par Pip manipulant la mémoire. Cependant, un aspect critique du noyau devait encore être conçu : le transfert de flot d'exécution d'une partition de mémoire à une autre.

La première contribution de cette thèse présente un nouveau service de Pip conçu pour supporter tous les transferts de flots d'exécution possibles au sein d'un système -- les interruptions, les fautes, et les appels explicites. Ce service gère de manière unifiée ces transferts de flot d'exécution afin de réduire au minimum l'effort de preuve. Une implémentation est proposée pour le noyau Pip.

La seconde contribution de cette thèse est la première implémentation au monde d'un ordonnanceur \emph{Earliest Deadline First} pour jobs arbitraires muni d'une preuve formelle de sa correction. La preuve garantit que la fonction d'élection respecte la politique \emph{EDF}, garantissant l'optimalité du planning sur les machines mono-processeur. La preuve a été conduite en partie en suivant la méthodologie habituelle de Pip, utilisant un \emph{shallow embedding} et une monade d'état. Elle a cependant été réalisée par raffinement. De plus, l'ordonnanceur se sert du service de transfert de flot d'exécution ; montrant la polyvalence et l'utilisabilité du service.

La dernière contribution présentée dans cette thèse est une preuve de concept libérant le code des services de Pip de ses liens avec le modèle d'isolation. Cette indépendance permet de créer des modèles alternatifs, permettant de raisonner sur le code à propos de nouvelles propriété tout en limitant l'effort de preuve. Cette contribution ouvre de nouvelles perspectives de recherche liées à la réduction du coup de raisonnement sur des propriétés additionnelles sur Pip. Cette preuve de concept n'apporte cependant pas que des avantages : en particulier sur la confiance accordée à la conjonction de propriétés formellement prouvées sur des modèles différents.

\vfil


    % Temporary switch of language for the abstract

    \selectlanguage{english}
    \addchap{Abstract}
    % This file should contain the English abstract.
The work decribed in this document is related to formal proofs on operating systems and more specifically tied to the Pip kernel.
Previous work on the Pip protokernel focused on providing an isolation proof to Pip's services manipulating the system's memory. Yet, another critical aspect of the kernel -- handling the execution flow transfer from a partition to another -- remained to be designed.

The first contribution of this thesis outlines the design of a single service able to handle all possible control flow transfers in a system ; namely interrupts, faults and explicit control flow transfers. The design focuses on minimalism and code factorization in order to reduce the overall proof effort. An implementation of the service is provided for the Pip kernel. We believe the idea behind the service is general enough to be implemented in other kernels and other architectures.

The second contribution outlined in this thesis is the first formally proven correct userland implementation of an earliest deadline first scheduler for arbitrary jobs. The formal proof guarantees that the election function of the scheduler respects the earliest deadline first scheduling policy, and is guaranteed to be optimal on mono-processor systems. This proof was conducted using Pip's usual methodology, leveraging a shallow embedding of the scheduler's code in Coq and a state monad. Nonetheless, while the Pip kernel properties were proven directly, the presented scheduler proofs include three refinement levels ; from the scheduling policy to the actual implementation. Furthermore, the scheduler uses the previously described service in order to pass the control flow to partitions and regain the execution flow through interrupts, showcasing its usability and versatility.

The last contribution presented in this thesis is a proof of concept of a generic monad applied to the Pip kernel. This generic monad allows to create independent models focusing on specific aspects of a single interface in order to prove specific properties on each model. This should greatly reduces the cost of proving new properties on the kernel. Nonetheless, using this technique has its own implications on the composition of those properties.


    % Switch back to French for the rest of the document

    \selectlanguage{french}
    % Unfortunately we have to redefine the parts commands because the language switch has redefined our redefined commands    
    \renewcommand*{\partformat}{}
% This command calls \partformat (#2) and displays the name of the part (#3).
\renewcommand*{\partlineswithprefixformat}[3]%
{%
    #2
    \thispagestyle{empty}
    \setlength{\mytmpa}{0.618\mypaperwidth}%
    \setlength{\mytmpb}{0.382\mypaperheight}%
    \ifprintVersion
        \ifprofessionalPrint
            \setlength{\mytmpa}{0.618\mypaperwidth + \mybindingcorrection + \extraborderlength}%
            \setlength{\mytmpb}{0.382\mypaperheight + \extraborderlength}%
        \fi
    \fi
    \begin{tikzpicture}[overlay, remember picture]%
        \node [inner sep = 0, outer sep = 0, anchor = north] at (current page.north west)%
        {%
            \begin{tikzpicture}[overlay, remember picture]%
            \draw[color = stroke1, line width = 0.7 mm] (\mytmpa, 0) -- (\mytmpa, -\mytmpb);%
            \end{tikzpicture}%
        };%
        \node (align) [align = right, below = \mytmpb - 2 ex, inner sep = 0, outer sep = 0, anchor = north west] at (current page.north west)%
        {%
            %\hspace{\mytmpa}\hspace{0.5 em}\partname\ \thepart\\[1 ex]
            \hspace{\mytmpa}\hspace{0.5 em}\partname\\[1 ex]
            \color{stroke1}#3%
        };%
    \end{tikzpicture}%
}


    \addchap{Remerciements}
    % Here you can write whom you want to thank.
{\color{white}
Ni se nourrir, ni se loger n'est gratuit.
Je crois qu'avec ça j'ai tout dit.
Ce monde est cruel
Ce monde est cruel.
Ça y est, j'ai tout dit.

Pas manger, ça fait mourir, et je suis habitué au chauffage.
Tes besoins vitaux sont payants : t'as compris la prise d'otage.
Depuis tout petit dans la merde, tu sais qu'il faudra mailler.
Au moins un peu pour le loyer, au moins un peu pour grailler.
Depuis tout petit dans la merde, on t'apprend à travailler ;
personne ne va te ravitailler à l'œil,
personne ne va s'apitoyer, pas de bol.

Ce monde est cruel.
Ce monde est cruel.
Je peux développer encore, je le fais sans aucun effort.
Pour travailler (donc pour manger), on te prend à trois ans
-- on te lâche à vingt-cinq (tes meilleures années).
Si tu pars avant, tu démarres en bas de la pyramide
et tu fermes ta gueule. Tu fais les pires des tâches,
tu gravis les étages au ralenti. Tu tapines en stage,
t'es sous-payé et on t'oblige à sourire --
car c'est une chance (merci !) déjà d'être là
avec tes vieux diplômes. Tiens, parlons des diplômes.

Personne n'est sûr, mais fais-le quand même pour la sécurité.
D'ailleurs, toute ta vie, pense à sécuriser :
même si tu amasses -- ne dépense pas,
on ne sait pas ce qui peut arriver.
Tu peux mourir, c'est vrai. Mais, si c'est pas le cas,
tu peux souffrir du manque puis être interdit par ta banque
et ça, ça fait peur. Les banques ça fait peur.
Des banques privées s'enrichissent, et des pays s'endettent.
De tout petits groupes très riches face au reste du monde,
face au bétail, face à la masse de salariés sans tête.

N'oublie jamais qui gagne quoi lorsque tu taffes.
Si ça te fâche et que tu ne veux plus,
n'oublie jamais : tu ne manges plus.
Ça ressemble à un choix...
Si c'est pas de l'esclavagisme,
c'est quand même pas vraiment humaniste.

[...]

Ce monde est cruel.
Ce monde est cruel.
Et j'ai tellement de chance à côté des autres,
je trouve ça tellement cruel.

Hein ? Comment ça ? Dieu donnerait de la chance, du talent,
à certains mais pas à d'autres ? Ça me rend parano.
Je ne sais plus si je me suis entraîné, si, tout ça, je le mérite ?
Si l'univers était avec moi ou si ça fait dix ans que je me bats...

[...]

En vrai, je ne sais pas comment ça se passe.
En vrai, je ne sais pas qui maintient le cap.
Si ça vient de moi ou si ça vient des astres.

Ce monde est cruel.
Ce monde est cruel.
Faut changer les choses, si ce monde est cruel,
c'est sûr qu'il y en a d'autres.
Je remercie les anges, je remercie les autres,
je remercie les miens, remerciez les vôtres.
Ce monde est cruel, mais je vous remercie quand même.

Merci pour tout
}


    \setuptoc{toc}{totoc}
    \tableofcontents

    \pagestyle{headings}
    \mainmatter

    %%%%%%%%%%%%%%%%%%%%%%%%%%%%%%%%%%%%%%%%%%%%%%%%%
    %% Please add the content of your thesis here. %%
    %%%%%%%%%%%%%%%%%%%%%%%%%%%%%%%%%%%%%%%%%%%%%%%%%

    \part{Le commencement}
    
    \chapter{Introduction}
    % Here you introduce your topic to the reader.

\section{Contexte}

\subsection{Technologique}

\subsection{Humain}

Cette thèse a été menée à l'Université de Lille, en collaboration avec le \emph{Centre de Recherche en Informatique, Signal et Automatique de Lille} (communement abrégé en laboratoire CRIStAL). Cette thèse a été financée par une dotation de l'Université de Lille.

Cette thèse a été dirigée par Gilles Grimaud, directeur de l'équipe <<~\emph{eXtra Small, eXtra Safe}~>> (abrégé 2XS) du CRIStAL. L'équipe se spécialise dans la conception de logiciels et matériels apportant sécurité, fiabilité et efficacité aux systèmes embarqués fortement contraints. Les travaux menés dans l'équipe portent sur la conception d'un noyau de système d'exploitation munis de preuves formelles de propriétés d'isolation de la mémoire, sur les moyens d'attaque physique sur du logiciel (Bluetooth, LoRa, analyse de la consommation, ...), sur la détection de malware et obfuscation d'applications Android, mais aussi sur des objets mathématiques plus théoriques comme par exemple les fonctions corécursives et leur représentation dans un assistant de preuve.

\emph{2XS} a des relations privilégiées avec d'autres équipes du laboratoire, notamment celles faisant partie du même groupe thématique <<~\emph{Systèmes embarqués adaptables et sécurisés}~>>. Cette thèse a notamment tiré profit d'une forte proximité avec l'équipe \emph{SyCoMoRES}, dont les travaux portent sur la conception et l’analyse des systèmes embarqués temps réel, basé sur l’analyse symbolique de composants paramétriques. La seconde contribution de cette thèse est le fruit de cette collaboration.

Par ailleurs, l'équipe \emph{2XS} est hébergée à l'\emph{Institut de Recherche sur les Composants logiciels et matériels pour l’Information et la Communication Avancée} (abrégé IRCICA). L'IRCICA est un établissement conçu pour favoriser la recherche interdisciplinaire, ce qui a notamment permis à l'équipe de saisir de nombreuses opportunités de collaboration avec l'\emph{Institut d'Électronique, de Microélectronique et de Nanotechnologies} (abrégé laboratoire IEMN), et plus particulièrement avec le groupe de recherche \emph{CSAM} notamment sur les travaux relatifs à l'attaque de logiciel au travers de moyens physiques.

Les travaux présentés dans cette thèse sont liés au noyau de système d'exploitation nommé Pip développé dans l'équipe \emph{2XS}.

\subsection{Pip}

Pip est un noyau de système d'exploitation \emph{minimal} dont le seul but est de garantir l'isolation d'applications s'exécutant sur le système. Pour ce faire, Pip est muni de preuves formelles que ses services préservent les propriétés d'isolation lors de leur exécution. Pip utilise la mémoire virtuelle comme moyen de garantir ces propriétés.

Le projet Pip a démarré avec trois thèses fondatrices :
\begin{itemize}
	\item La thèse de Narjes Jomaa, soutenue en décembre 2018, a porté sur l'aspect formel du noyau. Narjes a développé une méthodologie permettant de raisonner sur le code des services de Pip, ainsi qu'une méthodologie de co-design du code des services avec les preuves formelles afin d'alléger l'effort de preuve global. Narjes est à l'origine des preuves de préservation de l'isolation fournies par Pip ;
	\item La thèse de Quentin Bergougnoux, soutenue en juin 2019, a porté sur l'implémentation du noyau sur l'architecture Intel x86, en particulier sur le code des services actuellement présents dans le noyau. Ses travaux ont aussi porté sur des preuves de concept explorant les possibilités de portage de Pip sur un environnement multicœur ;
	\item La thèse de Mahiedinne Yaker, soutenue en décembre 2019, a porté sur l'implémentation de Pip sur une plateforme embarquée basée sur l'architecture Intel, offrant des perspectives de travail sur les systèmes embarqués. Ces travaux ont aussi portés sur des réflexions autour de la conception de systèmes où les entités y demeurant ne se font pas mutuellement confiance.
\end{itemize}

De ces travaux fondateurs ont émergé de nouvelles opportunités de recherche, dont certains se sont transformés en sujets de thèse. Trois nouvelles thèses ont été pourvues, portant sur des sujets étendants les travaux fondateurs :
\begin{itemize}
	\item La thèse de Nicolas Dejon, soutenue en décembre 2022, qui porte sur l'application des propriétés d'isolation de Pip aux systèmes dépourvus de mémoire virtuelle, mais pouvant restreindre l'accès à certaines portions de mémoire grâce à une \emph{MPU}. Ces caractéristiques sont courantes sur des systèmes beaucoup plus modestes, et se prêtent particulièrement bien à de l'\emph{IoT} ;
	\item Les travaux initiaux de Sofia Santiago Fernandez qui portent sur la preuve de préservation de la sémantique du code des services lors de la compilation du code Gallina \emph{shallow-embedded} vers du code C ;
	\item Mes propres travaux de thèse, présentés dans ce document, portant sur la formalisation du transfert de flôt d'exécution au sein du noyau et de travaux préliminaires relatifs à l'ajout de nouvelles propriétés non relatives à l'isolation.
\end{itemize}

Les doctorants n'ont pas été les seules personnes recrutées pour participer au développement de Pip : c'est par exemple le cas de Damien Amara, recruté en tant qu'ingénieur de recherche. Damien a contribué de manière significative à l'implémentation de Pip sur l'architecture Armv7, ainsi qu'à la version de Pip pour les systèmes munis d'une \emph{MPU}. Pip a aussi été au cœur de nombreuses collaborations industrielles par exemple dans le cadre de projets européens, notamment avec Orange.

\section{Objets d'étude}

Cette thèse est développée autour de trois objets d'études principaux qui transparaissent dans l'état de l'art et les chapitres de contribution.

\subsection{Ordonnancement}

Un des objets d'étude de cette thèse est l'ordonnancement. Dans un système où des entités ont besoin de ressources pour accomplir certaines actions et où les ressources disponibles sont limitées, l'ordonnancement est le fait d'arbitrer quelles entités disposeront d'un accès aux ressources ainsi que les périodes à laquelle elles en disposeront. Le logiciel réalisant l'ordonnancement au sein d'un système est appelé l'ordonnanceur. Le processus d'ordonnancement est omniprésent dans nos ordinateurs modernes, par exemple lorsqu'ils doivent choisir -- plusieurs centaines de fois par seconde -- le prochain programme à exécuter parmi la centaine de programmes attendant leur tour. Cette décision est orientée par la politique d'ordonnancement qui dicte à l'ordonnanceur selon quels critères distribuer les ressources. Certaines politiques s'attachent plus particulièrement au respect de contraintes temporelles : on parle alors de politique d'ordonnancement temps réel.

\subsection{Transfert de flot d'exécution}

Un autre objet d'étude connexe dont il sera abondamment question dans ce document est le transfert de flôt d'exécution. La notion de transfert de flot d'execution est similaire à celle de la commutation de contexte. Lorsqu'un programme est interrompu et qu'un autre s'exécute à sa place, par exemple sous l'effet de l'ordonnancement des programmes du système, il se déroule un transfert de flot d'exécution. Le système se charge de sauvegarder tous les paramètres qui permettront de restaurer le programme interrompu comme s'il n'avait jamais été interrompu, puis charge le programme qui continuera son exécution.

Ce mécanisme et le processus d'ordonnancement au sein d'un système permettent d'exécuter de multiples programmes de manière concurrente et efficace. Ils sont notamment à l'oeuvre lorsque vous travaillez avec de nombreux logiciels simultanément sur votre ordinateur, lorsque vous visionnez une vidéo sur votre téléphone, ou lorsque vous prenez l'avion et qu'il navigue en pilote automatique.

\subsection{Preuve de programmes}

Le dernier objet d'étude principal, omniprésent dans ce document, est la preuve de programmes. La preuve de programmes est une technique permettant d'apporter de très fortes garanties sur le fonctionnement du programme étudié par le biais de démonstrations mathématiques. Apporter de telles garanties sur un programme est néanmoins \emph{extrêmement} coûteux, et n'est généralement entrepris que pour les systèmes dont un dysfonctionnement logiciel pourrait mettre en péril des vies humaines, ou pourrait engendrer la perte d'une quantité astronomique d'argent. De tels systèmes sont courants dans certains secteurs d'activités tels que l'aérospatial ou le transport.

\section{Présentation du document}

\subsection{Plan}

La lecture de ce document a été découpée en 5 chapitres principaux. Le premier chapitre, que vous êtes en train de lire, est un chapitre d'introduction proposant une mise en contexte des travaux de thèses ainsi qu'une brève introduction aux objets d'étude principaux. Ce chapitre se termine par une exposition du plan et propose différents axes de lecture en fonction des sujets d'intérêts du lecteur.

Le second chapitre fait un état de l'art des sujets abordés dans cette thèse. Cet état de l'art se contente de décrire les notions nécessaires à la compréhension des travaux de thèse. Il est décomposé en trois parties, qui reflètent les objets d'étude principaux. La première partie de l'état de l'art concerne les transferts de flot d'exécution. Elle en donnera une définition qui les classifiera et décrira ses implications en terme de logiciel et de sécurité, plus particulièrement au travers du prisme de l'architecture Intel x86. La seconde partie de l'état de l'art, plus modeste, sera dédiée à l'ordonnancement. Elle en fera une présentation générale, discutera des politiques d'ordonnancement en proposant des métriques pour les évaluer, puis discutera plus particulièrement des systèmes temps réel, dont elle développera juste la théorie nécessaire à la compréhension des travaux de cette thèse. Cet état de l'art s'achève sur une partie concernant la preuve de programmes. L'état de l'art sur cette partie commencera par décrire le processus de raisonnement mathématique dans sa généralité, axée sur la vérification automatique du raisonnement grâce aux assistants de preuve, en prenant l'exemple de l'assistant de preuve Coq. Dans un second temps, cette partie décrira les méthodes particulières permettant de raisonner sur des programmes, puis concluera sur les exemples les plus probants de l'application de ces techniques sur des systèmes d'exploitation.

Le troisième chapitre présente la première contribution décrite dans cette thèse, un service de transfert de flot d'exécution unifié au sein du noyau Pip. Le chapitre commence par donner quelques motivations à l'écriture de ce nouveau service. Il décrit ensuite les idées derrière le service puis en décrit l'implémentation détaillée au sein de l'architecture Intel x86, en montrant comment les différents transferts de flot d'exécution ont été unifiés. Le chapitre se poursuit sur l'établissement de la preuve d'isolation traditionnelle de Pip sur ce service en décrivant les ajouts à l'interface avec la monade et détaillant fortement certaines portions du raisonnement, points clés de la preuve complète. Ce chapitre se terminera sur une courte section de retours d'expérience, proposant certaines métriques à propos de ce service et proposant des réflexions sur le résultat produit. 


\subsection{Axes de lecture}



    \chapter{État de l'art (20-30 pages)}

	Ce chapitre a pour intention de définir et préciser les différentes notions nécessaires à la lecture des travaux de thèses, ainsi que de définir le contexte scientifique du travail. Il portera, dans une première section, sur les détails des différents transferts de flot d'exécution dans les systèmes modernes, ainsi que les changements d'états inhérents à ces transferts de flot d'exécution. Cette section abordera ensuite les problèmes de sécurité liés au transfert de flot d'exécution, ainsi que les techniques de mitigation de ces problèmes. Cette section terminera sur les problématiques temps réel touchant au transfert de flot d'exécution.
	La seconde section de ce chapitre fera un état des lieux de la preuve de programme. Elle commencera par discuter de ce qu'est une preuve et de leur vérificaton automatique, ainsi que des stratégies de conduite de preuve. Cette section continuera sur la preuve de programme, en particulier comment raisonner sur un programme impératif. Elle abordera aussi les notions de représentation du langage. Enfin, elle terminera sur les exemples de systèmes vérifiés formellement.

	\section{Transfert de flôt d'exécution}

		Cette section va détailler les différents transferts de flot d'exécution mis à disposition dans les cpus modernes.
		Dans cette section, nous détaillerons les transferts de flot d'exécution qui impliquent une reconfiguration explicite de l'état de la machine ayant un impact sur les droits d'accès aux ressources. Les appels d'une fonction d'un programme vers une autre fonction ne seront pas considérés dans cette section, même s'ils pourraient être considérés comme un transfert de flôt d'exécution.


		\subsection{Hardware}

			\subsubsection{Appels explicites}
			Les transferts les plus courants sont les transferts de flot d'exécution explicites, c'est-à-dire dont la cible est explicitement fournie lors de l’appel, ou clairement établie dans la documentation.

Par exemple, dans Linux, un processus peut demander l’ouverture d’un fichier avec l’appel système open(). Cet appel transfère le flot d’exécution d’un processus non privilégié vers le noyau Linux disposant du plus haut niveau de privilèges. Les fonctions appelables par des transferts explicites servent d’interface entre des logiciels disposant de droits distincts.



Ce type de transfert de flôt d'exécution, d'apparence assez anodine, est pourtant l'objet d'attaques multiples, dont le but est de faire dévier l'exécution (de préférence en mode privilégié) vers du code choisi par l'attaquant. Pour y arriver, un attaquant doit exploiter une vulnérabilité dans une portion de code, qui lui donnera le contrôle d'une zone de mémoire d'intérêt (la pile, le tas, ou même le code). Une fois qu'il contrôle cette zone mémoire, il lui suffit d'écrire un \emph{shellcode}, et d'exploiter une vulnérabilité dans du code privilégié pour que l'exécution du \texttt{return} de la fonction compromise saute dans le shellcode. L'attaquant gagne à ce moment le contrôle de la machine.

Commence alors un jeu du chat et de la souris pour essayer de mitiger l'impact de ces vulnérabilités. Pour compliquer la vie de l'attaquant, et qu'il lui soit plus difficile d'exécuter son shellcode, de nombreuses stratégies ont été entreprises par les fabriquant de matériels ainsi que par les développeurs de systèmes d'exploitation. 

\paragraph{Canaries}
Une première statégie est l'ajout de \emph{canary} qui visent à détecter les corruptions mémoires. Les canaries sont des valeurs écrites dans la pile ou le tas et qui sont générées aléatoirement à chaque exécution. Lors de la sortie de la frame protégée par le canary, le code vérifie que la valeur du canary correspond bien à celle qui avait été écrite initialement ; si ce n'est pas le cas, c'est qu'une corruption mémoire a eu lieu et une faute est levée.

Une des techniques permettant de vaincre les canaries est de lire la valeur initiale du canary avant de corrompre la mémoire. En effet, la canary \textbf{reste la même pour l'intégralité de l'exécution}. Une fois cette valeur récupérée, il suffit de corrompre la mémoire en réécrivant cette valeur au bon endroit pour échapper à la détection. De plus, si l'exploitation de la vulnérabilité permet de corrompre la mémoire de manière fine, il suffit d'éviter d'écrire sur la canary.

\paragraph{Droits fins pour les zones mémoires}
\label{memory_rights}
Une autre stratégie a été de définir des droits fins concernant l'accès aux différentes zones mémoires de l'espace d'adressage des processus. Le mécanisme de mémoire virtuelle permet de définir des droits d'accès propres à chaque page mémoire configurée (lecture, écriture, éxecution, accessible en mode non priviligié). Par exemple, les pages mémoire contenant du code sont typiquement configurées pour des accès en lecture et exécution, alors que les pages contenant des données (pour la pile, le tas, les sections de données d'un binaire) sont configurées pour des accès en lecture/écriture.

Cette stratégie de défense empêche un attaquant d'exploiter une vulnérabilité pour écrire un shellcode code dans la mémoire si on considère que chaque page de mémoire est soit exécutable, soit accessible en écriture. Cependant, il existe des cas d'usage légitimes qui violent cette contrainte, par exemple lors de compilation à la volée (ou JIT, pour Just-In-Time). Fatalement, de tels logiciels sont devenus la cible privilégiée des attaquants, on pourra par exemple citer Webkit \cite{webkitexploit}.
Heureusement, il est peu probable que de tels logiciels aient besoin de s'exécuter en mode privilégié. 

Pour affaiblir ce vecteur d'attaque, cette stratégie de défense est renforcée par des mécanismes de sécurité supplémentaires tels que le \emph{Supervisor Mode Access Prevention} (SMAP) et le \emph{Supervisor Mode Execution Prevention} (SMEP). SMAP permet au processeur de lever une faute lorsque qu'il exécute du code privilégié et qu'il essaie d'accéder (en lecture ou en écriture) à des données présentes dans l'espace utilisateur. SMEP permet en complément de lever une faute lorsque le processeur essaie d'exécuter du code dans l'espace utilisateur alors qu'il se trouve dans un mode d'exécution privilégié.

Ces mécanismes permettent d'isoler le code privilégié de potentiels shellcodes écrit en espace utilisateur. Ainsi, pour compromettre intégralement un système, l'attaquant doit à présent exploiter une vulnérabilité dans le code privilégié, ayant à sa disposition des pages mémoire soit accessibles en écriture soit exécutables et qui, de surcrois, ne font pas partie de l'espace utilisateur.
Nait alors une nouvelle technique d'exploitation de vulnérabilité. 

\paragraph{Return Oriented Programming}
\label{ROP}
Le ROP (pour \emph{Return Oriented Programming}) consiste à attaquer du code vulnérable en n'utilisant que le code déjà accessible dans l'environnement d'origine, mais en exécutant des portions arbitraires de celui-ci. L'attaque consiste à repérer des \emph{gadgets} : de brèves portions de code ayant un effet spécifique sur la mémoire ou les registres, suivi d'une instruction \texttt{return}. Pour l'attaquant, il suffit de dévier le flot d'exécution sur l'un de ces gadgets et de manipuler la mémoire, de manière à ce que l'exécution du gadget entraine l'exécution du suivant. L'attaquant parvient au final à exécuter son shellcode constitué d'une succession de gadgets, contournant les mécanismes de sécurité mentionnés dans le paragraphe précédent.\\

Plusieurs contre-mesures ont émergé pour rendre plus difficile le ROP.

\paragraph{Address Space Layout Randomization}
\label{aslr}
L'ASLR (pour \emph{Address Space Layout Randomization}) rend imprédictible l'adresse des différentes zones de mémoire au sein d'un espace d'adressage virtuel. Les adresses du binaire, de la pile, du tas, des librairies, du noyau, etc. sont rendus aléatoires à chaque nouvelle exécution. L'ASLR est un de ce fait un frein considérable au développement d'un shellcode en ROP, puisqu'il est impossible de prédire où se situeront les gadgets lors de la prochaine exécution.

L'ASLR n'est cependant pas parfait. Les adresses des zones mémoire restantes peuvent être révélées par des pointeurs dans les zones mémoires controlées par l'attaquant \cite{bypasskaslr}, ou grâce à des attaques micro-architecturales \cite{microarchitecturalbypass}.

\paragraph{Vérification de l'intégrité du flôt d'exécution}
\label{cfi}
Une autre approche permettant de réduire la marge de manoeuvre de l'attaquant et de vérifier que le flot d'exécution est conforme à celui attendu. À chaque appel et à chaque retour de fonction, le processeur vérifie si la cible du saut est valide. Plusieurs implémentations existent, notamment des implémentations matérielles au sein des processeurs \cite{intelpointerauth, armpointerauth}, mais aussi certaines implémentations logicielles notamment provenant de compilateurs \cite{compilerpointerauth}. Windows, macOS, Android, iOS utilisent déjà un mécanisme de vérification du flot d'exécution.

\paragraph{eXecute Only Memory}
\label{execute_only_memory}
Le XOM (pour \emph{eXecute Only Memory}), est une fonctionnalité de certain processeurs permettant de déclencher une faute lorsque qu'un accès en lecture est fait sur les pages mémoires configurées comme étant exécutables. Avant cette fonctionnalité, aucune distinction n'était faite entre le processus de récupération des instructions par le processeur et la lecture de données par l'utilisateur. Cette fonctionnalité rend considérablement pour difficile la recherche de gadgets, puisqu'il est impossible pour l'attaquant de lire le code qu'il souhaite compromettre directement sur la cible.
On pourrait cependant argumenter que cette fonctionnalité relève de la sécurité par l'obscurité, et qu'elle n'est pas réellement efficace.

\paragraph{\blockquote{Mieux vaut prévenir que guérir}}

Ces contre-mesures, sans cesse contournées par de nouvelles méthodes d'attaque, supposent qu'il existera toujours des vulnérabilités dans le logiciel comme dans le matériel et tentent donc de limiter au maximum leur impact sur les systèmes affectés. Une toute autre classe de mesures essaie de régler le problème en s'attaquant à l'existence même des vulnérabilités, plutôt que d'essayer minimiser leurs conséquences.

On pourrait citer les méthodes d’analyse statique, les méthodes d’exécution symbolique, de fuzzing, et plus particulièrement le langage Rust conçu pour éradiquer ces vulnérabilités par conception. Par ailleurs, des travaux ont été entamés pour prouver formellement les fonctionnalités de Rust.


			\subsubsection{Fautes}

Les différentes formes de fautes logicielles constituent aussi une forme de transfert de flot d'exécution avec élévation de privilèges. Les logiciels sont susceptibles de déclencher des fautes logicielles de différentes façons, par exemple :
\begin{itemize}
  \item décodage impossible de la prochaine instruction ;
  \item demande d'exécution d'une instruction impossible (division par zéro...) ; 
  \item demande d'accès à une adresse mémoire protégée, résultat de l'activité d'une MMU ;  
  \item demande d'exécution d'une instruction privilégiée en mode non-privilégié.
\end{itemize}
Dans ces différentes situations il s'agit de transferts implicites depuis le logiciel en faute vers une fonction d'un logiciel en charge du traitement de cette faute. Les différentes fonctions de gestion des fautes sont généralement définies par des éléments de configuration du matériel, et, le plus souvent, par l'intermédiaire d'une table (ou vecteur) dont le nom change d'une architecture de microprocesseur à l'autre. Ce vecteur précise généralement le niveau d'élévation de privilèges associé à l'exécution de la fonction de traitement de la faute. Sur les architectures Intel cette table est appelée \emph{IDT} (pour \emph{Interrupt Descriptor Table}).

\subsubsection{Interruptions matérielles}
Les interruptions matérielles sont des transferts non explicites à priori non contrôlés par le code non privilégié. Elles sont déclenchées par le matériel, signalant un événement important à traiter, tel que l'arrivée d'un paquet réseau par exemple. Les fonctions de traitement des interruptions matérielles ainsi que leur niveau de privilèges sont aussi définis dans l'\emph{IDT}.

Puisque les fautes et interruptions déclenchent un changement de privilèges, elles sont un vecteur d'attaque supplémentaire d'intérêt pour un attaquant cherchant à s'octroyer de nouveaux droits. En effet, les mêmes types de failles peuvent résider dans les routines de gestion de ces portions de logiciel. De plus, les interruptions et fautes brisent le flot d'exécution et modifient potentiellement l'environnement d'exécution du code interrompu. Cela les rends d'autant plus susceptibles de contenir des vulnérabilités, qui tombent alors dans la catégorie des vulnérabilités de \emph{concurrence critique}.

Même en ayant pleinement conscience des différentes interactions et dépendances entre les différents composants d'un système, les vulnérabilités de concurrence critique sont \textbf{notoirement difficiles à cerner}, principalement à cause d'un phénomène d'explosion combinatoire. Il peut s'avérer difficile de détecter une telle vulnérabilité par les tests, puisqu'ils sont souvent effectués dans des environnement très controlés où les mêmes conditions d'exécution sont artificiellement répétées, occultant d'autres fils d'exécution possibles. Malgré cela, si le développeur parvient à exhiber un fil d'exécution contenant un comportement anormal, il peut alors être délicat de reproduire le fil d'exécution ayant conduit à ce comportement. En effet, le fil d'exécution peut etre le résultat de nombreuses interactions - parfois non-déterministes - du programme avec son environnement. De plus, attacher un debuggueur tel que \texttt{gdb} peut modifier subtilement ces interactions, de manière telle qu'il soit impossible d'exhiber à nouveau le comportement anormal : on parle alors d'Heisenbug.

Pour illustrer la difficulté à cerner cette catégorie de bugs, on pourrait citer un problème d'incohérence de cache dans le noyau de système d'exploitation de la Nintendo Switch après une interruption matérielle ayant mené à un changement de coeur. Les effets de ce bug avaient été observés dès la sortie de la console ; il n'a cependant été trouvé et corrigé qu'à la sortie du firmware 14.0.0 de la console, soit plus de 5 ans après les premiers rapports \cite{switchbug}.
On pourrait aussi citer une vulnérabilité exploitée dans la pile IPV6 du noyau FreeBSD de la console Playstation 4 de Sony, profitant d'une situation de concurrence critique pour déclecher un Use-After-Free et compromettant le système d'exploitation. Cette faille présente sur tous les firmwares de la console depuis son lancement en 2013 a été découverte en 2018 puis divulguée et patchée en 2020, soit 7 ans après sa mise sur le marché \cite{ps4bug}. La même vulnérabilité était présente jusqu'à la version 5.00 du firmware de la Playstation 5, soit plus de deux ans après le rapport de bug concernant l'ancienne console \cite{ps5bug}.

Certains débuggers (notamment \texttt{rr}~\cite{mozRR}) ont implémenté une fonctionnalité "\emph{record and replay}" (enregistre et rejoue), permettant de capturer une trace du programme inspecté, puis de rejouer dynamiquement cette même trace à la demande. Cette fonctionnalité résoud le problème de la reproductibilité des comportement anormaux des programmes, et permet de surcrois de revenir à un état précédent de l'exécution lors d'une session de débuggage, ce qui est impossible avec les débuggers classiques. Certains émulateurs tels que Xen ou Qemu proposent des fonctionnalités "record and replay" sur les machines virtualisées. Malheureusement, les fonctionnalités "record and replay" pour des programmes sur plusieurs coeurs sont actuellement extrêmement lents, et profiteraient grandement d'implémentation matérielles si elles venaient à exister \cite{mozRR}.

De nombreux travaux ont été menés afin de détecter les situations de concurrence critique, par exemple par analyse statique~\cite{racerX}. D'autres travaux ont développés des méthodes plus particulières permettant de découvrir des situations de concurrence critique liées aux interruptions matérielles~\cite{sdracer}. Parallèlement, dans le monde de la preuve formelle, on pourrait citer les travaux ayant abouti à la logique de séparation~\cite{separationlogic}.

		\subsection{Software}

			\subsubsection{Changement de droits}

Sur l'architecture Intel x86, les différents transferts de flot d'exécution peuvent s'opérer par le biais de différentes instructions et événements matériels. Néanmoins, lorsqu'un changement de droits est requis lors d'un transfert, les différents chemins sont régis par un ensemble de mécanismes de contrôle relativement homogènes.

\paragraph{Changement de droits sous x86}
\label{ring}

Les privilèges attribués au code s'exécutant actuellement sur la machine sont ceux du \emph{segment} chargé dans le registre \texttt{CS} (pour \emph{code segment}). Les segments sont définis dans la \emph{GDT} (pour \emph{Global Descriptor Table}) à l'initialisation de la machine. La \emph{GDT} est une table globale spécifiée par Intel, dont l'adresse est accessible par un registre dédié. Dans cette table par exemple, Linux se contente de définir deux segments de code. Un premier segment associé au niveau de privilèges maximum du microprocesseur, nommé par Intel \emph{ring 0}, utilisé pour le code responsable du système ; et un second segment non-privilégié, associé au niveau de privilèges \emph{ring 3}, pour le reste du code. 

Contrairement au code s'exécutant avec le segment privilégié, le code s'exécutant sans privilège ne peut pas changer de segment à sa guise. Des mécanismes de contrôle du processeur déclenchent une faute si du code non-privilégié essaie de modifier son segment de code. Pour y parvenir, il est possible d'utiliser les \emph{gates}, qui sont des tremplins définis dans les tables globales du système (comme l'\emph{IDT} ou la \emph{GDT}), permettant au code non privilégié d'appeler une fonction prédéfinie qui s'exécute avec d'autres droits.

Lorsqu'un changement de segment déclenche un changement de niveau de privilèges, le processeur change de pile. Ce changement de pile permet d'éviter aux routines privilégiées les échecs dus à un manque de place sur la pile, ainsi qu'à les prémunir d'éventuelles interférences avec les procédures non privilégiées~\cite{intel_stack_switch}. 
Une pile doit être définie par niveau de privilèges (\emph{ring}) utilisé par le système ; leurs adresses doivent être renseignées dans une structure appelée \emph{TSS} (pour \emph{Task State Segment}). Cette structure est initialisée conjointement avec la \emph{GDT} qui contient son descripteur. Un registre dédié indique au processeur la position de ce descripteur dans la \emph{GDT}.

\paragraph{Appels systèmes sur x86}

Afin d'obtenir un transfert de flot d'exécution avec élévation de privilèges, le logiciel appelant non-privilégié peut exécuter des instructions dédiés aux différentes \emph{gates} des tables globales. Tout d'abord, l'instruction \texttt{int} permet d'appeler les \emph{gates} situées dans l'\emph{IDT}. Ces \emph{gates} sont soit des \emph{interrupt gates}, des \emph{trap gates} ou des \emph{task gates}~\cite{intel_idt_gates}. \texttt{int} s'accompagne d'un argument correspondant à l'index de la \emph{gate} ciblée dans l'\emph{IDT}. Le code ainsi appelé sera exécuté avec le niveau de privilèges spécifié par le segment de code indiqué dans la gate (et chargé dans le registre \texttt{CS}).

L'instruction \texttt{callf} permet d'utiliser les \emph{gates} situées dans la \emph{GDT}. Ces gates sont soit des \emph{call gates} ou des \emph{task gates}. Ces gates permettent de copier un nombre fixé d'arguments depuis la pile de l'appelant dans la pile du code privilégié à l'appel de l'instruction \texttt{callf}. Le nombre d'arguments à copier est renseigné dans la \emph{gate} ciblée par l'instruction. Là aussi, l'élevation de privilèges est spécifiée par le segment de code indiqué dans la gate et chargé dans \texttt{CS} lors de l'appel.

La troisième manière de déclencher une élévation de privilèges est l'instruction \texttt{sysenter}. Cette 
instruction ne sollicite aucune \emph{gate} : à la place, elle utilise les \emph{MSR} (pour \emph{model-specific registers}), qui sont des registres de contrôle du processeur. Ces \emph{MSR} sont manipulables grâce aux instructions \texttt{wrmsr} et \texttt{rdmsr}, qui sont des instructions privilégiées et qui permettent d'écrire et de lire dans ces registres respectivement. Un appel à \texttt{sysenter} utilise les MSR \texttt{0x174}, \texttt{0x175} et \texttt{0x176} pour charger \texttt{CS} \texttt{EIP} \texttt{SS} \texttt{ESP}. Le système doit donc avoir initialisé ces registres avant l'utilisation de \texttt{sysenter}. De plus, \texttt{sysenter} ne sauvegarde pas l'adresse de retour ni l'adresse de la pile lors d'un appel, qui doivent être placés dans les registres \texttt{ECX} et \texttt{EDX} au moment de l'appel à \texttt{sysexit} pour retourner dans le code appelant.

\paragraph{Interruptions et fautes sur x86}

Les interruptions liées au matériel sur l'architecture x86 étaient autrefois gérées par un coprocesseur (le PIC 8259 \emph{pour Programmable Interrupt Controller} ou plus récemment, l'APIC pour \emph{Advanced Programmable Interrupt Controller}). Ce coprocesseur est maintenant intégré au processeur, mais nous continuerons de parler de coprocesseur pour honorer l'histoire. Ce coprocesseur utilise les \emph{gates} situées dans l'IDT de la même manière que l'instruction \texttt{int}. Il est possible de configurer ce coprocesseur
pour qu'il utilise une certaine plage de niveaux d'interruption, ou pour qu'il masque temporairement la venue de nouvelles interruptions. Les fautes utilisent elles aussi l'\emph{IDT}, et utilisent les trente-deux premières \emph{gates} de la table, en fonction de la faute à déclencher. Les fautes et interruptions déclenchées par le processeur ou le coprocesseur ont toujours le droit d'utiliser les \emph{gates}, peu importe le niveau de privilèges du code s'exécutant au moment de l'interruption.

\paragraph{Fonctionnement de la \emph{MMU} sur l'architecture Intel 32 bits}

Sur l'architecture Intel x86, mais aussi sur toutes les autres architectures supportant une \emph{MMU} (pour \emph{Memory Management Unit}), il est possible d'associer des droits d'accès spécifiques à chaque pages de mémoire configurée dans l'espace d'adressage virtuel.

Le concept de traduction d'adresse virtuelle vers l'adresse réelle est le suivant. Les bits de poids forts de l'adresse virtuelle servent à traverser les tables de la MMU (sur l'architecture Intel x86, le \emph{Page Directory} et les \emph{Page Tables}). Les bits de poids faible correspondent à l'emplacement de l'adresse désirée dans la page réelle obtenue après traduction (souvent appelé \emph{offset}).

En particulier, sur Intel x86 et en mode de pagination 32 bits pour des pages de 4 Kio, les espaces d'adressage sont configurés par une structure de données arborescente de pages de 4Kio. Cette structure de données a deux étages : la racine appelée PD pour \emph{Page Directory}, et les feuilles appelées PT (pour \emph{Page Tables}). Le développeur renseigne l'adresse du Page Directory à utiliser dans le registre \texttt{CR3} du processeur ; cette adresse est alignée sur 4Kio, les 12 bits de poids faibles (11-0) sont ignorés ou sont réservés pour un autre usage.

Plus précisement, le \emph{Page Directory} est constitué de 1024 entrées de 32 bits appelées les \emph{PDE} (pour \emph{Page Table Entries}). Les 20 bits de poids fort (31-12) de ces entrées déterminent l'adresse de la \emph{Page Table} à utiliser, alignée sur 4Kio. Les \emph{Page Tables} sont aussi constituée de 1024 entrées de 32 bits appelées \emph{PTE} (pour \emph{Page Table Entries}). De la même manière, les 20 bits de poids forts (31-12) déterminent l'adresse de la page de mémoire réelle, alignée sur 4Kio. (Il est aussi possible de configurer des pages de 4Mio plutot que des pages de 4Kio en modifiants certains bits de controle des \emph{PDE}.)

Lors de la traduction d'une adresse virtuelle, les 10 bits de poids forts de l'adresse virtuelle (31-22) déterminent le numéro de \emph{PDE} à utiliser, les bits (21-12) déterminent le numéro de \emph{PTE}, et les 12 bits de poids faible restants (11-0) déterminent l'\emph{offset} de l'adresse cible dans la page réelle.~\cite{intel_32bits_paging}

\paragraph{Contrôle d'accès par la MMU sur Intel x86}
Dans le mode de pagination 32 bits d'Intel, les droits associés à chaque page sont présents dans les \emph{PTE}, dans les 12 bits de poids faible. Le bit 1 (\texttt{R/W}) permet d'empêcher les accès en écriture sur la page. Le bit 2 (\texttt{U/S}) permet d'empêcher n'importe quel accès utilisateur à la page - en lecture ou en écriture. Le niveau de privilège de l'accès dépend du \emph{CPL} (pour \emph{Current Privilege Level}) de l'instruction courante, souvent déterminée par le segment de code actuel.
L'architecture Intel permet de restreindre la récupération des instructions en discriminant chaque page de mémoire. Cette fonctionnalité n'est cependant pas disponible dans le mode de pagination 32 bits, puisqu'elle nécessite des \emph{PDE} ou \emph{PTE} longues de 64 bits. Elle est par exemple disponible dans le mode de pagination \emph{PAE}. Cette fonctionnalité est activable au travers du bit \texttt{NXE} du registre \texttt{IA32\_EFER}. Pour chaque page de mémoire, mettre le bit 63 (\texttt{XD}) à 1 dans une \emph{PTE} empêche la récupération d'instruction depuis cette page mémoire.

Cependant, d'autres fonctionnalités de contrôle d'accès globaux liées à la \emph{MMU} existent. L'architecture Intel propose aussi les mécanismes \emph{SMAP} et \emph{SMEP} comme décris dans le paragraphe \ref{memory_rights}.
Le bit \emph{SMAP} (pour \emph{Supervisor Mode Access Protection} présent dans le registre CR4 permet d'empêcher du code utilisant un segment de données privilégié d'accéder aux pages mémoire annoncées comme étant des pages mémoire utilisateur (bit \texttt{U/S} à 1).
Le bit \emph{SMEP} pour \emph{Supervisor Mode Execution Protection} présent dans le registre CR4 permet d'empecher la récupération de code lorsque le segment actuel octroie des accès privilégiés et que la page mémoire contenant les instructions est annoncée comme une page mémoire utilisateur (bit \texttt{U/S} à 1).

%				Espace d'adressage, niveau de privilèges

			\subsubsection{Capture de l'état d'exécution}

\textcolor{red}{J'arrive pas à introduire le concept}

\paragraph{Le contexte d'exécution}

Le contexte d'exécution d'un programme est constitué d'informations permettant de reprendre de manière saine l'exécution d'un programme à un moment arbitraire de son exécution après qu'il ait été interrompu. Le contexte d'exécution est donc intimement lié à l'état de la machine lors de l'exécution du programme. 

\textcolor{red}{J'ai envie de parler de la limite des contextes d'exécution, notamment avec les attaques micro architecturales mais je n'arrive pas à le formuler} Il est difficile de déterminer ce qui fait partie de l'état d'un programme : le contenu des registres du processeur et de la mémoire, mais aussi l'état des composants logiciels ou matériels intéragissant avec le programme (tels que les périphériques, le cache, etc.).

Le contexte d'exécution d'un programme doit contenir toute partie de l'état du programme susceptible d'être modifiée légitimement par d'autres programmes du système. De ce fait, il n'est pas nécessaire de copier la mémoire utilisée par le programme. Dans les systèmes d'exploitation modernes, chaque programme dispose d'une portion de mémoire qui lui est dédiée. Cela peut être garanti par conception dans le cas de systèmes collaboratif ou plus strictement par des mécanismes de controle d'accès relatifs à une \emph{MMU} ou une \emph{MPU} par exemple. 
Puisqu'il n'est pas nécessaire de copier le contenu de la mémoire, il suffit donc de copier les registres du processeur susceptibles d'être modifiés. Lors d'appels sans changement de droit, des règles de sauvegarde des registres existent (par exemple \cite{arm32_bit_callconv}). Ces règles décrivent notamment quels sont les registres susceptibles d'être modifiés par l'appelé, et peuvent permettre d'économiser la sauvegarde de certain registres.
Cependant, puisque les transferts de flot d'exécution avec changement d'espace d'adressage et de droits peuvent être implicites, le système ne peut faire aucune hypothèse sur les registres préalablement sauvegardés par le programme interrompu. De plus, il est courant de continuer l'exécution d'un autre programme susceptible de modifier n'importe quel registre à sa disposition. En toute généricité, il est donc impossible de prévoir quels registres resteront intacts lors d'un tel transfert : il est donc nécessaire de sauvegarder tous les registres modifiables par le programme interrompu.

Le processeur nous assiste dans ce travail. En effet, sans l'aide du processeur, le contenu de certains registres comme le pointeur d'instruction ou le pointeur de pile seraient instantanément perdus au moment du transfert. Voici comment se passe le transfert pour chaque mode de transfert différent sous l'architecture Intel x86 : 


\paragraph{Changement de pile et registres sauvés lors d'un appel à \texttt{callf}}

Lors d'un appel à \texttt{callf}, le processeur accède à une call gate située dans la \emph{GDT}. Cette callgate indique entre autres, le segment de code à utiliser (et donc le niveau de privilège associé), le code à exécuter, le nombre d'arguments à copier depuis la pile utilisateur. La pile ainsi que le segment à utiliser sont eux renseignés dans la \emph{TSS} et sont liés au niveau de privilège du segment de code de la callgate.

Tout d'abord, le processeur sauve de manière temporaire le segment de pile et le pointeur de pile dans un tampon interne. Il remplace les registres de segment de pile et de pointeur de pile par ceux renseignés dans la \emph{TSS}, changeant de pile. Il pousse ensuite le segment de pile et le pointeur de pile de l'utilisateur dans la nouvelle pile. Il copie ensuite les arguments depuis la pile utilisateur vers la nouvelle pile, leur nombre étant renseigné dans la call gate. Le processeur pousse ensuite sur la pile le registre de segment de code et le pointeur d'instruction, avant de les remplacer par ceux renseignés dans la call gate.

\paragraph{Changement de pile et registres sauvés lors d'une interruption, d'une faute ou d'un appel à \texttt{int}}

Lors d'une faute, d'une interruption ou d'un appel à \texttt{int}, le processeur accède à une interrupt gate ou une trap gate dans l'\emph{IDT}. Comme pour la call gate, la gate contient le segment de code à utiliser, le code à exécuter. Ce transfert de flot d'exécution n'entraine pas une copie par le processeur d'éventuels arguments depuis la pile utilisateur sur la nouvelle pile.

Tout d'abord, le processeur change de pile, en sauvant temporairement les valeurs utilisateurs du segment de pile \texttt{SS} et de la pile \texttt{ESP}, et en les écrivants sur la pile renseignée dans la nouvelle pile renseignée dans la \emph{TSS}. Le processeur sauve ensuite l'état des registres \texttt{EFLAGS}, du segment de code \texttt{CS} et du pointeur d'instruction \texttt{EIP}. EFLAGS contient entre autres l'état des drapeaux d'overflow, de retenue, de parité, de conditions, mais aussi d'activation des interruptions. Le processeur peut pousser un code d'erreur supplémentaire sur la pile dans le cas de déclenchement de certaines fautes afin de préciser leur cause.

		\subsection{Failles de sécurités associées}
			%https://google.github.io/security-research/pocs/linux/bleedingtooth/writeup.html#achieving-rip-control
			%https://google.github.io/security-research/pocs/linux/cve-2021-22555/writeup.html
			%https://github.com/Bonfee/CVE-2022-0995
			CVE historiques ? :D

			%https://pointer-authentication.github.io/

		\subsection{Ordonnancement}

		Le transfert de flôt d'exécution est au coeur du fonctionnement de systèmes complexes, par exemple lors de l'ordonnancement au sein d'un système informatique. L'ordonnancement dans un système d'exploitation permet à plusieurs programmes de s'exécuter de manière concurrentielle sur le même processeur, en alternant leur exécution. L'ordonnanceur décide quel programme sera le prochain à s'exécuter sur une unité de calcul donnée, qu'il décide selon une \emph{politique d'ordonnancement}. Ces politiques sont multiples et visent à satisfaire des contraintes diverses, pouvant par exemple viser à exécuter les programmes interactifs de manière prioritaire, ou plus simplement à exécuter chaque programme l'un après l'autre. L'ordonnancement permet ainsi d'optimiser l'utilisation du processeur pour un objectif particulier. 

			\subsubsection{Partage équitable du CPU}
		Une des fonctions principales de l'ordonnancement est le partage du temps CPU. En effet, les programmes s'exécutant à sur un système ne collaborent pas forcément avec le système pour permettre aux autres programmes de s'exécuter. Il revient alors au système d'exploitation d'interrompre les programmes à intervalles réguliers, grâce aux interruptions déclenchées par l'horloge par exemple, pour ne pas créer de situation de \emph{famine}. Lorsque le programme est interrompu, le système appelle l'ordonnanceur, qui choisira le meilleur programme à exécuter selon sa politique.
		Plusieurs indicateurs permettent d'évaluer les politiques d'ordonnancement dans les systèmes classiques, notamment :
		\begin{itemize}
			\item{le débit, mesurant le nombre de tâches terminées sur une certaine période}
			\item{le temps d'attente, mesurant le temps moyen entre le moment où la tâche a été créée et le moment où elle a commencé à être exécutée}
			\item{l'équité, mesurant la différence entre les temps CPU accordés à chaque processus}
			\item{la latence, mesurant le temps d'attente moyen entre la soumission de la tache et la production des premières sorties par la tache}
		\end{itemize}

			\subsubsection{Respect des contraintes de temps}

		L'ordonnancement est un élément clé des systèmes temps réel. Les systèmes temps réel ont des contraintes de temps associées à chaque unité de travail. Les systèmes temps réel \emph{souples} sont munis de contraintes de temps indicatives, le non respect des contraintes temporelles pouvant mener à une dégradation de la qualité des résultats produits par le système, par exemple dans le cas d'applications multimédia (audio, vidéo, etc.) ou dans le cas de systèmes de surveillance collectant des données (par exemple météorologiques). Nous nous attarderons sur les systèmes temps réel \emph{stricts}, qui doivent impérativement compléter chaque unité de travail demandée avant leur expiration sous peine de dysfonctionnement critique du système. Ces systèmes s'appuient sur des modèles mathématiques décrivant les actions à accomplir par le système.

		\paragraph{Tâches et jobs}
		Les systèmes temps réel sont conçus pour réaliser des actions dans certaines limites temporelles. En toute généricité, ces actions sont uniques au sein des systèmes, et sont désignées en anglais par le terme \emph{job} (utilisé dans la suite du document faute d'un équivalent français adéquat). Les \emph{jobs} ont au minimum une action qui leur est associée, une date à partir de laquelle il est possible de commencer à réaliser l'action (appelée \emph{release date}), une durée (appelée \emph{duration}), et une échéance (appelée \emph{deadline}.
		Cependant, les actions à réaliser par les systèmes temps réels sont rarement unique en pratique : les systèmes peuvent être amenés à répéter certaines actions (ou \emph{job}) tout au long de leur fonctionnement. On parle alors d'une \emph{tâche} (ou \emph{task} en anglais), produisant un nouveau job unique à chaque fois que l'action doit être répétée. Par exemple, un système temps réel pourrait être amené à vérifier périodiquement la valeur produite par une sonde pour vérifier son bon fonctionnement : on parlerait alors de tâche de vérification des valeurs de la sonde ; chaque vérification indépendante de la valeur produisant un \emph{job}.

		\paragraph{Modèles de tâches}
		Les tâches d'un système temps réel sont traditionnellement par trois modèles mathématiques différents. Les tâches \emph{périodiques} permettent de représenter les tâches qui doivent produire un nouveau job après chaque période qui leur est propre. Les tâches \emph{sporadiques} permettent de représenter les tâches qui peuvent produire un nouveau job à un moment aléatoire, mais qui ne peuvent produire un nouveau job tant qu'après un certain délai d'attente. Les tâches \emph{apériodiques} sont des tâches qui peuvent produire un nouveau job à un moment aléatoire, sans délai particulier.

		\paragraph{Vérification du respect des échéances}
		La vérification du respect des contraintes temporelles s'effectue \emph{avant} la mise en fonctionnement du système temps réel. Lors de la conception du système, un modèle du fonctionnement du système doit être créé, associant une durée (ou une borne supérieure associée à la durée) à chaque action réalisée par le système, ainsi qu'un modèle spécifiant à quel moment chaque action doit être effectuée. Une fois ce modèle créé, il doit faire l'objet d'une \emph{analyse d'ordonnançabilité}, vérifiant que chaque action entreprise pourra se terminer dans le temps imparti.

		Lorsque l'analyse d'ordonnançabilité a été effectuée, et qu'elle atteste qu'il sera toujours possible d'ordonnancer les différents jobs du système en respectant les échéances, il existe deux méthodes permettant d'ordonnancer les jobs. La première méthode consiste à précalculer l'ordonnancement des jobs, et à l'inclure de manière statique dans le système. Cette méthode de calcul hors ligne (ou \emph{offline} en anglais) du plan d'ordonnancement, a pour avantages d'affranchir le système temps réel du coût de l'ordonnancement "en direct". Le système n'aura pas à choisir lui même la prochaine tâche à effectuer ; elle lui a été précalculée. Cependant, il n'est pas toujours possible de savoir à chaque instant du fonctionnement d'un système temps réel quels seront les jobs à exécuter par exemple dans le cas de tâches sporadiques ou apériodiques. En effet, certaines actions des sytèmes temps réels peuvent être liées à des stimuli externes, comme par exemple l'action de maintenir l'assiette d'un avion en vol lorsqu'il tangue sous l'effet de turbulences. Dans ce genre de cas, le système temps réel doit pouvoir s'accomoder de telles variations, et en conséquence le plan d'ordonnancement doit pouvoir être modifié pendant le fonctionnement du système. Il n'est alors plus possible de précalculer le plan d'ordonnancement ; le système temps réel doit intégrer un ordonnanceur pour faire face à ces aléas.

		\paragraph{Vérification des résultats produits par l'ordonnanceur}
		De nombreuses politiques d'ordonnancement existent :
		\begin{itemize}
			\item{tourniquet, fair-share, foreground-background }
			\item{Rate monotonic (ancien et fondateur, liu et layland)}
			\item{Deadline monotonic (optimal sous quelques hypothèses)}
			\item{earliest deadline first (optimal)}
			\item{shortest job next, shortest remaining time (famine)}
			\item{Highest response ratio next (variante de shortest job next proposant une réponse aux problèmes de famine)}
			\item{Multilevel feedback queue (ancien, prix turing)}
			\item{YDS (politique visant à minimiser la consommation énergétique, optimal quelques conditions)}
		\end{itemize}
		\textcolor{red}{Est ce que je dois détailler ? Est ce que c'est pas hors sujet et qu'on s'en foutrait pas un peu}

	\section{Preuve de code}

	Cette section va décrire les notions et travaux nécessaires à la compréhension des contributions sur la preuve formelle sur du code décrits dans les chapitres suivants. La première sous-section décrira le processus de vérification automatique de preuves, en décrivant d'abord le processus de raisonnement automatique et introduisant les notions d'axiomes, d'hypothèses, ainsi que le déroulement de la preuve avec les règles d'inférence. Cette description sera ensuite illustrée par l'exemple de l'assistant de preuve Coq, outil de l'état de l'art que j'ai utilisé pour mes contributions. Enfin, la sous-section décrira les deux méthodes principales de raisonnement : la preuve directe et la preuve par raffinement.

	La seconde sous-section décrira plus spécifiquement le processus de raisonnement formel sur du code. Elle commencera par discuter du raisonnement sur des programmes impératifs, et en particulier de la \emph{logique de Hoare}. En seconde partie de cette sous section seront décrits les aspects concernant la représentation du programme dans l'assistant de preuve, ainsi que le principe des \emph{monades}, permettant de capturer les effets de bords des programmes impératifs dans les langages fonctionnels des assistants de preuve.

	Enfin, la dernière sous-section décrira les travaux les plus reconnus concernant l'application du raisonnement formel sur les systèmes d'exploitation, en finissant par le noyau développé au sein de l'équipe.

		\subsection{Vérification automatique d'une preuve}

			\subsubsection{Qu'est ce qu'une preuve formelle?}
			Une preuve ou une démonstration formelle se place dans le cadre d'un système formel, qui définit les règles permettant de formuler des propositions mathématiques valides ainsi que les règles de transformations pouvant s'appliquer aux propositions. Une démonstration est l'ensemble des transformations effectuées sur une proposition initiale pour la transformer en une proposition finale. Cette proposition finale peut porter plusieurs noms suivant l'importance du résultat obtenu : \emph{proposition}, \emph{lemme}, ou encore \emph{théorème}. Une preuve est correcte si chaque transformation fait parti des règles de transformation (ou \emph{règles d'inférence}) du système formel dans lequel se place la démonstration. L'utilisation d'un système formel permet la vérification automatique des preuves par un ordinateur.

			\paragraph{Axiomes} Le système formel peut parfois inclure des \emph{axiomes} qui sont des propositions qui ne peuvent pas être prouvées, et qui servent de point de départ au raisonnement au sein de ce système formel. Pour qu'un système formel soit intéressant, il faut que ses axiomes n'amènent pas à une \emph{contradiction}.

			\paragraph{Hypothèses} Les hypothèses sont des propositions qui n'ont pas été prouvées. Les hypothèses peuvent servir à établir des preuves d'autres propositions. Cependant, toute démonstration utilisant une hypothèse n'est pas complète tant que l'hypothèse elle même n'a pas été prouvée, contrairement aux axiomes qui n'ont pas vocation à être prouvés. Les hypothèses peuvent par exemple servir à structurer les longues preuves.

			\paragraph{Raisonnement}
			Le raisonnement au sein d'une preuve est déroulée grâce à l'application de \emph{règles d'inférence}. Une règle d'inférence s'applique sur une ou plusieurs propositions initiales appelées \emph{prémisses} et crée une nouvelle proposition en retour qu'on appelle la \emph{conclusion}.

			\textcolor{red}{Petit exemple sur la forme des règles d'inférences ? type : $ \text{prémisses} \vdash \text{conclusion}$ ?}

			\paragraph{Lien entre preuve et véracité de la proposition finale} Il est important de distinguer le fait qu'une preuve soit correcte et la valeur de vérité de la proposition finale. En effet, la valeur de vérité d'une proposition résultant d'une démonstration est liée à la valeur de vérité des prémisses. Entre d'autres termes, toutes les hypothèses utilisées dans la démonstration doivent être valides pour que la proposition finale soit valide.

			\subsubsection{Exemple de Coq}
			L'assistant de preuve Coq est un logiciel open-source français initiallement développé à l'INRIA permettant de vérifier automatiquement des preuves. Le langage de Coq est Gallina, un langage fonctionnel proche d'OCaml permettant de décrire à la fois preuves, programmes, et prédicats. En effet, Coq repose sur le calcul des constructions : un système formel dérivé du lambda calcul et utilisant l'isomorphisme de Curry-Howard pour établir un lien entre les démonstrations et les programmes. Ainsi, le calcul des constructions s'inscrit dans la lignée des logiques intuitionnistes formant des preuves \emph{constructives}. 

			\paragraph{Logique intuitionniste et preuves constructives} La logique intuitionniste et des preuves constructives rejettent le principe de \emph{tiers exclu} qui stipule que soit une proposition est vraie, soit sa négation est vraie. En particulier, il n'est pas possible d'établir une preuve d'une proposition par l'absurde, c'est à dire en montrant que la négation d'une proposition aboutie à une contradiction ($ \neg P \implies \bot \vdash P$). Cette règle, uniquement valide en logique classique est à distinguer de la règle de réfutation ($P \implies \bot \vdash \neg P$). Cette règle, valide à la fois en logique classique et en logique intuitionniste, peut par exemple être utilisée pour prouver $\sqrt{2}$ n'est pas rationnel.
			Pour en faciliter la compréhension, certaines propositions en logique intuitionniste peuvent être interprétées d'une manière différente de la logique classique, par exemple:
			\begin{itemize}
				\item{$A$ se lit << $A$ est prouvable >>}
				\item{$\neg A$ se lit << $A$ est contradictoire >>}
				\item{$\exists x, A(x)$ se lit << On peut exhiber un élément $x$ tel que $A(x)$ est prouvable >>}
				\item{$\forall x, A(x)$ se lit << Pour n'importe quel élément $x$, $A(x)$ est prouvable >>}
			\end{itemize}

			\paragraph{Autres assistants de preuve} D'autre assistants de preuve existent tels que Isabelle/HOL, F*, Agda, Lean... \textcolor{red}{Que dois-je détailler ici ?}

			\subsubsection{Stratégie de conduite de preuve}
			Tout comme la conception de logiciel, la conduite de preuve nécessite parfois un effort d'ingénierie pour structurer la preuve. Savoir si une preuve mérite d'être structurée relève  de l'expertise : on parle alors de génie de la preuve. Il existe deux méthodes principales permettant de compléter la preuve d'une proposition : la preuve directe et la preuve par raffinement.

			\paragraph{Preuve directe} La méthode de preuve directe est à opposer à la méthode de preuve par raffinement. Elle consiste à utiliser les éléments directement fournis par le langage et éventuellement quelques \emph{lemmes} ou \emph{théorèmes} intermédiaires pour arriver à la preuve finale. La méthode de preuve directe est utilisée pour la majorité des preuves concernant le noyau Pip développé dans l'équipe. La principale force de la preuve directe est qu'elle se contente de prouver uniquement la propriété finale ; chaque proposition intermédiaire ne dépasse pas le cadre de la preuve initiale. La contrepartie d'une telle approche est qu'il est fastidieux de modifier une preuve si des modifications mineures venaient à être apportées aux prémisses.
			
			\paragraph{Preuve par raffinement} La preuve par raffinement est plutôt employée lorsque les étapes de la preuve à conduire ne sont pas immédiates, souvent de par la complexité de la preuve. La méthode de preuve par raffinement consiste à appliquer la stratégie << diviser pour régner >> en trouvant des abstractions intermédiaires permettant de découper la preuve en morceaux indépendants, ce qui permet par ailleurs de rendre la preuve modulaire. Ces avantages viennent au prix d'un effort d'abstraction supplémentaire non requis par la preuve directe. Certaines de ces abstractions intermédiaires peuvent néanmoins avoir été définies par des travaux préliminaires, réduisant le coût de recherche des abstractions. Par ailleurs, nous défendons la thèse que la preuve par raffinement et l'utilisation d'abstractions de manière générale éloigne la preuve conduite de l'objet d'étude initial, et mène plus facilement à l'oubli de certaines contraintes. On pourrait notamment citer le paradoxe du raffinement\textcolor{red}{C'est une opinion, a-t'elle sa place dans l'état de l'art ?}

			De nombreux travaux munis de preuves formelles sur les systèmes d'exploitation utilisent le raffinement. CertikOS et seL4 en sont les exemples les plus connus. Le projet CompCert, compilateur de code source C garantissant la préservation de la sémantique lors de la compilation, utilise aussi le raffinement pour montrer que le processus de compilation est correct.

		\subsection{Preuve de programme}
			
			Cette sous-section traite de la manière de raisonner sur les programmes, en particulier sur les programmes impératifs dont le paradigme est à priori incompatible avec les langages fonctionnels des assistants de preuves tels que Coq et son langage Gallina. La première partie de la sous-section décrira la logique de Hoare : un système formel permettant de raisonner sur les programmes séquentiels. La seconde sous-section décrira les méthodes de représentation du programme à prouver dans l'assistant de preuve, ainsi que des monades d'état, permettant de représenter les effets de bords dans les langages fonctionnels, et servant d'interface entre le monde mathématique et le monde réel. La dernière partie de la sous-section présentera la problématique de la préservation de la preuve lors du processus de compilation du code source prouvé. 

			\subsubsection{Logique de Hoare}

			La logique de Hoare est un système formel permettant de raisonner sur les programmes séquentiels non-interruptibles.
			La logique de Hoare raisonne sur des \emph{triplets de Hoare} qui sont définis comme suit :
			
			\begin{subequations}
			\begin{gather}
			    \{P\}\ c\ \{Q\}
			\end{gather}
			
			où :
			\begin{itemize}
				\item $P$ représente les \emph{préconditions} qui sont les propriétés sur l'état de la machine supposées prouvables \emph{avant} l'exécution du programme à vérifier - ses \emph{prémisses}.
			    \item $c$ représente le code du programme à exécuter sur la machine et qu'on souhaite vérifier.
			    \item $Q$ représente les \emph{postconditions} qui sont les propriétés sur l'état de la machine qu'on souhaite pouvoir prouver \emph{après} l'exécution du code $c$ du programme à vérifier.
			\end{itemize}
			Par définition, un triplet de Hoare est \emph{prouvable si et seulement si}, quel que soit l'état du système initial $E_0$ satisfaisant les propriétés $P$ alors $c$ fait entrer le système dans un nouvel état $E_n$ dans lequel les propriétés $Q$ sont prouvables.
			Si le nouvel état $E_n$ engendré par $c$ viole les \emph{postconditions} $Q$, alors le triplet $\{P\}~c~\{Q\}$ est contradictoire.
			Au travers du prisme intuitionniste, on pourrait lire le triplet de Hoare $\{P\}~c~\{Q\}$ comme << Si les propriétés $P$ sur l'état de la machine sont supposées prouvables avant l'exécution du code du programme $c$, alors les propriétés $Q$ sur l'état de la machine après exécution de $c$ sont prouvables >>.
			
			Pour garantir formellement que $c$ respecte les propriétés qu'on souhaite garantir sur le système, on peut décomposer le code du programme $c$ en parties $c_i$ aussi élémentaires que souhaité.\\
			Le triplet devient alors : 
			
			\begin{gather}
			    \{P\}~c_1 ; c_2 ; \hdots ; c_n~\{Q\}
			\end{gather}

			On pourrait par exemple considérer que ces parties élémentaires sont les instructions assembleur de la machine.

			La logique de Hoare nous permet alors de décomposer la preuve en observant un enchaînement d'états intermédiaires $E_i$ résultant de l'exécution des instructions $c_i$.

			\begin{gather}
				E_0 \overset{c_1}{\rightarrow} E_1 \overset{c_2}{\rightarrow} \hdots \overset{c_n}{\rightarrow} E_n
			\end{gather}

			 Chaque instruction $c_i$ amène de nouvelles propriétés sur l'état $E_i$ nouvellement créé et permet de créer des triplets de Hoare intermédiaires. Ces nouvelles propriétés servent à la fois de postconditions $Q_i$ pour le triplet concernant l'instruction $c_i$ et de préconditions pour le triplet de l'instruction suivante.
			
			\begin{gather}
			    \{P\}~c_1~\{Q_1\}, \{Q_1\}~c_2~\{Q_2\},~\hdots~, \{Q_{n-1}\}~c_n~\{Q_n\}
			\end{gather}
			\end{subequations}
		
			Ainsi, au fur et à mesure de l'exécution du programme et de l'enchaînement des états, les propriétés sur l'état de la machine changent et s'étoffent. Pour montrer que les postconditions $Q$ souhaitées sur l'état final $E_n$ sont prouvables, il faut montrer qu'elles sont prouvables si les postconditions $Q_n$ résultant de l'exécution de $c$ sont prouvables, soit $Q_n \implies Q$.

			\paragraph{Modèle de la machine et sémantique opérationnelle} Pour pouvoir raisonner grâce à la logique de Hoare, un modèle de la machine et des instructions élémentaires doit être établi. Ces modèles permettent de décrire un système de transition d'état, donnant une signification formelle au programme. Cette sémantique du programme, décrivant des états successifs de la machine, est appelée \emph{sémantique opérationnelle}.

			Par ailleurs, il est important de garder à l'esprit que ces modèles sont \textbf{arbitraires}, et qu'ils ne sont pas intrinséquement équivalents à la machine et aux instructions modélisées. Chaque choix de modèle pour l'étude d'un objet réel vient avec sa part d'incertitude, quelles que soient les précautions prises lors de ce choix.	La preuve d'un programme n'échappe pas à cette règle : elle est aussi contestable que les modèles sur lesquels elle repose.

			\subsubsection{Langage}
			L'assistant de preuve Coq utilise un langage de programmation fonctionnel nommé Gallina. Étant un langage fonctionnel, il est peu adapté à l'écriture de certains programmes tels que les systèmes d'exploitation qui sont des programmes impératifs séquentiels. La sous-section précédente a introduit la logique de Hoare, permettant de raisonner sur de tels programmes. Il faut alors réussir à représenter les programmes séquentiels dans un langage fonctionnel. \textcolor{red}{J'arrive pas à introduire le concept} Deux méthodes existent : le \emph{shallow embedding} et le \emph{deep embedding}.

				\paragraph{Deep embedding} Le deep embedding permet de représenter un langage cible en modélisant la syntaxe (et plus particulièrement l'\emph{AST} du langage cible ainsi que sa sémantique. Il permet de ce fait d'exprimer des propriétés sur la structure du programme. Ces propriétés sont parfois intéressantes, notamment pour procéder à la preuve de propriétés concernant la compilation du langage ciblé. Le projet CompCert utilise un deep embedding de C pour garantir la préservation de la sémantique lors de la compilation vers l'assembleur Intel.
				\paragraph{Shallow embedding} Le shallow embedding au contraire ne définit pas la syntaxe du langage cible, et se contente de ne définir que la sémantique des programmes à prouver. Cette approche est plus légère, mais ne permet cependant pas de prouver de propriétés relatives à la structure du programme. Un shallow embedding se sert de la syntaxe du langage hôte.

				\label{monad}
				\paragraph{Monade d'état} Les langages purement fonctionnels n'ont pas de notion d'état. Ainsi pour modéliser fidèlement des programmes impératifs et leurs effets de bords sur l'état de la machine, on peut le simuler. Dans le cas de Pip, une monade d'état est toute indiquée : elle permet en quelque sorte d'<< enrober >> des valeurs avec un état. Pip utilise des fonctions manipulant ces valeurs et qui sont des fonctions partielles sur l'état, indiquant comment produire la valeur de retour mais aussi quelles modifications éventuelles sont apportées à l'état. Ces fonctions sont dites \emph{monadiques}. Cette construction explicite l'état implicite des programmes impératifs qui transite de fonction en fonction grâce à la fonction \texttt{bind}. Le passage d'état est transparent dans le code grâce à du sucre syntaxique. En plus de la fonction \texttt{bind}, chaque monade nécessite un élément neutre (ou \emph{unit}). Les effets des deux fonctions sont expliqués ci-après.

				Voyons comment la monade d'état a été définie dans Pip. Tout d'abord, Pip définit un type \texttt{state} représentant l'état de la machine qui transitera de fonction en fonction. Pip définit aussi un type \texttt{result} représentant soit une valeur \texttt{val} soit un comportement indéfini \texttt{undef}.

\begin{figure}[!h]
	\coqcode{code/monad.v}
	\caption{Définition de la monade d'état avec ses fonctions \texttt{bind} et \texttt{ret}}
	\label{code:monad}
\end{figure}

Ces types sont utilisés pour définir la monade d'état \texttt{LLI}, présentée en figure \ref{code:monad}. La monade \texttt{LLI} est définie comme une fonction sur des types, prenant un type \texttt{state} en argument, et renvoyant une valeur de retour de type \texttt{result} composé d'un type arbitraire \texttt{A} et d'un \texttt{state}. Les fonctions partielles sur l'état renvoyant un type \texttt{LLI} sont donc \emph{monadiques}. La monade \texttt{LLI} << enrobe >> des valeurs quelconques avec le type \texttt{state}.

Lorsqu'on lui donne une fonction monadique \texttt{f} de \texttt{A} vers \texttt{LLI B} et une valeur monadique de type \texttt{LLI A}, la fonction \texttt{bind} permet de récupérer la valeur << enrobée >> \texttt{A} et de lui appliquer la fonction \texttt{f}, produisant une valeur monadique \texttt{LLI B}. La fonction \texttt{ret} sert d'élément neutre, qui à partir d'un état et d'une valeur, renvoie cette même valeur << enrobée >> avec l'état.

Peut être devrais-je donner une définition plus formelle d'une monade qui pourrait s'avérer beaucoup plus parlante. La définition repose sur la théorie des catégories et est attribuée à Saunders Mac Lane \cite[134]{mac2013categories} que l'on pourrait traduire de la sorte :

\blockquote{Une monade dans $X$ est juste un monoïde dans la catégorie des endofoncteurs de $X$, ayant pour produit $\times$ la composition d'endofoncteurs et pour élément neutre l'endofoncteur identité.}

			\subsubsection{Compilation et préservation de la sémantique}
			\label{compilation}
				Une problématique subsiste lorsque des propriétés formelles ont été prouvées sur du code source. Comment peut-on s'assurer que ces propriétés resteront prouvables une fois que le code source aura été compilé ?

				\paragraph{CompCert} La recherche dans ce domaine a mené au projet CompCert\cite{Leroy-backend}, un compilateur de code source C dont la préservation de la sémantique est garantie formellement jusqu'au code assembleur.

	La propriété de préservation de la sémantique est enoncée comme suit :

\begin{theorem}	
	Pour tout programme source $S$ et pour tout code $C$ généré par le compilateur, si le compilateur a produit le code $C$ à partir du programme source $S$ sans remonter d'erreur de compilation, alors le comportement observable de $C$ est l'un des comportements observables possibles de $S$.
\end{theorem}

CompCert compile le programme source $S$ à partir de son \emph{Abstract Syntax Tree} (ou \emph{AST}) issu de la passe du préprocesseur, de l'analyseur syntaxique et de la phase de vérification des erreurs de type (\emph{type checking}). CompCert produit un code assembleur $C$ sous la forme d'un \emph{AST} du langage assembleur ciblé.

À ce jour, le projet CompCert a prouvé 90\% de la chaîne de compilation, notamment les algorithmes d'optimisation et de génération de code assembleur. Les 10\% restants à prouver incluent l'assemblage et la phase d'édition des liens \cite{compcert_online}.

Les projets ayant recourt à la vérification de programme par \emph{deep embedding} ont l'avantage de pouvoir aisément reconstruire l'\emph{AST} du programme et de le compiler grâce à CompCert. Cependant, pour les projets comme Pip utilisant un \emph{shallow embedding} et n'explicitant pas l'\emph{AST} du programme vérifié, l'étape de reconstruction du programme n'est pas triviale.

Auparavant, la chaîne de compilation de Pip utilisait \emph{Digger}~\cite{digger}, un outil écrit en Haskell qui transforme le code C shallow embedded écrit en Gallina en code C. Digger n'est pas muni de preuve formelle de préservation de la sémantique. Ainsi, l'argument de la préservation de la sémantique de Pip résidait dans l'extrême simplicité apparente de la conversion.

		\paragraph{$\partial x$} $\partial x$ est le remplaçant de \emph{Digger} dans la chaîne de compilation de Pip. Comme \emph{Digger}, il permet de transformer le code C écrit en Gallina vers un code source C, sous une forme attendue par CompCert.

		$\partial x$ comporte deux phases de fonctionnement : la première phase consiste à extraire l'\emph{AST} du programme écrit en Gallina sous la forme d'une représentation intermédiaire ; la seconde transforme cet \emph{AST} en \emph{AST} CompCert \texttt{Csyntax}. La première phase est écrite en Elpi~\cite{elpi} car Coq ne propose pas nativement de mécanisme de \emph{réflexion}. La seconde phase est écrite directement dans Coq. Il est de ce fait désormais possible de raisonner sur la phase de transformation de l'\emph{AST}.~~$\partial x$ permet en outre d'afficher le code source C produit, en utilisant le \emph{pretty-printer} de CompCert.

		\subsection{Illustration système}	

			\subsubsection{seL4}
	seL4 \cite{sel4website} est un noyau de système d'exploitation de la famille des noyaux L4 pour de nombreuses architectures (Armv6, Armv7, Armv8, x86, x86\_64 et RISC-V RV64) \cite{sel4hardware}. seL4 propose des mécanismes de gestion de la mémoire virtuelle, de gestion des interruptions, de communication inter-processus (\emph{IPC}) qui reposent sur un système gestion des droits par \emph{capacités}~\cite{capabilities}.

	seL4 offre en outre, pour certaines plateformes, une vérification formelle de son implémentation. Cette vérification peut inclure :
	\begin{itemize}
		\item{une preuve formelle fonctionnelle du code C (c'est à dire une preuve que le code respecte sa spécification) \cite{sel4}}
		\item{une preuve formelle de la propagation de la preuve jusqu'au binaire exécutable\cite{sel4binary}}
		\item{une preuve du maintient de l'\emph{intégrité} et de la \emph{confidentialité} \cite{sel4integrity}}
		\item{des propriétés temps-réel notamment sur le respect de bornes sur le temps d'exécution (\emph{WCET})\cite{sel4wcet}}
	\end{itemize}

	Pour arriver au code C exécutable, seL4 définit tout d'abord une spécification abstraite, définissant la fonction du code à produire. Ensuite, un prototype en Haskell est implémenté, respectant à priori cette spécification. Ce prototype permet de générer automatiquement une spécification exécutable, définissant comment le code doit remplir sa fonction. L'implémentation réelle en C est réalisée manuellement et doit respecter la spécification exécutable.

	La preuve fonctionnelle de code dans le noyau seL4 est donc naturellement découpée en un raffinement en trois couches : la spécification abstraite, la spécification exécutable et l'implémentation réelle. La première étape pour établir une preuve fonctionnelle est de montrer que la spécification exécutable dérivée du prototype Haskell raffine la spécification abstraite. La seconde étape est de montrer que l'implémentation en C raffine la spécification exécutable \cite{sel4}.

			\subsubsection{CertiKOS}

	CertiKOS \cite{certikoswebsite} est un outil d'aide à la conception de noyau de système d'exploitation. CertiKOS propose une conception du noyau par de multiples couches d'abstractions propices au raffinement.

	Plus particulièrement, CertiKOS utilise le concept de \emph{raffinement contextuel}. En quelques mots, la méthodologie de CertiKOS consiste à définir des triplets $(L_1, M, L_2)$ représentant chacun une couche d'abstraction définissant une interface. $L_2$ représente l'interface que l'on souhaite certifier. $M$ représente l'implémentation de l'interface $L_2$ s'appuyant sur l'interface sous-jacente $L_1$. L'idée de la méthologie de CertiKOS est que chaque couche d'abstraction $L_2$ est assez précise pour capturer tous les comportements observables de l'implémentation $M$. Ce type de spécification est appelé \emph{spécification profonde}. Ainsi, une fois que la preuve que $M$ implémente l'interface $L_2$ a été établie, il est possible de raisonner exclusivement sur $L_2$ sans jamais avoir à raisonner sur $M$ de nouveau \cite{gu2015deep}.

	CertiKOS a certifié mCertiKOS, un noyau de système d'exploitation et hyperviseur, capable de faire tourner Linux et divisé en 40 couches d'abstraction \cite{gu2011certikos}. Plus récemment, mC2, un noyau de système d'exploitation concurrent \cite{concurrentcertikos, gu2016certikos} a été présenté à la communauté.

			\subsubsection{Pip}

	Pip est un noyau de système d'exploitation \emph{minimal} dont le seul but est la gestion de portions isolées de mémoire appelées \emph{partitions}, et du transfert de flôt d'exécution entre ces partitions. Pip est muni d'une preuve formelle garantissant que ses appels systèmes ne brisent pas \emph{l'isolation} des partitions. Pip a été initialement conçu pour fonctionner avec de la mémoire virtuelle en manipulant une \emph{MMU}. Cependant, de récents travaux ont fait évoluer Pip pour qu'il supporte la mémoire physique par le biais d'une \emph{MPU}.

	Dans Pip, les partitions forment une structure arborescente qui déterminent les droits d'accès à la mémoire. Au démarrage du système, une partition responsable de la totalité de la mémoire du système est créée. Pip permet à chaque partition de créer une autre partition en partageant une partie de sa propre mémoire avec la partition nouvellement créée. La nouvelle partition est appelée \emph{partition enfant}, la partition ayant partagé sa propre mémoire est appelée \emph{partition parent}. Cette relation parent/enfant crée la structure arborescente : les enfants pouvant créer à leur tour de nouvelles partitions en partageant leur mémoire.

	\paragraph{Propriété d'isolation} La propriété d'isolation de Pip est divisée en trois sous-propriétés :
	\begin{itemize}
		\item La première, la propriété de \textbf{partage vertical}, stipule que la mémoire partagée par une partition parent avec ses enfants reste accessible au parent par conception.

		\item La seconde, la propriété d'\textbf{isolation horizontale}. Dans Pip, chaque partition peut créer plusieurs partitions enfant ; cependant les portions de mémoire partagées avec chacune doit être strictement disjointe. Autrement dit, une partition ne peut pas partager une même portion de mémoire avec deux partitions enfant simultanément. Ainsi, deux partitions enfant issues d'une même partition parent sont \emph{isolées} : la mémoire accessible dans l'une d'entre elle est nécessairement inaccessible dans l'autre.

		\item La dernière, la propriété d'\textbf{isolation noyau}. À chaque création de partition, Pip réserve une petite portion de mémoire afin d'y stocker les structures nécessaires au contrôle des droits. Ces portions de mémoire deviennent inaccessibles à n'importe quelle partition.
	\end{itemize}


	\paragraph{Méthodologie de preuve} Pip a pris le contrepied des projets majeurs du domaine en utilisant un \emph{shallow embedding} de C plutôt que d'utiliser un \emph{deep embedding}. Ce \emph{shallow embedding} nécessite l'utilisation d'une monade d'état en Coq (voir \ref{monad}) et d'un outil spécifique (voir $\partial x$ \ref{compilation}) pour reconstruire l'AST ou le code source du noyau pour le compiler.
	Le pari de cette méthodologie est de pouvoir se concentrer sur la sémantique des programmes à prouver afin d'alléger l'effort de preuve général par rapport aux méthodologies utilisant un \emph{deep embedding}. Aussi, les proriétés de préservation de l'isolation de Pip ont été prouvées directement, plutôt qu'en passant par un raffinement.



    \chapter{Demo Chapter}
    \begin{jointwork}
    This is where you can write some meta information about your chapter. For example, this chapter is based on one of my publications~\cite{firstDemoReference}, and I just blindly copied everything without adjusting it. Just a heads-up warning.
    
    Sadly, if you cite your own publications, they will appear in the bibliography. Thus, make sure to cite your papers with yourself as one of the authors.
\end{jointwork}

This chapter shows off some of the basic formats of this thesis. Many packages are included in order for you to be able to start immediately without having to manually add all of the important things. The features deemed most important are now presented.

Here is just some filler text.\footnote[-0.2][][15]{Here is a footnote with a strange number (if that floats your boat). Note how the footnote mark is \emph{above} the period at the end of the sentence.} The following citations use the command \textsf{textcite}: \textcite{firstDemoReference}; \textcite{secondDemoReference}. The first reference has a short list of authors, the second one a long list.\todo{Also note that you can include to-do notes if necessary. Delete this chapter!}

\newcommand*{\filtration}{\mathOrText{\mathcal{F}}}
\newcommand*{\randomProcess}{\mathOrText{X}}
\newcommand*{\timePoint}{\mathOrText{t}}
\newcommand*{\target}{\mathOrText{x_{\min}}}
\newcommand*{\firstHittingTime}{\mathOrText{T}}
\createFunction{\driftFunction}{h}
\newcommand*{\functionDomain}{\mathOrText{D}}
We now state a theorem and restate it later on again. Have a look at the source code in order to see how the theorem is written. Many macros are used, and all of them can be used without using math mode explicitly. Note that we can refer to \cref{eq:variableDrift} as an inequality through the magic of an option in its label.
\begin{restatable}[Variable Drift]{theorem}{variableDrift}
    \label{thm:variableDrift}
	Let $(\filtration_\timePoint)_{\timePoint \in \mathds{N}}$ be a filtration, $(\randomProcess_\timePoint)_{\timePoint \in \mathds{N}}$ be a random process over~$\mathds{R}^+_0$ adapted to~\filtration\!\!, $\target > 0$, and let $\firstHittingTime = \inf\set{\timePoint}{\randomProcess_\timePoint < \target}$. Additionally, let~\functionDomain denote the smallest real interval that contains at least all values $x \geq \target$ that, for all $\timePoint \leq \firstHittingTime$, any $\randomProcess_\timePoint$ can take. Furthermore, suppose that
    \begin{enumerate}
        \item $\randomProcess_0 \geq \target$ and that
        
	\item there is a monotonically increasing function $\driftFunction\colon \functionDomain \to \mathds{R}^+$ such that, for all $\timePoint < \firstHittingTime$, we have $\randomProcess_t - \E{\randomProcess_{\timePoint + 1}}[\filtration_\timePoint] \geq \driftFunction[\randomProcess_\timePoint]$.
    \end{enumerate}
    Then
    \begin{equation}
        \E{\firstHittingTime}[\filtration_0] \leq \frac{\target}{\driftFunction[\target]} + \int^{\randomProcess_0}_{\target} \frac{1}{\driftFunction[z]} \d z\ .\label[ineq]{eq:variableDrift}\qedhere
    \end{equation}
\end{restatable}

Please shift your attention to \Cref{fig:logos}. This reference was created using the package \textsf{cleveref}, which knows in what environment the label is defined in. This way, you can easily change a theorem into a lemma, and the name of the reference will be adjusted automatically. A wrapfigure like \Cref{fig:HPIwrap} is referenced just like a normal figure.

\begin{figure}
    \centering
    \begin{subfigure}[b]{0.45\textwidth}
        \centering
        \def\svgwidth{0.90\textwidth}
        \input{images/ULille_black.pdf_tex}

        \caption{This is the caption of the subfigure that displays the logo of the ULille.}
        \label{fig:HPI}
    \end{subfigure}
    \hfil
    \begin{subfigure}[b]{0.45\textwidth}
        \centering
        \def\svgwidth{0.90\textwidth}
        \input{images/ULille_black.pdf_tex}

        \caption{This is the caption of the subfigure that displays the logo of the ULille.}
        \label{fig:UP}
    \end{subfigure}
    \caption{These are the two logos featured on the title page. \Cref{fig:HPI} belongs to the HPI, whereas \Cref{fig:UP} belongs to the UP.}
    \label{fig:logos}
\end{figure}

Of course, you can also use tables in a fancy style. See, for example, \Cref{tab:textAndNumbers}. This document already contains packages in order to also handle larger tables. Hence, it is possible to use tables spanning multiple pages or to rotate a page into landscape in order to fit in a wider table.

Before we continue, consider the following obvious theorem. We conjecture that it also holds for $n = 2$.
\begin{theorem}
    \label{thm:fermatsTheorem}
    Let $a, b, c, n \in \mathds{N}^+$ with $n > 2$. Then
    \[
        a^n + b^n \neq c^n\ .\qedhere
    \]
\end{theorem}

Since the proof is straightforward, it is omitted. Nonetheless, we present a proof in order to show off the proof environment.

\begin{proof}[Proof of \Cref{thm:fermatsTheorem}]
    Unfortunately, there is too little space in this PDF for the proof.
\end{proof}

You can have very expressive and fancy enumerations from the package \textsf{enumitem}. Again, we can easily reference an item like \cref{item:talkingAboutLabels}.
\begin{enumerate}[label = (\roman*), align = left, labelwidth = 2 em, labelsep = 0 em]
    \item The labels of the items can be nicely chosen.\label{item:talkingAboutLabels}
    
    \item Note how the labels are left-aligned. This does not look good but should demonstrate what is easily possible.
\end{enumerate}
We can even interrupt this enumeration and easily resume it immediately.
\begin{enumerate}[resume*]
    \item We continue where we left off.
\end{enumerate}

\begin{table}[t]
    \centering
    \caption{This is a nicely formatted table. Thus, the caption is \emph{above} the content. If not, the data could not be interpreted meaningfully. As a rule of thumb, never use vertical lines\footnote[0][-0.2]{Except you know what you are doing.}, and use horizontal lines sparingly. If you think that a table is illegible and thus needs vertical lines, then your spacing between columns is wrong and should be increased. Always use some whitespace first before you use some additional lines.}
    \label{tab:textAndNumbers}
    \begin{tabular}{p{0.75\textwidth}>{\bfseries}r}
        \toprule
        Text                                                                & Number\\\midrule
        This is some text. Thus, it is left-aligned.                        & 0\\
        Numbers are right-aligned.                                          & 1\\
        The numbers are formatted in bold using the package \textsf{array}. & 2\\\bottomrule
    \end{tabular}
\end{table}

Recall that \Cref{thm:variableDrift} was as follows:

\variableDrift*

Note that the reference above still refers to the first occurrence of the theorem. However, the theorem is repeated without any noise. That is, it is identical to the other occurrence.

From the next page on, other than a warp figure and some filler text, there is not much more to see. Thank you very much for taking your time and reading so far. I hope you got an impression of what this template is capable of. Have fun using it, and create a great thesis!

\section{Senseless Section}

Lorem ipsum dolor sit amet, consetetur sadipscing elitr, sed diam nonumy eirmod tempor invidunt ut labore et dolore magna aliquyam erat, sed diam voluptua. At vero eos et accusam et justo duo dolores et ea rebum. Stet clita kasd gubergren, no sea takimata sanctus est Lorem ipsum dolor sit amet. Lorem ipsum dolor sit amet, consetetur sadipscing elitr, sed diam nonumy eirmod tempor invidunt ut labore et dolore magna aliquyam erat, sed diam voluptua. At vero eos et accusam et justo duo dolores et ea rebum. Stet clita kasd gubergren, no sea takimata sanctus est Lorem ipsum dolor sit amet. Lorem ipsum dolor sit amet, consetetur sadipscing elitr, sed diam nonumy eirmod tempor invidunt ut labore et dolore magna aliquyam erat, sed diam voluptua. At vero eos et accusam et justo duo dolores et ea rebum. Stet clita kasd gubergren, no sea takimata sanctus est Lorem ipsum dolor sit amet.

\begin{wrapfigure}{r}{0.25\textwidth}
    \def\svgwidth{0.25\textwidth}
    \input{images/ULille_black.pdf_tex}

    \caption{The ULille logo is sneaked in between text.}
    \label{fig:HPIwrap}
\end{wrapfigure}

Duis autem vel eum iriure dolor in hendrerit in vulputate velit esse molestie consequat, vel illum dolore eu feugiat nulla facilisis at vero eros et accumsan et iusto odio dignissim qui blandit praesent luptatum zzril delenit augue duis dolore te feugait nulla facilisi. Lorem ipsum dolor sit amet, consectetuer adipiscing elit, sed diam nonummy nibh euismod tincidunt ut laoreet dolore magna aliquam erat volutpat.

Ut wisi enim ad minim veniam, quis nostrud exerci tation ullamcorper suscipit lobortis nisl ut aliquip ex ea commodo consequat. Duis autem vel eum iriure dolor in hendrerit in vulputate velit esse molestie consequat, vel illum dolore eu feugiat nulla facilisis at vero eros et accumsan et iusto odio dignissim qui blandit praesent luptatum zzril delenit augue duis dolore te feugait nulla facilisi.

Nam liber tempor cum soluta nobis eleifend option congue nihil imperdiet doming id quod mazim placerat facer possim assum. Lorem ipsum dolor sit amet, consectetuer adipiscing elit, sed diam nonummy nibh euismod tincidunt ut laoreet dolore magna aliquam erat volutpat. Ut wisi enim ad minim veniam, quis nostrud exerci tation ullamcorper suscipit lobortis nisl ut aliquip ex ea commodo consequat.

Duis autem vel eum iriure dolor in hendrerit in vulputate velit esse molestie consequat, vel illum dolore eu feugiat nulla facilisis.

At vero eos et accusam et justo duo dolores et ea rebum. Stet clita kasd gubergren, no sea takimata sanctus est Lorem ipsum dolor sit amet. Lorem ipsum dolor sit amet, consetetur sadipscing elitr, sed diam nonumy eirmod tempor invidunt ut labore et dolore magna aliquyam erat, sed diam voluptua. At vero eos et accusam et justo duo dolores et ea rebum. Stet clita kasd gubergren, no sea takimata sanctus est Lorem ipsum dolor sit amet. Lorem ipsum dolor sit amet, consetetur sadipscing elitr, At accusam aliquyam diam diam dolore dolores duo eirmod eos erat, et nonumy sed tempor et et invidunt justo labore Stet clita ea et gubergren, kasd magna no rebum. sanctus sea sed takimata ut vero voluptua. est Lorem ipsum dolor sit amet. Lorem ipsum dolor sit amet, consetetur sadipscing elitr, sed diam nonumy eirmod tempor invidunt ut labore et dolore magna aliquyam erat.

Consetetur sadipscing elitr, sed diam nonumy eirmod tempor invidunt ut labore et dolore magna aliquyam erat, sed diam voluptua. At vero eos et accusam et justo duo dolores et ea rebum. Stet clita kasd gubergren, no sea takimata sanctus est Lorem ipsum dolor sit amet. Lorem ipsum dolor sit amet, consetetur sadipscing elitr, sed diam nonumy eirmod tempor invidunt ut labore et dolore magna aliquyam erat, sed diam voluptua. At vero eos et accusam et justo duo dolores et ea rebum. Stet clita kasd gubergren, no sea takimata sanctus est Lorem ipsum dolor sit amet. Lorem ipsum dolor sit amet, consetetur sadipscing elitr, sed diam nonumy eirmod tempor invidunt ut labore et dolore magna aliquyam erat, sed diam voluptua. At vero eos et accusam et justo duo dolores et ea rebum. Stet clita kasd gubergren, no sea takimata sanctus.

Lorem ipsum dolor sit amet, consetetur sadipscing elitr, sed diam nonumy eirmod tempor invidunt ut labore et dolore magna aliquyam erat, sed diam voluptua. At vero eos et accusam et justo duo dolores et ea rebum. Stet clita kasd gubergren, no sea takimata sanctus est Lorem ipsum dolor sit amet. Lorem ipsum dolor sit amet, consetetur sadipscing elitr, sed diam nonumy eirmod tempor invidunt ut labore et dolore magna aliquyam erat, sed diam voluptua. At vero eos et accusam et justo duo dolores et ea rebum. Stet clita kasd gubergren, no sea takimata sanctus est Lorem ipsum dolor sit amet. Lorem ipsum dolor sit amet, consetetur sadipscing elitr, sed diam nonumy eirmod tempor invidunt ut labore et dolore magna aliquyam erat, sed diam voluptua. At vero eos et accusam et justo duo dolores et ea rebum. Stet clita kasd gubergren, no sea takimata sanctus est Lorem ipsum dolor sit amet.

Duis autem vel eum iriure dolor in hendrerit in vulputate velit esse molestie consequat, vel illum dolore eu feugiat nulla facilisis at vero eros et accumsan et iusto odio dignissim qui blandit praesent luptatum zzril delenit augue duis dolore te feugait nulla facilisi. Lorem ipsum dolor sit amet, consectetuer adipiscing elit, sed diam nonummy nibh euismod tincidunt ut laoreet dolore magna aliquam erat volutpat.

Ut wisi enim ad minim veniam, quis nostrud exerci tation ullamcorper suscipit lobortis nisl ut aliquip ex ea commodo consequat. Duis autem vel eum iriure dolor in hendrerit in vulputate velit esse molestie consequat, vel illum dolore eu feugiat nulla facilisis at vero eros et accumsan et iusto odio dignissim qui blandit praesent luptatum zzril delenit augue duis dolore te feugait nulla facilisi.

Nam liber tempor cum soluta nobis eleifend option congue nihil imperdiet doming id quod mazim placerat facer possim assum. Lorem ipsum dolor sit amet, consectetuer adipiscing elit, sed diam nonummy nibh euismod tincidunt ut laoreet dolore magna aliquam erat volutpat. Ut wisi enim ad minim veniam, quis nostrud exerci tation ullamcorper suscipit lobortis nisl ut aliquip ex ea commodo.


    %%%%%%%%%%%%%%%%%%%%%%%%%%%%%%%%%
    %% End of adding your content. %%
    %%%%%%%%%%%%%%%%%%%%%%%%%%%%%%%%%


    % Add the following chapters not to the current ›part‹ but one level above instead.
    \makeatletter
        \def\toclevel@chapter{-1}
        \def\toclevel@section{0}
    \makeatother

    \chapter{Conclusion}
    % This is where you conclude your thesis.

	\section{Conclusion}
		%\subsection{Résumé des contributions}

	\section{Perspectives}
		\subsection{Preuve fonctionelle de Pip}
		\subsection{Preuve du backend}
		\subsection{Logique de séparation}
		\subsection{Event-driven Earliest Deadline First}

	\section{Retour d'expérience ?}
		% Conseils à mon moi de début de thèse


    % Following are the files and commands for the bibliography and the author’s publications.
    \pagestyle{plain}

    \renewcommand*{\bibfont}{\small}
    \printbibheading
    \addcontentsline{toc}{chapter}{Bibliographie}
    \printbibliography[heading = none]

\end{document}
