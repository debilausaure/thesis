\documentclass
[
    twoside,                 % The thesis is formatted like a book. That is, odd and even pages are handled differently.
    openright,               % Starts a new chapter on an odd page number (right side).
    cleardoublepage = empty, % Clear pages inserted in order to have new chapters appear on odd pages are formatted with an empty style.
    fontsize = 11 pt,        % The size of the font.
    french,                % Support for American English.
    captions = tableheading, % Places the correct amount of space when the caption of a table is above the table.
    numbers = noenddot,      % Does not use a period at the end of numbered titles, such as sections or figures.
    footheight = 35 pt,      % Defines the height of the foot. Due to the line, it needs extra height.
%    draft,                   % Only displays boxes of figures. This option is useful if compilation takes a long time.
]
{scrbook}


% This file contains all sorts of commands that are used in order to specify certain options for the document.

\newif\ifprintVersion   % Defines a binary variable that signals whether the document is prepared for physical or digital print.
\newif\ifprofessionalPrint % Defines a binary variable that signals whether the print will be done by a professional printing service that requests extra margin for page cutting and is not bound to paper formats like A4.
\newif\iffancyTheorems  % Defines a binary variable that signals whether theorems are formatted in the classical style or in a new format that better suits the overall flavor of this thesis.
\newif\ifboldNumberSets % Defines a binary variable that signals whether the variables for number sets (like N or R) should be in bold. If not, they are in blackboard bold instead.

% Set all variables to their default values.
\printVersionfalse
\professionalPrintfalse
\fancyTheoremstrue
\boldNumberSetstrue

%%%%%%%%%%%%%%%%%%%%%%%%
% The following commands define certain strings that provide important information for the document.

% The title of the thesis.
\newcommand*{\printTitle}{}
\newcommand*{\myTitle}[1]{\renewcommand*{\printTitle}{#1}}
\newcommand*{\printTitleBold}{\textbf{\printTitle}}

% The author’s name.
\newcommand*{\printAuthor}{}
\newcommand*{\myName}[1]{\renewcommand*{\printAuthor}{#1}}

% The name of the author’s program.
\newcommand*{\printProgram}{}
\newcommand*{\myProgram}[1]{\renewcommand*{\printProgram}{#1}}

% A short description of the topic of the thesis. This string will be used for the PDF metadata.
\newcommand*{\printSubject}{}
\newcommand*{\mySubject}[1]{\renewcommand*{\printSubject}{#1}}

% A short description of the topic of the thesis. This string will be used for the PDF metadata.
\newcommand*{\printKeywords}{}
\newcommand*{\myKeywords}[1]{\renewcommand*{\printKeywords}{#1}}

% Defines the extra length added to each side for the print version.
\newlength{\extraborderlength}
\newcommand*{\extraBorder}[1]{\setlength{\extraborderlength}{#1}}

% Defines the length of the binding correction. (The class ›scrbook‹ has a binding correction but it does not work due to all the other packages that are loaded.)
\newlength{\mybindingcorrection}
\newcommand*{\bindingCorrection}[1]{\setlength{\mybindingcorrection}{#1}}
 % Contains commands that are used for certain information that is printed.


%%%%%%%%%%%%%%%%%%%%%%%%%%%%%%%%%%%%%%
%% Please adjust your options here. %%
%%%%%%%%%%%%%%%%%%%%%%%%%%%%%%%%%%%%%%

    % This section contains commands with important data for your thesis. Please adjust them in order for the document to be printed correctly.

    % Defines the length of the amount that a printed page will be cut from EACH side (including the inner side). This option only takes effect with \printVersiontrue and \professionalPrinttrue.
    \extraBorder{3 mm}

    % Shifts the inner margin outward by the amount specified. When the book is bound, part of the page will not be seen anymore. This option compensates for this loss. It only takes effect with \printVersiontrue.
    \bindingCorrection{6 mm}

    % Use the following command if this is a master thesis.

%    \printVersiontrue      % Use this value if you want to prepare your thesis for physical printing. In this case, links will not be colored. Without \professionalPrinttrue, the content will be moved outward by the binding correction, increasing the inner margin and decreasing the outer margin.
%    \professionalPrinttrue % Use this value if you want to have extra border for cutting and are not bound to paper formats (like A4). This option will increase the page size by the extra border on every side plus the binding correction once for the width. It only takes effect in combination with \printVersiontrue.
%    \fancyTheoremsfalse  % Use this value if you want to use the classical theorem style, where the text is italic. Further, with this style, the QED symbol is colorless.
%    \boldNumberSetsfalse % Use this value if you want variables for number sets (like N or R) to appear in blackboard bold rather than bold.

    % The title of the thesis.
    \myTitle{Conception, implémentation et preuve d'un service de transfert de flot d'exécution au sein d'un noyau de système d'exploitation}

    % The author’s name.
    \myName{Florian Vanhems}

    % The author’s program.
    \myProgram{Informatique}

    % A short summary of the thesis. These information will be used for the PDF meta data.
    \mySubject{A cool bachelor/master thesis.}

    % Some keywords of the thesis. These information will be used for the PDF meta data. Please use | as a separator and try to avoid commas.
    \myKeywords{bachelor--master thesis | world-changing | very important | please like and share and subscribe}

%%%%%%%%%%%%%%%%%%%%%%%%%%%%%%%%%%%%%%
%% End of options to adjust. %%%%%%%%%
%%%%%%%%%%%%%%%%%%%%%%%%%%%%%%%%%%%%%%


% This file includes all of the code that is used to format the thesis.
% Some packages are included if they are needed. This is done in the respective part and not at the beginning of this file.
%
% This file contains the following parts:
%   • Language an Character Set
%   • Penalties
%   • Indentation
%   • Footnotes
%   • Colors
%   • Size and Position of the Text Body
%   • Position of the Head and the Foot
%   • Margin Position and Width
%   • Header and Footer Format
%   • Caption Format
%   • Part Format
%   • Chapter Format
%   • Table of Contents


%%%%%%%%%%%%%%%%%%%%%%%%%%%%%%%%
%% Language and Character Set %%
%%%%%%%%%%%%%%%%%%%%%%%%%%%%%%%%

\usepackage
[
    english,         % English is used for the English abstract.
    main = french, % This is the main language of the thesis.
]
{babel}                     % Is responsible for sensible hyphenations.

%%%%%%%%%%%%%%%
%% Penalties %%
%%%%%%%%%%%%%%%

\widowpenalties 2 10000 0


%%%%%%%%%%%%%%%%%
%% Indentation %%
%%%%%%%%%%%%%%%%%

\usepackage{calc} % Makes it easer to do math with TeX measurements.

\newlength{\myparindent}
\newlength{\myparskip}
\setlength{\myparindent}{1 em}
\setlength{\myparskip}{0 em}

\setlength{\parindent}{\myparindent}
\setlength{\parskip}{\myparskip}
\setlength{\parskip}{0 pt plus 1 pt minus 0 pt}


%%%%%%%%%%%%%%%
%% Footnotes %%
%%%%%%%%%%%%%%%

% Remove the footnote rule.
\setfootnoterule{0 cm}

% The footnote number is made bold and not in superscript.
\deffootnote[1.2 em]{1.2 em}{0 em}{\makebox[1.4 em][l]{\textbf{\thefootnotemark}}}

% The footnote number will not be reset after every chapter.
\makeatletter%
    \@removefromreset{footnote}{chapter}%
\makeatother


%%%%%%%%%%%%
%% Colors %%
%%%%%%%%%%%%

\usepackage[dvipsnames]{xcolor} % Allows it to define colors. The option says that common names can be used.

% Dark blue.
\definecolor{stroke1}{HTML}{2574A9} % This color is used as the standard color to highlight things.


% Coloring various different labels.
\colorlet{captionlabel}{black}
\colorlet{footerpagenr}{black}
\colorlet{footerchapter}{stroke1}
\colorlet{footerchaptername}{black}
\colorlet{footersection}{stroke1}
\colorlet{footersectionname}{black}
\colorlet{chapternumber}{stroke1}


%%%%%%%%%%%%%%%%%%%%%%%%%%%%%%%%%%%%%%%%
%% Size and Position of the Text Body %%
%%%%%%%%%%%%%%%%%%%%%%%%%%%%%%%%%%%%%%%%

% The new paper dimensions that are exclusively used.
\newlength{\mypaperwidth}
\setlength{\mypaperwidth}{210 mm}

\newlength{\mypaperheight}
\setlength{\mypaperheight}{297 mm}

% The text area uses aesthetically pleasing measurements in the same ratio as the page.
% These dimensions are always used, as the text area should be the same in the printed and digital version of the thesis.
\newlength{\mybodywidth}
\setlength{\mybodywidth}{140 mm}

\newlength{\mybodyheight}
\setlength{\mybodyheight}{198 mm}

\newlength{\myoutermargin}
\ifprintVersion
    \ifprofessionalPrint
        \setlength{\myoutermargin}{(\mypaperwidth - \mybodywidth) / \real{1.5} + \extraborderlength}
    \else
        \setlength{\myoutermargin}{(\mypaperwidth - \mybodywidth) / \real{1.5} - \mybindingcorrection}
    \fi
\else
    \setlength{\myoutermargin}{(\mypaperwidth - \mybodywidth) / \real{1.5}}
\fi

\newlength{\mytopmargin}
\setlength{\mytopmargin}{(\mypaperheight - \mybodyheight) / 3}
\ifprintVersion
    \ifprofessionalPrint
        \setlength{\mytopmargin}{(\mypaperheight - \mybodyheight) / 3 + \extraborderlength}
    \fi
\fi

\newlength{\myinnermargin}
\setlength{\myinnermargin}{\mypaperwidth - \mybodywidth - \myoutermargin}
\ifprintVersion
    \ifprofessionalPrint
        \setlength{\myinnermargin}{\mypaperwidth + \mybindingcorrection + 2\extraborderlength - \mybodywidth - \myoutermargin}
    \fi
\fi

\newlength{\mybottommargin}
\setlength{\mybottommargin}{\mypaperheight - \mybodyheight - \mytopmargin}
\ifprintVersion
    \ifprofessionalPrint
        \setlength{\mybottommargin}{\mypaperheight + 2\extraborderlength - \mybodyheight - \mytopmargin}
    \fi
\fi


%%%%%%%%%%%%%%%%%%%%%%%%%%%%%%%%%%%
%% Position of the Head And Foot %%
%%%%%%%%%%%%%%%%%%%%%%%%%%%%%%%%%%%

\newcommand{\goldenratio}{1.618}

\newlength{\myheadsep} % Distance from the header to the body.
\setlength{\myheadsep}{\mytopmargin / \real{\goldenratio} / \real{\goldenratio} - 1 ex}
\ifprintVersion
    \ifprofessionalPrint
        \setlength{\myheadsep}{(\mytopmargin - \extraborderlength) / \real{\goldenratio} / \real{\goldenratio} - 1 ex}
    \fi
\fi

\newlength{\myfootskip} % Distance from the body to the footer.
\setlength{\myfootskip}{\mybottommargin / \real{\goldenratio} - 1 ex}
\ifprintVersion
    \ifprofessionalPrint
        \setlength{\myfootskip}{(\mybottommargin - \extraborderlength) / \real{\goldenratio} - 1 ex}
    \fi
\fi


%%%%%%%%%%%%%%%%%%%%%%%%%%%%%%%
%% Margin Position And Width %%
%%%%%%%%%%%%%%%%%%%%%%%%%%%%%%%

\newlength{\mymargininnersep} % Distance between the margin and the body.
\setlength{\mymargininnersep}{7 mm}

\newlength{\mymarginoutersep} % Distance between the margin and the paper border.
\setlength{\mymarginoutersep}{12 mm}
\ifprintVersion
    \ifprofessionalPrint
        \setlength{\mymarginoutersep}{12 mm + \extraborderlength}
    \fi
\fi

\newlength{\mymarginwidth} % Width of the margin.
\setlength{\mymarginwidth}{\myoutermargin - \mymargininnersep - \mymarginoutersep}

\newlength{\mymarginwidthwithinnersep} % Width of the margin.
\setlength{\mymarginwidthwithinnersep}{\mymarginwidth + \mymargininnersep}

\usepackage
[
    % In the printed version, we add an extra border to each side as well as the binding correction for the width.
    \ifprintVersion
        \ifprofessionalPrint
            paperwidth = \mypaperwidth + 2\extraborderlength + \mybindingcorrection,
            paperheight = \mypaperheight + 2\extraborderlength,
        \else
            paperwidth = \mypaperwidth,
            paperheight = \mypaperheight,
        \fi
    \else
        paperwidth = \mypaperwidth,
        paperheight = \mypaperheight,
    \fi
    textwidth = \mybodywidth,
    textheight = \mybodyheight,
    outer = \myoutermargin,
    top = \mytopmargin,
    headsep = \myheadsep,
    footskip = \myfootskip,
    marginparsep = \mymargininnersep,
    marginparwidth = \mymarginwidth,
%    showframe, % Use this option for debugging purposes in order to the an outline of all of the different parts of the page layout.
]
{geometry} % Used in order to define the dimensions of the page and its layout.


%%%%%%%%%%%%%%%%%%%%%%%%%%%%%%
%% Header and Footer Format %%
%%%%%%%%%%%%%%%%%%%%%%%%%%%%%%

\usepackage
[
%    draft, % Shows a lot of rules denoting the dimensions of the head and foot. Use this option only for debugging.
]
{scrlayer-scrpage} % Allows to adjust the definitions of the head and foot of a page.

%%%%%%%%%%%%%%%%%%%%%%%%%%%%%%
% Dimensions and formats are defined.

% Define the dimensions of the head and the foot. Since we want some information to appear in the margin, we extend the head and the foot by the respective lengths.
\KOMAoptions
{%
    headwidth = \textwidth + \mymarginwidthwithinnersep,%
    footwidth = \myoutermargin : \textwidth,%
}

% Defines the formats for the chapter and section titles in the marks of the head.
\renewcommand*{\chaptermarkformat}{\normalfont\sffamily\small\color{footerchaptername}}
\renewcommand*{\sectionmarkformat}{\normalfont\sffamily\small\color{footersectionname}}

% Displays the chapter names in the head of both odd and even pages.
\automark[chapter]{chapter}
% Replaces the chapter name to the head of right pages with the section name if a section is present.
\automark*[section]{}

%%%%%%%%%%%%%%%%%%%%%%%%%%%%%%
% The head is defined.

% Head for even pages.
% Puts ›Chapter‹ followed by the current chapter number.
\lehead%
{%
    \begin{minipage}[b]{\mymarginwidth}%
        \small\raggedleft\normalfont\textsf{\textbf{\color{footerchapter}\chaptername\ \thechapter}}
    \end{minipage}
}
% Put the title of the current chapter/section into the center of the head but push it to the border.
\cehead{\hspace*{\mymarginwidthwithinnersep}\parbox{\textwidth}{\raggedright\leftmark}}

% Head for odd pages.
\rohead%
{%
    % Check whether a section has already started or not.
    \Ifstr{\rightmark}{\leftmark}%
    {%
        \begin{minipage}[b]{\mymarginwidth}%
            \small\raggedright\normalfont\textsf{\textbf{\color{footersection}Chapter\ \thechapter}}%
        \end{minipage}%
    }%
    {%
        \begin{minipage}[b]{\mymarginwidth}%
            \small\raggedright\normalfont\textsf{\textbf{\color{footersection}Section\ \thesection}}%
        \end{minipage}%
    }%
}
\cohead{\hspace*{-\mymarginwidthwithinnersep}\parbox{\textwidth}{\raggedleft\rightmark}}

%%%%%%%%%%%%%%%%%%%%%%%%%%%%%%
% The foot is defined.

% Displays the page number in bold in the margin, aligned toward the center. Further, a blue line is drawn above number.
% The starred variant is used, since we want the format of the foot to also apply to the pagestyle ›plain‹.
\lefoot*%
{%
    \vspace*{1 ex}%
    {\color{stroke1}\rule{\myoutermargin - \mymargininnersep}{0.5 mm}}\\
    \begin{minipage}[b]{\myoutermargin - \mymargininnersep}%
        \raggedleft\normalfont\color{footerpagenr}\textbf{\thepage}%
    \end{minipage}%
}
\rofoot*%
{%
    {\color{stroke1}\rule{\myoutermargin - \mymargininnersep}{0.5 mm}}\\
    \begin{minipage}[b]{\myoutermargin - \mymargininnersep}%
        \raggedright\normalfont\color{footerpagenr}\textbf{\thepage}%
    \end{minipage}%
}


%%%%%%%%%%%%%%%%%%%%
%% Caption Format %%
%%%%%%%%%%%%%%%%%%%%

\usepackage[justification=centering]{caption}
\captionsetup
{
    font = small,
    labelfont = {bf, sf, color = captionlabel},
    format = plain,
    singlelinecheck = off,
}

%%%%%%%%%%%%%%%%%
%% Part Format %%
%%%%%%%%%%%%%%%%%
\usepackage{tikz} % Used in order to draw the stylistic elements.

\usetikzlibrary{arrows,calc,positioning,patterns,shadows} 
\tikzset{
    %Define standard arrow tip
    >=stealth'
}

%define hatched pattern
\pgfdeclarepatternformonly{south west lines}{\pgfqpoint{-0pt}{-0pt}}{\pgfqpoint{3pt}{3pt}}{\pgfqpoint{3pt}{3pt}}{
        \pgfsetlinewidth{0.4pt}
        \pgfpathmoveto{\pgfqpoint{0pt}{0pt}}
        \pgfpathlineto{\pgfqpoint{3pt}{3pt}}
        \pgfpathmoveto{\pgfqpoint{2.8pt}{-.2pt}}
        \pgfpathlineto{\pgfqpoint{3.2pt}{.2pt}}
        \pgfpathmoveto{\pgfqpoint{-.2pt}{2.8pt}}
        \pgfpathlineto{\pgfqpoint{.2pt}{3.2pt}}
        \pgfusepath{stroke}}

\newlength{\mytmpa}
\setlength{\mytmpa}{1 mm}
\newlength{\mytmpb}
\newlength{\mytmpc}

%%%%%%%%%%%%%%%%%
% The following code draws the outline for a ›part‹ of the thesis.
% This command is used before the name of the part is displayed. It is void, as the part is added via \partlineswithprefixformat.
% Unfortunately this redefinition is needed every time the language switched back to french so it has its own file
\renewcommand*{\partformat}{}
% This command calls \partformat (#2) and displays the name of the part (#3).
\renewcommand*{\partlineswithprefixformat}[3]%
{%
    #2
    \thispagestyle{empty}
    \setlength{\mytmpa}{0.618\mypaperwidth}%
    \setlength{\mytmpb}{0.382\mypaperheight}%
    \ifprintVersion
        \ifprofessionalPrint
            \setlength{\mytmpa}{0.618\mypaperwidth + \mybindingcorrection + \extraborderlength}%
            \setlength{\mytmpb}{0.382\mypaperheight + \extraborderlength}%
        \fi
    \fi
    \begin{tikzpicture}[overlay, remember picture]%
        \node [inner sep = 0, outer sep = 0, anchor = north] at (current page.north west)%
        {%
            \begin{tikzpicture}[overlay, remember picture]%
            \draw[color = stroke1, line width = 0.7 mm] (\mytmpa, 0) -- (\mytmpa, -\mytmpb);%
            \end{tikzpicture}%
        };%
        \node (align) [align = right, below = \mytmpb - 2 ex, inner sep = 0, outer sep = 0, anchor = north west] at (current page.north west)%
        {%
            %\hspace{\mytmpa}\hspace{0.5 em}\partname\ \thepart\\[1 ex]
            \hspace{\mytmpa}\hspace{0.5 em}\partname\\[1 ex]
            \color{stroke1}#3%
        };%
    \end{tikzpicture}%
}


% This command defines various parameters for the ›part‹ format.
\RedeclareSectionCommand%
[%
    font = \normalfont\Huge\sffamily,
    prefixfont = \normalfont\Huge\sffamily,
]
{part}

%%%%%%%%%%%%%%%%%%%%%
%%% Chapter Format %%
%%%%%%%%%%%%%%%%%%%%%

\usepackage{etoolbox}

\newbool{chapterHasANumber}
\newbool{chapterHasAStar}
\renewcommand*{\chapterlinesformat}[3]%
{%
    % Check whether \chapter of \addchap has been used.
    \Ifnumbered{#1}{\setbool{chapterHasANumber}{true}}{\setbool{chapterHasANumber}{false}}%
    % Check whether \chapter* or \chapter has been used.
    \Ifstr{#2}{}{\setbool{chapterHasAStar}{true}}{\setbool{chapterHasAStar}{false}}%
    % Check whether a normal \chapter or something else is used.
    \ifboolexpr{bool{chapterHasANumber} and not bool{chapterHasAStar}}%
    {%
        \begin{tikzpicture}[overlay, remember picture]%
            \node [right = \myinnermargin, below = \mytopmargin, inner sep = 0, outer sep = 0, anchor = north west] (numbernode) at (current page.north west)%
            {%
                \hspace{\myinnermargin}%
                \sffamily\fontsize{60}{60}\selectfont%
                \color{chapternumber}%
                \thechapter%
            };%
            \node [inner sep = 0, outer sep = 0, anchor = north west] at (numbernode.south west)%
            {%
                \begin{tikzpicture}[overlay, remember picture]%
                    \draw[color = stroke1, line width = 0.7 mm] (\myinnermargin, -1 ex) -- (\paperwidth, -1 ex);%
                \end{tikzpicture}%
            };%
            \node (align) [text width = \textwidth - 2 cm, align = right, right = \myinnermargin + \mybodywidth, inner sep = 0, outer sep = 0, anchor = east] at (numbernode.west)%
            {%
                #3%
            };%
        \end{tikzpicture}%
    }%
    {%
        \begin{tikzpicture}[overlay, remember picture]%
            \node [right = \myinnermargin, below = \mytopmargin, inner sep = 0, outer sep = 0, anchor = north west] (numbernode) at (current page.north west)%
            {%
                \hspace{\myinnermargin}%
                \sffamily\fontsize{60}{60}\selectfont%
                \color{white}%
                \thechapter%
            };%
            \node [inner sep = 0, outer sep = 0, anchor = north west] at (numbernode.south west)%
            {%
                \begin{tikzpicture}[overlay, remember picture]%
                    \draw[color = stroke1, line width = 0.7 mm] (\myinnermargin, -1 ex) -- (\paperwidth, -1 ex);%
                \end{tikzpicture}%
            };%
            \node (align) [align = left, right = \myinnermargin, inner sep = 0, outer sep = 0, anchor = south west] at (numbernode.south west)%
            {%
                #3%
            };%
        \end{tikzpicture}%
    }%
}
\RedeclareSectionCommand%
[%
    font = \color{stroke1}\normalfont\huge\sffamily,
    afterskip = 20 pt,
]
{chapter}


%%%%%%%%%%%%%%%%%%%%%%%%
%%% Table of Contents %%
%%%%%%%%%%%%%%%%%%%%%%%%

% Format the table of contents to have a ›plain‹ page style.
\BeforeStartingTOC[toc]{\pagestyle{plain}}
\AfterStartingTOC{\thispagestyle{plain}}
                        % Contains commands that define the general format and layout of the thesis.
% This file contains all of the code that formats the bibliography. Since be package ›biblatex‹ is used, the bibliography needs to be compiled with ›biber‹.
%
% This file contains the following parts:
%   • Resources
%   • Redefined Keywords
%   • Coloring
%   • Format of the Entries
%   • Format of the Own Publications


\usepackage
[
    sortcites,              % Sort multiple references when citing them together.
    style = alphabetic,     % The style of a citation mark.
    defernumbers,           % Makes sure that references always have unique numbers. This is important if you use multiple bibliographies.
    safeinputenc,           % Allows to use UTF8 characters in the bibliography and tries to translate them into TeX automatically.
    backref = true,         % Creates back references in the bibliography.
    backrefstyle = three,   % Compresses three or more consecutive pages in the back references into a range.
    hyperref = true,        % Makes links generated by biblatex clickable. If hyperref is not used, a warning is issued.
    maxbibnames = 99,       % The maximum number of names displayed in the bibliography.
    maxcitenames = 2,       % The maximum number of names displayed when using commands like ›textcite‹. The default is 3. After that, ›et al.‹ is used.
%    useprefix,              % Prints name prefixes, such as ›von‹. The default is false. This means that prefixes are not considered to be part of the last name.
]
{biblatex} % Used in order to format the bibliography.

% The following command changes the space between the list of authors and the citation mark into a non-breaking space.
\renewcommand\namelabeldelim{\addnbspace}


%%%%%%%%%%%%%%%
%% Resources %%
%%%%%%%%%%%%%%%

\addbibresource{references/strings.bib}                     % Contains many strings for common conference names etc. These strings can then be used in the references.
\addbibresource{references/references.bib}                  % The file that contains the references that are used for the thesis.


%%%%%%%%%%%%%%%%%%%%%%%%
%% Redefined Keywords %%
%%%%%%%%%%%%%%%%%%%%%%%%

\renewbibmacro{in:}%
{%
    \ifentrytype{article}{}{\printtext{\bibstring{in}\intitlepunct}}%
}
% \renewcommand*{\volumenumberdelim}{\addcolon}

\renewbibmacro*{volume+number+eid}%
{%
    \printfield{volume}%
    \iffieldundef{number}{}{\addcolon}%
    %  \setunit*{\addnbthinspace}%
    \printfield{number}%
    \setunit*{\addcomma\space}%
    \printfield{eid}%
}

\DefineBibliographyStrings{english}%
{%
    backrefpage  = {\lowercase{s}ee page}, % For a single page number.
    backrefpages = {\lowercase{s}ee pages} % For multiple page numbers.
}


%%%%%%%%%%%%%%
%% Coloring %%
%%%%%%%%%%%%%%

\DeclareFieldFormat[article]{title}{\textbf{\color{stroke1}#1}}
\DeclareFieldFormat[inproceedings]{title}{\textbf{\color{stroke1}#1}}
\DeclareFieldFormat[thesis]{title}{\textbf{\color{stroke1}#1}}
\DeclareFieldFormat[book]{title}{\textbf{\color{stroke1}#1}}
\DeclareFieldFormat[unpublished]{title}{\textbf{\color{stroke1}#1}}
\DeclareFieldFormat[report]{title}{\textbf{\color{stroke1}#1}}
\DeclareFieldFormat[inbook]{chapter}{\textbf{\color{stroke1}#1}}
\DeclareFieldFormat[inbook]{title}{#1}
\DeclareFieldFormat{pages}{#1}


%%%%%%%%%%%%%%%%%%%%%%%%%%%
%% Format of the Entries %%
%%%%%%%%%%%%%%%%%%%%%%%%%%%

% The following toggle defines how the citation mark formats the author names. If this toggle is true, more information is used.
\newtoggle{authorend}
\togglefalse{authorend}

% Article
\DeclareBibliographyDriver{article}%
{%
  \usebibmacro{bibindex}%
  \usebibmacro{begentry}%
  \iftoggle{authorend}{}{\usebibmacro{author/translator+others}}%
  \setunit{\labelnamepunct}\newblock
  \usebibmacro{title}%
  \newunit
  \printlist{language}%
  \newunit\newblock
  \usebibmacro{byauthor}%
  \newunit\newblock
  \usebibmacro{bytranslator+others}%
  \newunit\newblock
  \printfield{version}%
  \newunit\newblock
  \usebibmacro{in:}%
  \usebibmacro{journal+issuetitle}%
  \newunit
  \usebibmacro{byeditor+others}%
  \newunit
  \usebibmacro{note+pages}%
  \newunit\newblock
  \iftoggle{bbx:isbn}
  {\printfield{issn}}
  {}%
  \newunit\newblock
  \usebibmacro{doi+eprint+url}%
  \newunit\newblock
  \usebibmacro{addendum+pubstate}%
  \setunit{\bibpagerefpunct}\newblock
  \usebibmacro{pageref}%
  \newunit\newblock
  \iftoggle{bbx:related}
  {\usebibmacro{related:init}%
    \usebibmacro{related}}
  {}%
  \usebibmacro{finentry}%
  \iftoggle{authorend}{\usebibmacro{author/translator+others}}{}%
}

% Book Chapter
\DeclareBibliographyDriver{inbook}%
{%
  \usebibmacro{bibindex}%
  \usebibmacro{begentry}%
  \iftoggle{authorend}{}{\usebibmacro{author/translator+others}}%
  \setunit{\labelnamepunct}\newblock
  % \usebibmacro{title}%
  \usebibmacro{chapter+pages}%
  % \printfield{chapter}%
  \newunit
  \printlist{language}%
  \newunit\newblock
  \usebibmacro{byauthor}%
  \newunit\newblock
  \usebibmacro{in:}%
  \usebibmacro{bybookauthor}%
  \newunit\newblock
  \usebibmacro{maintitle+booktitle}%
  \newunit\newblock
  \usebibmacro{byeditor+others}%
  \newunit\newblock
  \printfield{edition}%
  \newunit
  \iffieldundef{maintitle}
  {\printfield{volume}%
    \printfield{part}}
  {}%
  \newunit
  \printfield{volumes}%
  \newunit\newblock
  \usebibmacro{series+number}%
  \newunit\newblock
  \printfield{note}%
  \newunit\newblock
  \usebibmacro{publisher+location+date}%
  \newunit\newblock
  % \usebibmacro{chapter+pages}%
  \newunit\newblock
  \iftoggle{bbx:isbn}
  {\printfield{isbn}}
  {}%
  \newunit\newblock
  \usebibmacro{doi+eprint+url}%
  \newunit\newblock
  \usebibmacro{addendum+pubstate}%
  \setunit{\bibpagerefpunct}\newblock
  \usebibmacro{pageref}%
  \newunit\newblock
  \iftoggle{bbx:related}
  {\usebibmacro{related:init}%
    \usebibmacro{related}}
  {}%
  \usebibmacro{finentry}%
  \iftoggle{authorend}{\usebibmacro{author/translator+others}}{}%
}

% Proceedings Article
\DeclareBibliographyDriver{inproceedings}%
{%
  \usebibmacro{bibindex}%
  \usebibmacro{begentry}%
  \iftoggle{authorend}{}{\usebibmacro{author/translator+others}}%
  \setunit{\labelnamepunct}\newblock
  \usebibmacro{title}%
  \newunit
  \printlist{language}%
  \newunit\newblock
  \usebibmacro{byauthor}%
  \newunit\newblock
  \usebibmacro{in:}%
  \usebibmacro{maintitle+booktitle}%
  \newunit\newblock
  \usebibmacro{event+venue+date}%
  \newunit\newblock
  \usebibmacro{byeditor+others}%
  \newunit\newblock
  \iffieldundef{maintitle}
  {\printfield{volume}%
    \printfield{part}}
  {}%
  \newunit
  \printfield{volumes}%
  \newunit\newblock
  \usebibmacro{series+number}%
  \newunit\newblock
  \printfield{note}%
  \newunit\newblock
  \printlist{organization}%
  \newunit
  \usebibmacro{publisher+location+date}%
  \newunit\newblock
  \usebibmacro{chapter+pages}%
  \newunit\newblock
  \iftoggle{bbx:isbn}
  {\printfield{isbn}}
  {}%
  \newunit\newblock
  \usebibmacro{doi+eprint+url}%
  \newunit\newblock
  \usebibmacro{addendum+pubstate}%
  \setunit{\bibpagerefpunct}\newblock
  \usebibmacro{pageref}%
  \newunit\newblock
  \iftoggle{bbx:related}
  {\usebibmacro{related:init}%
    \usebibmacro{related}}
  {}%
  \usebibmacro{finentry}%
  \iftoggle{authorend}{\usebibmacro{author/translator+others}}{}%
}

% Thesis
\DeclareBibliographyDriver{thesis}%
{%
  \usebibmacro{bibindex}%
  \usebibmacro{begentry}%
  \iftoggle{authorend}{}{\usebibmacro{author}}%
  \setunit{\labelnamepunct}\newblock
  \usebibmacro{title}%
  \newunit
  \printlist{language}%
  \newunit\newblock
  \usebibmacro{byauthor}%
  \newunit\newblock
  \printfield{note}%
  \newunit\newblock
  \printfield{type}%
  \newunit
  \usebibmacro{institution+location+date}%
  \newunit\newblock
  \usebibmacro{chapter+pages}%
  \newunit
  \printfield{pagetotal}%
  \newunit\newblock
  \iftoggle{bbx:isbn}
  {\printfield{isbn}}
  {}%
  \newunit\newblock
  \usebibmacro{doi+eprint+url}%
  \newunit\newblock
  \usebibmacro{addendum+pubstate}%
  \setunit{\bibpagerefpunct}\newblock
  \usebibmacro{pageref}%
  \newunit\newblock
  \iftoggle{bbx:related}
  {\usebibmacro{related:init}%
    \usebibmacro{related}}
  {}%
  \usebibmacro{finentry}%
  \iftoggle{authorend}{\usebibmacro{author}}{}%
}


%%%%%%%%%%%%%%%%%%%%%%%%%%%%%%%%%%%%
%% Format of the Own Publications %%
%%%%%%%%%%%%%%%%%%%%%%%%%%%%%%%%%%%%

% The own publications are formatted using a numeric list, whereas the bibliography of the thesis uses an alphanumeric style.

% Copied from numeric.cbx in order to imitate numerical citations.
\providebool{bbx:subentry}
\newbibmacro*{citenum}%
{% Note: the original macro was called ›cite‹. I did not redefine ›cite‹ but instead defined a new macro ›citenum‹ because the author-year citations use the ›cite‹ macro too. Using ›\renewbibmacro*{cite}‹ would have caused all the author-year citations to become numeric too.
  \printtext[bibhyperref]{% If you ever want to use hyperref.
    \printfield{prefixnumber}%
    \printfield{labelnumber}%
    \ifbool{bbx:subentry}
    {\printfield{entrysetcount}}
    {}}%
}

% Copied from numeric.cbx to define a new numeric citation command for @online entries.
\DeclareCiteCommand{\conline}[\mkbibbrackets]
{\usebibmacro{prenote}}
{\usebibmacro{citeindex}%
  \usebibmacro{citenum}}% Note: this was originally "cite" but I changed it to "citenum" to avoid clashes with the author-year style.
{\multicitedelim}
{\usebibmacro{postnote}}
       % Contains commands for the layout of the bibliography.
% This file contains most of the packages used for this document. If you want to add a package, do it below the appropriate ribbon.
% Some packages are already included in other files in the ›settings‹ folder if they were already necessary. Thus, make sure to go through these files too if you want to know whether a certain package is already included.
%
% This file contains the following parts:
%   • Typography
%   • Math
%   • Fonts
%   • Graphics
%   • Tables
%   • Enumerations
%   • Algorithms
%   • Spaces and Special Characters
%   • Miscellaneous
%   • Additional Packages
%   • Hyperlinks

%%%%%%%%%%
%% Math %%
%%%%%%%%%%

% The following packages are the standard packages used in order to typeset math. They contain a lot of useful commands.
\usepackage{amsmath}
\usepackage{amssymb}
\usepackage{amsthm}
\usepackage{thmtools}
\usepackage{mathtools}
\usepackage{thm-restate}
\usepackage{dsfont}        % Yields far better blackboard-bold letters than \mathbb. Use \mathds in order to write such letters.
\usepackage{braceMnSymbol} % Adjusts overbraces and underbraces such that longer versions are put together seamlessly.


%%%%%%%%%%%
%% Fonts %%
%%%%%%%%%%%

\usepackage
[
    mono=false, % Disables the mono/typewriter font.
]
{libertinus-otf} % The main font used in this thesis.

\usepackage{fontspec}  % Package required to load custom fonts
\setmonofont{FiraCode} % Use Fira Code as default mono font
[
	Extension = .ttf,
	Path = fonts/Fira-Code/,
	UprightFont = *-Regular,
	BoldFont = *-Bold,
	ItalicFont = *-Light,
	Scale = 0.85
]

\usepackage{url} % Responsible for URL formatting.
\usepackage{bm}  % Allows to use sensible bold letters in math mode. This package has to go after the font packages. Otherwise it does not work correctly!

%%%%%%%%%%%%%%
%% Graphics %%
%%%%%%%%%%%%%%

\usepackage{graphicx} % The standard package for including graphics into your document.
\usepackage
[
    subrefformat = simple, % Formats the label of the \subref command without parentheses.
    labelformat = simple,  % Formats the mark of a subfigure without parentheses.
]
{subcaption}         % Enables it to have subfigures inside of a single figure.
\usepackage{wrapfig} % Allows to put figures next to text.

% Changing the \columnsep adds some space next to a warpfigure.
\columnsep = \mymargininnersep
% The reference label of a subfigure is redefined to have a non-breaking space and parentheses. (Thus, the subfigures show parentheses although the package options removed parentheses; otherwise, two pairs of brackets would be seen.)
\renewcommand*{\thesubfigure}{~(\alph{subfigure})}

% tikz has already been included in the file "settings/format.tex"

%%%%%%%%%%%%
%% Tables %%
%%%%%%%%%%%%

\usepackage{array}     % Improves the way that tables can be formatted.
\usepackage{booktabs}  % Adds lines (called ›rules‹) that can be used in tables and improves spacing.
\usepackage{longtable} % Allows to make tables that span multiple pages.
\usepackage{pdflscape} % Allows to change a page into landscape. This is handy if a table is very wide.


%%%%%%%%%%%%%%%%%%
%% Enumerations %%
%%%%%%%%%%%%%%%%%%

\usepackage{enumitem} % Adds tons of useful features to enumeration environments.


%%%%%%%%%%%%%%%%
%% Algorithms %%
%%%%%%%%%%%%%%%%

\usepackage[outputdir=build, newfloat=true]{minted} % WARNING : Requires shell escape
\newmintedfile[coqcode]{coq}{
	fontfamily=tt, % select the correct font family
	linenos=false,
	numberblanklines=true,
	numbersep=5pt,
	gobble=0,
	frame=lines,
	framerule=0.4pt,
	framesep=2mm,
	funcnamehighlighting=true,
	tabsize=4,
	obeytabs=false,
	mathescape=true
	samepage=false, %with this setting you can force the list to appear on the same page
	showspaces=false,
	showtabs =false,
	texcl=false,
}

%%%%%%%%%%%%%%%%%%%%%%%%%%%%%%%%%%%
%% Spaces and Special Characters %%
%%%%%%%%%%%%%%%%%%%%%%%%%%%%%%%%%%%

\usepackage{xspace}   % Adds the functionality that a space after a command will be shown as a space in the output.
\usepackage
[
    shortcuts, % Allows to use short symbols for non-breaking hyphens and dashes instead of lengthy commands.
]
{extdash}             % Adds non-breaking hyphens and dashes.
\usepackage{setspace} % Allows to easily chnage the spacing inside of the document.

%%%%%%%%%%%%%%%%
%% Typography %%
%%%%%%%%%%%%%%%%

\usepackage
[
    babel = true, % Enables language-specific tuning.
]
{microtype}           % Uses the text space more efficiently.
\usepackage{csquotes} % Uses the correct quotes according to the current language.

%%%%%%%%%%%%%%%%%%%
%% Miscellaneous %%
%%%%%%%%%%%%%%%%%%%

\usepackage{xparse}    % Is used in order to define reasonable commands.
\usepackage{footnote}  % Allows it to extend the environments footnotes can be used in. It is said that this package is in conflict with ›hyperref‹. I did not note any troubles. However, if something is fishy, it is probably best to not use this package.
\usepackage{afterpage} % Adds the \afterpage command, which specifies that the provided argument shall be processed after the current page is finished.
\usepackage
[
    textsize = scriptsize, % Determines the text size of the TODO note.
]
{todonotes}            % Adds TODO notes to the document. These are small text areas inside of the margin of a page.

%%%%%%%%%%%%%%%%%%%%%%%%%%%%%%%%%%%%%%%%%%%%%%%%%%%%%%%%%%%%%%%%%
%% If you want to add new packages, add them below this ribbon %%
%%%%%%%%%%%%%%%%%%%%%%%%%%%%%%%%%%%%%%%%%%%%%%%%%%%%%%%%%%%%%%%%%


%%%%%%%%%%%%%%%%
%% Hyperlinks %%
%%%%%%%%%%%%%%%%

\usepackage
[
    bookmarks = true,                 % Generates boodmarks for the PDF.
    bookmarksopen = false,            % The bookmarks are closed by default.
    bookmarksnumbered = true,         % The bookmarks use the numbers of the corresponding headline.
    pdfstartpage = 1,                 % The first page seen when opening the PDF.
    pdftitle = {{\printTitle}},       % The PDF’s title in the meta data.
    pdfauthor = {{\printAuthor}},     % The PDF’s author name in the meta data.
    pdfsubject = {{\printSubject}},   % The PDF’s subject in the meta data.
    pdfkeywords = {{\printKeywords}}, % The PDF’s keywords in the meta data.
    breaklinks = true,                % Allows it to break links.
    \ifprintVersion
        hidelinks,                    % In the printed version, links are not highlighted, as they are not clickable.
    \else
    colorlinks = true,            % The text of hyperlinks is colored instead of having a colored box around it.
    allcolors = stroke1,          % Every hyperlink uses the same color. If you want to change specific colors, use the commands below.
    %        linkcolor = stroke1,          % The color of an in-document hyperlink.
    %        citecolor = stroke1,          % The color of a citation.
    %        filecolor = stroke1,          % The color of a file link.
    %        pagecolor = stroke1,          % The color of a reference to a page.
    %        urlcolor = stroke1,           % The color of a weblink.
    \fi
]
{hyperref} % The standard package that is used for creating hyperlinks inside of a document.

\usepackage
[
    %    capitalise, % Capitalizes the words in front of the labels. This can also be done by simply using \Cref instead of \cref. In order to have a greater variety, this option is not used.
    noabbrev,   % The words in front of the labels are not abbreviated.
    nameinlink, % Extends the link of a reference to the word in front of it.
]
{cleveref} % This package must be included after ›hyperref‹. It creates clever references that know what they refer to.
     % Contains the packages that this template provides.
% This file contains all sorts of macros that are globally used. Further, certain options made available through packages are set here as well.
%
% This file contains the following parts:
%   • Type of Degree
%   • Miscellaneous
%   • Footnotes
%   • Theorem Environments
%   • Meta Commands
%   • Common Commands


%%%%%%%%%%%%%%%%%%%%
%% Type of Degree %%
%%%%%%%%%%%%%%%%%%%%

% The colloquial term of the degree.
\newcommand*{\colloquialDegreeName}{Philosophal Doctorate}
\newcommand*{\colloquialDegreeNameLowercase}{philosophal doctorate}

% The abbreviation of the degree.
\newcommand*{\degreeAbbreviation}{Ph. D.}


%%%%%%%%%%%%%%%%%%%
%% Miscellaneous %%
%%%%%%%%%%%%%%%%%%%

% Defines the environment used at the beginning of each chapter.
\newenvironment{jointwork}
{\itshape}
{\ignorespacesafterend\bigskip}

% Defines the IfEmptyTF command. This is useful for optional arguments provided as [].
\makeatletter
    \def\IfEmptyTF#1%
    {%
        \if\relax\detokenize{#1}\relax%
            \expandafter\@firstoftwo%
        \else%
            \expandafter\@secondoftwo%
        \fi%
    }
\makeatother

% Creates an environment that automatically uses math mode if necessary and creates a space afterward if wanted. Basically, if the command \example is defined to use this environment, you can use \example without mathe mode in normal text as if it were ordinary text.
\NewDocumentCommand{\mathOrText}{m}
{%
    \ensuremath{#1}\xspace%
}

% Reduces the space around scaling bracekts.
\let\originalleft\left
\let\originalright\right
\renewcommand{\left}{\mathopen{}\mathclose\bgroup\originalleft}
\renewcommand{\right}{\aftergroup\egroup\originalright}

% Lets math text in an environment of bold text also appear bold.
\makeatletter
    \DeclareRobustCommand{\bfseries}%
    {%
        \not@math@alphabet\bfseries\mathbf%
        \fontseries\bfdefault\selectfont%
        \boldmath%
    }
\makeatother

% Adds square and curly brackets to the exceptions for xspace such that no space is used right in front of them.
\xspaceaddexceptions{]\}}

% Formats URLs by using the normal font (not the typewriter font).
\urlstyle{rm}

% Allows large display formulas to span multiple pages.
\allowdisplaybreaks

% Defines an optional argument for labels named ›ineq‹ that signals that cleveref should name the respective reference ›inequality‹ instead of its actual name.
\crefname{ineq}{inequality}{inequalities}
\creflabelformat{ineq}{#2{\upshape(#1)}#3} 

% Defines an optional argument for labels named ›term‹ that signals that cleveref should name the respective reference ›term‹ instead of its actual name.
\crefname{term}{term}{terms}
\creflabelformat{term}{#2{\upshape(#1)}#3}


%%%%%%%%%%%%%%%
%% Footnotes %%
%%%%%%%%%%%%%%%

% In the following, the command ›footnote‹ is redefined such that the footnote mark can be more easily adjusted.
\let\oldfootnote\footnote

% The following are variables used by the command.
\newlength{\spaceBeforeFootnote} % Denotes the space before the footnote mark in em.
\newlength{\spaceAfterFootnote}  % Denotes the space after the footnote mark in em.

% The new footnote command. The first three arguments are optional, the fourth mandatory. Its arguments have the following meaning:
%   1. The amount of space before the footnote mark in em. The default is 0.
%   2. The amount of space after the footnote mark in em. The default is 0.
%   3. The number of the footnote mark.
%   4. The text of the footnote.
\RenewDocumentCommand{\footnote}{o o o m}%
{%
    \IfNoValueTF{#1}%
    {%
        \oldfootnote{#4}%
    }%
    {%
        \setlength{\spaceBeforeFootnote}{\IfEmptyTF{#1}{0}{#1} em}%
        \IfNoValueTF{#2}%
        {%
            \hspace*{\spaceBeforeFootnote}\oldfootnote{#4}%
        }%
        {%
            \setlength{\spaceAfterFootnote}{\IfEmptyTF{#2}{0}{#2} em}%
            \hspace*{\spaceBeforeFootnote}\IfNoValueTF{#3}{\oldfootnote{#4}}{\oldfootnote[#3]{#4}}\hspace*{\spaceAfterFootnote}%
        }%
    }%
}

% The following commands enable it such that footnotes can be used in various other environments other than simple text.
\makesavenoteenv{figure}
\makesavenoteenv{table}
\makesavenoteenv{tabular}


%%%%%%%%%%%%%%%%%%%%%%%%%%
%% Theorem Environments %%
%%%%%%%%%%%%%%%%%%%%%%%%%%

\iffancyTheorems
    % The following theorem style uses a bold heading for the theorem and normal (upright) text. The environment begins with a triangle of color ›stroke1‹ pointing to the right and uses a QED symbol that is a triangle of the same color pointing to the left. Thus, the environment is enclosed by triangles.
    \declaretheoremstyle
    [
        spaceabove = \topsep,
        spacebelow = \topsep,
        headfont = \bfseries,
        headformat = \textcolor{stroke1}{$\blacktriangleright$} \NAME~\NUMBER \NOTE,
        notefont = \bfseries,
        notebraces = {(}{)},
        bodyfont = \normalfont,
        postheadspace = 0.5 em,
        qed = \textcolor{stroke1}{\bfseries$\blacktriangleleft$},
    ]
    {myTheoremStyle}
    
    % The QED symbol used in proofs is a squre with color ›stroke1‹ in order to look similar to the theorem environments.
    \renewcommand*{\qedsymbol}{\textcolor{stroke1}{$\blacksquare$}}
    
    \declaretheorem
    [
        style = myTheoremStyle,
        name = Conjecture,
        numberwithin = chapter,
    ]
    {conjecture}
    \declaretheorem
    [
        style = myTheoremStyle,
        name = Proposition,
        sharenumber = conjecture,
    ]
    {proposition}
    \declaretheorem
    [
        style = myTheoremStyle,
        name = Claim,
        sharenumber = conjecture,
    ]
    {claim}
    \declaretheorem
    [
        style = myTheoremStyle,
        name = Lemma,
        sharenumber = conjecture,
    ]
    {lemma}
    \declaretheorem
    [
        style = myTheoremStyle,
        name = Corollary,
        sharenumber = conjecture,
    ]
    {corollary}
    \declaretheorem
    [
        style = myTheoremStyle,
        name = Theorem,
        sharenumber = conjecture,
    ]
    {theorem}
    \declaretheorem
    [
        style = myTheoremStyle,
        name = Definition,
        sharenumber = conjecture,
    ]
    {definition}
    \declaretheorem
    [
        style = myTheoremStyle,
        name = Example,
        sharenumber = conjecture,
    ]
    {example}
    \declaretheorem
    [
        style = myTheoremStyle,
        name = Remark,
        sharenumber = conjecture,
    ]
    {remark}
\else
    % This is the default style. That is, the head is bold, the rest is italic, and there is no symbol to denote the end of the environment.
    \theoremstyle{plain}
    
    \newtheorem{conjecture}{Conjecture}[chapter]
    \newtheorem{proposition}[conjecture]{Proposition}
    \newtheorem{claim}[conjecture]{Claim}
    \newtheorem{lemma}[conjecture]{Lemma}
    \newtheorem{corollary}[conjecture]{Corollary}
    \newtheorem{theorem}[conjecture]{Theorem}
    \newtheorem{definition}[conjecture]{Definition}
    \newtheorem{example}[conjecture]{Example}
    \newtheorem{remark}[conjecture]{Remark}
\fi


%%%%%%%%%%%%%%%%%%%
%% Meta Commands %%
%%%%%%%%%%%%%%%%%%%

% A template for a function that can use an optional variable bracket size. Its arguments have the following meaning:
%   1. The name of the function.
%   2. The type of the left bracket. This should be a bracket symbol, as it will be forwarded to the command \left.
%   3. The type of the right bracket. The same restrictions as with parameter 2 hold here.
%   4. The arguments that the function takes, that is, the things that are enclosed by the brackets.
%   5. The size of the brackets. This should be a value like \big or similar, as it will be forwarded to the command \left.
\NewDocumentCommand{\functionTemplate}{m m m m o}%
{%
    \IfNoValueTF{#5}%
    {%
        \mathOrText{#1\left#2{#4}\right#3}%
    }%
    {%
        \mathOrText{#1#5#2{#4}#5#3}%
    }%
}

% The following two commands are used as variables for the following command.
\newcommand*{\leftBracketType}{(}
\newcommand*{\rightBracketType}{)}

% This is a command that creates a command that is a function as defined by the command \functionTemplate. Its arguments have the following meaning:
%   1. The name of the function command.
%   2. The name of the function itself.
%   3. The type of the left bracket. This will be forwarded to parameter 2 of \functionTemplate. The default is (. Use \lbrack for [ and \{ for }.
%   4. The type of the right bracket. This will be forwarded to parameter 3 of \functionTemplate. The default is ). The rest is similar to parameter 3.
% The command created has two optional arguments, which are as follows:
%   1. The arguments of the function. If this is empty, only the name of the function will be used.
%   2. The size of the brackets. This will be forwarded to parameter 5 of \functionTemplate.
\NewDocumentCommand{\createFunction}{m m o o}%
{%
    \renewcommand*{\leftBracketType}{\IfNoValueTF{#3}{(}{#3}}%
    \renewcommand*{\rightBracketType}{\IfNoValueTF{#4}{)}{#4}}%
    \NewDocumentCommand{#1}{o o}%
    {%
        \IfNoValueTF{##1}%
        {%
            \mathOrText{#2}%
        }%
        {%
            \functionTemplate{#2}{\leftBracketType}{\rightBracketType}{##1}[##2]%
        }%
    }%
}

% A template for a probabilistic symbol, which can make use of a condition denoted by |. Its arguments have the following meaning:
%   1. The name of the function.
%   2. The argument of the function.
%   3. The condition of the function. The default is that there is no condition.
%   4. The size of the brackets. This will be forwarded to parameter 5 of \functionTemplate.
\DeclareDocumentCommand{\probabilisticFunctionTemplate}{m m O{} o}
{%
    \functionTemplate{#1}%
    {\lbrack}%
    {\rbrack}%
    {#2\IfEmptyTF{#3}{}{\ \IfNoValueTF{#4}{\left}{#4}\vert\ \vphantom{#2}#3\IfNoValueTF{#4}{\right.}{}}}%
    [#4]%
}


%%%%%%%%%%%%%%%%%%%%%
%% Common Commands %%
%%%%%%%%%%%%%%%%%%%%%

%%%%%%%%%%%%%%%%%%%%%
% Number Sets

% Number sets appear in bold by default. The other option is to make them appear in blackboard bold.
\ifboldNumberSets
    \newcommand*{\N}{\mathOrText{\mathbf{N}}}
    \newcommand*{\Z}{\mathOrText{\mathbf{Z}}}
    \newcommand*{\Q}{\mathOrText{\mathbf{Q}}}
    \newcommand*{\R}{\mathOrText{\mathbf{R}}}
    \newcommand*{\C}{\mathOrText{\mathbf{C}}}
    \newcommand*{\indicatorFunctionSymbol}{\mathbf{1}}
\else
    \newcommand*{\N}{\mathOrText{\mathds{N}}}
    \newcommand*{\Z}{\mathOrText{\mathds{Z}}}
    \newcommand*{\Q}{\mathOrText{\mathds{Q}}}
    \newcommand*{\R}{\mathOrText{\mathds{R}}}
    \newcommand*{\C}{\mathOrText{\mathds{C}}}
    \newcommand*{\indicatorFunctionSymbol}{\mathds{1}}
\fi

%%%%%%%%%%%%%%%%%%%%%
% Probabilistic Functions
% All of these functions follow the outline of \probabilisticFunctionTemplate. That is, the syntax is, for example, \Pr{A}[B][\big], which would be shown as Pr[A | B] with \big brackets.

% Probability measure
\RenewDocumentCommand{\Pr}{m O{} o}%
{%
    \probabilisticFunctionTemplate{\mathrm{Pr}}{#1}[#2][#3]%
}

% Expected value
\NewDocumentCommand{\E}{m O{} o}%
{%
    \probabilisticFunctionTemplate{\mathrm{E}}{#1}[#2][#3]%
}

% Variance
\NewDocumentCommand{\Var}{m O{} o}%
{%
    \probabilisticFunctionTemplate{\mathrm{Var}}{#1}[#2][#3]%
}

%%%%%%%%%%%%%%%%%%%%%
% Landau Notation
% The following commands all take a mandatory argument, which is the term of the Landau notation, as well as an optional argument, which determines the size of the brackets.

% Big O
\DeclareDocumentCommand{\bigO}{m o}%
{%
    \functionTemplate{\mathrm{O}}{(}{)}{#1}[#2]%
}

% Small O
\DeclareDocumentCommand{\smallO}{m o}%
{%
    \functionTemplate{\mathrm{o}}{(}{)}{#1}[#2]%
}

% Big Theta
\DeclareDocumentCommand{\bigTheta}{m o}%
{%
    \functionTemplate{\upTheta}{(}{)}{#1}[#2]%
}

% Big Omega
\DeclareDocumentCommand{\bigOmega}{m o}%
{%
    \functionTemplate{\upOmega}{(}{)}{#1}[#2]%
}

% Small Omega
\DeclareDocumentCommand{\smallOmega}{m o}%
{%
    \functionTemplate{\upomega}{(}{)}{#1}[#2]%
}

%%%%%%%%%%%%%%%%%%%%%
% Constants

% Pi; ratio of a circle’s circumference to its diameter
\newcommand*{\circlePi}{\mathOrText{\uppi}}

% Euler’s constant. This command takes an optional parameter, which becomes the exponent of this constant.
\DeclareDocumentCommand{\eulerE}{o}%
{%
    \mathOrText{\mathrm{e}\IfNoValueTF{#1}{}{^{#1}}}%
}

% i; the imaginary unit
\newcommand*{\imaginaryUnit}{\mathOrText{\mathrm{i}}}

%%%%%%%%%%%%%%%%%%%%%
% Other

% A polynomial function. The mandatory parameter is the argument of the function, the optional one is the size of the brackets.
\DeclareDocumentCommand{\poly}{m o}%
{%
    \functionTemplate{\mathrm{poly}}{(}{)}{#1}[#2]%
}

% The identity function
\createFunction{\id}{\mathrm{id}}

% An indicator function. The first parameter is set as an index, the second is the argument of the function, and the third is the size of the brackets.
\NewDocumentCommand{\ind}{m o o}%
{%
    \IfNoValueTF{#2}%
    {%
        \mathOrText{\indicatorFunctionSymbol_{#1}}%
    }%
    {%
        \functionTemplate{\indicatorFunctionSymbol_{#1}}{(}{)}{#2}[#3]%
    }%
}

% The domain of a function. Its parameters are the same as for \poly.
\DeclareDocumentCommand{\dom}{m o}%
{%
    \functionTemplate{\mathrm{dom}}{(}{)}{#1}[#2]%
}

% The range of a function. Its parameters are the same as for \poly.
\DeclareDocumentCommand{\rng}{m o}%
{%
    \functionTemplate{\mathrm{rng}}{(}{)}{#1}[#2]%
}

% The d for an integral. The optional parameter becomes the exponent/degree of the operator.
\DeclareDocumentCommand{\d}{o}%
{%
    \mathrm{d}\IfNoValueTF{#1}{}{^{#1}}%
}

% A command that creates sets. The first parameter is the left-hand side, the second is the right-hand side, and the third (optional) parameter is the size of the brackets.
\DeclareDocumentCommand{\set}{m m o}%
{
    \mathOrText{\IfNoValueTF{#3}{\left}{#3}\{#1\ \IfNoValueTF{#3}{\left}{#3}\vert\
    \vphantom{#1}#2\IfNoValueTF{#3}{\right.}{}\IfNoValueTF{#3}{\right}{#3}\}}
}
      % Contains newly defined commands useful for mathematics.

% This is the thesis. The front matter as well as the references should not be changed. The main matter can be edited freely.
\begin{document}


    \frontmatter
    % This file contains the layout of the title page.

% As taken from the MADIS doctoral school page : 
% https://lilliad.univ-lille.fr/doctorant/conseils-redaction-page-garde

%La page de garde (ou page de titre) de votre thèse doit comporter au moins les éléments suivants : 
%    le nom de l’université et son logo : Université de Lille ;
%    le nom de votre école doctorale ;
%    le nom de votre laboratoire ;
%    le titre de la thèse ;
%    la mention «Thèse préparée et soutenue publiquement par Votre Nom le XX/XX/20XX, pour obtenir le grade de Docteur en Votre discipline de thèse» (ou toute autre formulation équivalente) ;
%    la liste des membres du jury, avec leur fonction et leur affiliation.

%Votre discipline ou votre spécialité doit être indiquée telle qu’elle a été saisie lors de l’enregistrement du jury de thèse auprès du Service des affaires doctorales - Bureau des soutenances de l’Université. 

%Si vous avez rédigé votre thèse en anglais, n’oubliez pas de faire figurer les titres en anglais et en français sur votre page de garde. Tous deux sont nécessaires.


% This page uses a different geometry, as the content will be centered (not including the binding correction).
\ifprintVersion
    \ifprofessionalPrint
        \newgeometry
        {
            textwidth = 134 mm,
            textheight = 220 mm,
            top = 38 mm + \extraborderlength,
            inner = 38 mm + \mybindingcorrection + \extraborderlength,
        }
    \else
        \newgeometry
        {
            textwidth = 134 mm,
            textheight = 220 mm,
            top = 38 mm,
            inner = 38 mm + \mybindingcorrection,
        }
    \fi
\else
    \newgeometry
    {
        textwidth = 134 mm,
        textheight = 220 mm,
        top = 38 mm,
        inner = 38 mm,
    }
\fi

% The format of the title page.
\begin{titlepage}
    \sffamily
    \begin{center}
	{
        	\def\svgwidth{20 em}
		\input{images/ULille_black.pdf_tex}\\
		{\tiny{CRIStAL - UMR 9189}\hfil{ED MADIS - 631}}
	}
        \vfil
	{
		{
			{\Huge{\textsc{\textbf{Thèse}}}}
		}\\[1 em]
		{
			{présentée et soutenue publiquement le}\\
			{\textbf{T.B.D.}}
		}\\[1 em]
		{
			{pour l'obtention du grade de}\\
			{\LARGE{\textsc{\textbf{Docteur de l'Université de Lille}}}}\\
			{\large\textit{spécialité \printProgram}}
		}\\[1 em]
		{
			{par}\\[0.3 em]
			{\Large\textbf{\printAuthor}}
		}
	}
        \vfil
        {\LARGE
            \rule[1 ex]{\textwidth}{1.5 pt}
            \onehalfspacing\printTitleBold\\[1 ex]
            \rule[-1 ex]{\textwidth}{1.5 pt}
        }
    \end{center}
    
    \vfil
	{\small \centering \underline{Composition du jury :}}
    \begin{table}[h]
	\small
        \sffamily 
        {%\def\arraystretch{1.2}
	    \begin{tabular}{
		>{\raggedright\arraybackslash}p{0.32\textwidth}
		>{\bfseries\raggedright\arraybackslash}p{0.262\textwidth}% 0.738
		>{\itshape\raggedleft\arraybackslash}p{0.32\textwidth}
	    }
		Gilles Grimaud	& Directeur de thèse	& Professeur des universités, Université de Lille\\
		XXXXXX XXXXXXX	& Rapportrice		& Maître de conférences, Université de La Rochelle\\
		XXXXXX XXXXXXX	& Rapporteur		& Maître de conférences, Université de La Rochelle\\
		XXXXXX XXXXXXX	& Examinatrice		& Professeur des universités, Université de Lille\\
		XXXXXX XXXXXXX	& Examinateur		& Professeur des universités, Université de Lille\\
		XXXXXX XXXXXXX	& Invité		& Professeur des universités, Université de Lille\\
            \end{tabular}
        }
    \end{table}
\end{titlepage}

\restoregeometry


    \pagestyle{plain}

    \addchap{Abstract}
    % This file should contain the abstract.

\vfil

Les travaux présentés dans ce document de thèse sont liés à la vérification formelle de propriétés sur des composants de systèmes d'exploitation. Les premiers travaux piliers de ce domaine sont ceux du projet seL4 ; démontrant que la vérification de propriétés formelles sur un micro noyau est réalisable, malgré un coût élevé. Pour réduire le coût de la preuve, le projet CertikOS a proposé une méthode de preuve plus étagée et plus modulaire, en tirant à l'extrême la méthode de preuve par raffinement. L'équipe du noyau Pip a pris le contrepied de ces travaux, en prônant le minimalisme, en utilisant une méthodologie reposant sur un \emph{shallow embedding} et en prouvant les propriétés désirées directement plutôt qu'en utilisant la méthode par raffinement.

Les travaux présentés dans cette thèse sont plus spécifiquement liés au noyau Pip. Les travaux précédents sur le noyau Pip ont porté sur une preuve de préservation de l'isolation des services fournis par Pip manipulant la mémoire. Cependant, un aspect critique du noyau devait encore être conçu : le transfert de flot d'exécution d'une partition de mémoire à une autre.

La première contribution de cette thèse présente un nouveau service de Pip conçu pour supporter tous les transferts de flots d'exécution possibles au sein d'un système -- les interruptions, les fautes, et les appels explicites. Ce service gère de manière unifiée ces transferts de flot d'exécution afin de réduire au minimum l'effort de preuve. Une implémentation est proposée pour le noyau Pip.

La seconde contribution de cette thèse est la première implémentation au monde d'un ordonnanceur \emph{Earliest Deadline First} pour jobs arbitraires muni d'une preuve formelle de sa correction. La preuve garantit que la fonction d'élection respecte la politique \emph{EDF}, garantissant l'optimalité du planning sur les machines mono-processeur. La preuve a été conduite en partie en suivant la méthodologie habituelle de Pip, utilisant un \emph{shallow embedding} et une monade d'état. Elle a cependant été réalisée par raffinement. De plus, l'ordonnanceur se sert du service de transfert de flot d'exécution ; montrant la polyvalence et l'utilisabilité du service.

La dernière contribution présentée dans cette thèse est une preuve de concept libérant le code des services de Pip de ses liens avec le modèle d'isolation. Cette indépendance permet de créer des modèles alternatifs, permettant de raisonner sur le code à propos de nouvelles propriété tout en limitant l'effort de preuve. Cette contribution ouvre de nouvelles perspectives de recherche liées à la réduction du coup de raisonnement sur des propriétés additionnelles sur Pip. Cette preuve de concept n'apporte cependant pas que des avantages : en particulier sur la confiance accordée à la conjonction de propriétés formellement prouvées sur des modèles différents.

\vfil


    % Temporary switch of language for the abstract

    \selectlanguage{english}
    \addchap{Abstract}
    % This file should contain the English abstract.
The work decribed in this document is related to formal proofs on operating systems and more specifically tied to the Pip kernel.
Previous work on the Pip protokernel focused on providing an isolation proof to Pip's services manipulating the system's memory. Yet, another critical aspect of the kernel -- handling the execution flow transfer from a partition to another -- remained to be designed.

The first contribution of this thesis outlines the design of a single service able to handle all possible control flow transfers in a system ; namely interrupts, faults and explicit control flow transfers. The design focuses on minimalism and code factorization in order to reduce the overall proof effort. An implementation of the service is provided for the Pip kernel. We believe the idea behind the service is general enough to be implemented in other kernels and other architectures.

The second contribution outlined in this thesis is the first formally proven correct userland implementation of an earliest deadline first scheduler for arbitrary jobs. The formal proof guarantees that the election function of the scheduler respects the earliest deadline first scheduling policy, and is guaranteed to be optimal on mono-processor systems. This proof was conducted using Pip's usual methodology, leveraging a shallow embedding of the scheduler's code in Coq and a state monad. Nonetheless, while the Pip kernel properties were proven directly, the presented scheduler proofs include three refinement levels ; from the scheduling policy to the actual implementation. Furthermore, the scheduler uses the previously described service in order to pass the control flow to partitions and regain the execution flow through interrupts, showcasing its usability and versatility.

The last contribution presented in this thesis is a proof of concept of a generic monad applied to the Pip kernel. This generic monad allows to create independent models focusing on specific aspects of a single interface in order to prove specific properties on each model. This should greatly reduces the cost of proving new properties on the kernel. Nonetheless, using this technique has its own implications on the composition of those properties.


    % Switch back to French for the rest of the document

    \selectlanguage{french}
    % Unfortunately we have to redefine the parts commands because the language switch has redefined our redefined commands    
    \renewcommand*{\partformat}{}
% This command calls \partformat (#2) and displays the name of the part (#3).
\renewcommand*{\partlineswithprefixformat}[3]%
{%
    #2
    \thispagestyle{empty}
    \setlength{\mytmpa}{0.618\mypaperwidth}%
    \setlength{\mytmpb}{0.382\mypaperheight}%
    \ifprintVersion
        \ifprofessionalPrint
            \setlength{\mytmpa}{0.618\mypaperwidth + \mybindingcorrection + \extraborderlength}%
            \setlength{\mytmpb}{0.382\mypaperheight + \extraborderlength}%
        \fi
    \fi
    \begin{tikzpicture}[overlay, remember picture]%
        \node [inner sep = 0, outer sep = 0, anchor = north] at (current page.north west)%
        {%
            \begin{tikzpicture}[overlay, remember picture]%
            \draw[color = stroke1, line width = 0.7 mm] (\mytmpa, 0) -- (\mytmpa, -\mytmpb);%
            \end{tikzpicture}%
        };%
        \node (align) [align = right, below = \mytmpb - 2 ex, inner sep = 0, outer sep = 0, anchor = north west] at (current page.north west)%
        {%
            %\hspace{\mytmpa}\hspace{0.5 em}\partname\ \thepart\\[1 ex]
            \hspace{\mytmpa}\hspace{0.5 em}\partname\\[1 ex]
            \color{stroke1}#3%
        };%
    \end{tikzpicture}%
}


    \addchap{Remerciements}
    % Here you can write whom you want to thank.
{\color{white}
Ni se nourrir, ni se loger n'est gratuit.
Je crois qu'avec ça j'ai tout dit.
Ce monde est cruel
Ce monde est cruel.
Ça y est, j'ai tout dit.

Pas manger, ça fait mourir, et je suis habitué au chauffage.
Tes besoins vitaux sont payants : t'as compris la prise d'otage.
Depuis tout petit dans la merde, tu sais qu'il faudra mailler.
Au moins un peu pour le loyer, au moins un peu pour grailler.
Depuis tout petit dans la merde, on t'apprend à travailler ;
personne ne va te ravitailler à l'œil,
personne ne va s'apitoyer, pas de bol.

Ce monde est cruel.
Ce monde est cruel.
Je peux développer encore, je le fais sans aucun effort.
Pour travailler (donc pour manger), on te prend à trois ans
-- on te lâche à vingt-cinq (tes meilleures années).
Si tu pars avant, tu démarres en bas de la pyramide
et tu fermes ta gueule. Tu fais les pires des tâches,
tu gravis les étages au ralenti. Tu tapines en stage,
t'es sous-payé et on t'oblige à sourire --
car c'est une chance (merci !) déjà d'être là
avec tes vieux diplômes. Tiens, parlons des diplômes.

Personne n'est sûr, mais fais-le quand même pour la sécurité.
D'ailleurs, toute ta vie, pense à sécuriser :
même si tu amasses -- ne dépense pas,
on ne sait pas ce qui peut arriver.
Tu peux mourir, c'est vrai. Mais, si c'est pas le cas,
tu peux souffrir du manque puis être interdit par ta banque
et ça, ça fait peur. Les banques ça fait peur.
Des banques privées s'enrichissent, et des pays s'endettent.
De tout petits groupes très riches face au reste du monde,
face au bétail, face à la masse de salariés sans tête.

N'oublie jamais qui gagne quoi lorsque tu taffes.
Si ça te fâche et que tu ne veux plus,
n'oublie jamais : tu ne manges plus.
Ça ressemble à un choix...
Si c'est pas de l'esclavagisme,
c'est quand même pas vraiment humaniste.

[...]

Ce monde est cruel.
Ce monde est cruel.
Et j'ai tellement de chance à côté des autres,
je trouve ça tellement cruel.

Hein ? Comment ça ? Dieu donnerait de la chance, du talent,
à certains mais pas à d'autres ? Ça me rend parano.
Je ne sais plus si je me suis entraîné, si, tout ça, je le mérite ?
Si l'univers était avec moi ou si ça fait dix ans que je me bats...

[...]

En vrai, je ne sais pas comment ça se passe.
En vrai, je ne sais pas qui maintient le cap.
Si ça vient de moi ou si ça vient des astres.

Ce monde est cruel.
Ce monde est cruel.
Faut changer les choses, si ce monde est cruel,
c'est sûr qu'il y en a d'autres.
Je remercie les anges, je remercie les autres,
je remercie les miens, remerciez les vôtres.
Ce monde est cruel, mais je vous remercie quand même.

Merci pour tout
}


    \setuptoc{toc}{totoc}
    \tableofcontents

    \pagestyle{headings}
    \mainmatter

    %%%%%%%%%%%%%%%%%%%%%%%%%%%%%%%%%%%%%%%%%%%%%%%%%
    %% Please add the content of your thesis here. %%
    %%%%%%%%%%%%%%%%%%%%%%%%%%%%%%%%%%%%%%%%%%%%%%%%%

    \part{Corps du document}
    
    \chapter{Introduction}
    % Here you introduce your topic to the reader.

\section{Contexte}

\subsection{Technologique}

\subsection{Humain}

Cette thèse a été menée à l'Université de Lille, en collaboration avec le \emph{Centre de Recherche en Informatique, Signal et Automatique de Lille} (communement abrégé en laboratoire CRIStAL). Cette thèse a été financée par une dotation de l'Université de Lille.

Cette thèse a été dirigée par Gilles Grimaud, directeur de l'équipe <<~\emph{eXtra Small, eXtra Safe}~>> (abrégé 2XS) du CRIStAL. L'équipe se spécialise dans la conception de logiciels et matériels apportant sécurité, fiabilité et efficacité aux systèmes embarqués fortement contraints. Les travaux menés dans l'équipe portent sur la conception d'un noyau de système d'exploitation munis de preuves formelles de propriétés d'isolation de la mémoire, sur les moyens d'attaque physique sur du logiciel (Bluetooth, LoRa, analyse de la consommation, ...), sur la détection de malware et obfuscation d'applications Android, mais aussi sur des objets mathématiques plus théoriques comme par exemple les fonctions corécursives et leur représentation dans un assistant de preuve.

\emph{2XS} a des relations privilégiées avec d'autres équipes du laboratoire, notamment celles faisant partie du même groupe thématique <<~\emph{Systèmes embarqués adaptables et sécurisés}~>>. Cette thèse a notamment tiré profit d'une forte proximité avec l'équipe \emph{SyCoMoRES}, dont les travaux portent sur la conception et l’analyse des systèmes embarqués temps réel, basé sur l’analyse symbolique de composants paramétriques. La seconde contribution de cette thèse est le fruit de cette collaboration.

Par ailleurs, l'équipe \emph{2XS} est hébergée à l'\emph{Institut de Recherche sur les Composants logiciels et matériels pour l’Information et la Communication Avancée} (abrégé IRCICA). L'IRCICA est un établissement conçu pour favoriser la recherche interdisciplinaire, ce qui a notamment permis à l'équipe de saisir de nombreuses opportunités de collaboration avec l'\emph{Institut d'Électronique, de Microélectronique et de Nanotechnologies} (abrégé laboratoire IEMN), et plus particulièrement avec le groupe de recherche \emph{CSAM} notamment sur les travaux relatifs à l'attaque de logiciel au travers de moyens physiques.

Les travaux présentés dans cette thèse sont liés au noyau de système d'exploitation nommé Pip développé dans l'équipe \emph{2XS}.

\subsection{Pip}

Pip est un noyau de système d'exploitation \emph{minimal} dont le seul but est de garantir l'isolation d'applications s'exécutant sur le système. Pour ce faire, Pip est muni de preuves formelles que ses services préservent les propriétés d'isolation lors de leur exécution. Pip utilise la mémoire virtuelle comme moyen de garantir ces propriétés.

Le projet Pip a démarré avec trois thèses fondatrices :
\begin{itemize}
	\item La thèse de Narjes Jomaa, soutenue en décembre 2018, a porté sur l'aspect formel du noyau. Narjes a développé une méthodologie permettant de raisonner sur le code des services de Pip, ainsi qu'une méthodologie de co-design du code des services avec les preuves formelles afin d'alléger l'effort de preuve global. Narjes est à l'origine des preuves de préservation de l'isolation fournies par Pip ;
	\item La thèse de Quentin Bergougnoux, soutenue en juin 2019, a porté sur l'implémentation du noyau sur l'architecture Intel x86, en particulier sur le code des services actuellement présents dans le noyau. Ses travaux ont aussi porté sur des preuves de concept explorant les possibilités de portage de Pip sur un environnement multicœur ;
	\item La thèse de Mahiedinne Yaker, soutenue en décembre 2019, a porté sur l'implémentation de Pip sur une plateforme embarquée basée sur l'architecture Intel, offrant des perspectives de travail sur les systèmes embarqués. Ces travaux ont aussi portés sur des réflexions autour de la conception de systèmes où les entités y demeurant ne se font pas mutuellement confiance.
\end{itemize}

De ces travaux fondateurs ont émergé de nouvelles opportunités de recherche, dont certains se sont transformés en sujets de thèse. Trois nouvelles thèses ont été pourvues, portant sur des sujets étendants les travaux fondateurs :
\begin{itemize}
	\item La thèse de Nicolas Dejon, soutenue en décembre 2022, qui porte sur l'application des propriétés d'isolation de Pip aux systèmes dépourvus de mémoire virtuelle, mais pouvant restreindre l'accès à certaines portions de mémoire grâce à une \emph{MPU}. Ces caractéristiques sont courantes sur des systèmes beaucoup plus modestes, et se prêtent particulièrement bien à de l'\emph{IoT} ;
	\item Les travaux initiaux de Sofia Santiago Fernandez qui portent sur la preuve de préservation de la sémantique du code des services lors de la compilation du code Gallina \emph{shallow-embedded} vers du code C ;
	\item Mes propres travaux de thèse, présentés dans ce document, portant sur la formalisation du transfert de flôt d'exécution au sein du noyau et de travaux préliminaires relatifs à l'ajout de nouvelles propriétés non relatives à l'isolation.
\end{itemize}

Les doctorants n'ont pas été les seules personnes recrutées pour participer au développement de Pip : c'est par exemple le cas de Damien Amara, recruté en tant qu'ingénieur de recherche. Damien a contribué de manière significative à l'implémentation de Pip sur l'architecture Armv7, ainsi qu'à la version de Pip pour les systèmes munis d'une \emph{MPU}. Pip a aussi été au cœur de nombreuses collaborations industrielles par exemple dans le cadre de projets européens, notamment avec Orange.

\section{Objets d'étude}

Cette thèse est développée autour de trois objets d'études principaux qui transparaissent dans l'état de l'art et les chapitres de contribution.

\subsection{Ordonnancement}

Un des objets d'étude de cette thèse est l'ordonnancement. Dans un système où des entités ont besoin de ressources pour accomplir certaines actions et où les ressources disponibles sont limitées, l'ordonnancement est le fait d'arbitrer quelles entités disposeront d'un accès aux ressources ainsi que les périodes à laquelle elles en disposeront. Le logiciel réalisant l'ordonnancement au sein d'un système est appelé l'ordonnanceur. Le processus d'ordonnancement est omniprésent dans nos ordinateurs modernes, par exemple lorsqu'ils doivent choisir -- plusieurs centaines de fois par seconde -- le prochain programme à exécuter parmi la centaine de programmes attendant leur tour. Cette décision est orientée par la politique d'ordonnancement qui dicte à l'ordonnanceur selon quels critères distribuer les ressources. Certaines politiques s'attachent plus particulièrement au respect de contraintes temporelles : on parle alors de politique d'ordonnancement temps réel.

\subsection{Transfert de flot d'exécution}

Un autre objet d'étude connexe dont il sera abondamment question dans ce document est le transfert de flôt d'exécution. La notion de transfert de flot d'execution est similaire à celle de la commutation de contexte. Lorsqu'un programme est interrompu et qu'un autre s'exécute à sa place, par exemple sous l'effet de l'ordonnancement des programmes du système, il se déroule un transfert de flot d'exécution. Le système se charge de sauvegarder tous les paramètres qui permettront de restaurer le programme interrompu comme s'il n'avait jamais été interrompu, puis charge le programme qui continuera son exécution.

Ce mécanisme et le processus d'ordonnancement au sein d'un système permettent d'exécuter de multiples programmes de manière concurrente et efficace. Ils sont notamment à l'oeuvre lorsque vous travaillez avec de nombreux logiciels simultanément sur votre ordinateur, lorsque vous visionnez une vidéo sur votre téléphone, ou lorsque vous prenez l'avion et qu'il navigue en pilote automatique.

\subsection{Preuve de programmes}

Le dernier objet d'étude principal, omniprésent dans ce document, est la preuve de programmes. La preuve de programmes est une technique permettant d'apporter de très fortes garanties sur le fonctionnement du programme étudié par le biais de démonstrations mathématiques. Apporter de telles garanties sur un programme est néanmoins \emph{extrêmement} coûteux, et n'est généralement entrepris que pour les systèmes dont un dysfonctionnement logiciel pourrait mettre en péril des vies humaines, ou pourrait engendrer la perte d'une quantité astronomique d'argent. De tels systèmes sont courants dans certains secteurs d'activités tels que l'aérospatial ou le transport.

\section{Présentation du document}

\subsection{Plan}

La lecture de ce document a été découpée en 5 chapitres principaux. Le premier chapitre, que vous êtes en train de lire, est un chapitre d'introduction proposant une mise en contexte des travaux de thèses ainsi qu'une brève introduction aux objets d'étude principaux. Ce chapitre se termine par une exposition du plan et propose différents axes de lecture en fonction des sujets d'intérêts du lecteur.

Le second chapitre fait un état de l'art des sujets abordés dans cette thèse. Cet état de l'art se contente de décrire les notions nécessaires à la compréhension des travaux de thèse. Il est décomposé en trois parties, qui reflètent les objets d'étude principaux. La première partie de l'état de l'art concerne les transferts de flot d'exécution. Elle en donnera une définition qui les classifiera et décrira ses implications en terme de logiciel et de sécurité, plus particulièrement au travers du prisme de l'architecture Intel x86. La seconde partie de l'état de l'art, plus modeste, sera dédiée à l'ordonnancement. Elle en fera une présentation générale, discutera des politiques d'ordonnancement en proposant des métriques pour les évaluer, puis discutera plus particulièrement des systèmes temps réel, dont elle développera juste la théorie nécessaire à la compréhension des travaux de cette thèse. Cet état de l'art s'achève sur une partie concernant la preuve de programmes. L'état de l'art sur cette partie commencera par décrire le processus de raisonnement mathématique dans sa généralité, axée sur la vérification automatique du raisonnement grâce aux assistants de preuve, en prenant l'exemple de l'assistant de preuve Coq. Dans un second temps, cette partie décrira les méthodes particulières permettant de raisonner sur des programmes, puis concluera sur les exemples les plus probants de l'application de ces techniques sur des systèmes d'exploitation.

Le troisième chapitre présente la première contribution décrite dans cette thèse, un service de transfert de flot d'exécution unifié au sein du noyau Pip. Le chapitre commence par donner quelques motivations à l'écriture de ce nouveau service. Il décrit ensuite les idées derrière le service puis en décrit l'implémentation détaillée au sein de l'architecture Intel x86, en montrant comment les différents transferts de flot d'exécution ont été unifiés. Le chapitre se poursuit sur l'établissement de la preuve d'isolation traditionnelle de Pip sur ce service en décrivant les ajouts à l'interface avec la monade et détaillant fortement certaines portions du raisonnement, points clés de la preuve complète. Ce chapitre se terminera sur une courte section de retours d'expérience, proposant certaines métriques à propos de ce service et proposant des réflexions sur le résultat produit. 


\subsection{Axes de lecture}



    \chapter{État de l'art (20-30 pages)}

	Ce chapitre a pour intention de définir et préciser les différentes notions nécessaires à la lecture des travaux de thèses, ainsi que de définir le contexte scientifique du travail. Il portera, dans une première section, sur les détails des différents transferts de flot d'exécution dans les systèmes modernes, ainsi que les changements d'états inhérents à ces transferts de flot d'exécution. Cette section abordera ensuite les problèmes de sécurité liés au transfert de flot d'exécution, ainsi que les techniques de mitigation de ces problèmes. Cette section terminera sur les problématiques temps réel touchant au transfert de flot d'exécution.
	La seconde section de ce chapitre fera un état des lieux de la preuve de programme. Elle commencera par discuter de ce qu'est une preuve et de leur vérificaton automatique, ainsi que des stratégies de conduite de preuve. Cette section continuera sur la preuve de programme, en particulier comment raisonner sur un programme impératif. Elle abordera aussi les notions de représentation du langage. Enfin, elle terminera sur les exemples de systèmes vérifiés formellement.

	\section{Transfert de flôt d'exécution}

		Cette section va détailler les différents transferts de flot d'exécution mis à disposition dans les cpus modernes.
		Dans cette section, nous détaillerons les transferts de flot d'exécution qui impliquent une reconfiguration explicite de l'état de la machine ayant un impact sur les droits d'accès aux ressources. Les appels d'une fonction d'un programme vers une autre fonction ne seront pas considérés dans cette section, même s'ils pourraient être considérés comme un transfert de flôt d'exécution.


		\subsection{Hardware}

			\subsubsection{Appels explicites}
			Les transferts les plus courants sont les transferts de flot d'exécution explicites, c'est-à-dire dont la cible est explicitement fournie lors de l’appel, ou clairement établie dans la documentation.

Par exemple, dans Linux, un processus peut demander l’ouverture d’un fichier avec l’appel système open(). Cet appel transfère le flot d’exécution d’un processus non privilégié vers le noyau Linux disposant du plus haut niveau de privilèges. Les fonctions appelables par des transferts explicites servent d’interface entre des logiciels disposant de droits distincts.



Ce type de transfert de flôt d'exécution, d'apparence assez anodine, est pourtant l'objet d'attaques multiples, dont le but est de faire dévier l'exécution (de préférence en mode privilégié) vers du code choisi par l'attaquant. Pour y arriver, un attaquant doit exploiter une vulnérabilité dans une portion de code, qui lui donnera le contrôle d'une zone de mémoire d'intérêt (la pile, le tas, ou même le code). Une fois qu'il contrôle cette zone mémoire, il lui suffit d'écrire un \emph{shellcode}, et d'exploiter une vulnérabilité dans du code privilégié pour que l'exécution du \texttt{return} de la fonction compromise saute dans le shellcode. L'attaquant gagne à ce moment le contrôle de la machine.

Commence alors un jeu du chat et de la souris pour essayer de mitiger l'impact de ces vulnérabilités. Pour compliquer la vie de l'attaquant, et qu'il lui soit plus difficile d'exécuter son shellcode, de nombreuses stratégies ont été entreprises par les fabriquant de matériels ainsi que par les développeurs de systèmes d'exploitation. 

\paragraph{Canaries}
Une première statégie est l'ajout de \emph{canary} qui visent à détecter les corruptions mémoires. Les canaries sont des valeurs écrites dans la pile ou le tas et qui sont générées aléatoirement à chaque exécution. Lors de la sortie de la frame protégée par le canary, le code vérifie que la valeur du canary correspond bien à celle qui avait été écrite initialement ; si ce n'est pas le cas, c'est qu'une corruption mémoire a eu lieu et une faute est levée.

Une des techniques permettant de vaincre les canaries est de lire la valeur initiale du canary avant de corrompre la mémoire. En effet, la canary \textbf{reste la même pour l'intégralité de l'exécution}. Une fois cette valeur récupérée, il suffit de corrompre la mémoire en réécrivant cette valeur au bon endroit pour échapper à la détection. De plus, si l'exploitation de la vulnérabilité permet de corrompre la mémoire de manière fine, il suffit d'éviter d'écrire sur la canary.

\paragraph{Droits fins pour les zones mémoires}
\label{memory_rights}
Une autre stratégie a été de définir des droits fins concernant l'accès aux différentes zones mémoires de l'espace d'adressage des processus. Le mécanisme de mémoire virtuelle permet de définir des droits d'accès propres à chaque page mémoire configurée (lecture, écriture, éxecution, accessible en mode non priviligié). Par exemple, les pages mémoire contenant du code sont typiquement configurées pour des accès en lecture et exécution, alors que les pages contenant des données (pour la pile, le tas, les sections de données d'un binaire) sont configurées pour des accès en lecture/écriture.

Cette stratégie de défense empêche un attaquant d'exploiter une vulnérabilité pour écrire un shellcode code dans la mémoire si on considère que chaque page de mémoire est soit exécutable, soit accessible en écriture. Cependant, il existe des cas d'usage légitimes qui violent cette contrainte, par exemple lors de compilation à la volée (ou JIT, pour Just-In-Time). Fatalement, de tels logiciels sont devenus la cible privilégiée des attaquants, on pourra par exemple citer Webkit \cite{webkitexploit}.
Heureusement, il est peu probable que de tels logiciels aient besoin de s'exécuter en mode privilégié. 

Pour affaiblir ce vecteur d'attaque, cette stratégie de défense est renforcée par des mécanismes de sécurité supplémentaires tels que le \emph{Supervisor Mode Access Prevention} (SMAP) et le \emph{Supervisor Mode Execution Prevention} (SMEP). SMAP permet au processeur de lever une faute lorsque qu'il exécute du code privilégié et qu'il essaie d'accéder (en lecture ou en écriture) à des données présentes dans l'espace utilisateur. SMEP permet en complément de lever une faute lorsque le processeur essaie d'exécuter du code dans l'espace utilisateur alors qu'il se trouve dans un mode d'exécution privilégié.

Ces mécanismes permettent d'isoler le code privilégié de potentiels shellcodes écrit en espace utilisateur. Ainsi, pour compromettre intégralement un système, l'attaquant doit à présent exploiter une vulnérabilité dans le code privilégié, ayant à sa disposition des pages mémoire soit accessibles en écriture soit exécutables et qui, de surcrois, ne font pas partie de l'espace utilisateur.
Nait alors une nouvelle technique d'exploitation de vulnérabilité. 

\paragraph{Return Oriented Programming}
\label{ROP}
Le ROP (pour \emph{Return Oriented Programming}) consiste à attaquer du code vulnérable en n'utilisant que le code déjà accessible dans l'environnement d'origine, mais en exécutant des portions arbitraires de celui-ci. L'attaque consiste à repérer des \emph{gadgets} : de brèves portions de code ayant un effet spécifique sur la mémoire ou les registres, suivi d'une instruction \texttt{return}. Pour l'attaquant, il suffit de dévier le flot d'exécution sur l'un de ces gadgets et de manipuler la mémoire, de manière à ce que l'exécution du gadget entraine l'exécution du suivant. L'attaquant parvient au final à exécuter son shellcode constitué d'une succession de gadgets, contournant les mécanismes de sécurité mentionnés dans le paragraphe précédent.\\

Plusieurs contre-mesures ont émergé pour rendre plus difficile le ROP.

\paragraph{Address Space Layout Randomization}
\label{aslr}
L'ASLR (pour \emph{Address Space Layout Randomization}) rend imprédictible l'adresse des différentes zones de mémoire au sein d'un espace d'adressage virtuel. Les adresses du binaire, de la pile, du tas, des librairies, du noyau, etc. sont rendus aléatoires à chaque nouvelle exécution. L'ASLR est un de ce fait un frein considérable au développement d'un shellcode en ROP, puisqu'il est impossible de prédire où se situeront les gadgets lors de la prochaine exécution.

L'ASLR n'est cependant pas parfait. Les adresses des zones mémoire restantes peuvent être révélées par des pointeurs dans les zones mémoires controlées par l'attaquant \cite{bypasskaslr}, ou grâce à des attaques micro-architecturales \cite{microarchitecturalbypass}.

\paragraph{Vérification de l'intégrité du flôt d'exécution}
\label{cfi}
Une autre approche permettant de réduire la marge de manoeuvre de l'attaquant et de vérifier que le flot d'exécution est conforme à celui attendu. À chaque appel et à chaque retour de fonction, le processeur vérifie si la cible du saut est valide. Plusieurs implémentations existent, notamment des implémentations matérielles au sein des processeurs \cite{intelpointerauth, armpointerauth}, mais aussi certaines implémentations logicielles notamment provenant de compilateurs \cite{compilerpointerauth}. Windows, macOS, Android, iOS utilisent déjà un mécanisme de vérification du flot d'exécution.

\paragraph{eXecute Only Memory}
\label{execute_only_memory}
Le XOM (pour \emph{eXecute Only Memory}), est une fonctionnalité de certain processeurs permettant de déclencher une faute lorsque qu'un accès en lecture est fait sur les pages mémoires configurées comme étant exécutables. Avant cette fonctionnalité, aucune distinction n'était faite entre le processus de récupération des instructions par le processeur et la lecture de données par l'utilisateur. Cette fonctionnalité rend considérablement pour difficile la recherche de gadgets, puisqu'il est impossible pour l'attaquant de lire le code qu'il souhaite compromettre directement sur la cible.
On pourrait cependant argumenter que cette fonctionnalité relève de la sécurité par l'obscurité, et qu'elle n'est pas réellement efficace.

\paragraph{\blockquote{Mieux vaut prévenir que guérir}}

Ces contre-mesures, sans cesse contournées par de nouvelles méthodes d'attaque, supposent qu'il existera toujours des vulnérabilités dans le logiciel comme dans le matériel et tentent donc de limiter au maximum leur impact sur les systèmes affectés. Une toute autre classe de mesures essaie de régler le problème en s'attaquant à l'existence même des vulnérabilités, plutôt que d'essayer minimiser leurs conséquences.

On pourrait citer les méthodes d’analyse statique, les méthodes d’exécution symbolique, de fuzzing, et plus particulièrement le langage Rust conçu pour éradiquer ces vulnérabilités par conception. Par ailleurs, des travaux ont été entamés pour prouver formellement les fonctionnalités de Rust.


			\subsubsection{Fautes}

Les différentes formes de fautes logicielles constituent aussi une forme de transfert de flot d'exécution avec élévation de privilèges. Les logiciels sont susceptibles de déclencher des fautes logicielles de différentes façons, par exemple :
\begin{itemize}
  \item décodage impossible de la prochaine instruction ;
  \item demande d'exécution d'une instruction impossible (division par zéro...) ; 
  \item demande d'accès à une adresse mémoire protégée, résultat de l'activité d'une MMU ;  
  \item demande d'exécution d'une instruction privilégiée en mode non-privilégié.
\end{itemize}
Dans ces différentes situations il s'agit de transferts implicites depuis le logiciel en faute vers une fonction d'un logiciel en charge du traitement de cette faute. Les différentes fonctions de gestion des fautes sont généralement définies par des éléments de configuration du matériel, et, le plus souvent, par l'intermédiaire d'une table (ou vecteur) dont le nom change d'une architecture de microprocesseur à l'autre. Ce vecteur précise généralement le niveau d'élévation de privilèges associé à l'exécution de la fonction de traitement de la faute. Sur les architectures Intel cette table est appelée \emph{IDT} (pour \emph{Interrupt Descriptor Table}).

\subsubsection{Interruptions matérielles}
Les interruptions matérielles sont des transferts non explicites à priori non contrôlés par le code non privilégié. Elles sont déclenchées par le matériel, signalant un événement important à traiter, tel que l'arrivée d'un paquet réseau par exemple. Les fonctions de traitement des interruptions matérielles ainsi que leur niveau de privilèges sont aussi définis dans l'\emph{IDT}.

Puisque les fautes et interruptions déclenchent un changement de privilèges, elles sont un vecteur d'attaque supplémentaire d'intérêt pour un attaquant cherchant à s'octroyer de nouveaux droits. En effet, les mêmes types de failles peuvent résider dans les routines de gestion de ces portions de logiciel. De plus, les interruptions et fautes brisent le flot d'exécution et modifient potentiellement l'environnement d'exécution du code interrompu. Cela les rends d'autant plus susceptibles de contenir des vulnérabilités, qui tombent alors dans la catégorie des vulnérabilités de \emph{concurrence critique}.

Même en ayant pleinement conscience des différentes interactions et dépendances entre les différents composants d'un système, les vulnérabilités de concurrence critique sont \textbf{notoirement difficiles à cerner}, principalement à cause d'un phénomène d'explosion combinatoire. Il peut s'avérer difficile de détecter une telle vulnérabilité par les tests, puisqu'ils sont souvent effectués dans des environnement très controlés où les mêmes conditions d'exécution sont artificiellement répétées, occultant d'autres fils d'exécution possibles. Malgré cela, si le développeur parvient à exhiber un fil d'exécution contenant un comportement anormal, il peut alors être délicat de reproduire le fil d'exécution ayant conduit à ce comportement. En effet, le fil d'exécution peut etre le résultat de nombreuses interactions - parfois non-déterministes - du programme avec son environnement. De plus, attacher un debuggueur tel que \texttt{gdb} peut modifier subtilement ces interactions, de manière telle qu'il soit impossible d'exhiber à nouveau le comportement anormal : on parle alors d'Heisenbug.

Pour illustrer la difficulté à cerner cette catégorie de bugs, on pourrait citer un problème d'incohérence de cache dans le noyau de système d'exploitation de la Nintendo Switch après une interruption matérielle ayant mené à un changement de coeur. Les effets de ce bug avaient été observés dès la sortie de la console ; il n'a cependant été trouvé et corrigé qu'à la sortie du firmware 14.0.0 de la console, soit plus de 5 ans après les premiers rapports \cite{switchbug}.
On pourrait aussi citer une vulnérabilité exploitée dans la pile IPV6 du noyau FreeBSD de la console Playstation 4 de Sony, profitant d'une situation de concurrence critique pour déclecher un Use-After-Free et compromettant le système d'exploitation. Cette faille présente sur tous les firmwares de la console depuis son lancement en 2013 a été découverte en 2018 puis divulguée et patchée en 2020, soit 7 ans après sa mise sur le marché \cite{ps4bug}. La même vulnérabilité était présente jusqu'à la version 5.00 du firmware de la Playstation 5, soit plus de deux ans après le rapport de bug concernant l'ancienne console \cite{ps5bug}.

Certains débuggers (notamment \texttt{rr}~\cite{mozRR}) ont implémenté une fonctionnalité "\emph{record and replay}" (enregistre et rejoue), permettant de capturer une trace du programme inspecté, puis de rejouer dynamiquement cette même trace à la demande. Cette fonctionnalité résoud le problème de la reproductibilité des comportement anormaux des programmes, et permet de surcrois de revenir à un état précédent de l'exécution lors d'une session de débuggage, ce qui est impossible avec les débuggers classiques. Certains émulateurs tels que Xen ou Qemu proposent des fonctionnalités "record and replay" sur les machines virtualisées. Malheureusement, les fonctionnalités "record and replay" pour des programmes sur plusieurs coeurs sont actuellement extrêmement lents, et profiteraient grandement d'implémentation matérielles si elles venaient à exister \cite{mozRR}.

De nombreux travaux ont été menés afin de détecter les situations de concurrence critique, par exemple par analyse statique~\cite{racerX}. D'autres travaux ont développés des méthodes plus particulières permettant de découvrir des situations de concurrence critique liées aux interruptions matérielles~\cite{sdracer}. Parallèlement, dans le monde de la preuve formelle, on pourrait citer les travaux ayant abouti à la logique de séparation~\cite{separationlogic}.

		\subsection{Software}

			\subsubsection{Changement de droits}

Sur l'architecture Intel x86, les différents transferts de flot d'exécution peuvent s'opérer par le biais de différentes instructions et événements matériels. Néanmoins, lorsqu'un changement de droits est requis lors d'un transfert, les différents chemins sont régis par un ensemble de mécanismes de contrôle relativement homogènes.

\paragraph{Changement de droits sous x86}
\label{ring}

Les privilèges attribués au code s'exécutant actuellement sur la machine sont ceux du \emph{segment} chargé dans le registre \texttt{CS} (pour \emph{code segment}). Les segments sont définis dans la \emph{GDT} (pour \emph{Global Descriptor Table}) à l'initialisation de la machine. La \emph{GDT} est une table globale spécifiée par Intel, dont l'adresse est accessible par un registre dédié. Dans cette table par exemple, Linux se contente de définir deux segments de code. Un premier segment associé au niveau de privilèges maximum du microprocesseur, nommé par Intel \emph{ring 0}, utilisé pour le code responsable du système ; et un second segment non-privilégié, associé au niveau de privilèges \emph{ring 3}, pour le reste du code. 

Contrairement au code s'exécutant avec le segment privilégié, le code s'exécutant sans privilège ne peut pas changer de segment à sa guise. Des mécanismes de contrôle du processeur déclenchent une faute si du code non-privilégié essaie de modifier son segment de code. Pour y parvenir, il est possible d'utiliser les \emph{gates}, qui sont des tremplins définis dans les tables globales du système (comme l'\emph{IDT} ou la \emph{GDT}), permettant au code non privilégié d'appeler une fonction prédéfinie qui s'exécute avec d'autres droits.

Lorsqu'un changement de segment déclenche un changement de niveau de privilèges, le processeur change de pile. Ce changement de pile permet d'éviter aux routines privilégiées les échecs dus à un manque de place sur la pile, ainsi qu'à les prémunir d'éventuelles interférences avec les procédures non privilégiées~\cite{intel_stack_switch}. 
Une pile doit être définie par niveau de privilèges (\emph{ring}) utilisé par le système ; leurs adresses doivent être renseignées dans une structure appelée \emph{TSS} (pour \emph{Task State Segment}). Cette structure est initialisée conjointement avec la \emph{GDT} qui contient son descripteur. Un registre dédié indique au processeur la position de ce descripteur dans la \emph{GDT}.

\paragraph{Appels systèmes sur x86}

Afin d'obtenir un transfert de flot d'exécution avec élévation de privilèges, le logiciel appelant non-privilégié peut exécuter des instructions dédiés aux différentes \emph{gates} des tables globales. Tout d'abord, l'instruction \texttt{int} permet d'appeler les \emph{gates} situées dans l'\emph{IDT}. Ces \emph{gates} sont soit des \emph{interrupt gates}, des \emph{trap gates} ou des \emph{task gates}~\cite{intel_idt_gates}. \texttt{int} s'accompagne d'un argument correspondant à l'index de la \emph{gate} ciblée dans l'\emph{IDT}. Le code ainsi appelé sera exécuté avec le niveau de privilèges spécifié par le segment de code indiqué dans la gate (et chargé dans le registre \texttt{CS}).

L'instruction \texttt{callf} permet d'utiliser les \emph{gates} situées dans la \emph{GDT}. Ces gates sont soit des \emph{call gates} ou des \emph{task gates}. Ces gates permettent de copier un nombre fixé d'arguments depuis la pile de l'appelant dans la pile du code privilégié à l'appel de l'instruction \texttt{callf}. Le nombre d'arguments à copier est renseigné dans la \emph{gate} ciblée par l'instruction. Là aussi, l'élevation de privilèges est spécifiée par le segment de code indiqué dans la gate et chargé dans \texttt{CS} lors de l'appel.

La troisième manière de déclencher une élévation de privilèges est l'instruction \texttt{sysenter}. Cette 
instruction ne sollicite aucune \emph{gate} : à la place, elle utilise les \emph{MSR} (pour \emph{model-specific registers}), qui sont des registres de contrôle du processeur. Ces \emph{MSR} sont manipulables grâce aux instructions \texttt{wrmsr} et \texttt{rdmsr}, qui sont des instructions privilégiées et qui permettent d'écrire et de lire dans ces registres respectivement. Un appel à \texttt{sysenter} utilise les MSR \texttt{0x174}, \texttt{0x175} et \texttt{0x176} pour charger \texttt{CS} \texttt{EIP} \texttt{SS} \texttt{ESP}. Le système doit donc avoir initialisé ces registres avant l'utilisation de \texttt{sysenter}. De plus, \texttt{sysenter} ne sauvegarde pas l'adresse de retour ni l'adresse de la pile lors d'un appel, qui doivent être placés dans les registres \texttt{ECX} et \texttt{EDX} au moment de l'appel à \texttt{sysexit} pour retourner dans le code appelant.

\paragraph{Interruptions et fautes sur x86}

Les interruptions liées au matériel sur l'architecture x86 étaient autrefois gérées par un coprocesseur (le PIC 8259 \emph{pour Programmable Interrupt Controller} ou plus récemment, l'APIC pour \emph{Advanced Programmable Interrupt Controller}). Ce coprocesseur est maintenant intégré au processeur, mais nous continuerons de parler de coprocesseur pour honorer l'histoire. Ce coprocesseur utilise les \emph{gates} situées dans l'IDT de la même manière que l'instruction \texttt{int}. Il est possible de configurer ce coprocesseur
pour qu'il utilise une certaine plage de niveaux d'interruption, ou pour qu'il masque temporairement la venue de nouvelles interruptions. Les fautes utilisent elles aussi l'\emph{IDT}, et utilisent les trente-deux premières \emph{gates} de la table, en fonction de la faute à déclencher. Les fautes et interruptions déclenchées par le processeur ou le coprocesseur ont toujours le droit d'utiliser les \emph{gates}, peu importe le niveau de privilèges du code s'exécutant au moment de l'interruption.

\paragraph{Fonctionnement de la \emph{MMU} sur l'architecture Intel 32 bits}

Sur l'architecture Intel x86, mais aussi sur toutes les autres architectures supportant une \emph{MMU} (pour \emph{Memory Management Unit}), il est possible d'associer des droits d'accès spécifiques à chaque pages de mémoire configurée dans l'espace d'adressage virtuel.

Le concept de traduction d'adresse virtuelle vers l'adresse réelle est le suivant. Les bits de poids forts de l'adresse virtuelle servent à traverser les tables de la MMU (sur l'architecture Intel x86, le \emph{Page Directory} et les \emph{Page Tables}). Les bits de poids faible correspondent à l'emplacement de l'adresse désirée dans la page réelle obtenue après traduction (souvent appelé \emph{offset}).

En particulier, sur Intel x86 et en mode de pagination 32 bits pour des pages de 4 Kio, les espaces d'adressage sont configurés par une structure de données arborescente de pages de 4Kio. Cette structure de données a deux étages : la racine appelée PD pour \emph{Page Directory}, et les feuilles appelées PT (pour \emph{Page Tables}). Le développeur renseigne l'adresse du Page Directory à utiliser dans le registre \texttt{CR3} du processeur ; cette adresse est alignée sur 4Kio, les 12 bits de poids faibles (11-0) sont ignorés ou sont réservés pour un autre usage.

Plus précisement, le \emph{Page Directory} est constitué de 1024 entrées de 32 bits appelées les \emph{PDE} (pour \emph{Page Table Entries}). Les 20 bits de poids fort (31-12) de ces entrées déterminent l'adresse de la \emph{Page Table} à utiliser, alignée sur 4Kio. Les \emph{Page Tables} sont aussi constituée de 1024 entrées de 32 bits appelées \emph{PTE} (pour \emph{Page Table Entries}). De la même manière, les 20 bits de poids forts (31-12) déterminent l'adresse de la page de mémoire réelle, alignée sur 4Kio. (Il est aussi possible de configurer des pages de 4Mio plutot que des pages de 4Kio en modifiants certains bits de controle des \emph{PDE}.)

Lors de la traduction d'une adresse virtuelle, les 10 bits de poids forts de l'adresse virtuelle (31-22) déterminent le numéro de \emph{PDE} à utiliser, les bits (21-12) déterminent le numéro de \emph{PTE}, et les 12 bits de poids faible restants (11-0) déterminent l'\emph{offset} de l'adresse cible dans la page réelle.~\cite{intel_32bits_paging}

\paragraph{Contrôle d'accès par la MMU sur Intel x86}
Dans le mode de pagination 32 bits d'Intel, les droits associés à chaque page sont présents dans les \emph{PTE}, dans les 12 bits de poids faible. Le bit 1 (\texttt{R/W}) permet d'empêcher les accès en écriture sur la page. Le bit 2 (\texttt{U/S}) permet d'empêcher n'importe quel accès utilisateur à la page - en lecture ou en écriture. Le niveau de privilège de l'accès dépend du \emph{CPL} (pour \emph{Current Privilege Level}) de l'instruction courante, souvent déterminée par le segment de code actuel.
L'architecture Intel permet de restreindre la récupération des instructions en discriminant chaque page de mémoire. Cette fonctionnalité n'est cependant pas disponible dans le mode de pagination 32 bits, puisqu'elle nécessite des \emph{PDE} ou \emph{PTE} longues de 64 bits. Elle est par exemple disponible dans le mode de pagination \emph{PAE}. Cette fonctionnalité est activable au travers du bit \texttt{NXE} du registre \texttt{IA32\_EFER}. Pour chaque page de mémoire, mettre le bit 63 (\texttt{XD}) à 1 dans une \emph{PTE} empêche la récupération d'instruction depuis cette page mémoire.

Cependant, d'autres fonctionnalités de contrôle d'accès globaux liées à la \emph{MMU} existent. L'architecture Intel propose aussi les mécanismes \emph{SMAP} et \emph{SMEP} comme décris dans le paragraphe \ref{memory_rights}.
Le bit \emph{SMAP} (pour \emph{Supervisor Mode Access Protection} présent dans le registre CR4 permet d'empêcher du code utilisant un segment de données privilégié d'accéder aux pages mémoire annoncées comme étant des pages mémoire utilisateur (bit \texttt{U/S} à 1).
Le bit \emph{SMEP} pour \emph{Supervisor Mode Execution Protection} présent dans le registre CR4 permet d'empecher la récupération de code lorsque le segment actuel octroie des accès privilégiés et que la page mémoire contenant les instructions est annoncée comme une page mémoire utilisateur (bit \texttt{U/S} à 1).

%				Espace d'adressage, niveau de privilèges

			\subsubsection{Capture de l'état d'exécution}

\textcolor{red}{J'arrive pas à introduire le concept}

\paragraph{Le contexte d'exécution}

Le contexte d'exécution d'un programme est constitué d'informations permettant de reprendre de manière saine l'exécution d'un programme à un moment arbitraire de son exécution après qu'il ait été interrompu. Le contexte d'exécution est donc intimement lié à l'état de la machine lors de l'exécution du programme. 

\textcolor{red}{J'ai envie de parler de la limite des contextes d'exécution, notamment avec les attaques micro architecturales mais je n'arrive pas à le formuler} Il est difficile de déterminer ce qui fait partie de l'état d'un programme : le contenu des registres du processeur et de la mémoire, mais aussi l'état des composants logiciels ou matériels intéragissant avec le programme (tels que les périphériques, le cache, etc.).

Le contexte d'exécution d'un programme doit contenir toute partie de l'état du programme susceptible d'être modifiée légitimement par d'autres programmes du système. De ce fait, il n'est pas nécessaire de copier la mémoire utilisée par le programme. Dans les systèmes d'exploitation modernes, chaque programme dispose d'une portion de mémoire qui lui est dédiée. Cela peut être garanti par conception dans le cas de systèmes collaboratif ou plus strictement par des mécanismes de controle d'accès relatifs à une \emph{MMU} ou une \emph{MPU} par exemple. 
Puisqu'il n'est pas nécessaire de copier le contenu de la mémoire, il suffit donc de copier les registres du processeur susceptibles d'être modifiés. Lors d'appels sans changement de droit, des règles de sauvegarde des registres existent (par exemple \cite{arm32_bit_callconv}). Ces règles décrivent notamment quels sont les registres susceptibles d'être modifiés par l'appelé, et peuvent permettre d'économiser la sauvegarde de certain registres.
Cependant, puisque les transferts de flot d'exécution avec changement d'espace d'adressage et de droits peuvent être implicites, le système ne peut faire aucune hypothèse sur les registres préalablement sauvegardés par le programme interrompu. De plus, il est courant de continuer l'exécution d'un autre programme susceptible de modifier n'importe quel registre à sa disposition. En toute généricité, il est donc impossible de prévoir quels registres resteront intacts lors d'un tel transfert : il est donc nécessaire de sauvegarder tous les registres modifiables par le programme interrompu.

Le processeur nous assiste dans ce travail. En effet, sans l'aide du processeur, le contenu de certains registres comme le pointeur d'instruction ou le pointeur de pile seraient instantanément perdus au moment du transfert. Voici comment se passe le transfert pour chaque mode de transfert différent sous l'architecture Intel x86 : 


\paragraph{Changement de pile et registres sauvés lors d'un appel à \texttt{callf}}

Lors d'un appel à \texttt{callf}, le processeur accède à une call gate située dans la \emph{GDT}. Cette callgate indique entre autres, le segment de code à utiliser (et donc le niveau de privilège associé), le code à exécuter, le nombre d'arguments à copier depuis la pile utilisateur. La pile ainsi que le segment à utiliser sont eux renseignés dans la \emph{TSS} et sont liés au niveau de privilège du segment de code de la callgate.

Tout d'abord, le processeur sauve de manière temporaire le segment de pile et le pointeur de pile dans un tampon interne. Il remplace les registres de segment de pile et de pointeur de pile par ceux renseignés dans la \emph{TSS}, changeant de pile. Il pousse ensuite le segment de pile et le pointeur de pile de l'utilisateur dans la nouvelle pile. Il copie ensuite les arguments depuis la pile utilisateur vers la nouvelle pile, leur nombre étant renseigné dans la call gate. Le processeur pousse ensuite sur la pile le registre de segment de code et le pointeur d'instruction, avant de les remplacer par ceux renseignés dans la call gate.

\paragraph{Changement de pile et registres sauvés lors d'une interruption, d'une faute ou d'un appel à \texttt{int}}

Lors d'une faute, d'une interruption ou d'un appel à \texttt{int}, le processeur accède à une interrupt gate ou une trap gate dans l'\emph{IDT}. Comme pour la call gate, la gate contient le segment de code à utiliser, le code à exécuter. Ce transfert de flot d'exécution n'entraine pas une copie par le processeur d'éventuels arguments depuis la pile utilisateur sur la nouvelle pile.

Tout d'abord, le processeur change de pile, en sauvant temporairement les valeurs utilisateurs du segment de pile \texttt{SS} et de la pile \texttt{ESP}, et en les écrivants sur la pile renseignée dans la nouvelle pile renseignée dans la \emph{TSS}. Le processeur sauve ensuite l'état des registres \texttt{EFLAGS}, du segment de code \texttt{CS} et du pointeur d'instruction \texttt{EIP}. EFLAGS contient entre autres l'état des drapeaux d'overflow, de retenue, de parité, de conditions, mais aussi d'activation des interruptions. Le processeur peut pousser un code d'erreur supplémentaire sur la pile dans le cas de déclenchement de certaines fautes afin de préciser leur cause.

		\subsection{Failles de sécurités associées}
			%https://google.github.io/security-research/pocs/linux/bleedingtooth/writeup.html#achieving-rip-control
			%https://google.github.io/security-research/pocs/linux/cve-2021-22555/writeup.html
			%https://github.com/Bonfee/CVE-2022-0995
			CVE historiques ? :D

			%https://pointer-authentication.github.io/

		\subsection{Ordonnancement}

		Le transfert de flôt d'exécution est au coeur du fonctionnement de systèmes complexes, par exemple lors de l'ordonnancement au sein d'un système informatique. L'ordonnancement dans un système d'exploitation permet à plusieurs programmes de s'exécuter de manière concurrentielle sur le même processeur, en alternant leur exécution. L'ordonnanceur décide quel programme sera le prochain à s'exécuter sur une unité de calcul donnée, qu'il décide selon une \emph{politique d'ordonnancement}. Ces politiques sont multiples et visent à satisfaire des contraintes diverses, pouvant par exemple viser à exécuter les programmes interactifs de manière prioritaire, ou plus simplement à exécuter chaque programme l'un après l'autre. L'ordonnancement permet ainsi d'optimiser l'utilisation du processeur pour un objectif particulier. 

			\subsubsection{Partage équitable du CPU}
		Une des fonctions principales de l'ordonnancement est le partage du temps CPU. En effet, les programmes s'exécutant à sur un système ne collaborent pas forcément avec le système pour permettre aux autres programmes de s'exécuter. Il revient alors au système d'exploitation d'interrompre les programmes à intervalles réguliers, grâce aux interruptions déclenchées par l'horloge par exemple, pour ne pas créer de situation de \emph{famine}. Lorsque le programme est interrompu, le système appelle l'ordonnanceur, qui choisira le meilleur programme à exécuter selon sa politique.
		Plusieurs indicateurs permettent d'évaluer les politiques d'ordonnancement dans les systèmes classiques, notamment :
		\begin{itemize}
			\item{le débit, mesurant le nombre de tâches terminées sur une certaine période}
			\item{le temps d'attente, mesurant le temps moyen entre le moment où la tâche a été créée et le moment où elle a commencé à être exécutée}
			\item{l'équité, mesurant la différence entre les temps CPU accordés à chaque processus}
			\item{la latence, mesurant le temps d'attente moyen entre la soumission de la tache et la production des premières sorties par la tache}
		\end{itemize}

			\subsubsection{Respect des contraintes de temps}

		L'ordonnancement est un élément clé des systèmes temps réel. Les systèmes temps réel ont des contraintes de temps associées à chaque unité de travail. Les systèmes temps réel \emph{souples} sont munis de contraintes de temps indicatives, le non respect des contraintes temporelles pouvant mener à une dégradation de la qualité des résultats produits par le système, par exemple dans le cas d'applications multimédia (audio, vidéo, etc.) ou dans le cas de systèmes de surveillance collectant des données (par exemple météorologiques). Nous nous attarderons sur les systèmes temps réel \emph{stricts}, qui doivent impérativement compléter chaque unité de travail demandée avant leur expiration sous peine de dysfonctionnement critique du système. Ces systèmes s'appuient sur des modèles mathématiques décrivant les actions à accomplir par le système.

		\paragraph{Tâches et jobs}
		Les systèmes temps réel sont conçus pour réaliser des actions dans certaines limites temporelles. En toute généricité, ces actions sont uniques au sein des systèmes, et sont désignées en anglais par le terme \emph{job} (utilisé dans la suite du document faute d'un équivalent français adéquat). Les \emph{jobs} ont au minimum une action qui leur est associée, une date à partir de laquelle il est possible de commencer à réaliser l'action (appelée \emph{release date}), une durée (appelée \emph{duration}), et une échéance (appelée \emph{deadline}.
		Cependant, les actions à réaliser par les systèmes temps réels sont rarement unique en pratique : les systèmes peuvent être amenés à répéter certaines actions (ou \emph{job}) tout au long de leur fonctionnement. On parle alors d'une \emph{tâche} (ou \emph{task} en anglais), produisant un nouveau job unique à chaque fois que l'action doit être répétée. Par exemple, un système temps réel pourrait être amené à vérifier périodiquement la valeur produite par une sonde pour vérifier son bon fonctionnement : on parlerait alors de tâche de vérification des valeurs de la sonde ; chaque vérification indépendante de la valeur produisant un \emph{job}.

		\paragraph{Modèles de tâches}
		Les tâches d'un système temps réel sont traditionnellement par trois modèles mathématiques différents. Les tâches \emph{périodiques} permettent de représenter les tâches qui doivent produire un nouveau job après chaque période qui leur est propre. Les tâches \emph{sporadiques} permettent de représenter les tâches qui peuvent produire un nouveau job à un moment aléatoire, mais qui ne peuvent produire un nouveau job tant qu'après un certain délai d'attente. Les tâches \emph{apériodiques} sont des tâches qui peuvent produire un nouveau job à un moment aléatoire, sans délai particulier.

		\paragraph{Vérification du respect des échéances}
		La vérification du respect des contraintes temporelles s'effectue \emph{avant} la mise en fonctionnement du système temps réel. Lors de la conception du système, un modèle du fonctionnement du système doit être créé, associant une durée (ou une borne supérieure associée à la durée) à chaque action réalisée par le système, ainsi qu'un modèle spécifiant à quel moment chaque action doit être effectuée. Une fois ce modèle créé, il doit faire l'objet d'une \emph{analyse d'ordonnançabilité}, vérifiant que chaque action entreprise pourra se terminer dans le temps imparti.

		Lorsque l'analyse d'ordonnançabilité a été effectuée, et qu'elle atteste qu'il sera toujours possible d'ordonnancer les différents jobs du système en respectant les échéances, il existe deux méthodes permettant d'ordonnancer les jobs. La première méthode consiste à précalculer l'ordonnancement des jobs, et à l'inclure de manière statique dans le système. Cette méthode de calcul hors ligne (ou \emph{offline} en anglais) du plan d'ordonnancement, a pour avantages d'affranchir le système temps réel du coût de l'ordonnancement "en direct". Le système n'aura pas à choisir lui même la prochaine tâche à effectuer ; elle lui a été précalculée. Cependant, il n'est pas toujours possible de savoir à chaque instant du fonctionnement d'un système temps réel quels seront les jobs à exécuter par exemple dans le cas de tâches sporadiques ou apériodiques. En effet, certaines actions des sytèmes temps réels peuvent être liées à des stimuli externes, comme par exemple l'action de maintenir l'assiette d'un avion en vol lorsqu'il tangue sous l'effet de turbulences. Dans ce genre de cas, le système temps réel doit pouvoir s'accomoder de telles variations, et en conséquence le plan d'ordonnancement doit pouvoir être modifié pendant le fonctionnement du système. Il n'est alors plus possible de précalculer le plan d'ordonnancement ; le système temps réel doit intégrer un ordonnanceur pour faire face à ces aléas.

		\paragraph{Vérification des résultats produits par l'ordonnanceur}
		De nombreuses politiques d'ordonnancement existent :
		\begin{itemize}
			\item{tourniquet, fair-share, foreground-background }
			\item{Rate monotonic (ancien et fondateur, liu et layland)}
			\item{Deadline monotonic (optimal sous quelques hypothèses)}
			\item{earliest deadline first (optimal)}
			\item{shortest job next, shortest remaining time (famine)}
			\item{Highest response ratio next (variante de shortest job next proposant une réponse aux problèmes de famine)}
			\item{Multilevel feedback queue (ancien, prix turing)}
			\item{YDS (politique visant à minimiser la consommation énergétique, optimal quelques conditions)}
		\end{itemize}
		\textcolor{red}{Est ce que je dois détailler ? Est ce que c'est pas hors sujet et qu'on s'en foutrait pas un peu}

	\section{Preuve de code}

	Cette section va décrire les notions et travaux nécessaires à la compréhension des contributions sur la preuve formelle sur du code décrits dans les chapitres suivants. La première sous-section décrira le processus de vérification automatique de preuves, en décrivant d'abord le processus de raisonnement automatique et introduisant les notions d'axiomes, d'hypothèses, ainsi que le déroulement de la preuve avec les règles d'inférence. Cette description sera ensuite illustrée par l'exemple de l'assistant de preuve Coq, outil de l'état de l'art que j'ai utilisé pour mes contributions. Enfin, la sous-section décrira les deux méthodes principales de raisonnement : la preuve directe et la preuve par raffinement.

	La seconde sous-section décrira plus spécifiquement le processus de raisonnement formel sur du code. Elle commencera par discuter du raisonnement sur des programmes impératifs, et en particulier de la \emph{logique de Hoare}. En seconde partie de cette sous section seront décrits les aspects concernant la représentation du programme dans l'assistant de preuve, ainsi que le principe des \emph{monades}, permettant de capturer les effets de bords des programmes impératifs dans les langages fonctionnels des assistants de preuve.

	Enfin, la dernière sous-section décrira les travaux les plus reconnus concernant l'application du raisonnement formel sur les systèmes d'exploitation, en finissant par le noyau développé au sein de l'équipe.

		\subsection{Vérification automatique d'une preuve}

			\subsubsection{Qu'est ce qu'une preuve formelle?}
			Une preuve ou une démonstration formelle se place dans le cadre d'un système formel, qui définit les règles permettant de formuler des propositions mathématiques valides ainsi que les règles de transformations pouvant s'appliquer aux propositions. Une démonstration est l'ensemble des transformations effectuées sur une proposition initiale pour la transformer en une proposition finale. Cette proposition finale peut porter plusieurs noms suivant l'importance du résultat obtenu : \emph{proposition}, \emph{lemme}, ou encore \emph{théorème}. Une preuve est correcte si chaque transformation fait parti des règles de transformation (ou \emph{règles d'inférence}) du système formel dans lequel se place la démonstration. L'utilisation d'un système formel permet la vérification automatique des preuves par un ordinateur.

			\paragraph{Axiomes} Le système formel peut parfois inclure des \emph{axiomes} qui sont des propositions qui ne peuvent pas être prouvées, et qui servent de point de départ au raisonnement au sein de ce système formel. Pour qu'un système formel soit intéressant, il faut que ses axiomes n'amènent pas à une \emph{contradiction}.

			\paragraph{Hypothèses} Les hypothèses sont des propositions qui n'ont pas été prouvées. Les hypothèses peuvent servir à établir des preuves d'autres propositions. Cependant, toute démonstration utilisant une hypothèse n'est pas complète tant que l'hypothèse elle même n'a pas été prouvée, contrairement aux axiomes qui n'ont pas vocation à être prouvés. Les hypothèses peuvent par exemple servir à structurer les longues preuves.

			\paragraph{Raisonnement}
			Le raisonnement au sein d'une preuve est déroulée grâce à l'application de \emph{règles d'inférence}. Une règle d'inférence s'applique sur une ou plusieurs propositions initiales appelées \emph{prémisses} et crée une nouvelle proposition en retour qu'on appelle la \emph{conclusion}.

			\textcolor{red}{Petit exemple sur la forme des règles d'inférences ? type : $ \text{prémisses} \vdash \text{conclusion}$ ?}

			\paragraph{Lien entre preuve et véracité de la proposition finale} Il est important de distinguer le fait qu'une preuve soit correcte et la valeur de vérité de la proposition finale. En effet, la valeur de vérité d'une proposition résultant d'une démonstration est liée à la valeur de vérité des prémisses. Entre d'autres termes, toutes les hypothèses utilisées dans la démonstration doivent être valides pour que la proposition finale soit valide.

			\subsubsection{Exemple de Coq}
			L'assistant de preuve Coq est un logiciel open-source français initiallement développé à l'INRIA permettant de vérifier automatiquement des preuves. Le langage de Coq est Gallina, un langage fonctionnel proche d'OCaml permettant de décrire à la fois preuves, programmes, et prédicats. En effet, Coq repose sur le calcul des constructions : un système formel dérivé du lambda calcul et utilisant l'isomorphisme de Curry-Howard pour établir un lien entre les démonstrations et les programmes. Ainsi, le calcul des constructions s'inscrit dans la lignée des logiques intuitionnistes formant des preuves \emph{constructives}. 

			\paragraph{Logique intuitionniste et preuves constructives} La logique intuitionniste et des preuves constructives rejettent le principe de \emph{tiers exclu} qui stipule que soit une proposition est vraie, soit sa négation est vraie. En particulier, il n'est pas possible d'établir une preuve d'une proposition par l'absurde, c'est à dire en montrant que la négation d'une proposition aboutie à une contradiction ($ \neg P \implies \bot \vdash P$). Cette règle, uniquement valide en logique classique est à distinguer de la règle de réfutation ($P \implies \bot \vdash \neg P$). Cette règle, valide à la fois en logique classique et en logique intuitionniste, peut par exemple être utilisée pour prouver $\sqrt{2}$ n'est pas rationnel.
			Pour en faciliter la compréhension, certaines propositions en logique intuitionniste peuvent être interprétées d'une manière différente de la logique classique, par exemple:
			\begin{itemize}
				\item{$A$ se lit << $A$ est prouvable >>}
				\item{$\neg A$ se lit << $A$ est contradictoire >>}
				\item{$\exists x, A(x)$ se lit << On peut exhiber un élément $x$ tel que $A(x)$ est prouvable >>}
				\item{$\forall x, A(x)$ se lit << Pour n'importe quel élément $x$, $A(x)$ est prouvable >>}
			\end{itemize}

			\paragraph{Autres assistants de preuve} D'autre assistants de preuve existent tels que Isabelle/HOL, F*, Agda, Lean... \textcolor{red}{Que dois-je détailler ici ?}

			\subsubsection{Stratégie de conduite de preuve}
			Tout comme la conception de logiciel, la conduite de preuve nécessite parfois un effort d'ingénierie pour structurer la preuve. Savoir si une preuve mérite d'être structurée relève  de l'expertise : on parle alors de génie de la preuve. Il existe deux méthodes principales permettant de compléter la preuve d'une proposition : la preuve directe et la preuve par raffinement.

			\paragraph{Preuve directe} La méthode de preuve directe est à opposer à la méthode de preuve par raffinement. Elle consiste à utiliser les éléments directement fournis par le langage et éventuellement quelques \emph{lemmes} ou \emph{théorèmes} intermédiaires pour arriver à la preuve finale. La méthode de preuve directe est utilisée pour la majorité des preuves concernant le noyau Pip développé dans l'équipe. La principale force de la preuve directe est qu'elle se contente de prouver uniquement la propriété finale ; chaque proposition intermédiaire ne dépasse pas le cadre de la preuve initiale. La contrepartie d'une telle approche est qu'il est fastidieux de modifier une preuve si des modifications mineures venaient à être apportées aux prémisses.
			
			\paragraph{Preuve par raffinement} La preuve par raffinement est plutôt employée lorsque les étapes de la preuve à conduire ne sont pas immédiates, souvent de par la complexité de la preuve. La méthode de preuve par raffinement consiste à appliquer la stratégie << diviser pour régner >> en trouvant des abstractions intermédiaires permettant de découper la preuve en morceaux indépendants, ce qui permet par ailleurs de rendre la preuve modulaire. Ces avantages viennent au prix d'un effort d'abstraction supplémentaire non requis par la preuve directe. Certaines de ces abstractions intermédiaires peuvent néanmoins avoir été définies par des travaux préliminaires, réduisant le coût de recherche des abstractions. Par ailleurs, nous défendons la thèse que la preuve par raffinement et l'utilisation d'abstractions de manière générale éloigne la preuve conduite de l'objet d'étude initial, et mène plus facilement à l'oubli de certaines contraintes. On pourrait notamment citer le paradoxe du raffinement\textcolor{red}{C'est une opinion, a-t'elle sa place dans l'état de l'art ?}

			De nombreux travaux munis de preuves formelles sur les systèmes d'exploitation utilisent le raffinement. CertikOS et seL4 en sont les exemples les plus connus. Le projet CompCert, compilateur de code source C garantissant la préservation de la sémantique lors de la compilation, utilise aussi le raffinement pour montrer que le processus de compilation est correct.

		\subsection{Preuve de programme}
			
			Cette sous-section traite de la manière de raisonner sur les programmes, en particulier sur les programmes impératifs dont le paradigme est à priori incompatible avec les langages fonctionnels des assistants de preuves tels que Coq et son langage Gallina. La première partie de la sous-section décrira la logique de Hoare : un système formel permettant de raisonner sur les programmes séquentiels. La seconde sous-section décrira les méthodes de représentation du programme à prouver dans l'assistant de preuve, ainsi que des monades d'état, permettant de représenter les effets de bords dans les langages fonctionnels, et servant d'interface entre le monde mathématique et le monde réel. La dernière partie de la sous-section présentera la problématique de la préservation de la preuve lors du processus de compilation du code source prouvé. 

			\subsubsection{Logique de Hoare}

			La logique de Hoare est un système formel permettant de raisonner sur les programmes séquentiels non-interruptibles.
			La logique de Hoare raisonne sur des \emph{triplets de Hoare} qui sont définis comme suit :
			
			\begin{subequations}
			\begin{gather}
			    \{P\}\ c\ \{Q\}
			\end{gather}
			
			où :
			\begin{itemize}
				\item $P$ représente les \emph{préconditions} qui sont les propriétés sur l'état de la machine supposées prouvables \emph{avant} l'exécution du programme à vérifier - ses \emph{prémisses}.
			    \item $c$ représente le code du programme à exécuter sur la machine et qu'on souhaite vérifier.
			    \item $Q$ représente les \emph{postconditions} qui sont les propriétés sur l'état de la machine qu'on souhaite pouvoir prouver \emph{après} l'exécution du code $c$ du programme à vérifier.
			\end{itemize}
			Par définition, un triplet de Hoare est \emph{prouvable si et seulement si}, quel que soit l'état du système initial $E_0$ satisfaisant les propriétés $P$ alors $c$ fait entrer le système dans un nouvel état $E_n$ dans lequel les propriétés $Q$ sont prouvables.
			Si le nouvel état $E_n$ engendré par $c$ viole les \emph{postconditions} $Q$, alors le triplet $\{P\}~c~\{Q\}$ est contradictoire.
			Au travers du prisme intuitionniste, on pourrait lire le triplet de Hoare $\{P\}~c~\{Q\}$ comme << Si les propriétés $P$ sur l'état de la machine sont supposées prouvables avant l'exécution du code du programme $c$, alors les propriétés $Q$ sur l'état de la machine après exécution de $c$ sont prouvables >>.
			
			Pour garantir formellement que $c$ respecte les propriétés qu'on souhaite garantir sur le système, on peut décomposer le code du programme $c$ en parties $c_i$ aussi élémentaires que souhaité.\\
			Le triplet devient alors : 
			
			\begin{gather}
			    \{P\}~c_1 ; c_2 ; \hdots ; c_n~\{Q\}
			\end{gather}

			On pourrait par exemple considérer que ces parties élémentaires sont les instructions assembleur de la machine.

			La logique de Hoare nous permet alors de décomposer la preuve en observant un enchaînement d'états intermédiaires $E_i$ résultant de l'exécution des instructions $c_i$.

			\begin{gather}
				E_0 \overset{c_1}{\rightarrow} E_1 \overset{c_2}{\rightarrow} \hdots \overset{c_n}{\rightarrow} E_n
			\end{gather}

			 Chaque instruction $c_i$ amène de nouvelles propriétés sur l'état $E_i$ nouvellement créé et permet de créer des triplets de Hoare intermédiaires. Ces nouvelles propriétés servent à la fois de postconditions $Q_i$ pour le triplet concernant l'instruction $c_i$ et de préconditions pour le triplet de l'instruction suivante.
			
			\begin{gather}
			    \{P\}~c_1~\{Q_1\}, \{Q_1\}~c_2~\{Q_2\},~\hdots~, \{Q_{n-1}\}~c_n~\{Q_n\}
			\end{gather}
			\end{subequations}
		
			Ainsi, au fur et à mesure de l'exécution du programme et de l'enchaînement des états, les propriétés sur l'état de la machine changent et s'étoffent. Pour montrer que les postconditions $Q$ souhaitées sur l'état final $E_n$ sont prouvables, il faut montrer qu'elles sont prouvables si les postconditions $Q_n$ résultant de l'exécution de $c$ sont prouvables, soit $Q_n \implies Q$.

			\paragraph{Modèle de la machine et sémantique opérationnelle} Pour pouvoir raisonner grâce à la logique de Hoare, un modèle de la machine et des instructions élémentaires doit être établi. Ces modèles permettent de décrire un système de transition d'état, donnant une signification formelle au programme. Cette sémantique du programme, décrivant des états successifs de la machine, est appelée \emph{sémantique opérationnelle}.

			Par ailleurs, il est important de garder à l'esprit que ces modèles sont \textbf{arbitraires}, et qu'ils ne sont pas intrinséquement équivalents à la machine et aux instructions modélisées. Chaque choix de modèle pour l'étude d'un objet réel vient avec sa part d'incertitude, quelles que soient les précautions prises lors de ce choix.	La preuve d'un programme n'échappe pas à cette règle : elle est aussi contestable que les modèles sur lesquels elle repose.

			\subsubsection{Langage}
			L'assistant de preuve Coq utilise un langage de programmation fonctionnel nommé Gallina. Étant un langage fonctionnel, il est peu adapté à l'écriture de certains programmes tels que les systèmes d'exploitation qui sont des programmes impératifs séquentiels. La sous-section précédente a introduit la logique de Hoare, permettant de raisonner sur de tels programmes. Il faut alors réussir à représenter les programmes séquentiels dans un langage fonctionnel. \textcolor{red}{J'arrive pas à introduire le concept} Deux méthodes existent : le \emph{shallow embedding} et le \emph{deep embedding}.

				\paragraph{Deep embedding} Le deep embedding permet de représenter un langage cible en modélisant la syntaxe (et plus particulièrement l'\emph{AST} du langage cible ainsi que sa sémantique. Il permet de ce fait d'exprimer des propriétés sur la structure du programme. Ces propriétés sont parfois intéressantes, notamment pour procéder à la preuve de propriétés concernant la compilation du langage ciblé. Le projet CompCert utilise un deep embedding de C pour garantir la préservation de la sémantique lors de la compilation vers l'assembleur Intel.
				\paragraph{Shallow embedding} Le shallow embedding au contraire ne définit pas la syntaxe du langage cible, et se contente de ne définir que la sémantique des programmes à prouver. Cette approche est plus légère, mais ne permet cependant pas de prouver de propriétés relatives à la structure du programme. Un shallow embedding se sert de la syntaxe du langage hôte.

				\label{monad}
				\paragraph{Monade d'état} Les langages purement fonctionnels n'ont pas de notion d'état. Ainsi pour modéliser fidèlement des programmes impératifs et leurs effets de bords sur l'état de la machine, on peut le simuler. Dans le cas de Pip, une monade d'état est toute indiquée : elle permet en quelque sorte d'<< enrober >> des valeurs avec un état. Pip utilise des fonctions manipulant ces valeurs et qui sont des fonctions partielles sur l'état, indiquant comment produire la valeur de retour mais aussi quelles modifications éventuelles sont apportées à l'état. Ces fonctions sont dites \emph{monadiques}. Cette construction explicite l'état implicite des programmes impératifs qui transite de fonction en fonction grâce à la fonction \texttt{bind}. Le passage d'état est transparent dans le code grâce à du sucre syntaxique. En plus de la fonction \texttt{bind}, chaque monade nécessite un élément neutre (ou \emph{unit}). Les effets des deux fonctions sont expliqués ci-après.

				Voyons comment la monade d'état a été définie dans Pip. Tout d'abord, Pip définit un type \texttt{state} représentant l'état de la machine qui transitera de fonction en fonction. Pip définit aussi un type \texttt{result} représentant soit une valeur \texttt{val} soit un comportement indéfini \texttt{undef}.

\begin{figure}[!h]
	\coqcode{code/monad.v}
	\caption{Définition de la monade d'état avec ses fonctions \texttt{bind} et \texttt{ret}}
	\label{code:monad}
\end{figure}

Ces types sont utilisés pour définir la monade d'état \texttt{LLI}, présentée en figure \ref{code:monad}. La monade \texttt{LLI} est définie comme une fonction sur des types, prenant un type \texttt{state} en argument, et renvoyant une valeur de retour de type \texttt{result} composé d'un type arbitraire \texttt{A} et d'un \texttt{state}. Les fonctions partielles sur l'état renvoyant un type \texttt{LLI} sont donc \emph{monadiques}. La monade \texttt{LLI} << enrobe >> des valeurs quelconques avec le type \texttt{state}.

Lorsqu'on lui donne une fonction monadique \texttt{f} de \texttt{A} vers \texttt{LLI B} et une valeur monadique de type \texttt{LLI A}, la fonction \texttt{bind} permet de récupérer la valeur << enrobée >> \texttt{A} et de lui appliquer la fonction \texttt{f}, produisant une valeur monadique \texttt{LLI B}. La fonction \texttt{ret} sert d'élément neutre, qui à partir d'un état et d'une valeur, renvoie cette même valeur << enrobée >> avec l'état.

Peut être devrais-je donner une définition plus formelle d'une monade qui pourrait s'avérer beaucoup plus parlante. La définition repose sur la théorie des catégories et est attribuée à Saunders Mac Lane \cite[134]{mac2013categories} que l'on pourrait traduire de la sorte :

\blockquote{Une monade dans $X$ est juste un monoïde dans la catégorie des endofoncteurs de $X$, ayant pour produit $\times$ la composition d'endofoncteurs et pour élément neutre l'endofoncteur identité.}

			\subsubsection{Compilation et préservation de la sémantique}
			\label{compilation}
				Une problématique subsiste lorsque des propriétés formelles ont été prouvées sur du code source. Comment peut-on s'assurer que ces propriétés resteront prouvables une fois que le code source aura été compilé ?

				\paragraph{CompCert} La recherche dans ce domaine a mené au projet CompCert\cite{Leroy-backend}, un compilateur de code source C dont la préservation de la sémantique est garantie formellement jusqu'au code assembleur.

	La propriété de préservation de la sémantique est enoncée comme suit :

\begin{theorem}	
	Pour tout programme source $S$ et pour tout code $C$ généré par le compilateur, si le compilateur a produit le code $C$ à partir du programme source $S$ sans remonter d'erreur de compilation, alors le comportement observable de $C$ est l'un des comportements observables possibles de $S$.
\end{theorem}

CompCert compile le programme source $S$ à partir de son \emph{Abstract Syntax Tree} (ou \emph{AST}) issu de la passe du préprocesseur, de l'analyseur syntaxique et de la phase de vérification des erreurs de type (\emph{type checking}). CompCert produit un code assembleur $C$ sous la forme d'un \emph{AST} du langage assembleur ciblé.

À ce jour, le projet CompCert a prouvé 90\% de la chaîne de compilation, notamment les algorithmes d'optimisation et de génération de code assembleur. Les 10\% restants à prouver incluent l'assemblage et la phase d'édition des liens \cite{compcert_online}.

Les projets ayant recourt à la vérification de programme par \emph{deep embedding} ont l'avantage de pouvoir aisément reconstruire l'\emph{AST} du programme et de le compiler grâce à CompCert. Cependant, pour les projets comme Pip utilisant un \emph{shallow embedding} et n'explicitant pas l'\emph{AST} du programme vérifié, l'étape de reconstruction du programme n'est pas triviale.

Auparavant, la chaîne de compilation de Pip utilisait \emph{Digger}~\cite{digger}, un outil écrit en Haskell qui transforme le code C shallow embedded écrit en Gallina en code C. Digger n'est pas muni de preuve formelle de préservation de la sémantique. Ainsi, l'argument de la préservation de la sémantique de Pip résidait dans l'extrême simplicité apparente de la conversion.

		\paragraph{$\partial x$} $\partial x$ est le remplaçant de \emph{Digger} dans la chaîne de compilation de Pip. Comme \emph{Digger}, il permet de transformer le code C écrit en Gallina vers un code source C, sous une forme attendue par CompCert.

		$\partial x$ comporte deux phases de fonctionnement : la première phase consiste à extraire l'\emph{AST} du programme écrit en Gallina sous la forme d'une représentation intermédiaire ; la seconde transforme cet \emph{AST} en \emph{AST} CompCert \texttt{Csyntax}. La première phase est écrite en Elpi~\cite{elpi} car Coq ne propose pas nativement de mécanisme de \emph{réflexion}. La seconde phase est écrite directement dans Coq. Il est de ce fait désormais possible de raisonner sur la phase de transformation de l'\emph{AST}.~~$\partial x$ permet en outre d'afficher le code source C produit, en utilisant le \emph{pretty-printer} de CompCert.

		\subsection{Illustration système}	

			\subsubsection{seL4}
	seL4 \cite{sel4website} est un noyau de système d'exploitation de la famille des noyaux L4 pour de nombreuses architectures (Armv6, Armv7, Armv8, x86, x86\_64 et RISC-V RV64) \cite{sel4hardware}. seL4 propose des mécanismes de gestion de la mémoire virtuelle, de gestion des interruptions, de communication inter-processus (\emph{IPC}) qui reposent sur un système gestion des droits par \emph{capacités}~\cite{capabilities}.

	seL4 offre en outre, pour certaines plateformes, une vérification formelle de son implémentation. Cette vérification peut inclure :
	\begin{itemize}
		\item{une preuve formelle fonctionnelle du code C (c'est à dire une preuve que le code respecte sa spécification) \cite{sel4}}
		\item{une preuve formelle de la propagation de la preuve jusqu'au binaire exécutable\cite{sel4binary}}
		\item{une preuve du maintient de l'\emph{intégrité} et de la \emph{confidentialité} \cite{sel4integrity}}
		\item{des propriétés temps-réel notamment sur le respect de bornes sur le temps d'exécution (\emph{WCET})\cite{sel4wcet}}
	\end{itemize}

	Pour arriver au code C exécutable, seL4 définit tout d'abord une spécification abstraite, définissant la fonction du code à produire. Ensuite, un prototype en Haskell est implémenté, respectant à priori cette spécification. Ce prototype permet de générer automatiquement une spécification exécutable, définissant comment le code doit remplir sa fonction. L'implémentation réelle en C est réalisée manuellement et doit respecter la spécification exécutable.

	La preuve fonctionnelle de code dans le noyau seL4 est donc naturellement découpée en un raffinement en trois couches : la spécification abstraite, la spécification exécutable et l'implémentation réelle. La première étape pour établir une preuve fonctionnelle est de montrer que la spécification exécutable dérivée du prototype Haskell raffine la spécification abstraite. La seconde étape est de montrer que l'implémentation en C raffine la spécification exécutable \cite{sel4}.

			\subsubsection{CertiKOS}

	CertiKOS \cite{certikoswebsite} est un outil d'aide à la conception de noyau de système d'exploitation. CertiKOS propose une conception du noyau par de multiples couches d'abstractions propices au raffinement.

	Plus particulièrement, CertiKOS utilise le concept de \emph{raffinement contextuel}. En quelques mots, la méthodologie de CertiKOS consiste à définir des triplets $(L_1, M, L_2)$ représentant chacun une couche d'abstraction définissant une interface. $L_2$ représente l'interface que l'on souhaite certifier. $M$ représente l'implémentation de l'interface $L_2$ s'appuyant sur l'interface sous-jacente $L_1$. L'idée de la méthologie de CertiKOS est que chaque couche d'abstraction $L_2$ est assez précise pour capturer tous les comportements observables de l'implémentation $M$. Ce type de spécification est appelé \emph{spécification profonde}. Ainsi, une fois que la preuve que $M$ implémente l'interface $L_2$ a été établie, il est possible de raisonner exclusivement sur $L_2$ sans jamais avoir à raisonner sur $M$ de nouveau \cite{gu2015deep}.

	CertiKOS a certifié mCertiKOS, un noyau de système d'exploitation et hyperviseur, capable de faire tourner Linux et divisé en 40 couches d'abstraction \cite{gu2011certikos}. Plus récemment, mC2, un noyau de système d'exploitation concurrent \cite{concurrentcertikos, gu2016certikos} a été présenté à la communauté.

			\subsubsection{Pip}

	Pip est un noyau de système d'exploitation \emph{minimal} dont le seul but est la gestion de portions isolées de mémoire appelées \emph{partitions}, et du transfert de flôt d'exécution entre ces partitions. Pip est muni d'une preuve formelle garantissant que ses appels systèmes ne brisent pas \emph{l'isolation} des partitions. Pip a été initialement conçu pour fonctionner avec de la mémoire virtuelle en manipulant une \emph{MMU}. Cependant, de récents travaux ont fait évoluer Pip pour qu'il supporte la mémoire physique par le biais d'une \emph{MPU}.

	Dans Pip, les partitions forment une structure arborescente qui déterminent les droits d'accès à la mémoire. Au démarrage du système, une partition responsable de la totalité de la mémoire du système est créée. Pip permet à chaque partition de créer une autre partition en partageant une partie de sa propre mémoire avec la partition nouvellement créée. La nouvelle partition est appelée \emph{partition enfant}, la partition ayant partagé sa propre mémoire est appelée \emph{partition parent}. Cette relation parent/enfant crée la structure arborescente : les enfants pouvant créer à leur tour de nouvelles partitions en partageant leur mémoire.

	\paragraph{Propriété d'isolation} La propriété d'isolation de Pip est divisée en trois sous-propriétés :
	\begin{itemize}
		\item La première, la propriété de \textbf{partage vertical}, stipule que la mémoire partagée par une partition parent avec ses enfants reste accessible au parent par conception.

		\item La seconde, la propriété d'\textbf{isolation horizontale}. Dans Pip, chaque partition peut créer plusieurs partitions enfant ; cependant les portions de mémoire partagées avec chacune doit être strictement disjointe. Autrement dit, une partition ne peut pas partager une même portion de mémoire avec deux partitions enfant simultanément. Ainsi, deux partitions enfant issues d'une même partition parent sont \emph{isolées} : la mémoire accessible dans l'une d'entre elle est nécessairement inaccessible dans l'autre.

		\item La dernière, la propriété d'\textbf{isolation noyau}. À chaque création de partition, Pip réserve une petite portion de mémoire afin d'y stocker les structures nécessaires au contrôle des droits. Ces portions de mémoire deviennent inaccessibles à n'importe quelle partition.
	\end{itemize}


	\paragraph{Méthodologie de preuve} Pip a pris le contrepied des projets majeurs du domaine en utilisant un \emph{shallow embedding} de C plutôt que d'utiliser un \emph{deep embedding}. Ce \emph{shallow embedding} nécessite l'utilisation d'une monade d'état en Coq (voir \ref{monad}) et d'un outil spécifique (voir $\partial x$ \ref{compilation}) pour reconstruire l'AST ou le code source du noyau pour le compiler.
	Le pari de cette méthodologie est de pouvoir se concentrer sur la sémantique des programmes à prouver afin d'alléger l'effort de preuve général par rapport aux méthodologies utilisant un \emph{deep embedding}. Aussi, les proriétés de préservation de l'isolation de Pip ont été prouvées directement, plutôt qu'en passant par un raffinement.


    \chapter{Service de transfert de flot d'exécution avec preuve d'isolation}

Ce chapitre décrit la première contribution de cette thèse : un service de transfert de flôt d'exécution pour Pip. Ce chapitre commencera par exposer les motivations qui ont conduit à ce service de transfert de flôt d'exécution.

La seconde section décrira le service tel qu'il a été conçu : en premier lieu, nous exposerons le principe général derrière le service, en explicitant notamment les structures de données et le prototype du service. Cette exposition du service sera suivie d'une illustration de l'utilisation du service sur les trois différents transferts de flot d'exécution au sein d'un système : les appels systèmes entre différents espaces d'adressages, ainsi que les transferts de flot d'exécution suite à une faute ou une interruption. Cette section s'achèvera sur une vue interne du service, décrivant les différents blocs unifiant ces trois différents transferts.

La troisième section expliquera le processus de preuve du service, en commençant par la définition des types nécessaire à l'écriture du service et plus généralement de la conception des ajouts à l'interface avec la monade. Cette section détaillera ensuite les différentes propriétés d'isolation, puis identifiera les points délicats de l'établissement de la preuve en s'appuyant sur les différents blocs détaillés dans la section précédente.

La dernière section de ce chapitre reviendra sur la conception de ce service d'un point de vue pragmatique, en s'intéressant à quelques métriques et en revenant sur la pertinence de la preuve.

% Réecrire le modele de writeContext qui devrait écrire dans le modèle si la page donnée est une page noyau

	\section{Motivations}

		{\Huge \textcolor{red}{OSCOUR}}

		Point de vue pragmatique :
		\begin{itemize}
			\item anciennement deux appels systèmes \texttt{dispatch} et \texttt{resume} disponibles, écrit en C sans documentation, qui ne couvraient pas la totalité des cas d'usage.
			  donc nécessité de (re-)conception d'un mécanisme de transfert de flot d'exécution car le changement d'espace d'adressage est une opération privilégiée.
		\end{itemize}

		Point de vue académique :
		\begin{itemize}
			\item compléter la preuve des appels système de Pip pour surenchérir sur la validité de la méthologie de Pip
			\item valeur intrinsèque de l'unification des diférents transferts de flot de controle
		\end{itemize}
		\subsection{Failles de sécurité}
		\subsection{Changement d'espace d'adressage opération privilégiée}
		\subsection{Arguments de co-design (minimaliste, générique)}
			

	\section{Description du service}

	Avant toute chose, il faut définir ce qu'on attend du service, sa spécification. En particulier, il s'agit de spécifier les transferts de flôt de contrôle valides au sein du système, que ce soit pour les transferts explicites tels que les appels systèmes ou les transferts implicites comme les fautes ou les interruptions. Cette spécification doit pouvoir accomoder tous les cas d'usage de transfert de flôt d'exécution au sein d'un noyau tel que Pip, conçu comme une tour de virtualisation.
	
	Pour rappel, Pip définit des partitions de mémoire qui sont responsables de la mémoire qui leur est attribuée. Chaque partition de mémoire a son propre espace d'adressage. Ces partitions peuvent engendrer des sous-partitions, en partageant en partie de leur propre mémoire. Les sous-partitions engendrées de cette manière sont appelées les partitions enfants. La partition ayant partagé sa mémoire avec son enfant est appelée la partition parent. Au démarrage du système, une seule partition est créée par Pip. Cette partition a accès à l'intégralité de la mémoire : c'est la partition racine.

	\subsubsection{Flôts d'exécution valides au sein d'une tour de virtualisation}

	Les transferts de flot de contrôle valides entre les différentes partitions reprennent les trois modalités présentées dans le chapitre précédent en section \ref{control_flow_transfer} en les voyant à travers le prisme d'une tour de virtualisation.

	La tour de virtualisation crée un système de délégation des fonctionnalités. L'intégralité des fonctions du système est initialement endossé par la partition racine, qui peut décider de déléguer certaines fonctionnalités à ses enfants. Les partitions enfants peuvent à leur tour déléguer ces fonctionnalités à leurs propres enfants ; la partition racine n'en a cependant pas forcément connaissance. C'est pourquoi les transferts de flôt d'exécution explicites ne sont nécessaires qu'entre parent et enfants ; chaque partition connait les fonctionnalités qui lui incombent, et peut donc diriger le flot d'exécution vers une autre partition si nécessaire. \textbf{Ainsi, chaque partition offre un certain nombre de services qui définissent son interface.}

	Lorsqu'une faute survient, une partition manque à ses responsabilités. La faute remonte la chaîne de responsabilité vers son parent qui peut alors gérer l'incident.

	Les interruptions matérielles signalent un évènement extérieur dont la responsabilité peut incomber à n'importe quelle partition, et seule la partition racine connait l'ensemble des chaînes de responsabilité. Ainsi, lorsqu'une interruption matérielle survient, la partition racine récupère le flôt d'exécution et peut -- si nécessaire -- diriger l'interruption vers la partition qui en a la responsabilité. Ceci est semblable à un superviseur muni d'une fonction de multiplexage.

Ainsi, même s'il existe un grand nombre de modalités de transfert de flôt d'exécution en pratique, l'architecture du proto-noyau Pip promeut un modèle unifié qui adapte à une tour de virtualisation les trois situations génériques. En réduisant à trois cas distincts l'ensemble des modalités de transferts de flot d'exécution, le travail de preuve de programme nécessaire pour établir une garantie de sécurité est simplifié. Cependant, la sous-section suivante s'attache à démontrer qu'il est possible de résumer ces trois cas distincts en un seul service dont la preuve de bon fonctionnement apporte les garanties de sécurité à l'ensemble des situations de transfert de flot d'exécution possibles au sein de l'architecture x86.

	\subsection{Principe de fonctionnement du service} 
	\label{service_idea}

	\subsubsection{Structures de données du service}

	\paragraph{VIDT} Les services exposés par les différentes partitions sont définis dans une structure appelée \emph{Virtual Interrupt Descriptor Table} ou \emph{VIDT}. Cette structure reprend les concepts de l'\emph{IDT} classique (voir \ref{IDT}), appliqués à chaque partition. Elle doit être placée - par convention - au début de la dernière page virtuelle de chaque partition. Cependant, contrairement à l'\emph{IDT} qui contient des \emph{gates} composées de pointeurs de fonctions et de contrôles de droits, la \emph{VIDT} de chaque partition contient des pointeurs vers des \emph{contextes} d'exécution, comme illustré sur la figure \ref{fig:vidt}.

\begin{figure}[!ht]
	\centering
	\begin{tikzpicture}
\node[minimum width=3cm, minimum height=3cm] (vidt) at (0,0) {};
\node[above=0.1cm of vidt] {Partition VIDT};
\node[above left=0.1cm of vidt] {No.};
%\draw[dashed] (3.5, -1.5) -- (3.5, 1.5);
%\draw[dashed] (6.5, -1.5) -- (6.5, 1.5);
\node[draw, semithick, minimum width=3cm, minimum height=0.5cm] (ctx_ptr1) at (0,1.30) {context pointer};
\node[left=0.2cm of ctx_ptr1] {\texttt{0}};
\node[draw, semithick, minimum width=3cm, minimum height=0.5cm] (ctx_ptr2) at (0,0.73) {context pointer};
\node[left=0.2cm of ctx_ptr2] {\texttt{1}};
\node[minimum width=3cm, minimum height=0.5cm] (dots) at (0,-0.2) {\vdots};
\node[left=0.2cm of dots] {\vdots};
\node[draw, minimum width=3cm, minimum height=1cm, pattern=south west lines] (ctx) at (5, 0.5) {};
\node[draw, semithick, minimum width=3cm, minimum height=0.5cm] (ctx_ptr3) at (0,-1.36) {context pointer};
\node[left=0.1cm of ctx_ptr3] {\texttt{255}};
\node[below=0.02cm of ctx] {Context};
%\node[above=0.6cm of ctx] {Partition memory};
\node[draw, color=black!25, pattern color=black!25, minimum width=3cm, minimum height=1cm, pattern=south west lines] (ctx2) at (5, 2) {};
\node[draw, color=black!25, pattern color=black!25, minimum width=3cm, minimum height=1cm, pattern=south west lines] (ctx3) at (5, -1.5) {};
%%%%%%%%%%%%%%%%%%%%%%%%%%%%%%%%%%%%%%%%%%%%%%%%%%%%%%%%%%%%%%%
\draw[dashed] (ctx_ptr2.south west) -- (ctx_ptr3.north west);
\draw[dashed] (ctx_ptr2.south east) -- (ctx_ptr3.north east);
\draw[->, dotted] (ctx_ptr1.east) -- (ctx2.west);
\draw[->] (ctx_ptr2.east) -- (ctx.west);
\draw[->, dotted] (ctx_ptr3.east) -- (ctx3.west);
\end{tikzpicture}

	\caption{La structure d'une VIDT}
	\label{fig:vidt}
\end{figure}

	\paragraph{Contexte d'exécution} Ces \emph{contextes} sont des instantanés de l'état du processeur au moment du transfert du flot d'exécution. Pour l'architecture Intel x86, ils sont partiellement générés par les mécanismes du matériel tels que détaillé dans le chapitre précédent dans la section \ref{context}, puis complétés par du logiciel. Ces contextes d'exécution peuvent aussi être créés ex-nihilo par les partitions afin de définir de nouveaux services.


Plus simplement, les services de chaque partition sont définis dans leur \emph{VIDT} au moyen de pointeurs vers des contextes d'exécution. Il est important de noter que ces contextes d'exécution sont situés dans l'espace d'adressage de chaque partition, et sont donc \textbf{accessibles et modifiables} par le code non privilégié.

	\subsubsection{Principe d'utilisation et prototype du service pour un transfert de flôt d'exécution}
	\label{sec:service_usage}
	\begin{listing}[!ht]
		\ccode{code/entrypoint_prototype.c}
		\caption{Prototype du point d'entrée du service tel qu'appelée par les partitions}
		\label{code:c_proto}
	\end{listing}

	Lors d'un appel explicite au service de transfert de flôt d'exécution dont le protoype est donné par le listing \ref{code:c_proto}, la partition appelante doit désigner une autre partition ainsi que le numéro de service désiré. La partition est désignée l'adresse virtuelle de son descripteur correspondant au paramètre \texttt{calleePartDescVAddr}. Dans le cas d'un appel vers la partition parent, l'adresse par défaut est utilisée. Le numéro de service n'est autre que la position du pointeur vers le contexte d'exécution à restaurer dans la VIDT de la partition ciblée, correspondant au paramètre \texttt{userTargetInterrupt}. Ces deux paramètres permettent de déterminer où transférer le flôt d'exécution.

	De plus, Pip permet à la partition appelante de sauvegarder son contexte d'exécution actuel afin qu'il puisse être restauré et que l'exécution puisse reprendre ultérieurement. La partition appelante doit avoir réservé préalablement de la mémoire pour que Pip puisse y placer un contexte, et renseigné un pointeur vers cet espace dans sa propre VIDT. Pour que Pip préserve le contexte d'exécution, la partition doit fournir l'entier \texttt{userContextSaveIndex} qui indique la position du pointeur dans sa VIDT pointant vers l'espace réservé. Si un pointeur nul se trouve à la position indiquée, le contexte n'est pas sauvegardé.

	Les deux derniers paramètres, \texttt{flagsOnYield} et \texttt{flagsOnWake} permettent à la partition de restreindre l'utilisation de certains de ses services. Ce sont en réalité des drapeaux vérifiés par le service de transfert de flôt d'exécution de Pip indiquant que certains services de la partition sont temporairement indisponibles, bien qu'ils soient correctement configurés. \texttt{flagsOnYield} sont les drapeaux qui seront appliqués immédiatement par Pip à la partition appelante au moment du transfert de flôt d'exécution. \texttt{flagsOnWake} sont les drapeaux qui seront appliqués au moment de la restauration du contexte d'exécution actuel de la partition.

	Enfin, un dernier paramètre contenant un pointeur vers le contexte d'exécution est généré par le code trampoline permettant d'exécuter le code du service écrit en Gallina. Ceci permet au service de sauvegarder le contexte d'exécution comme énoncé précédemment. La figure \ref{code:gallina_proto} montre le prototype attendu par le code prouvé.

		\begin{listing}[!ht]
			\coqcode{code/prototype.v}
			\caption{Prototype du point d'entrée du service en Gallina}
			\label{code:gallina_proto}
		\end{listing}
		
		\subsection{Illustration de mise en place du service sur l'architecture Intel x86 au travers d'un appel explicite}

			\subsubsection{Point d'entrée par \texttt{callgate}}

		Dans l'implémentation de Pip sur l'architecture Intel x86, les services de Pip sont appelables au travers de \emph{callgates} (voir \ref{sec:x86_syscall}). Ces callgates permettent au code non privilégié des partitions d'appeler les services privilégiés de Pip.
		Pour ce faire, la partition doit pousser les arguments décrits en section \ref{sec:service_usage} sur sa pile, puis utiliser un \emph{farcall}. Cet appel est implémenté au sein de la LibPip, la librairie utilisateur facilitant l'utilisation du noyau. Son code est disponible en annexe (voir listing \ref{code:libpip_yield}).

		Lorsque le processeur exécute l'instruction \texttt{lcall} de la partition, le flôt d'exécution est transféré vers Pip et le processeur passe en mode privilégié, copie les paramètres et pousse partiellement l'état précédent sur la pile (voir \ref{sec:intel_callgate}). Dans le cas de l'appel au service de transfert de flôt d'exécution, le processeur commence par exécuter une routine qui va sauver sur la pile noyau le contexte d'exécution de la partition encore partiellement présent dans les registres. Accessoirement, cette routine réordonne les éléments de la pile afin de regrouper les différentes parties du contexte et de pouvoir utiliser une structure \texttt{gate\_ctx\_t} pour le représenter. Cette routine étant un peu longue, elle est placée en annexe (voir Fragment de code \ref{code:cg_yieldGlue}).
		La figure \ref{fig:cg_stack} montre l'état de la pile après l'exécution de la routine.

		\begin{figure}[!ht]
			\centering
\begin{tikzpicture}
	\node[draw, semithick, minimum width=4cm, minimum height=0.6cm] (ss)         at (0, 3.3) {\texttt{SS}};
	\node[draw, semithick, minimum width=4cm, minimum height=0.6cm] (esp)        at (0, 2.7) {\texttt{ESP}};
	\node[draw, semithick, minimum width=4cm, minimum height=0.6cm] (eflags)     at (0, 2.1) {\texttt{EFLAGS}};
	\node[draw, semithick, minimum width=4cm, minimum height=0.6cm] (cs)         at (0, 1.5) {\texttt{CS}};
	\node[draw, semithick, minimum width=4cm, minimum height=0.6cm] (eip)        at (0, 0.9) {\texttt{EIP}};
	\node[draw, semithick, minimum width=4cm, minimum height=1.8cm, align=center] (greg)       at (0,-0.3) {Registres généraux\\\texttt{(8 dwords)}};
	\node[draw, semithick, minimum width=4cm, minimum height=0.6cm] (argn)       at (0,-1.5) {Argument $5$};
	\node[draw, semithick, minimum width=4cm, minimum height=0.6cm] (argx)       at (0,-2.1) {$\vdots$};
	\node[draw, semithick, minimum width=4cm, minimum height=0.6cm] (arg1)       at (0,-2.7) {Argument $1$};
	\node[draw, semithick, minimum width=4cm, minimum height=0.6cm] (ctx_ptr)    at (0,-3.3) {\texttt{gate\_ctx\_t *}};

	\node[minimum height=0.6cm]                                                  at (0,-4.1) {$\downarrow$};

	\draw[semithick] (-2, -3.6) -- (-2,-4.5);
	\draw[semithick] ( 2, -3.6) -- ( 2,-4.5);

	\node[left=0.5cm of eflags, minimum width=3cm] (iret_ctx_t) {\texttt{iret\_ctx\_t}};
	\draw[dashed, thin] (ss.north west) -- (iret_ctx_t.north |- ss.north west) -- (iret_ctx_t.north);
	\draw[dashed, thin] (eip.south west) -- (iret_ctx_t.south |- eip.south west) -- (iret_ctx_t.south);

	\node[left=0.5cm of greg,   minimum width=3cm] (pushad_regs_t) {\texttt{pushad\_regs\_t}};
	\draw[dashed, thin] (pushad_regs_t.north) -- (pushad_regs_t.north |- eip.south west);
	\draw[dashed, thin] (pushad_regs_t.south) -- (pushad_regs_t.south |- greg.south west) -- (greg.south west);

	\node[right=0.5cm of eip.north east, minimum width=3cm] (gate_ctx_t) {\texttt{gate\_ctx\_t}};
	\draw[dashed, thin] (ss.north east) -- (gate_ctx_t.north |- ss.north east) -- (gate_ctx_t.north);
	\draw[dashed, thin] (greg.south east) -- (gate_ctx_t.south |- greg.south east) -- (gate_ctx_t.south);

	\draw[dashed, thin] (ctx_ptr.east) -- (gate_ctx_t.320 |- ctx_ptr.east);
	\draw[dashed, thin, ->] (gate_ctx_t.320 |- ctx_ptr.east) -- (gate_ctx_t.320);

\end{tikzpicture}

			\caption{État de la pile du noyau après la routine assembleur exécutée après l'appel du service au travers d'une callgate}
			\label{fig:cg_stack}
		\end{figure}
	
		\paragraph{Harmonisation du contexte d'exécution et appel du service prouvé} \label{sec:context_harmonisation} Avant d'appeler le code prouvé, une dernière transformation est opérée sur le contexte d'exécution. Il est copié en haut de la pile, puis transformé en contexte générique de type \texttt{user\_ctx\_t} afin d'harmoniser les différentes représentations de contexte entre les différents points d'entrées du service. Le code est disponible en annexe (voir Fragment de code \ref{code:yieldGlue}).

			\subsubsection{Introduction générale des étapes du service}

			Le service écrit en Gallina permettant de transférer le flôt d'exécution procède en trois étapes.

			\paragraph{Étape préliminaire de validation et de récupération des données} Avant toute chose, la première étape du service vérifie la validité des arguments et des structures modifiables en espace utilisateur. En particulier, elle vérifie que l'adresse virtuelle fournie comme la cible du transfert correspond bien à une partition enfant ou parent, et récupère l'adresse réelle à laquelle débute son descripteur. Elle vérifie aussi que les \emph{VIDT} des partitions appelantes et appelées sont accessibles en espace utilisateur, et que les espaces de mémoires ciblés par l'appel sont eux aussi accessibles. Ceci permet par exemple de récupérer le contexte d'exécution de la partition ciblée par l'appel. Cette étape préliminaire permet de s'assurer que le service ne pourra pas rencontrer d'erreur dans les prochaines étapes.

			\paragraph{Étape de modification de l'état} La seconde partie du service est une étape procédant à la modification de l'état du système. Cette étape regroupe toutes les écritures en mémoire requises par le service. Tout d'abord, le service va procéder à l'écriture du contexte de la partition appelante dans son espace d'adressage (si demandé lors de l'appel). Le contexte est recopié depuis la pile noyau jusqu'à la zone de mémoire pointée par le pointeur dans la \emph{VIDT} de la partition appelante.
			Ensuite, le service met à jour l'espace d'adressage pour refléter l'espace d'adressage de la partition cible.
			Enfin, le service procède à la mise à jour des structures de données du noyau afin de préserver les propriétés de cohérence internes de Pip. Cette étape ne nécessite en fait qu'une unique écriture dans une variable globale, indiquant quelle partition s'exécutera lorsque le flôt d'exécution repassera en espace utilisateur.

			\paragraph{Étape de transfert de flôt d'exécution} La troisième et dernière étape du service transfère le flôt d'exécution vers la partition appelée au travers du contexte d'exécution récupéré lors de l'étape préliminaire. Cette étape est représentée dans le code prouvé par un appel à une fonction de l'interface.

		\subsection{Décomposition des opérations}

		Afin de pouvoir accomoder les différents modes de transfert de flôt d'exécution au sein d'un même service, le service est composé de plusieurs blocs de code remplissant leur propre fonction et appelant le bloc de code suivant. Ces blocs pourraient être assimilés à des \emph{continuations}. Les blocs composants le service sont illustrés à la fin de la sous-section sur la figure \ref{fig:callgraph}.

		\paragraph{Vérification du numéro de contexte ciblé}

		Le bloc de code \texttt{checkIntLevelCont} est le point d'entrée dans le service lors d'un appel explicite, tel que présenté dans le code \ref{code:gallina_proto}.
		Il se contente de vérifier que le numéro de contexte ciblé soit bien valide et transforme son type pour qu'il soit utilisable par le noyau. Il appelle ensuite le bloc de code \texttt{checkCtxSaveIdxCont}.

		%\begin{listing}[!ht]
		%	\coqcode{code/checkIntLevelCont.v}
		%	\caption{Prototype du point d'entrée du service en Gallina}
		%	\label{code:checkIntLevelCont}
		%\end{listing}

		\paragraph{Vérification du numéro de sauvegarde du contexte de la partition appelante}
		
		Le bloc de code \texttt{checkCtxSaveIdxCont} vérifie que le numéro de sauvegarde du contexte de la partition appelante est valide, et transforme son type pour qu'il soit utilisable par le noyau. Il récupère ensuite le descripteur de partition de la partition appelante, ainsi que son Page Directory (la page mémoire racine de la configuration de son espace d'adressage -- voir \ref{sec:intel_mmu}).

		Enfin, ce bloc de code va vérifier si la valeur de l'adresse virtuelle désignant la partition appelée est celle par défaut. Si c'est le cas, le prochain bloc de code sera \texttt{getParentPartDescCont}. Si ce n'est pas la valeur par défaut, il va appeler le bloc de code \texttt{getChildPartDescCont}.

		%\begin{listing}[!ht]
		%	\coqcode{code/checkCtxSaveIdxCont.v}
		%	\caption{Prototype du point d'entrée du service en Gallina}
		%	\label{code:checkCtxSaveIdxCont}
		%\end{listing}

		\paragraph{Récupération de la partition parent}

		Le bloc de code \texttt{getParentPartDescCont} est un des deux blocs d'exécution possibles concernant la cible du transfert de flôt d'exécution qui traite de l'appel vers le parent. Il vérifie uniquement que la partition appelante n'est pas la partition racine, puisque la partition racine n'a pas de parent. Il récupère ensuite le descripteur de partition de son parent, qu'il passera en tant que descripteur de partition appelée au prochain bloc de code \texttt{getSourceVidtCont}. Ce prochain bloc de code est commun aux deux fils d'exécutions alternatifs.

		%\begin{listing}[!ht]
		%	\coqcode{code/getParentPartDescCont.v}
		%	\caption{Prototype du point d'entrée du service en Gallina}
		%	\label{code:getParentPartDescCont}
		%\end{listing}

		\paragraph{Récupération de la partition enfant}

		Le bloc de code \texttt{getChildPartDescCont} est l'autre alternative d'exécution du transfert de flôt d'exécution qui traite de l'appel vers un enfant. Il vérifie que l'adresse virtuelle passée comme paramètre correspond au descripteur de partition d'une partition enfant. Pour le vérifier, Pip vérifie tout d'abord que l'adresse virtuelle réside bien dans l'espace d'adressage de la partition appelante, et vérifie ensuite dans ses structures de données internes que l'adresse correspond à un descripteur de partition.

		Ensuite, de la même manière que le bloc de code concernant l'appel à un parent, ce bloc de code passe le descripteur de partition au prochain bloc de code \texttt{getSourceVidtCont}.

		%\begin{listing}[!ht]
		%	\coqcode{code/getChildPartDescCont.v}
		%	\caption{Prototype du point d'entrée du service en Gallina}
		%	\label{code:getChildPartDescCont}
		%\end{listing}

		\paragraph{Récupération de la VIDT de la partition appelante}
		\label{sec:getSourceVidtCont}

		Le bloc de code \texttt{getSourceVidtCont} vérifie d'abord que la VIDT de la partition appelante est mappée et accessible dans son espace d'adressage. Une fois qu'il a déterminé qu'il était possible de lire dans la VIDT, il va récupérer l'adresse virtuelle de l'espace mémoire où sauvegarder le contexte de la partition appelante. Enfin, la page de mémoire contenant la VIDT, et l'adresse virtuelle de l'espace mémoire de sauvegarde sont passés au bloc de code suivant \texttt{getTargetVidtCont}.

		%\begin{listing}[!ht]
		%	\coqcode{code/getSourceVidtCont.v}
		%	\caption{Prototype du point d'entrée du service en Gallina}
		%	\label{code:getSourceVidtCont}
		%\end{listing}

		\paragraph{Récupération de la VIDT de la partition ciblée}
		\label{sec:getTargetVidtCont}
		Tout d'abord, le bloc de code \texttt{getTargetVidtCont} va récupérer le \emph{Page Directory} de la partition appelée à partir de son descripteur de partition. Ensuite, similairement au bloc de code précédent, il vérifie que la VIDT de la partition appelée est mappée et accessible dans son espace d'adressage. Ensuite, la page de mémoire contenant la VIDT ainsi que son \emph{Page Directory} sont passés au bloc de code suivant \texttt{getTargetContextCont}.
		%\begin{listing}[!ht]
		%	\coqcode{code/getTargetVidtCont.v}
		%	\caption{Prototype du point d'entrée du service en Gallina}
		%	\label{code:getTargetVidtCont}
		%\end{listing}

		\paragraph{Récupération du contexte de la partition ciblée}

		Le bloc de code \texttt{getTargetContextCont} vérifie que le pointeur vers le contexte d'exécution ciblé pointe bien dans l'espace d'adressage de la partition ciblée, et que cet espace est bien accessible à la partition. En supplément, le bloc va vérifier que l'adresse de fin du contexte ne va pas overflow, et que la dernière adresse du contexte est aussi accessible à la partition afin de s'assurer que l'entièreté du contexte pourra être lu sans déclencher de faute. Ceci introduit l'hypothèse qu'un contexte d'exécution a une taille inférieure à une page mémoire.
		Enfin, le bloc de code va comparer l'adresse virtuelle de sauvegarde de contexte récupérée au bloc \texttt{getSourceVidtCont} (voir \ref{sec:getSourceVidtCont}) avec l'adresse virtuelle par défaut. Si cette adresse n'est pas celle par défaut, c'est que la partition veut sauvegarder son contexte , et le bloc de code \texttt{saveSourceContextCont} est appelé. Sinon, le bloc de code \texttt{switchContextCont} est appelé.
		%\begin{listing}[!ht]
		%	\coqcode{code/getTargetContextCont.v}
		%	\caption{Prototype du point d'entrée du service en Gallina}
		%	\label{code:getTargetContextCont}
		%\end{listing}

		\paragraph{Sauvegarde du contexte de la partition appelante}

		Le bloc de code \texttt{saveSourceContextCont} va se contenter de vérifier que l'espace mémoire pointé par l'adresse récupérée dans la VIDT de la partition appelante est bien dans l'espace d'adressage de la partition appelante et qu'il est accessible en espace utilisateur. Comme dans le bloc précédent, il va vérifier en supplément que l'adresse de fin de cet espace mémoire n'overflow pas, et qu'il reste accessible dans son espace d'adressage, afin de s'assurer qu'une écriture dans cette zone mémoire ne déclenchera pas de faute. Une fois les vérifications faites, il écrit le contexte de la partition appelante dans la zone mémoire, puis appelle le dernier bloc de code \texttt{switchContextCont}.
		%\begin{listing}[!ht]
		%	\coqcode{code/saveSourceContextCont.v}
		%	\caption{Prototype du point d'entrée du service en Gallina}
		%	\label{code:saveSourceContextCont}
		%\end{listing}

		\paragraph{Changement d'espace d'adressage et chargement du contexte d'exécution}

		Le bloc de code \texttt{switchContextCont} ne procède plus à aucune vérification. Il commence par écrire les drapeaux \texttt{flagsOnYield} dans la partition appelante. Ensuite, l'espace d'adressage est changé par celui de la partition appelée -- son \emph{Page Directory} ayant été récupéré dans le bloc de code \texttt{getTargetVidtCont} (voir \ref{sec:getTargetVidtCont}). Ensuite, il met à jour la variable de Pip indiquant la partition qui s'exécutera lorsque le processeur repassera en mode utilisateur. Le bloc de code récupère ensuite les drapeaux \texttt{flagsOnWake} du contexte d'exécution de la partition à réveiller et les applique à la partition courante, sur le nouvel espace d'adressage. À partir de ces drapeaux, il détermine si la partition appelée peut demander à ne pas être interrompue, puis procède enfin au chargement de son contexte.
		%\begin{listing}[!ht]
		%	\coqcode{code/switchContextCont.v}
		%	\caption{Prototype du point d'entrée du service en Gallina}
		%	\label{code:switchContextCont}
		%\end{listing}

		\newpage

		\begin{figure}[!ht]
			\begin{tikzpicture}[>=triangle 45,font=\sffamily, every text node part/.style={align=center}, scale=1, every node/.style={transform shape}] {
	\node[draw, fill=white, ultra thin, drop shadow, minimum width=3cm, minimum height=1cm] (int_no_checks) at (0,9.5) {\footnotesize{Vérification du n° contexte appelé}};
	\node[draw, fill=white, ultra thin, drop shadow, minimum width=3cm, minimum height=1cm] (ctx_sav_checks) at (0,8.45) {\footnotesize{Vérification du n° de sauvegarde du contexte}};
	\node[draw, fill=white, ultra thin, drop shadow, minimum width=3cm, minimum height=1cm] (child_checks) at (-3, 6.95) {\footnotesize{Récupération partition enfant}};
	\node[draw, fill=white, ultra thin, drop shadow, minimum width=3cm, minimum height=1cm] (parent_checks) at (3, 6.95) {\footnotesize{Récupération partition parent}};
	\node[draw, fill=white, ultra thin, drop shadow, minimum width=3cm, minimum height=1cm] (source_checks) at (0, 5.45) {\footnotesize{Récupération VIDT appelant}};
	\node[draw, fill=white, ultra thin, drop shadow, minimum width=3cm, minimum height=1cm] (target_checks) at (0, 4.40) {\footnotesize{Récupération VIDT appelé}};
	\node[draw, fill=white, ultra thin, drop shadow, minimum width=3cm, minimum height=1cm] (target_ctx) at (0, 3.35) {\footnotesize{Récupération contexte appelé}};
	\node[draw, fill=white, ultra thin, drop shadow, minimum width=3cm, minimum height=1.5cm] (save_ctx) at (-3, 1.60) {\footnotesize{Vérification de l'espace de sauvegarde du contexte}\\\footnotesize{Sauvegarde du contexte}};
	\node[draw, fill=white, ultra thin, drop shadow, minimum width=3cm, minimum height=2cm] (switch_ctx) at (2, -0.65) {\footnotesize{Changement d'espace d'adressage}\\\footnotesize{Mise à jour des structures internes de Pip}\\\footnotesize{Chargement du contexte appelé}};
	\draw[->] (ctx_sav_checks) to[in=90, out=200] (child_checks.20);\draw[->] (ctx_sav_checks) to[in=90, out=340] (parent_checks.160);
	\draw[->] (child_checks) to[in=90, out=0] (source_checks.140);\draw[->] (parent_checks) to[in=90, out=180] (source_checks.40);
	\draw[->] (target_ctx) to[in=90, out=180] (save_ctx);\draw[->] (target_ctx) to[in=90, out=340] (switch_ctx);
	\draw[->] (save_ctx) to[in=180, out=270] (switch_ctx);

	%\node[above right=0.2cm of parent_checks.60, fill=red!30, drop shadow={shadow xshift=0.05cm,shadow yshift=-0.05cm}] (fault) {\small{\texttt{Entrée fautes}}};
	\node[right=0.2cm of parent_checks.07, fill=red!30, drop shadow={shadow xshift=0.05cm,shadow yshift=-0.05cm}] (fault) {\small{\texttt{Entrée fautes}}};
	\node[above right=0.2cm of parent_checks.45, fill=red!30, drop shadow={shadow xshift=0.05cm,shadow yshift=-0.05cm}] {\small{\texttt{Entrée int. logicielle}}};
	\node[right=0.2cm of target_checks.07, fill=red!30, drop shadow={shadow xshift=0.05cm,shadow yshift=-0.05cm}]{\small{\texttt{Entrée double fautes}}};
	\node[above=0.2cm of int_no_checks.90, fill=red!30, drop shadow={shadow xshift=0.05cm,shadow yshift=-0.05cm}]{\small{\texttt{Entrée explicite}}};
	\node[left=0.2cm of source_checks.173, fill=red!30, drop shadow={shadow xshift=0.05cm,shadow yshift=-0.05cm}]{\small{\texttt{Entrée int. matérielle}}};
}

\end{tikzpicture}

			\caption{Vue éclatée des blocs constituant le service}
			\label{fig:callgraph}
		\end{figure}
		\newpage

		\subsection{Généralisation du service aux fautes et aux interruptions}
		\label{sec:service_generalisation}

		L'idée principale derrière cette unification est qu'il est possible pour Pip de fixer certains paramètres et de commencer l'exécution à un endroit arbitraire du service. Il est notamment possible de commencer l'exécution directement après la validation des paramètres et de la traduction de l'adresse virtuelle de la partition cible. Ceci permet au système de passer en arguments les adresses réelles de partitions qui seront la cible des différents événements. La figure \ref{fig:callgraph} montre comment sont placés les différents points d'entrée du système dans le service.

		Cependant, les différents mécanismes décrits dans la section \ref{sec:x86_syscall} ne présentent pas d'interface commune ; c'est pourquoi, de petites portions de code C et assembleur sont placées juste avant les différents points d'entrée afin d'harmoniser les différents formats de contexte. Ces morceaux d'assembleur sont placés dans l'\emph{IDT} du système, afin qu'ils soient appelés lors d'une faute ou d'une interruption. Deux niveaux d'interruption sont réservés à la sauvegarde des contextes par Pip lorsqu'une ifaute ou une interruption survient. Sur l'architecture \texttt{x86} ce sont les niveaux 48 et 49 qui sont utilisés (les niveaux d'interruptions 0 à 31 sont réservés par Intel pour les fautes, les niveaux d'interrruptions de 32 à 47 ont été configurés pour correspondre aux interruptions matérielles).

		\subsubsection{Implémentation des fautes utilisant le service sur l'architecture x86}

		Pour rappel, les fautes doivent être transmises au parent de la partition fautive. Il est donc naturel de commencer l'exécution du service au bloc de code \texttt{getParentPartDescCont}. Le prototype de ce bloc de code est présenté en listing \ref{code:getParentPartDescCont}.

		\begin{listing}[!ht]
			\coqcode{code/getParentPartDescCont.v}
			\caption{Prototype du point d'entrée du service en Gallina}
			\label{code:getParentPartDescCont}
		\end{listing}

		Cependant, l'appel de ce bloc de code n'est pas trivial après une faute. Il s'agit d'abord de récupérer le contexte d'exécution de la partition fautive. Comme expliqué précédemment dans la section \ref{sec:faults}, lorsqu'une faute survient, le processeur va chercher la \emph{gate} installée dans l'\emph{IDT} dont le numéro correspond au niveau de la faute. Dans Pip, ces \emph{gates} sont toutes des \emph{interrupt gates}, qui s'exécutent en mode privilégié.

		De ce fait, le processeur change de pile. Il pousse le segment de pile \texttt{SS} ainsi que l'ancien pointeur vers le sommet de la pile \texttt{ESP}. Il pousse ensuite l'état des drapeaux du processeur \texttt{EFLAGS}, puis pousse le segment de code \texttt{CS} et le pointeur d'instruction \texttt{EIP} au moment de la faute. Enfin, en fonction de la faute qui a été déclenchée, le processeur peut éventuellement pousser un entier précisant la cause de la faute. La figure \ref{fig:proc_interrupt_stack} illustre l'état de la pile noyau après qu'une faute soit survenue.

		\begin{figure}[!ht]
			\centering
\begin{tikzpicture}

	\node[draw, semithick, minimum width=4cm, minimum height=0.6cm] (ss)         at (0, 3.3) {\texttt{SS}};
	\node[draw, semithick, minimum width=4cm, minimum height=0.6cm] (esp)        at (0, 2.7) {\texttt{ESP}};
	\node[draw, semithick, minimum width=4cm, minimum height=0.6cm] (eflags)     at (0, 2.1) {\texttt{EFLAGS}};
	\node[draw, semithick, minimum width=4cm, minimum height=0.6cm] (cs)         at (0, 1.5) {\texttt{CS}};
	\node[draw, semithick, minimum width=4cm, minimum height=0.6cm] (eip)        at (0, 0.9) {\texttt{EIP}};
	\node[draw, semithick, minimum width=4cm, minimum height=0.6cm] (error_code) at (0, 0.3) {\texttt{(Error Code)}};
	\node[                                    minimum height=0.6cm]              at (0,-0.5) {$\downarrow$};

	\draw[semithick] (-2, 0) -- (-2, -0.9);
	\draw[semithick] ( 2, 0) -- ( 2, -0.9);
\end{tikzpicture}

			\caption{État de la pile noyau après qu'une faute soit survenue en espace utilisateur\\Reproduction partielle du manuel Intel \cite{intel_interrupt_stack}}
			\label{fig:proc_interrupt_stack}
		\end{figure}

		Pour unifier la structure des données sur la pile du noyau entre toutes les fautes et interruptions, le bout d'assembleur s'exécutant en sortie de faute va pousser une valeur de bourrage sur la pile si le processeur n'a pas poussé de code d'erreur. Il va pousser le niveau d'interruption de la faute sur la pile à des fins informatives. La routine assembleur va poursuivre en complétant le contexte d'exécution qui avait été partiellement sauvé par le processeur en poussant les registres généraux sur la pile. Ceci complète la structure \texttt{int\_ctx\_t} représentant le contexte d'exécution de la partition fautive. Enfin, le bout d'assembleur va pousser un pointeur vers cette structure. Cette routine assembleur s'achève en appelant le code C qui sera chargé de récupérer les arguments pour appeler le bloc de code \texttt{getParentPartDescCont}.

		\begin{figure}[!ht]
			\centering
\begin{tikzpicture}

	\node[draw, semithick, minimum width=4cm, minimum height=0.6cm] (ss)         at (0, 3.3) {\texttt{SS}};
	\node[draw, semithick, minimum width=4cm, minimum height=0.6cm] (esp)        at (0, 2.7) {\texttt{ESP}};
	\node[draw, semithick, minimum width=4cm, minimum height=0.6cm] (eflags)     at (0, 2.1) {\texttt{EFLAGS}};
	\node[draw, semithick, minimum width=4cm, minimum height=0.6cm] (cs)         at (0, 1.5) {\texttt{CS}};
	\node[draw, semithick, minimum width=4cm, minimum height=0.6cm] (eip)        at (0, 0.9) {\texttt{EIP}};
	\node[draw, semithick, minimum width=4cm, minimum height=0.6cm] (error_code) at (0, 0.3) {\texttt{(Error Code)}};
	\node[draw, semithick, minimum width=4cm, minimum height=0.6cm] (int_level)  at (0,-0.3) {\texttt{Interrupt Level}};
	\node[draw, semithick, minimum width=4cm, minimum height=1.8cm, align=center] (greg) at (0,-1.5) {Registres généraux\\\texttt{(8 dwords)}};
	\node[draw, semithick, minimum width=4cm, minimum height=0.6cm] (ctx_ptr)    at (0,-2.7) {\texttt{int\_ctx\_t *}};
	\node[                                    minimum height=0.6cm]              at (0,-3.5) {$\downarrow$};

	\draw[semithick] (-2, -3) -- (-2, -3.9);
	\draw[semithick] ( 2, -3) -- ( 2, -3.9);

	\node[left=0.5cm of eflags, minimum width=3cm] (iret_ctx_t) {\texttt{iret\_ctx\_t}};
	\draw[dashed, thin] (ss.north west) -- (iret_ctx_t.north |- ss.north west) -- (iret_ctx_t.north);
	\draw[dashed, thin] (eip.south west) -- (iret_ctx_t.south |- eip.south west) -- (iret_ctx_t.south);

	\node[left=0.5cm of greg,   minimum width=3cm] (pushad_regs_t) {\texttt{pushad\_regs\_t}};
	\draw[dashed, thin] (pushad_regs_t.north) -- (pushad_regs_t.north |- greg.north west) -- (greg.north west);
	\draw[dashed, thin] (pushad_regs_t.south) -- (pushad_regs_t.south |- greg.south west) -- (greg.south west);

	\node[right=0.5cm of eip.south east, minimum width=3cm] (int_ctx_t) {\texttt{int\_ctx\_t}};
	\draw[dashed, thin] (ss.north east) -- (int_ctx_t.north |- ss.north east) -- (int_ctx_t.north);
	\draw[dashed, thin] (greg.south east) -- (int_ctx_t.south |- greg.south east) -- (int_ctx_t.south);

	\draw[dashed, thin] (ctx_ptr.east) -- (int_ctx_t.320 |- ctx_ptr.east);
	\draw[dashed, thin, ->] (int_ctx_t.320 |- ctx_ptr.east) -- (int_ctx_t.320);
\end{tikzpicture}

			\caption{État de la pile noyau après qu'une faute soit survenue en espace utilisateur\\Reproduction partielle du manuel Intel \cite{intel_interrupt_stack}}
			\label{fig:interrupt_stack}
		\end{figure}

		Cette fonction s'appelle \texttt{faultInterruptHandler} ; son prototype est donné dans le listing \ref{code:faultInterruptHandler_proto}. Son code est trop long pour être inclus dans ce chapitre mais vous pouvez le retrouver en annexe \ref{code:faultInterruptHandler}.

		\begin{listing}[!ht]
			\ccode{code/faultInterruptHandler_proto.c}
			\caption{Prototype de la fonction calculant les arguments du service lors d'une faute}
			\label{code:faultInterruptHandler_proto}
		\end{listing}

		Tout comme l'appel du service par la \emph{callgate}, la fonction \texttt{faultInterruptHandler} commence par créer un nouveau contexte générique de type \texttt{user\_ctx\_t} à partir du contexte fautif précédemment créé. Ceci permet d'avoir la même représentation du contexte entre les différents points d'entrée du service, comme évoqué dans la section \ref{sec:context_harmonisation}. Cependant, le niveau de faute sera utilisé par la fonction comme argument \texttt{targetInterrupt}, pour que le parent soit reveillé avec le contexte lié à la faute.
		La fonction récupère ensuite le descripteur de partition fautive grâce à la variable globale de Pip indiquant la partition s'exécutant en espace utilisateur, qui deviendra l'argument \texttt{sourcePartDesc}. Elle récupère aussi l'état de ses drapeaux, liés aux arguments \texttt{flagsOnYield} et \texttt{flagsOnWake}. La fonction va décider où sauvegarder le contexte de la partition fautive en fonction de ces drapeaux. L'implémentation actuelle considère que si le mot mémoire représentant les drapeaux est égal à zéro, la partition ne souhaitait pas être interrompue (on parle de \emph{Virtual CLI}, en référence à l'instruction assembleur désactivant les interruptions). L'état sera alors sauvegardé dans l'espace mémoire pointé à l'index \texttt{CLI\_SAVE\_INDEX}. Si les drapeaux sont différents de zéro, alors l'autre indice réservé, \texttt{STI\_SAVE\_INDEX} sera utilisé pour l'argument \texttt{sourceContextSaveIndex}.
		Enfin, la fonction récupère le \emph{Page Directory} de la partition à partir de son descripteur qui servira d'argument \texttt{sourcePageDir}.

		Lorsque tous les arguments ont été récupérés\footnote{l'argument \texttt{nbL} de l'appel a été omis du texte principal car il est peu intéressant. Cet argument est en fait un paramètre de l'implémentation précisant le nombre d'indirections dans les structures de configurations de la MMU, soit 2 dans l'implémentation Intel x86}, la fonction est prête à appeler le bloc de code \texttt{getParentPartDescCont}.

		\paragraph{Cas des doubles fautes} Il est possible que l'appel au service échoue lors des vérifications pour le transfert de flôt d'exécution. Lorsque ce transfert est dû à une faute, la partition l'ayant déclenchée n'est plus en mesure de recevoir le flôt d'exécution : elle déclencherait à nouveau la faute. Ainsi, il faut que le service propose une voie de transfert de flôt d'exécution dont le succès ne dépend que de la bonne configuration de la partition ayant la responsabilité de la partition fautive -- c'est à dire son parent, la partition recevant le flot d'exécution.
		Dans le cas où le parent ne serait pas non plus en mesure de recevoir le flôt d'exécution, le flôt d'exécution serait redirigé vers le parent de la partition parent et remonterai la chaîne de responsabilité, jusqu'à ce qu'une partition reçoive le flôt d'exécution sans erreur, ou jusqu'à ce que la partition racine elle même échoue à le recevoir.

		C'est pourquoi la fonction récupérant les arguments appelle à son tour une autre fonction appelée \texttt{propagateFault} (cette fonction est disponible en annexe, voir listing \ref{code:faultInterruptHandler}). Cette fonction est chargée de faire l'appel au bloc de code \texttt{getParentPartDescCont} et de gérer les éventuelles erreurs pouvant survenir après un appel à ce bloc. Il y a trois cas d'erreurs distincs gérés par la fonction :
		\begin{itemize}
			\item le cas où la fonction ayant réalisé la faute est la partition racine : dans ce cas là, il n'y a plus rien à rattraper, la partition racine étant la base de confiance absolue du système -- si elle échoue le système s'écroule. Le service s'arrête alors sur une boucle infinie.
			\item le cas où le service n'a pas réussi à récupérer la VIDT de la partition fautive, ou l'espace mémoire permettant de sauvegarder le contexte : dans ce cas, la sauvegarde du contexte de la partition fautive est omise, et l'exécution reprend au bloc de code \texttt{getTargetVidtCont}.
			\item dans tous les autres cas d'erreur, la partition parent n'est pas correctement configurée et n'est pas en mesure de rattraper la faute. Dans ce cas, la fonction \texttt{propagateFault} fait un appel récursif. La faute est redirigée sur le parent de la cible actuelle, et le niveau d'interruption de la faute est changé au niveau correspondant à la double faute.
		\end{itemize}

		Les différentes étapes logicielles permettant d'utiliser le service pour une faute sont résumés dans la figure \ref{fig:fault_software}.

		\begin{figure}[!ht]
			\centering
			\begin{tikzpicture}[>=triangle 45,font=\sffamily, every text node part/.style={align=center}, scale=1, every node/.style={transform shape}] {

	\node[draw, fill=white, ultra thin, drop shadow, minimum height=1cm] (assembly) at (9,5.5) {Complétion du contexte d'exécution};
	\node[draw, fill=white, ultra thin, drop shadow, minimum height=1cm] (args) at (7.5,4.25) {Récupération des arguments};
	\node[draw, fill=white, ultra thin, drop shadow, minimum height=1.5cm, align=center] (call) at (6,2.75) {Appel au service\\Gestion des erreurs};
	\node[draw, fill=white, ultra thin, drop shadow, minimum height=1cm] (parents_checks) at (0.5,2.75) {Récupération partition parent};
	\node[draw, fill=white, ultra thin, drop shadow, minimum height=1cm] (callee_vidt) at (0.5,1) {Récupération VIDT appelé};

	\node[draw, fill=white, ultra thin, drop shadow={shadow xshift=0.07cm,shadow yshift=-0.07cm}, minimum height=1cm, circle] (guru_meditation) at (10.5,1.5) {$\infty$};

	\node[above=0.2cm of assembly, fill=red!30, drop shadow={shadow xshift=0.05cm,shadow yshift=-0.05cm}] (interruptgate) {\small{\texttt{Faute - Interrupt gate}}};

	\draw[->] (assembly.345) to[in=0, out=270] (args);
	\draw[->] (args.340) to[in=0, out=270] (call);

	\draw[->] (call.173) to[in=0, out=180] (parents_checks.5);
	\draw[->, dashed, gray] (parents_checks.355) to[in=180, out=0] (call.187);
	\draw[->, dashed, gray] (call.220) to[in=0, out=270] (callee_vidt);
	\draw[->, dashed, gray] (call.260) to[in=240, out=280] (call.330);
	\draw[->, dashed, gray] (call) -- (guru_meditation);

	\node (separation) at (3.625,6) {};
	\draw[semithick, dotted] (separation) -- (3.625, 0.5);
	\node[below left=0.4cm of separation, fill=gray!30, drop shadow={shadow xshift=0.05cm,shadow yshift=-0.05cm}] (service) {\small{\texttt{Blocs du service}}};
}

\end{tikzpicture}

			\caption{Résumé des différents blocs logiciels permettant d'appeler le service après une faute}
			\label{fig:fault_software}
		\end{figure}

		\subsubsection{Implémentation des interruptions utilisant le service sur l'architecture x86}

		La même méthode a été employée pour gérer les interruptions matérielles avec le service. Pour rappel, les interruptions matérielles doivent arriver à la partition racine. Il n'y a pas de bloc de code récupérant la partition racine directement, il faut donc commencer à exécuter le service après les blocs de code récupérant le descripteur de partition. Ceci permet d'injecter dans les paramètres le descripteur de la partition racine à la place des partitions initialement prévues par les blocs précédents.

		Le bloc idéal pour devoir récupérer le moins d'arguments possibles est le bloc \texttt{getSourceVidtCont}. Le prototype du bloc est présenté en listing \ref{code:getSourceVidtCont_proto}.

		\begin{listing}[!ht]
			\coqcode{code/getSourceVidtCont.v}
			\caption{Prototype du bloc de code \texttt{getSourceVidtCont}, ciblé par les interruptions matérielles.}
			\label{code:getSourceVidtCont_proto}
		\end{listing}

		Le cheminement jusqu'à l'exécution du bloc du service est extrêmement similaire à celui concernant les fautes. De la même manière que pour une faute, lorsqu'une interruption matérielle arrive, le processeur va récupérer l'\emph{interrupt gate} correspondante dans l'\emph{IDT} (pour rappel, les niveaux d'interruptions correspondants aux interruptions matérielles vont de 32 à 47). Cette similitude continue jusqu'à la récupération des arguments : le processeur change de pile et pousse les mêmes éléments que ceux présentés précédemment en figure \ref{figure/interrupt_stack}, puis une routine assembleur complète le contexte d'exécution de type \texttt{int\_ctx\_t}, et appelle finalement la fonction de récupération des arguments \texttt{hardwareInterrupthandler}. Les arguments à récupérer par la fonction sont les mêmes que pour le bloc \texttt{getParentPartDescCont} et sont récupérés de la même manière, mis à part l'argument \texttt{targetPartDesc}. Cet argument est l'adresse réelle du descripteur de la partition recevant le flôt d'exécution, qui devra donc être le descripteur de la partition racine. La fonction récupère ce descripteur au travers d'une variable globale de Pip.

		L'appel du bloc de code \texttt{getSourceVidtCont} peut aussi se solder par une erreur : si l'origine de cette erreur est le fait que la VIDT de la partition n'est pas accessible ou que l'emplacement de sauvegarde de son contexte n'est pas accessible, alors la sauvegarde de contexte est omise et la fonction appelle le bloc de code \texttt{getTargetVidtCont}. Une autre erreur indiquerait que la partition racine n'est pas en mesure de récupérer le flôt d'exécution ; dans ce cas la fonction arrête l'exécution en rentrant dans une boucle infinie.

		\begin{figure}[!ht]
			\centering
			\begin{tikzpicture}[>=triangle 45,font=\sffamily, every text node part/.style={align=center}, scale=1, every node/.style={transform shape}] {

	\node[draw, fill=white, ultra thin, drop shadow, minimum height=1cm] (assembly) at (9,5.5) {Complétion du contexte d'exécution};
	\node[draw, fill=white, ultra thin, drop shadow, minimum height=1cm] (args) at (7.5,4.25) {Récupération des arguments};
	\node[draw, fill=white, ultra thin, drop shadow, minimum height=1.5cm, align=center] (call) at (6,2.75) {Appel au service\\Gestion des erreurs};
	\node[draw, fill=white, ultra thin, drop shadow, minimum height=1cm] (parents_checks) at (0.5,2.75) {Récupération VIDT appelant};
	\node[draw, fill=white, ultra thin, drop shadow, minimum height=1cm] (callee_vidt) at (0.5,1) {Récupération VIDT appelé};

	\node[draw, fill=white, ultra thin, drop shadow={shadow xshift=0.07cm,shadow yshift=-0.07cm}, minimum height=1cm, circle] (guru_meditation) at (10.5,1.5) {$\infty$};

	\node[above=0.2cm of assembly, fill=red!30, drop shadow={shadow xshift=0.05cm,shadow yshift=-0.05cm}] (interruptgate) {\small{\texttt{Interruption matérielle - Interrupt gate}}};

	\draw[->] (assembly.345) to[in=0, out=270] (args);
	\draw[->] (args.340) to[in=0, out=270] (call);

	\draw[->] (call.173) to[in=0, out=180] (parents_checks.5);
	\draw[->, dashed, gray] (parents_checks.355) to[in=180, out=0] (call.187);
	\draw[->, dashed, gray] (call.220) to[in=0, out=270] (callee_vidt);
	\draw[->, dashed, gray] (call) -- (guru_meditation);

	\node (separation) at (3.625,6) {};
	\draw[semithick, dotted] (separation) -- (3.625, 0.5);
	\node[below left=0.4cm of separation, fill=gray!30, drop shadow={shadow xshift=0.05cm,shadow yshift=-0.05cm}] (service) {\small{\texttt{Blocs du service}}};
}

\end{tikzpicture}

			\caption{Résumé des différents blocs logiciels permettant d'appeler le service après une interruption matérielle}
			\label{fig:interrupt_software}
		\end{figure}

	\section{Preuve d'isolation}

		Cette section a pour but de décrire le processus d'établissement de la preuve d'isolation de Pip sur le service. La première sous-section décrira tout d'abord les nouvelles fonctions introduites dans l'interface de la monade d'état ainsi que la raison de leur inclusion. Ces fonctions sont accompagnées de nouveaux types, qui seront aussi explicités. La seconde sous-section fera un rappel des propriétés d'isolation de Pip. La dernière sous-section décrira le processus d'établissement de la preuve.\\

		\textbf{Note importante :} Dans un soucis de transparence et d'intégrité scientifique, je souhaite préciser que la preuve d'isolation reposant sur les modèles présentés dans cette section n'est pas encore complète. En effet, nous avons récemment repéré une modélisation incorrecte de la fonction \texttt{writeContext}. Cette trouvaille a mené à la réécriture du modèle de la fonction et a cassé la preuve d'isolation précédemment établie sur l'ancien modèle. Dans cette section, nous présenterons néanmoins le nouveau modèle qui reflète le comportement attendu de la fonction. Ce nouveau modèle ne change en rien l'implémentation réelle de cette fonction : ces changements sont cantonnés au monde des mathématiques. De plus, ce nouveau modèle n'introduit pas d'obstacle particulier à l'établissement de la preuve d'isolation. Elle n'a pas été établie par manque de temps entre sa découverte et l'écriture de ce document. Nous reviendrons cependant sur cette erreur de modélisation dans la section suivante, qui révèle la subtilité de l'établissement de preuves au travers de modèles.

		\subsection{Définition de l'interface avec la monade}

		Cette sous-section est dédiée à la définition des nouveaux éléments de l'interface avec la monade. Nous discuterons d'abord des nouveaux types et de ce qu'ils représentent, puis nous discuterons des fonctions.

		\subsubsection{Nouveaux types intégrés à la monade}

		\paragraph{\texttt{userValue}} Le type \texttt{userValue} est un type \emph{opaque}, c'est à dire un type dont les valeurs n'ont pas vocation à être manipulées dans le modèle. Le type \texttt{userValue} indique une valeur arbitraire provenant de l'espace utilisateur. Cette valeur doit subir des tests avant de pouvoir être transformée en un type sain et utilisable par le noyau. Prenons par exemple le type \texttt{index}. \texttt{index} est un entier naturel dont la valeur ne peut dépasser le nombre de mots mémoire dans une page. Afin de s'assurer qu'un indice passé en paramètre par l'utilisateur respecte bien ces contraintes, le service va le traiter comme une \texttt{userValue}. Le service vérifie alors que la contrainte est bien respectée, puis raffine le type de la valeur en un \texttt{index}. Pour certains types de valeurs, il n'est pas nécessaire d'utiliser le type \texttt{userValue}. Ceci est vrai pour les types dont l'intégralité des valeurs représentables sont des valeurs valides, comme par exemple les adresses virtuelles.

		Le type \texttt{userValue} étant un type opaque, son modèle importe peu. Il est représenté par un \texttt{nat}.

		\paragraph{\texttt{interruptMask}} Le type \texttt{interruptMask} est un type \emph{opaque} représentant les drapeaux servant aux partitions à indiquer si elles souhaitent être interrompues ou non. Ce type est entouré d'une interface permettant de récupérer ces drapeaux depuis un contexte d'exécution, d'appliquer ces drapeaux à une partition, de sauver ces drapeaux dans un contexte d'exécution, ainsi que d'interprêter ces drapeaux. La mise en place d'un tel mécanisme est fortement lié à la plateforme et a été laissée libre à l'implémentation. Le type est representé comme une liste de booléens dont la longueur est égale au nombre de niveaux d'interruptions, cependant cette information n'est jamais exploitée dans le modèle.
		\begin{listing}[!ht]
			\begin{minted}{coq}
Record interruptMask := {
    m  :> list bool;
    Hm :  length m = maxVint+1;
}.
			\end{minted}
			\caption{Représentation du type \texttt{interruptMask} dans le modèle}
		\end{listing}

		\paragraph{\texttt{contextAddr}} Le type \texttt{contextAddr} est un type \emph{opaque} représentant l'adresse d'un contexte d'exécution. Il est muni de deux fonctions : une permettant d'écrire ce contexte à une adresse donnée, l'autre l'applicant au système qui reprendra le flôt d'exécution lié. Le type est représenté par un \texttt{nat}.


		\subsubsection{Nouvelles fonctions de l'interface}

		Certaines de ces nouvelles fonctions font partie de l'interface des nouveaux types présentés dans les paragraphes précédent. Tout d'abord le type \texttt{userValue} a engendré deux nouvelles fonctions :
		\paragraph{\mintinline{c}{bool checkIndexPropertyLTB(userValue userIndex);}}~\\
		Cette fonction retourne vrai si la valeur passée en paramètre respecte la contrainte du type \texttt{index} ; la valeur doit être inférieure au nombre de mots mémoire dans une page. Voici son modèle :

		\begin{listing}[!ht]
			\begin{minted}{coq}
Definition checkIndexPropertyLTB (userIndex : userValue) : LLI bool :=
ret (Nat.ltb userIndex tableSize).
			\end{minted}
			\caption{Modèle de la fonction \texttt{checkIndexPropertyLTB}}
		\end{listing}
		Cette fonction de l'interface a cependant été maladroitement choisie, car le choix de la valeur de la comparaison \texttt{tableSize} relève de la responsabilité de l'implémentation, alors qu'elle aurait pu rester dans le code du service. La fonction pertinente à ajouter dans l'interface au lieu de celle-ci aurait dû être la fonction de comparaison entre une \texttt{userValue} et un \texttt{index}.

		\paragraph{\mintinline{c}{index userValueToIndex(userValue userIndex);}}~\\
		Cette fonction transforme la valeur de type \texttt{userValue} en fonction de type \texttt{index}. Cette fonction n'a pas d'effet particulier sur la valeur.
		Son modèle est disponible en listing \ref{code:userValueToIndex}. Ce modèle met en lumière que la valeur \texttt{userIndex} doit être inférieure à la constante \texttt{tableSize}, qu'une preuve en soit établie, et que cette preuve soit passée au constructeur du type \texttt{index} (ici déterminée automatiquement par Coq au travers du symbole \texttt{\_}).
		\begin{listing}[!ht]
			\begin{minted}{coq}
Program Definition userValueToIndex (userIndex : userValue) : LLI index :=
    if lt_dec userIndex tableSize
    then
        ret (Build_index userIndex _ )
    else undefined 85.
			\end{minted}
			\caption{Modèle de la fonction \texttt{userValueToIndex}}
			\label{code:userValueToIndex}
		\end{listing}

		\newpage
		Le service a ajouté quatre fonctions dans l'interface qui intéragissent avec le type \texttt{interruptMask}.

		\mintinline{c}{interruptMask getInterruptMaskFromCtx(contextAddr context)} est la première fonction permettant de récupérer les drapeaux présents dans un contexte d'exécution. La seconde fonction~ \mintinline{c}{bool noInterruptRequest(interruptMask flagsOnWake)} permet d'interpréter les drapeaux passés en paramètres pour renvoyer si la partition ne souhaite pas être interrompue. La troisième fonction en rapport avec le type \texttt{interruptMask} est la fonction \mintinline{c}{void setInterruptMask(uint32_t interrupt_state)} qui permet d'appliquer les masques à la partition courante. La dernière fonction ajoutée,~~\mintinline{c}{uint32_t}\\ \mintinline{c}{get_self_int_state()}, permet de récupérer les drapeaux de la partition courante.

		Du point de vue du modèle, ces fonctions sont peu intéressantes. Elles n'ont pas été intégrées car la position des drapeaux n'a pas été définie dans l'interface et peut donc être placée arbitrairement par l'implémentation. L'implémentation Intel x86, place ces drapeaux dans le descripteur de la partition, afin que leur écriture soit atomique. Ainsi, ces fonctions renvoient soit une valeur arbitraire, soit n'ont aucun effet sur le modèle. La fonction \mintinline{c}{uint32_t get_self_int_state()} n'a pas été modélisée car elle n'apparait pas dans le service, mais dans les fonctions de récupération des arguments discutés en section \ref{sec:service_generalisation}.\\

%		\paragraph{\mintinline{c}{interruptMask getInterruptMaskFromCtx(contextAddr context);}}
%			Cette fonction récupère les drapeaux enregistrés dans un contexte d'exécution. Son modèle retourne des drapeaux arbitraires.
%			\begin{listing}[!ht]
%				\begin{minted}{coq}
%Definition getInterruptMaskFromCtx (context : contextAddr) :
%LLI interruptMask := ret int_mask_d.
%				\end{minted}
%				\caption{Modèle de la fonction \texttt{getInterruptMaskFromCtx}}
%				\label{code:getInterruptMaskFromCtx}
%			\end{listing}
%
%		\paragraph{\mintinline{c}{bool noInterruptRequest(interruptMask flagsOnWake);}}
%			Cette fonction renvoie vrai si les drapeaux passés en paramètres indiquent que la partition ne souhaite pas se faire interrompre. Le modèle de cette fonction renvoie une valeur arbitraire.
%			\begin{listing}[!ht]
%				\begin{minted}{coq}
%Definition noInterruptRequest (flags : interruptMask) : LLI bool :=
%    ret true.
%				\end{minted}
%				\caption{Modèle de la fonction \texttt{noInterruptRequest}}
%				\label{code:noInterruptRequest}
%			\end{listing}
%
%		\paragraph{\mintinline{c}{void setInterruptMask(uint32_t interrupt_state);}}
%			Cette fonction applique les drapeaux à la partition courante. Cette fonction n'a aucun effet dans le modèle.
%			\begin{listing}[!ht]
%				\begin{minted}{coq}
%Definition setInterruptMask (mask : interruptMask) : LLI unit :=
%    ret tt.
%				\end{minted}
%				\caption{Modèle de la fonction \texttt{setInterruptRequest}}
%				\label{code:setInterruptRequest}
%			\end{listing}
%
%		\paragraph{\mintinline{c}{uint32_t get_self_int_state();}}
%			Cette fonction permet de récupérer les drapeaux de la partition courante. Cette fonction n'a pas de modèle car elle n'est utilisée que dans les routines de récupération des arguments pour le transfert de flôt d'exécution lors de fautes ou d'interruptions.

		Les fonctions les plus intéressantes de cet ajout à l'interface sont celles qui manipulent le type \texttt{contextAddr}.

		\paragraph{\mintinline{c}{contextAddr vaddrToContextAddr(vaddr contextVAddr);}}~\\
		Cette fonction convertit une adresse virtuelle en pointeur vers un contexte d'exécution. Le modèle de cette fonction renvoie une valeur arbitraire.

		\paragraph{\mintinline{c}{void loadContext(contextAddr ctx, bool enforce_interrupts);}}~\\
		Cette fonction charge le contexte d'exécution pointé par \texttt{ctx}, en s'assurant que le futur flôt d'exécution sera interruptible si \texttt{enforce\_interrupts} est vrai.

		Cette fonction est intéressante car elle ne peut être écrite en Gallina, le langage de Coq. En effet, Coq s'assure que tous les programmes écrits en Gallina terminent, ce qui n'est pas le cas de cette fonction. Par exemple, dans l'implémentation Intel x86, cette fonction charge tout les registres généraux puis pousse une petite partie du contexte sur pile, et exécute l'instruction assembleur \texttt{iret}. À partir du moment où cette instruction a été exécutée, le flôt d'exécution a déjà changé : la pile et le pointeur d'instruction ne sont plus les mêmes. De ce fait cette fonction n'a aucun effet dans le modèle.

		\begin{listing}[!ht]
			\begin{minted}{coq}
Definition loadContext (contextToLoad : contextAddr)
                       (enforce_interrupt : bool) : LLI unit :=
    ret tt.
			\end{minted}
			\caption{Modèle de la fonction \texttt{loadContext}}
			\label{code:loadContext}
		\end{listing}

		Cependant, il n'est pas \emph{nécessaire} que cette fonction ne retourne pas. En effet, dans l'implémentation actuelle sur Intel x86, la fonction copie le contexte sur le dessus de la pile et le charge immédiatement. Il s'avère que ce code de chargement est aussi disponible dans les portions d'assembleur appelant le service, dans le cas où une erreur surviendrait pendant l'exécution du service. Dans ce cas, le service retourne prématurément pour signaler une erreur et la fonction \texttt{loadContext} n'est jamais atteinte. Ces portions de code restaurent le contexte appelant le service, mais il serait possible que la fonction \texttt{loadContext} vienne modifier ce contexte initial pour qu'il corresponde au nouveau contexte à charger. Ainsi, le transfert de flot d'exécution ne s'opérerait pas dans la fonction \texttt{loadContext}, mais dans la portion d'assembleur de retour au contexte initial. Ainsi, il serait possible d'éviter cette dualité entre le monde du modèle qui retourne et renvoie un succès, et le monde réel qui ne retourne pas et charge directement le contexte dans la fonction. Cette approche est néanmoins beaucoup plus compliquée à mettre en place comparée à une simple copie des données sur le sommet de la pile, et n'apporte pour le moment pas d'autre avantage que l'agréable sentiment d'harmonie dans la création.

		\paragraph{\mintinline{c}{void writeContext(contextAddr ctx, vaddr ctxSaveVAddr, interruptMask}}~~\\
		   \textbf{\mintinline{c}{flagsOnWake);}}~\\
			Cette fonction écrit le contexte d'exécution pointé par \texttt{ctx} à l'adresse virtuelle \texttt{ctxSaveVAddr}, en y écrivant les drapeaux \texttt{flagsOnWake}. Cette fonction est différente des autres fonctions d'écriture de l'interface avec la monade car elle effectue plusieurs écritures successives en mémoire. De plus, ces écritures sont faites en espace utilisateur. Son modèle se contente d'appeler une fonction récursive qui va itérer sur la taille d'un contexte, et écrire aux adresses virtuelles successives une valeur arbitraire.\\

		\begin{listing}[!ht]
			\begin{minted}{coq}
Definition writeContext (callingContextAddr : contextAddr)
                        (contextSaveAddr : vaddr)
                        (flagsOnWake : interruptMask) : LLI unit :=
    perform maxIdx := getMaxIndex in
    perform idxContextInPage := ret (List.last contextSaveAddr index_d) in
    writeContextAux contextSaveAddr idxContextInPage maxIdx contextSize.

Fixpoint writeContextAux (contextSaveAddr : vaddr) (currIdx : index)
                         (maxIdx : index) (bound : nat) : LLI unit :=
  match bound with
  | 0 => ret tt
  | S dec_bound =>
    storeVirtual contextSaveAddr currIdx vaddrDefault ;;
    if idxEq currIdx maxIdx then
      writeContextAux (getNextVaddr contextSaveAddr) idx0 maxIdx dec_bound
    else
      perform nextIdx := idxSuccM currIdx in
      writeContextAux (getNextVaddr contextSaveAddr) nextIdx maxIdx dec_bound
  end.
			\end{minted}
			\caption{Modèle de la fonction \texttt{writeContext} et sa fonction auxiliaire récursive}
			\label{code:loadContext}
		\end{listing}

		Les écritures multiples de la fonction \texttt{writeContext} ont mené à des vérifications supplémentaires qui ont nécessité l'ajout de deux fonctions dans l'interface. La première est \mintinline{c}{vaddr getNthVAddrFrom(vaddr base, uint32_t size)}. Cette fonction retourne l'adresse virtuelle se trouvant à l'adresse de base plus une certain nombre d'octets \texttt{size}. Cette fonction permet notamment de récupérer la dernière adresse d'un contexte d'exécution, afin de vérifier qu'elle est bien valide avant d'entamer l'écriture.\\
		La seconde est \mintinline{c}{bool firstVAddrGreaterThanSecond(vaddr vaddr1, vaddr vaddr2)}, qui retourne si une adresse virtuelle est plus grande que la seconde. Cela permet au service de vérifier qu'un overflow n'a pas eu lieu lors du calcul de la dernière adresse du contexte.

		La fonction \mintinline{c}{vaddr getVaddrVIDT()} fait aussi partie des nouveaux ajouts à l'interface, et retourne l'adresse fixe de la VIDT. Cette valeur est définie comme la dernière page de l'espace d'adressage virtuel.

		Enfin, la fonction \mintinline{c}{void updateMMURoot(page MMURoot)} a été ajoutée. Cette fonction applique un nouvel espace d'adressage décrit par \texttt{MMURoot}. Dans l'implémentation Intel, cette fonctionne écrit le Page Descriptor \texttt{MMURoot} -- structure racine de la configuration de la MMU -- dans le registre \texttt{CR3} du processeur. Cette fonction est spéciale car elle aurait pu être cachée dans l'implémentation, et n'a aucun sens dans le modèle d'isolation de Pip. Elle été ajoutée dans l'optique de produire une preuve fonctionnelle du service.

		C'est sur ces différents modèles de fonction que la preuve d'isolation sera établie. Dans la prochaine section nous préciserons les propriétés d'isolation, avant de procéder à l'explication de l'établissement de la preuve.

		\subsection{Rappel des propriétes d'isolation de Pip}

		Comme exprimé dans la section \textcolor{red}{insérer ref}, la preuve d'isolation de Pip repose sur trois propriétés fondamentales : la propriété d'\emph{isolation horizontale}, la propriété d'\emph{isolation noyau} ainsi que la propriété de \emph{partage verticale}. Ces trois propriétés principales sont accompagnées d'une multitude de propriétés liées à la cohérence du noyau. Certaines de ces propriétés seront mises en avant dans ce document, puisqu'elle seront discutées dans la section suivante traitant de l'établissement de la preuve du service. \textcolor{red}{Est ce vrai ?}.

			\subsubsection{Isolation horizontale}

			La propriété d'isolation horizontale stipule que deux enfants d'une même partition parent ne peuvent partager de page mémoire au travers de leur espace d'adressage ou de leurs pages de configuration. Les deux ensembles des pages concernant l'une et l'autre des partitions enfants doivent être strictement disjoints. Le listing \ref{code:horizontal_isolation} montre comment est exprimée formellement cette propriété dans l'assistant de preuve.

			\begin{listing}[!ht]
				\coqcode{code/horizontal_isolation.v}
				\caption{Propriété d'isolation horizontale telle qu'exprimée dans Coq}
				\label{code:horizontal_isolation}
			\end{listing}

			La fonction \texttt{getPartitions} récupère les pages contenant les descripteurs de partitions. La fonction \texttt{getChildren} recupère la liste des pages contenant les descripteurs de partitions de tous les enfants de la partition passée en paramètre. La fonction \texttt{getUsedPages} récupère l'ensemble des pages mémoires renseignées dans les structures de données de la partition passée en paramètre. Cela inclus les structures internes de Pip ainsi que la description de l'espace d'adressage de la partition.

			La propriété peut être lue de la sorte :
			\begin{theorem}
				Pour toutes pages de mémoire \texttt{parent}, \texttt{child1} et \texttt{child2}. Si \texttt{parent} est dans la liste des descripteurs de partitions, si \texttt{child1} et \texttt{child2} sont tous deux dans la liste des partitions enfant de \texttt{parent}, et si \texttt{child1} et \texttt{child2} sont différents, alors l'ensemble des pages utilisées par la partition \texttt{child1} est disjoint de l'ensemble des pages utilisées par la partition \texttt{child2}.
			\end{theorem}

			\subsubsection{Isolation noyau}

			La propriété d'isolation du noyau stipule que les pages de configuration d'une partition (c'est à dire les pages réquisitionnées par Pip pour y loger ses structures, notamment les structures relative à l'espace d'adressage de la partition) sont inaccessibles à \emph{n'importe quelle} partition.
			Cette propriété est exprimée formellement par le code décrit en listing \ref{code:kernel_isolation}.

			\begin{listing}[!ht]
				\coqcode{code/kernel_isolation.v}
				\caption{Propriété d'isolation du noyau telle qu'exprimée dans Coq}
				\label{code:kernel_isolation}
			\end{listing}

			La fonction \texttt{getAccessibleMappedPages} permet de récupérer les pages de mémoire renseignées comme accessibles sans privilège dans l'espace d'adressage de la partition passée en paramètre. La fonction \texttt{getConfigPages} permet de récupérer l'ensemble des pages de mémoire de la partition passée en paramètre contenant les structures internes à Pip.

			La propriété peut être lue de la sorte :
			\begin{theorem}
				Pour toutes pages de mémoire \texttt{partition1} et \texttt{partition2}. Si \texttt{partition1} et \texttt{partition2} sont toutes deux dans la liste des descripteurs de partitions, alors l'ensemble des pages contenant les structures internes de Pip relatives à la \texttt{partition2} est disjoint de l'ensemble des pages accessibles à la \texttt{partition1}.
			\end{theorem}

			\subsubsection{Partage vertical}

			La dernière propriété fondamentale d'isolation de Pip, appelée propriété de partage vertical, stipule que chaque page relative à une partition enfant fait partie de l'espace d'adressage de son parent. Ces pages concernent aussi bien les pages de configuration de la partition enfant reservées à Pip que les pages disponibles dans son espace d'adressage. Certaines pages ne sont cependant pas accessibles à la partition parent bien qu'elles soient mappées, en particulier les pages de configuration de la partition enfant réquisitionnées par Pip.

			\begin{listing}[!ht]
				\coqcode{code/vertical_sharing.v}
				\caption{Propriété de partage vertical de la mémoire telle qu'exprimée dans Coq}
				\label{code:vertical_sharing}
			\end{listing}

			La fonction \texttt{getMappedPages} permet de récupérer toutes les pages de mémoire mappées dans la partition passée en paramètre.

			La propriété peut être lue de la manière suivante :

			\begin{theorem}
				Pour toutes pages mémoire \texttt{parent} et \texttt{child}, si \texttt{parent} fait partie de la liste des descripteurs de partitions et que \texttt{child} est dans la liste des descripteurs de partitions enfant de \texttt{parent}, alors l'ensemble des pages relatives à \texttt{child} est inclus dans l'ensemble des pages mappées dans l'espace d'adressage de la partition \texttt{parent}.
			\end{theorem}

			\subsubsection{Propriétés de cohérence du noyau}

			Dans Pip, il y a 25 propriétés annexes, qui décrivent comment sont organisées les structures de données internes de Pip. Ces propriétés sont appelées les propriétés de cohérence du noyau. Nous n'allons pas passer en revue l'ensemble des propriétés de cohérence dans ce document, mais seulement celles liées aux données modifiées par le service.

			La première propriété de cohérence à laquelle nous nous intéresserons est la propriété \texttt{currentPartitionInPartitionsList}.

			\begin{listing}[!ht]
				\coqcode{code/currentPartitionInPartitionsList.v}
				\caption{Propriété de cohérence indiquant que la partition courante doit faire partie de la liste des partitions}
				\label{code:currentPartitionInPartitionsList}
			\end{listing}

			Cette propriété assure que la variable globale contenant la partition courante contiendra toujours un descripteur de partition valide provenant de la liste des partitions.

			La seconde propriété de cohérence à laquelle nous nous intéresserons est la propriété \texttt{currentPartitionIsNotDefaultPage}.

			\begin{listing}[!ht]
				\coqcode{code/currentPartitionIsNotDefaultPage.v}
				\caption{Propriété de cohérence indiquant que la partition courante ne doit pas être la page par défaut}
				\label{code:currentPartitionIsNotDefaultPage}
			\end{listing}

			Cette propriété assure que la variable reflétant la partition courante soit différente de la page par défaut.
		
		\subsection{Déroulement de la preuve}
			\subsubsection{Validation des paramètres}
			\subsubsection{Sauvegarde du contexte d'exécution de la partition appelante}
			\subsubsection{Modification de la partition courante}

			
			\begin{listing}[!ht]
				\coqcode{code/partitionTreeRemains.v}
				\caption{Preuve que la fonction \texttt{getPartitions} effectue les mêmes calculs peu importe la partition courante}
				\label{code:partitionTreeRemains}
			\end{listing}

			\begin{listing}[!ht]
				\coqcode{code/currentPartitionInPartitionsListActivate.v}
				\caption{Preuve de la propriété de cohérence \texttt{currentPartitionInPartitionsList}}
				\label{code:currentPartitionInPartitionsListActivate}
			\end{listing}


	\section{Retour d'expérience}
	% Remarques pragmatiques sur cette contribution
		\subsection{Métriques}
		\subsection{Prise de recul sur la nature de la preuve}
		\subsection{Limites du service}
			- parler de la limite de la taille des flags comparé à nombre d'interruptions


    \chapter{Politique d'ordonnancement prouvée}

	Le chapitre précédent a porté sur un service de transfert de flot d'exécution au sein de Pip. Ce service permet de gérer les transferts explicites de flot d'exécution, ainsi que les interruptions. De cette manière, Pip fournit les outils nécessaires à la préemption de flots d'exécution. La préemption est nécessaire à la mise en place d'ordonnancement au sein d'un système ; elle n'est cependant pas suffisante pour implémenter un ordonnanceur. Celui-ci doit être muni d'une politique d'ordonnancement, lui permettant de choisir le prochain flot d'exécution.

	Ce chapitre porte sur l'implémentation d'un ordonnanceur Earliest Deadline First (abrégé EDF) s'exécutant en espace utilisateur. L'ordonnanceur est conçu pour fonctionner dans une partition de Pip, et est muni d'une preuve formelle que sa fonction d'élection respecte la politique d'ordonnancement EDF. Ces travaux ont été menés conjointement avec Vlad Rusu, avec qui j'ai conçu les modèles sur lesquels reposent la preuve et qui a établit les preuves formelles sur cet ordonnanceur. Dans ce travail commun, ma contribution a été de concevoir les composants de l'ordonnanceur, leur interface ainsi qu'une partie de leur modèle, ainsi que d'écrire l'implémentation pour que cet ordonnanceur puisse être compilé et s'exécuter dans une partition de Pip.
	\footnote{David Nowak, Gilles Grimaud mais aussi Samuel Hym ont aussi contribué à ce projet de manière plus modeste. David a participé à l'élaboration des modèles, Gilles a participé à l'établissement des interfaces, et Samuel a participé à l'écriture de l'implémentation}.
	
	Ces travaux ont fait l'objet d'une publication et d'une présentation à RTAS2022, dont ce chapitre s'inspire fortement. Le papier est disponible à l'adresse suivante : \url{https://hal.archives-ouvertes.fr/hal-03671598}. La vidéo de la présentation est disponible à l'adresse suivante : \url{https://pip.univ-lille.fr/recordings/RTAS.mp4}. Le dépot contenant le code ainsi que les instructions pour exécuter l'ordonnanceur est hébergé sur Github : \url{https://github.com/2xs/pip_edf_scheduler}.

	\section{Motivations}
	
	% Autre type de preuve : garantir le respect des échéances / pourquoi est ce vraiment critique ?
	La motivation première de cette contribution est de proposer des garanties supplémentaires aux garanties d'isolation classiques de Pip. Cette motivation nous a mené à nous intéresser aux problématiques des systèmes temps réel, et notamment sur le respect des échéances, car nous pensons que la vision de Pip peut être pertinente. La preuve formelle du respect de la politique d'ordonnancement est un premier pas vers des garanties temps réel dans l'écosystème de Pip.

	% Intro : ce qu'on va étudier : utliser notre méthode de dévelopement de logiciel pour prouver un algorithme de selection des cibles du transfert de flot d'exécution autres propriétés que l'isolation
	Cette contribution est aussi un moyen d'éprouver la méthode de développement des logiciels et preuves de Pip, en l'appliquant à un autre objet d'étude. On peut noter deux différences fondamentales entre Pip et cet ordonnanceur. La première est que Pip s'exécute en mode privilégié, contrairement à l'ordonnanceur qui s'exécute en espace utilisateur. Cette différence a des répercussions sur la conception des interfaces ; la dépendance à du code non prouvé est moins critique en espace utilisateur. Cette différence a aussi un impact implicite sur les preuves : la logique de Hoare n'est valable que lorsque l'état ne change pas entre deux instructions -- il faut donc s'assurer que les interruptions ne perturbent pas l'ordonnanceur. La seconde réside dans le fait que les propriétés prouvées sur l'ordonnanceur relèvent de l'ordre des propriétés fonctionnelles, c'est à dire qui décrivent le comportement de l'ordonnanceur. Les propriétés d'isolation de Pip traditionnellement prouvées au travers de cette méthode ne sont pas des propriétés fonctionnelles. Aussi, nous ne chercherons pas à établir des propriétés non-fonctionnelles sur la préemption telles que des propriétés sur le partage du temps d'exécution du processeur. Ces deux différences font de cet ordonnanceur un objet d'étude intéressant du point de vue du développement logiciel.

	De plus, cette contribution utilise tous les aspects du service de transfert de flot d'exécution décrit dans le chapitre précédent. Cette contribution permet d'exhiber par l'expérience que l'utilisation de ce service est pertinente dans un composant commun de système d'exploitation.
	
	Enfin, cette contribution apporte une preuve inédite à la communauté des systèmes temps réels. En effet, les travaux montrant de manière formelle des propriétés sur les algorithmes d'ordonnancement sont fréquents dans la communauté. Néanmoins, peu de travaux s'intéressent à la preuve de \emph{l'implémentation} de l'algorithme, qui permettent par exemple de propager les preuves jusqu'au code compilé, et qui tiennent compte de détails supplémentaires (tels que les structures de données par exemple). Cette contribution fournit une preuve formelle que l'implémentation de la fonction d'élection respecte la politique d'ordonnancement EDF pour des \emph{jobs} arbitraires, ce qui n'avait été prouvé sur aucun autre ordonnanceur EDF auparavant. Des travaux indépendants et concomitants sur le projet CertiKOS ont montré une propriété similaire sur un ordonnanceur. Cependant leur ordonnanceur est limité à l'ordonnancement de \emph{jobs} provenant de tâches périodiques.

	\section{Description structurelle}

	\label{sec:sched_struct_desc}

		Cette section décrit la structure générale de l'ordonnanceur. Elle commencera par sa partie phare, la fonction d'élection. Cette section en donnera une description fonctionnelle, ainsi que son prototype. Cette section décrira ensuite la place de l'ordonnancement dans Pip. Elle discutera du choix de placer l'ordonnanceur dans une partition, et des différences principales avec un système d'exploitation classique tel que Linux. Enfin, cette section détaillera les composants de l'ordonnanceur, leur fonction ainsi que leurs interactions.

		\subsection{Définition prototype et oracle de la fonction d'élection}

		La fonction d'élection est le morceau de logiciel que l'ordonnanceur doit appeler à chaque préemption, afin d'élire le prochain flot d'exécution du système. La fonction d'élection fait ce choix en accord avec une politique d'ordonnancement qui lui permet de discriminer les flots d'exécution selon des critères spécifiques qui sont propres à chaque politique. En particulier, la fonction d'élection de notre contribution suit la politique d'ordonnancement \emph{Earliest Deadline First}, qui est une politique temps réel dont la priorité est d'exécuter les \emph{jobs} dont l'échéance est la plus proche.

		La fonction d'élection de notre ordonnanceur agit comme un oracle, ne prennant aucun argument et retournant un type composite. Ce type composite contient un booléen indiquant à l'ordonnanceur qu'il doit attendre la prochaine préemption car il n'y a aucun \emph{job} à exécuter. Ce type contient l'identifiant du \emph{job} à exécuter dans le cas contraire, ainsi qu'un autre booléen indiquant si le \emph{job} a dépassé son échéance. Ce dernier booléen sert à informer l'ordonnanceur que l'ensemble des \emph{jobs} n'est pas ordonnançable.

		\begin{listing}[!ht]
			\coqcode{code/electionfunction.v}
			\caption{Prototype de la fonction d'élection et définition de son type de retour }
		\end{listing}

b\begin{figure}[!b]
		\centering{
			\definecolor{job1color}{RGB}{88 ,88,184}
\definecolor{job2color}{RGB}{234,80,90}

\begin{tikzpicture}[>=triangle 45,font=\sffamily, every text node part/.style={align=center}, every node/.style={transform shape}]

	%timeline
	\node (tstarta) at (0, 0) {};
	\node (tenda) at (9, 0) {};
	\draw[->] (tstarta.center) -- (tenda.center);
	\node[below=0.2cm of tenda] {temps (u.a.)};

	\node (tstartb) at (0, 2) {};
	\node (tendb) at (9, 2) {};
	\draw[->] (tstartb.center) -- (tendb.center);
	\node[below=0.2cm of tendb] {temps (u.a.)};

	\node[draw, fill=job1color, minimum height = 1cm, minimum width = 4cm] (job1) at (3.5, 4.5) {};
	\node[left=0.3cm of job1] {Job \small{$a$}};
	\draw[<->] (1.5, 3.8) -- (5.5, 3.8);
	\node[below=0.2cm of job1] {$c_a$};
	\node[draw, fill=job2color, minimum width = 1cm, minimum height = 1cm] (job2) at (7, 4.5) {};
	\node[right=0.3cm of job2] {Job \small{$b$}};
	\draw[<->] (6.5, 3.8) -- (7.5, 3.8);
	\node[below=0.2cm of job2] {$c_b$};

	\node (tjob1arrival) at (2, 0) {};
	\node[above left=0cm of tjob1arrival] {$r_a$};
	\node (tjob1arrival2) at (2, -0.5) {$t=2$};
	\draw[dashed] (tjob1arrival) -- (tjob1arrival2);
	\node (tjob1deadline) at (7, 0) {};
	\node[above right=0cm of tjob1deadline] {$d_a$};
	\node (tjob1deadline2) at (7, -0.5) {$t=7$};
	\draw[dashed] (tjob1deadline) -- (tjob1deadline2);
	\node (job1height) at (2, 1) {};
	\draw (tjob1arrival2 |- job1height) -- (tjob1deadline2 |- tenda) node [midway, draw, fill=white, minimum width = 5cm, minimum height = 1cm] (job1execperiod) {};
	\node[draw, pattern color=job1color, pattern=south west lines, minimum width = 5cm, minimum height = 1cm] at (job1execperiod) {};

	\node (tjob2arrival) at (3, 2) {};
	\node[above left=0cm of tjob2arrival] {$r_b$};
	\node (tjob2arrival2) at (3, 1.5) {$t=3$};
	\draw[dashed] (tjob2arrival) -- (tjob2arrival2);
	\node (tjob2deadline) at (6, 2) {};
	\node[above right=0cm of tjob2deadline] {$d_b$};
	\node (tjob2deadline2) at (6, 1.5) {$t=6$};
	\draw[dashed] (tjob2deadline) -- (tjob2deadline2);
	\node (job2height) at (2, 3) {};
	\draw (tjob2arrival2 |- job2height) -- (tjob2deadline2 |- tendb) node [midway, draw, fill=white, minimum width = 3cm, minimum height = 1cm] (job2execperiod) {};
	\node[draw, pattern color=job2color, pattern=south west lines, minimum width = 3cm, minimum height = 1cm] at (job2execperiod) {};
\end{tikzpicture}

			\caption{Exemple de jobs pour l'illustration du fonctionnement de la fonction d'élection}
			\label{fig:electionfunction_jobs}
		}
		\end{figure}
			\subsubsection{Exemple d'appel à la fonction d'élection}

			Pour rappel, chaque \emph{job} est muni au minimum des informations suivantes : une \emph{release date}, notée $r$, indiquant l'instant à partir duquel il est possible d'exécuter le \emph{job}, une échéance, notée $d$, pour laquelle le \emph{job} doit avoir terminé son exécution, et un budget d'exécution, noté $c$, indiquant le nombre de périodes d'exécution allouées pour ce \emph{job}.

			Dans cet exemple, on considère l'ensemble de \emph{jobs} constitué des \emph{jobs} $a$ et $b$ représenté sur la Figure \ref{fig:electionfunction_jobs}. Le \emph{job} $a$ représenté par un rectangle bleu sur la Figure \ref{fig:electionfunction_jobs} a pour \emph{release date} $r_a = 2$, pour échéance $d_a = 7$, et pour budget d'exécution $c_a = 4$ (indiqué par la longueur du rectangle bleu le représentant). Le \emph{job} $b$ représenté par le rectangle rouge a pour \emph{release date} $r_b = 3$, pour échéance $d_b = 6$, et pour budget d'exécution $c_b = 1$ (indiqué par la longueur du rectangle rouge). La période d'exécution potentielle de chaque \emph{job}, s'étalant de leur \emph{release date} jusqu'à leur échéance, et représentée par les rectangles hachurés de la Figure \ref{fig:electionfunction_jobs}.

			\begin{figure}[!ht]
			\centering{
				\definecolor{job1color}{RGB}{88 ,88,184}
\definecolor{job2color}{RGB}{234,80,90}

\begin{tikzpicture}[>=triangle 45,font=\sffamily, every text node part/.style={align=center}, every node/.style={transform shape}]

	%timeline
	\node (tstart) at (0, 0) {};
	\node (tend) at (9, 0) {};
	\draw[->] (tstart.center) -- (tend.center);
	\node[below=0cm of tend] {temps (u.a.)};

	\node (tjob1arrival) at (2, -0.5) {$r_a$};
	\node (tjob1deadline) at (7, -0.5) {$d_a$};
	\node (job1height) at (2, 1) {};
	\draw (tjob1arrival |- job1height) -- (tjob1deadline |- tend) node [midway, draw, fill=white, minimum width = 5cm, minimum height = 1cm] (job1execperiod) {};
	\node[draw, pattern color=job1color, pattern=south west lines, minimum width = 5cm, minimum height = 1cm] at (job1execperiod) {};
	\draw (tjob1arrival |- job1height) -- (tjob2deadline |- tend) node [midway, draw, fill=job1color, minimum width = 4cm, minimum height = 1cm] (job1) {};

	\node (tjob2arrival) at (3, 2.5) {$r_b$};
	\node (tjob2deadline) at (6, 2.5) {$d_b$};
	\node (job2height) at (2, 2) {};
	\draw (tjob2arrival |- job2height) -- (tjob2deadline |- job1height) node [midway, draw, fill=white, minimum width = 3cm, minimum height = 1cm] (job2execperiod) {};
	\node[draw, fill=white, pattern color=job2color, pattern=south west lines, minimum width = 3cm, minimum height = 1cm] at (job2execperiod) {};

	\node (tjob2end) at (4, 2.5) {};
	\draw (tjob2arrival |- job2height) -- (tjob2end |- job1height) node [midway, draw, fill=job2color, minimum width = 1cm, minimum height = 1cm] (job2) {};
	\node [fill=job1color, text=job2color] at (3.5, 0.5) {\huge{!}};
	\draw[dashed] (tjob1arrival |- tend) -- (tjob1arrival);
	\draw[dashed] (tjob1deadline |- tend) -- (tjob1deadline);
	\draw[dashed] (tjob2arrival) -- (tjob2arrival |- tend);
	\draw[dashed] (tjob2deadline) -- (tjob2deadline |- tend);
	\draw[dashed] (tjob2end) -- (tjob2end |- tend);

	\node (tstart2) at (0, -3.5) {};
	\node (tend2) at (9, -3.5) {};
	\draw[->] (tstart2.center) -- (tend2.center);
	\node[below=0cm of tend2] {temps (u.a.)};

	\node (tjob1arrival2) at (2, -4) {$r_a$};
	\node (tjob1deadline2) at (7, -4) {$d_a$};
	\node (job1height2) at (2, -2.5) {};
	\draw (tjob1arrival2 |- job1height2) -- (tjob1deadline2 |- tend2) node [midway, draw, fill=white, minimum width = 5cm, minimum height = 1cm] (job1execperiod2) {};
	\node[draw, pattern color=job1color, pattern=south west lines, minimum width = 5cm, minimum height = 1cm] at (job1execperiod2) {};

	\node (tjob2arrival2) at (3, -1) {$r_b$};
	\node (tjob2deadline2) at (6,-1) {$d_b$};
	\node (job2height2) at (2, -1.5) {};
	\draw (tjob2arrival2 |- job2height2) -- (tjob2deadline2 |- job1height2) node [midway, draw, fill=white, minimum width = 3cm, minimum height = 1cm] (job2execperiod2) {};
	\node[draw, fill=white, pattern color=job2color, pattern=south west lines, minimum width = 3cm, minimum height = 1cm] at (job2execperiod2) {};

	\node (tjob2end2) at (4, -1) {};
	\draw (tjob2arrival2 |- job2height2) -- (tjob2end2 |- job1height2) node [midway, draw, fill=job2color, minimum width = 1cm, minimum height = 1cm] (job22) {};
	\draw (tjob1arrival2 |- job1height2) -- (tjob2arrival2 |- tend2) node [midway, draw, fill=job1color, minimum width = 1cm, minimum height = 1cm] (job121) {};
	\draw[dashed] (tjob1arrival2 |- tend2) -- (tjob1arrival2);
	\draw[dashed] (tjob1deadline2 |- tend2) -- (tjob1deadline2);
	\draw[dashed] (tjob2arrival2) -- (tjob2arrival2 |- tend2);
	\draw[dashed] (tjob2deadline2) -- (tjob2deadline2 |- tend2);
	\draw[dashed] (tjob2end2) -- (tjob2end2 |- tend2);
	\draw (tjob2end2 |- job1height2) -- (tjob1deadline2 |- tend2) node [midway, draw, fill=job1color, minimum width = 3cm, minimum height = 1cm] (job122) {};
\end{tikzpicture}

				\caption{Ordonnancement des jobs selon la politique d'ordonnancement \emph{Earliest Deadline First}}
				\label{fig:electionfunction_schedule}
			}
			\end{figure}

			La figure \ref{fig:electionfunction_schedule} présente comment la politique va ordonnancer les \emph{jobs} $a$ et $b$. La partie supérieure de la figure présente un plan d'ordonnancement erroné des deux \emph{jobs} : les \emph{jobs} sont exécutés dès l'arrivée de leur \emph{release date}. Cet ordonnancement n'est pas correct, car il est prévu que les deux \emph{jobs} s'exécutent simultanément sur la période suivant la \emph{release date} du \emph{job} $b$ (ce conflit est indiqué par un point d'exclamation rouge). Ainsi, la politique doit choisir entre l'exécution du \emph{job} $a$ et du \emph{job} $b$ sur cette période. La politique \emph{Earliest Deadline First} privilégie alors le \emph{job} ayant l'échéance la plus proche, dans notre exemple le \emph{job} $b$. L'exécution du \emph{job} $a$ est alors suspendue jusqu'à la terminaison du \emph{job} $b$, qui survient avant la période d'exécution suivante puisque son budget d'exécution $c_b$ n'est que d'une seule période d'exécution. L'exécution du \emph{job} $a$ reprend alors, comme indiqué dans la partie inférieure de la figure \ref{fig:electionfunction_schedule}.

			La figure \ref{fig:electionfunction_timeline} montre les résultats retournés par la fonction d'élection lors d'appels à des instants $t$ différents. À $t = 1$, ni le \emph{job} $a$, ni le \emph{job} $b$ ne sont prêts à être exécutés ; la fonction d'élection retourne alors une valeur nulle. À $t = 2$, la \emph{release date} du \emph{job} $a$ est atteinte, ainsi il est prêt à être exécuté ; l'appel à la fonction d'élection retourne le \emph{job} $a$. À $t = 3$, la \emph{release date} du \emph{job} $b$ est atteinte, ainsi le \emph{job} $a$ et le \emph{job} $b$ sont prêts à être exécutés. C'est alors que le rôle de la fonction d'élection prend son sens et choisi le \emph{job} $b$, conformément à la politique \emph{Earliest Deadline First}. À $t = 4$, le \emph{job} $b$ a terminé son exécution : il a épuisé son budget d'exécution $c_b$. Ainsi, la fonction d'élection retourne le \emph{job} $a$, seul \emph{job} pouvant encore être exécuté.

			\begin{figure}[!ht]
			\centering{
				\definecolor{job1color}{RGB}{88 ,88,184}
\definecolor{job2color}{RGB}{234,80,90}

\begin{tikzpicture}[>=triangle 45,font=\sffamily, every text node part/.style={align=center}, every node/.style={transform shape}]

	\node (tstart0) at (-1, 0) {};
	\node (tend0) at (9, 0) {};
	\node (t0) at (0, 1) {};
	\draw[->] (tstart0.center) -- (tend0.center);
	\draw[semithick] (t0.center) -- (t0 |- tend0);
	\node[above left=0.2cm of tstart0] {$t=0$};

	\node (tstart1) at (-1, -2) {};
	\node (tend1) at (9, -2) {};
	\node (t1) at (0, -1) {};
	\node[above=0.0cm of t1] {\texttt{Fonction d'élection (1) = $\emptyset$}};
	\draw[->] (tstart1.center) -- (tend1.center);
	\draw[semithick] (t1.center) -- (t1 |- tend1);
	\node[above left=0.2cm of tstart1] {$t=1$};

	\node (tstart2) at (-1, -4) {};
	\node (tend2) at (9, -4) {};
	\node (t2) at (0, -3) {};
	\node[above=0.0cm of t2] {\texttt{Fonction d'élection (2) = }Job $a$};
	\draw[->] (tstart2.center) -- (tend2.center);
	\draw[thick] (t2.center) -- (t2 |- tend2);
	\node[above left=0.2cm of tstart2] {$t=2$};
	\draw (t2.center) -- (4, -4) node [midway, fill=white, minimum width = 4cm, minimum height = 1cm] (jobaexecperiod) {};
	\node[draw, pattern color=job1color, pattern=south west lines, minimum width = 4cm, minimum height = 1cm] at (jobaexecperiod) {};

	\node (tstart3) at (-1, -6) {};
	\node (tend3) at (9, -6) {};
	\node (t3) at (0, -5) {};
	\node[above=0.0cm of t3] {\texttt{Fonction d'élection (3) = }Job $b$};
	\draw[->] (tstart3.center) -- (tend3.center);
	\draw[semithick] (t3.center) -- (t3 |- tend3);
	\node[above left=0.2cm of tstart3] {$t=3$};

	\node (tstart4) at (-1, -8) {};
	\node (tend4) at (9, -8) {};
	\node (t4) at (0, -7) {};
	\node[above=0.0cm of t4] {\texttt{Fonction d'élection (4) = }Job $a$};
	\draw[->] (tstart4.center) -- (tend4.center);
	\draw[semithick] (t4.center) -- (t4 |- tend4);
	\node[above left=0.2cm of tstart4] {$t=4$};
	\node[below=0.2cm of tend4] {temps (u.a.)};
	%\node[draw, fill=job1color, minimum height = 1cm, minimum width = 4cm] (job1) at (3.5, 4) {};
	%\node[left=0.3cm of job1] {Job \small{$a$}};
	%\draw[<->] (1.5, 3.3) -- (5.5, 3.3);
	%\node[below=0.2cm of job1] {$c_a$};
	%\node[draw, fill=job2color, minimum width = 1cm, minimum height = 1cm] (job2) at (7, 4) {};
	%\node[right=0.3cm of job2] {Job \small{$b$}};
	%\draw[<->] (6.5, 3.3) -- (7.5, 3.3);
	%\node[below=0.2cm of job2] {$c_b$};

	%\node (tjob1arrival) at (2, 1.5) {$r_a$};
	%\node (tjob1arrival2) at (2, -0.5) {$t=2$};
	%\draw[dashed] (tjob1arrival) -- (tjob1arrival2);
	%\node (tjob1deadline) at (7, 1.5) {$d_a$};
	%\node (tjob1deadline2) at (7, -0.5) {$t=7$};
	%\draw[dashed] (tjob1deadline) -- (tjob1deadline2);
	%\node (job1height) at (2, 1) {};
	%\draw (tjob1arrival |- job1height) -- (tjob1deadline |- tend) node [midway, draw, fill=white, minimum width = 5cm, minimum height = 1cm] (job1execperiod) {};
	%\node[draw, pattern color=job1color, pattern=south west lines, minimum width = 5cm, minimum height = 1cm] at (job1execperiod) {};

	%\node (tjob2arrival) at (3, 2.5) {$r_b$};
	%\node (tjob2arrival2) at (3, -1) {$t=3$};
	%\draw[dashed] (tjob2arrival) -- (tjob2arrival2);
	%\node (tjob2deadline) at (6, 2.5) {$d_b$};
	%\node (tjob2deadline2) at (6, -1) {$t=6$};
	%\draw[dashed] (tjob2deadline) -- (tjob2deadline2);
	%\node (job2height) at (2, 2) {};
	%\draw (tjob2arrival |- job2height) -- (tjob2deadline |- job1height) node [midway, draw, fill=white, minimum width = 3cm, minimum height = 1cm] (job2execperiod) {};
	%\node[draw, pattern color=job2color, pattern=south west lines, minimum width = 3cm, minimum height = 1cm] at (job2execperiod) {};
\end{tikzpicture}

				\caption{Illustration d'un appel à la fonction d'élection sur les \emph{jobs} $a$ et $b$}
				\label{fig:electionfunction_timeline}
			}
			\end{figure}

		\subsection{Place de l'ordonnancement dans Pip}

		Cet ordonnanceur ne s'exécute pas en espace privilégié, comme c'est le cas dans les systèmes d'exploitation traditionnels tels que Linux ou même dans des systèmes d'exploitation de niche tels que seL4. 

		La première raison en faveur d'une implémentation de l'ordonnanceur dans une partition de Pip concerne la méthode de conception des logiciels autour de Pip. En effet, placer l'ordonnanceur en espace utilisateur lui fait implicitement profiter des propriétés d'isolation des partitions de Pip. Ce choix est en adéquation avec la vision de Pip du système, conçu comme une tour de virtualisation. Une tour de virtualisation dont chaque étage devient partie prenante de la base de confiance (TCB) des étages supérieurs. Dans le modèle de Pip, les différents logiciels se construisent les uns au dessus des autres, profitant des propriétés prouvées sur le logiciel sous-jacent.

		La seconde raison est une conséquence directe de ce choix, relatif à l'effort de preuve. Si l'ordonnanceur avait été implémenté dans Pip, alors il aurait fallu montrer que l'ordonnanceur respecte les propriétés d'isolation de Pip. À titre de comparaison, la preuve de préservation de l'isolation initialement écrite sur le service décrit dans le chapitre précédent a nécessité plus de 2500 lignes de preuves, alors même que le service ne modifie pas les structures du noyau, et n'introduit pas de nouvelle propriété de consistence. Produire une telle preuve sur l'ordonnanceur aurait sans doute requis un effort d'un tout autre ordre de grandeur, simplement pour prouver la propriété d'isolation. De plus, l'ordonnanceur développé dans cet contribution a été conçu autour de la politique d'ordonnancement EDF. Ce choix est arbitraire, et il serait tout à fait pertinent de le munir d'une autre politique d'ordonnancement. Ainsi, cet effort aurait été requis pour chaque autre variante de l'ordonnanceur engendrée par chaque politique souhaitée -- encore une fois uniquement pour garantir l'isolation. Cette propriété est \emph{donnée} dès lors que l'ordonnanceur est sorti du code du noyau.

		\subsection{Décomposition des éléments de l'ordonnanceur}
		\label{sec:project_overview}
			\subsubsection{Vue générale}
			\begin{figure}[!ht]
			    \centering
			    \begin{tikzpicture}[>=triangle 45,font=\sffamily, every text node part/.style={align=center}, scale=0.8, every node/.style={transform shape}]

	\node (task1) at (-8, 2) [fill=white, draw,  semithick, minimum height=3cm, minimum width=2cm] {\emph{Job}};
    \node[below=0.75cm of task1] {$\vdots$};
	\node (task2) at (-8, -3) [fill=white, draw,  semithick, minimum height=3cm, minimum width=2cm] {\emph{Job}};
    
    \node (kernel) at (-4, -6) [fill=white, draw, semithick, minimum height=2cm, minimum width=3cm] {Noyau};
    
    \node (scheduler) at (-0.25, -0.5) [fill=white, draw, semithick, minimum height=8cm, minimum width=11.5cm, pattern=south west lines] {};
    \node[above=0.2cm of scheduler] {\large{Ordonnanceur}};
    
	\node (back-end) at (-4, -0.5) [fill=white, draw, semithick, minimum height=7cm, minimum width=3cm] {\emph{Back-end}};
    
    \node (election_function) at (1.5, 2) [fill=white, draw, semithick, minimum height=2cm, minimum width=6cm] {Fonction d'élection vérifiée};
    
    \node (oracles) at (-0.5, -0.5) [fill=white, draw, semithick, minimum height=1.2cm, minimum width=3cm] {Oracles};
    
    \node (interface) at (3.5, -0.5) [fill=white, draw, semithick, minimum height=1.2cm, minimum width=3cm] {Interface avec l'état};
    
    \node (mutable_state) at (3, -3) [fill=white, draw, semithick, minimum height=1.5cm, minimum width=3cm] {État mutable};
    
    \node (immutable_state) at (0, -3) [fill=white, draw, semithick, minimum height=1.5cm, minimum width=3cm] {Environnement};
    
    \draw [->] (kernel) -- (back-end);

    \draw [->] (back-end.301) -- (immutable_state.west);
    \draw [->] (back-end.59) -- (election_function);
    \draw [->] (back-end) -- (task1);
    \draw [->] (back-end) -- (task2);
    
    \draw [->] (oracles) -- (immutable_state);
    \draw [->] (interface) -- (mutable_state);
    \draw [->] (election_function) -- (oracles);
    \draw [->] (election_function) -- (interface);

\end{tikzpicture}

			    \caption{Vue générale de l'ensemble des composants de l'ordonnanceur et de leurs interactions}
			    \label{fig:project_overview}
			\end{figure}


			La figure \ref{fig:project_overview} donne une vue générale des composants de l'ordonnanceur et de leurs interactions. Le \emph{back-end} est le point d'entrée de l'ordonnanceur. Il est appelé par le noyau lorsqu'une interruption survient, appelle la fonction d'élection puis exécute le \emph{job} choisi par la fonction d'élection. Le \emph{back-end} doit aussi mettre à jour une partie de l'état sur laquelle repose la fonction d'élection, que nous appelons l'environnement. L'environnement est la partie de l'état disponible en lecture seule à la fonction d'élection, que la fonction d'élection peut interroger au travers des \emph{oracles}. La seconde partie est la partie mutable de l'état, que la fonction d'élection ne peut manipuler qu'au travers de l'\emph{interface}, principalement composée de types de données et de fonction pour les manipuler.

		\subsubsection{État interne de l'ordonnanceur}

			L'état interne de l'ordonnanceur est chargé de stocker les informations nécessaires à la fonction d'élection pour calculer le prochain \emph{job} à élire. Il existe deux représentations de cet état : une implémentation exécutable, et un modèle utilisé pour la preuve.

			L'état est -- de sucroit -- divisé en deux parties, qui se distinguent par leur fonction par rapport à la fonction d'élection. La première partie de l'état est maintenue par la fonction d'élection, qui peut y lire et y écrire des données à sa guise en utilisant l'interface de l'état prévue à cet effet. Cette partie de l'état est appelée l'état mutable. La seconde partie est la partie de l'état que la fonction d'élection ne peut qu'observer, qui est appelé l'environnement. L'environnement contient les informations relatives aux évènements extérieurs essentiels au bon fonctionnement de l'ordonnanceur, tels que l'arrivée de nouveaux \emph{jobs} à ordonnancer, ou la complétion d'un \emph{job} précédemment exécuté. Cette partie de l'état est donc accessible en lecture seule à la fonction d'élection. La fonction d'élection peut consulter l'environnement aux travers d'oracles, qui servent d'interface avec l'environnement.

		\subsubsection{Interface avec l'état -- Oracles, Types opaques et fonctions associées}
			
			Comme indiqué dans la sous-section précédente, la fonction d'élection effectue des modifications sur l'état ou observe l'environnement grâce à l'interface et aux oracles. Cette interface est \emph{nécessaire} car elle permet de définir les opérations qui doivent être supportées par l'état, et donc par ses deux représentations (le code exécutable et le modèle). Sans cette interface, certaines opérations pourraient être possible dans une des deux représentations sans qu'elle soit possible dans l'autre.
			L'interface ainsi que les oracles sont composés de types opaques, implémentés de manière indépendante dans les modèles ou dans le code exécutable. Ces types sont munis de fonctions elles aussi pourvues d'une implémentation exécutable ainsi que d'un modèle.

			Les oracles sont une partie de l'interface spécifique. Les oracles sont la partie de l'interface qui permet à la fonction d'élection d'observer les phénomènes extérieurs, plus précisément la terminaison du \emph{job} précédemment exécuté et la liste des nouveaux \emph{jobs} à ordonnancer. Bien qu'ils disposent aussi de deux représentations, leur modèle ne décrit pas réellement les résultats que l'oracle doit produire, mais plutôt des contraintes sur les résultats produits. Le modèle va se contenter de décrire l'ensemble des résultats possibles, plutôt que d'en désigner un en particulier. Par exemple, une des contraintes sur la liste des nouveaux jobs est que leur date d'arrivée ne peut pas être supérieure à la date courante. 

		\subsubsection{Back-ends}
		\label{sec:back-ends}

		Nous avons vu que l'état est composé en partie d'une partie non mutable, appelée l'environnement. Les informations contenues dans l'environnement sont mises à jour par le \emph{back-end}. Le \emph{back-end} est aussi chargé d'appeler la fonction d'élection et d'exécuter le \emph{job} qui aura été élu.

		Cette contribution est munie de deux \emph{back-ends} à choisir au moment de la compilation de l'ordonnanceur.
		\begin{itemize}
			\item Le premier back-end est conçu comme une simulation permettant d'exécuter la fonction d'élection dans un simple processus Linux, afin de fournir des informations sur l'état interne de l'ordonnanceur. Ce back-end appelle la fonction d'élection, affiche le \emph{job} choisi pour exécution et autres informations d'intérêt sur l'état interne de la fonction d'élection tels que la liste des \emph{jobs} en attente, etc ;

			\item Le second back-end fourni est l'implémentation d'un ordonnanceur Earliest Dealine First dans une partition de Pip. Cette implémentation exécute la fonction d'élection et transfère le flot d'exécution vers le \emph{job} élu par la fonction d'élection. Chaque job se trouve dans sa propre partition, avec un espace d'adressage qui lui est propre. Le flot d'exécution revient au back-end soit lorsqu'une interruption d'horloge survient, interrompant le \emph{job} élu, soit lorsque le \emph{job} lui signale sa terminaison. Le back-end attend alors la prochaine interruption pour réélire un nouveau \emph{job}.
		\end{itemize}

		Les back-ends ne font pas partie du modèle mathématique de l'ordonnanceur, si ce n'est au travers des \emph{oracles}. Ils ne font pas partie de la politique d'ordonnancement ; ils appliquent simplement ses arbitrages et lui fournissent les informations dont elle a besoin.

		\subsubsection{Fonction d'élection}

		La fonction d'élection est divisée en deux parties principales. La première partie maintient la liste des \emph{jobs} en attente et sélection le prochain \emph{job} à exécuter, la seconde maintient la cohérence de son état interne.

		Dans la première partie, la fonction d'élection invoque un oracle, qui lui retourne la liste des nouveaux \emph{jobs} à ordonnancer. Ces jobs sont ajoutés à la liste des jobs à ordonnancer. Ensuite, elle invoque un autre oracle qui lui renvoie l'état du dernier job exécuté, afin de vérifier s'il a complété son exécution ou s'il a dépassé son échéance. S'il remplit l'une de ces deux conditions, il est enlevé de la liste des jobs à ordonnancer. Si il a excédé son échéance, l'erreur sera remontée lors du retour de la fonction. Enfin, la fonction récupère l'identifiant du premier \emph{job} de la liste (si la liste n'est pas vide) et le retourne avec le drapeau indiquant si le dernier \emph{job} exécuté a dépassé son échéance. Une valeur par défaut est retournée si aucun \emph{job} n'est disponible.

		La seconde partie de la fonction d'élection maintient la cohérence de l'état interne de la fonction d'élection pour les futurs appels. Elle commence par décrémenter le budget d'exécution du \emph{job} élu. Elle décrémente ensuite l'échéance relative des \emph{jobs} en attente, pour tenir compte du temps qui sera passé à exécuter le \emph{job} élu. Finalement, la fonction d'élection incrémente le compteur de temps, qui garde le compte de périodes de temps s'étant écoulées depuis le premier appel à la fonction d'élection.

	\section{Conduite de la preuve}

	\label{sec:proof}
	Cette section met en avant la preuve fonctionnelle de la fonction d'élection de l'ordonnanceur EDF, écrit sous la forme d'un programme monadique dans l'assistant de preuve Coq.

	Cette section commencera par décrire la méthodologie de preuve, qui utilise un raffinement en trois niveaux d'abstraction. La section décrira ensuite les différents niveaux d'abstraction et le raffinement qui les relie, en détaillant particulièrement pourquoi le raffinement préserve les propriétés de correction du niveaux d'abstraction supérieur. Enfin, la section conclura sur quelques détails d'élaboration de la preuve en Coq.

	\subsection{Méthodologie de preuve}

	Cette preuve est conduite par raffinement. Cette méthode permet d'appliquer la stratégie ``diviser pour régner'' pour gérer la complexité de la preuve.
	Cette preuve commence par une définition abstraite de la politique d'ordonnancement Earliest Deadline First, et montre que la politique est \emph{correcte}. Cette étape signifie que, sous réserve de quelques hypothèses, n'importe quel ensemble de \emph{jobs} ordonnançable ordonnancé selon cette politique sera ordonnancé tel que chaque \emph{job} complètera son exécution sans excéder son échéance.
	La preuve continue en définissant un algorithme de fonction d'élection idéalisé, et montre que cet algorithme implémente la politique d'ordonnancement définie précédemment. Il suit que l'algorithme idéalisé est aussi \emph{correct}.
	L'étape finale de la preuve consiste à écrire la fonction d'élection qui sera finalement compilée en code source C. Il faut ensuite montrer que la fonction d'élection qui sera traduite en C exhibe les mêmes comportements que l'algorithme idéalisé de la fonction d'élection (en particulier les effets de bord et les valeurs de retour). La preuve est alors complète, la preuve de correction a été propagée jusqu'au code C.
	%The properties proved in our work act as layered invariants that extend from the EDF policy model to the monadic code. To prove a new refinement layer, we start from the invariant of the previous (more abstract) layer and show that the new invariant \emph{refines} the previous one, i.e., the new invariant has a more detailed specification that remains compatible with its more abstract version.
	%All the proofs that we performed rely on the following assumptions.

	\subsubsection{Modèle de \emph{job}}
	\label{sec:jobmodel}
	Les \emph{jobs} sont modélisés comme des tuples $j = (i_j, r_j,d_j,c_j, \delta_j)$ d'entiers naturels, où $i_j$ est l'identifiant du \emph{job}, $r_j$ représente l'instant à partir duquel il est possible de l'ordonnancer (\emph{release date}), $d_j$ représente son échéance (\emph{deadline}), et $c_j$ représente son budget temporel, qui est une borne supérieure sur sa durée d'exécution. $\delta_j$ représente sa durée d'exécution maximum réelle (Worst Case Execution Time -- \emph{WCET}).

	Les \emph{jobs} doivent respecter certaines contraintes de construction :
	\begin{itemize}
		\item $r_i + c_i \leq d_i$, l'échéance est assez tardive pour que le \emph{job} puisse terminer son exécution sans la dépasser s'il est le seul à s'exécuter sur le processeur ;
		\item $0 < \delta_j \leq c_j$, le budget temporel alloué est bien une borne supérieure au temps d'exécution maximum réel (WCET), lui-même strictement positif ;
		\item l'identifiant de chaque \emph{job} doit être unique ;
		\item il n'existe qu'une seule occurrence de chaque \emph{job}.
	\end{itemize}

	\subsubsection{Hypothèse d'ordonnançabilité}

	Soient deux instants quelconques $t$ et $t'$ tels que $t < t'$, soit $\Gamma_{t, t'}$ l'ensemble des \emph{jobs} à ordonnancer dans l'intervalle $[t, t']$ (c'est à dire les \emph{jobs} $j$ ayant une release date $r_j$ supérieure à $t$ et une échéance $d_j$ inférieure à $t'$).
	L'hypothèse d'ordonnançabilité stipule que la somme des durées $\delta_j$ des \emph{jobs} à ordonnancer dans l'intervalle $[t, t']$ est inférieure à la durée $t'-t$ de l'intervalle.

	\begin{gather} \label{eq:schedulability}
    		\forall t, t'. ~t < t' \implies
    		\sum_{j \in \Gamma_{t, t'}}\delta_j \leq t' - t
	\end{gather}
%% \text{with:}
%% \begin{align*}
%%               \rho, \delta&\text{, arbitrary time instants},\\
%%               \Gamma_{\rho, \delta}&\text{, the set of jobs to schedule in the interval $[\rho, \delta]$},\\
%%               J_i&\text{, a job from }\Gamma_{\rho, \delta},\\
%%               r_i,~d_i,~C_i &\text{, respectively the release time, the deadline,}\\&\text{~~and the WCET of }J_i
%% \end{align*}
	Le fait qu'un ensemble de jobs ordonnançable puisse être ordonnancé par un ordonnanceur Earliest Deadline First sur un unique processeur est un résultat classique de la théorie de l'ordonnancement (\cite{stankovic2012deadline}, pp. 33-34). Les conditions d'ordonnançabilité d'un ensemble peuvent parfois être représentées plus simplement, par exemple lorsque les \emph{jobs} sont des instances de tâches périodiques. Cependant cette contribution porte sur un ordonnanceur de \emph{jobs arbitraires}, ce qui ne nous permet pas d'utiliser les hypothèses d'ordonnançabilité de cas spécifiques.

	\subsection{Couches d'abstraction et étapes de raffinement}

	\subsubsection{Politique d'ordonnancement Earliest Deadline First}
	\label{sec:policy}
	La première couche d'abstraction, la plus abstraite de toutes, est celle de la politique d'ordonnancement. Elle est définie de la manière suivante : pour tout \emph{job} $j$, et à n'importe quel instant $t$, si le \emph{job} $j$ s'exécute à l'instant $t$, alors pour chaque autre \emph{job} $j'$ qui aurait pu être exécuté à la place, on a $d_j \leq d_{j'}$.

	De manière formelle,
	\begin{equation}
		\forall j, \forall t, [...]~\mathtt{run}~j~t~~\implies~~\forall j',~\mathtt{waiting}~j'~t~~\implies~~d_j \leq d_{j'}.
	\end{equation}
\indent avec :
	\begin{align*}
		j, j'&\text{, deux identifiants de job},\\
		t&\text{, un instant arbitraire},\\
		\mathtt{run}~i~t&\text{, prouvable si le \emph{job} $j$ s'exécute à l'instant $t$},\\ 
		\text{waiting}~x~t&\text{, prouvable si le \emph{job} $j'$ est prêt à s'exécuter à l'instant $t$}
	\end{align*}

	La \emph{propriété de correction} pour la politique d'ordonnancement EDF est décrite de la manière suivante.
	Étant donné un ensemble ordonnançable de \emph{jobs} (c'est à dire satisfaisant l'hypothèse d'ordonnancement \ref{eq:schedulability}), si la politique d'ordonnancement EDF est appliquée, alors aucun \emph{job} de l'ensemble ne dépassera son échéance.
	
	\begin{equation}
	\begin{split}
		\textit{or}&\textit{donnançable} \implies\\
		\forall j, \forall t.~~&~\text{EdfPolicyUpTo}~t \implies\\
		           &\neg \text{overdue}~j~t.
	\end{split}
	\end{equation}
\indent	avec :
	\begin{align*}
		\text{EdfPolicyUpTo}~t&\text{ signifie que la politique a été appliquée jusqu'à l'instant}~t\\
		\text{overdue}~j~t&\text{, prouvable si le \emph{job} $j$ a dépassé son échéance à l'instant}~t
	\end{align*}

	\subsubsection{Fonction d'élection idéalisée implémentant la politique d'ordonnancement}
	\label{sec:functional}

	La première étape du raffinement est de montrer que la fonction d'élection idéalisée \texttt{functional\_scheduler\_star} implémente la politique d'ordonnancement EDF. La propriété suivante établie que si la fonction d'élection idéalisée est exécutée jusqu'à un certain instant $t$, alors la politique EDF a été appliquée jusqu'à cet instant.
	\begin{equation}
	\begin{split}
		\forall t, \forall o, \forall s.~~~\texttt{funct}&\texttt{ional\_scheduler\_star}~(t) = (o,s) \implies\\
		&\text{EdfPolicyUpTo}~(\text{now}~s).
	\end{split}
	\end{equation}
\indent	avec :
	\begin{align*}
		s&\text{, l'état du programme après avoir exécuté la fonction}\\
		&~~\text{d'élection idéalisée après $t$ unités de temps},\\
		%\texttt{functional\_scheduler\_star}~(t)&\text{, appel à la fonction d'élection idéalisée pour $t$ unités de temps}\\
		\text{now}~s&\text{, extrait le compteur d'unité de temps de l'état $s$}\\
		o~&\text{, l'identifiant du \emph{job} élu à l'instant $t$}
	\end{align*}

	De cela, on peut déduire la propriété de correction de la fonction d'élection idéalisée :
	étant donné un ensemble de \emph{job} ordonnançable, alors pour tout \emph{job} $j$ de cet ensemble, ce \emph{job} ne dépassera pas son échéance à l'instant $t$ si la fonction a été appelée à chaque instant jusqu'à l'instant $t$

	\begin{equation}
	\begin{split}
		&\textit{ordonnançable} \implies\\
		\forall t, \forall o, \forall s.~~~\texttt{functio}&\texttt{nal\_scheduler\_star}~(t) = (o,s) \implies\\
		\forall i.&~~~\neg \text{overdue}~i~(\text{now}~s).
	\end{split}
	\end{equation}

	\subsubsection{Fonction d'élection monadique raffinant la fonction d'élection idéalisée}
	\label{sec:monadic}

	La prochaine étape consiste à prouver que la fonction d'élection monadique raffine la fonction d'élection idéalisée. Cette propriété contient des triplets de Hoare. 
	Les préconditions de la propriété sont paramétrés par un environnement observable \texttt{env} et par un état mutable $s$. Les postconditions sont paramétrées par la valeur de retour $o$ du programme $x$, et un état $s'$, le résultat de l'exécution du programme $c$ sur l'état $s$.

	Cette étape de raffinement peut être décrite de la manière suivante : pour tout instant $t$, si $(o, s')$ sont les valeurs retournées par la fonction d'élection monadique \texttt{scheduler\_star} sur l'environnement $E$ et l'état initial \texttt{init}, alors $(o, s')$ sont aussi les valeurs retournées par la fonction d'élection idéalisée \texttt{functional\_scheduler\_star} à l'instant $t$. Le triplet de Hoare correspondant est le suivant :
	\begin{equation}
	\begin{split}
		\forall& t.\\
		\{
		~\lambda~\mathtt{env}~s.~~~\mathtt{env} =&~E~~\land~~s = \mathtt{init}~
		\}\\
		\texttt{schedule}&\texttt{r\_star}~(t)\\
		\{
		~\lambda~o~s'.~~~\texttt{functional\_sch}&\texttt{eduler\_star}~(t) = (o,s')~
		\}
	\end{split}
	\end{equation}

	À partir de ce triplet, on peut déduire le triplet de Hoare exprimant la propriété de correction de la fonction d'élection monadique.

	\begin{equation}
	\begin{split}
		\textit{ordonnan}&\textit{çable} \implies\\
		\forall& t.\\
		\{
		~\lambda~\mathtt{env}~s.~~~\mathtt{env} =& ~E~~\land~~s = \mathtt{init}~
		\}\\
		\texttt{schedule}&\texttt{r\_star}~(t)\\
		\{
		~\lambda~\_~s'.~~~\forall i.~~~\neg \text{ov}&\text{erdue}~i~(\text{now}~s') ~
		\}
	\end{split}
	\end{equation}

	\subsection{Hypothèses et déroulement de la preuve en Coq}

	Les détails de la preuve sont disponibles sur le dépot, et nous avons décrit la structure principale de la preuve. Nous allons ici développer les faits les plus caractéristiques de cette preuve, donnant une vision plus précise de son déroulement. 

	Les preuves Coq suivent de près l'approche par raffinement décrite dans la sous-section précédente. Tout d'abord, le modèle choisi pour les jobs, décrit sous forme mathématiquement dans la section \ref{sec:jobmodel} est implémenté dans l'assistant de preuve. Les hypothèses listées dans la même section \ref{sec:jobmodel} sont des hypothèses globales, qui s'appliquent donc à l'ensemble de la preuve. Il est facile d'argumenter que ces hypothèses sont des hypothèses raisonnables sur l'ensemble étudié, c'est à dire qui ne restreignent pas les ensembles considérés, ou sans lesquelles l'ordonnancement d'un tel ensemble est impossible.

	La première couche d'abstraction modélisant la politique d'ordonnancement nécessite des hypothèses supplémentaires. Ces hypothèses -- appelées hypothèses \emph{locales} -- ne sont utilisées que pour établir la preuve de correction de la politique. Elles disparaissent dès le premier raffinement vers la fonction d'élection idéalisée ; ces hypothèses deviennent alors des définitions et des lemmes Coq.

	On suppose localement que deux fonctions sont fournies avec la politique : \texttt{run} qui permet de déterminer quel \emph{job} sera exécuté à n'importe quel instant, et \texttt{rem}, qui garde le compte du temps d'exécution restant de chaque \emph{job} à n'importe quel moment.
	Ces fonctions ont besoin elles aussi d'hypothèses locales supplémentaires, qui représentent des propriétés générales sur l'ordonnancement mono-processeur :
	
	\begin{itemize}
		\item à chaque instant, au maximum un \emph{job} est exécuté;
		\item le temps d'exécution restant d'un \emph{job} est égal à sa durée au moins jusqu'à sa \emph{release date};
		\item le temps d'exécution restant d'un \emph{job} diminue lorsqu'il est exécuté, et reste constant lorsqu'il ne l'est pas;
		\item il y a un \emph{job} qui s'exécute à chaque instant où au moins un \emph{job} est prêt à être exécuté.
	\end{itemize}

	\label{sec:proof_insight}

	Le premier raffinement représente la plus grosse portion de la preuve. Ce raffinement consiste à définir les fonctions abstraites \texttt{run} et \texttt{rem} supposées fournies dans la couche d'abstraction de la politique, puis à prouver les hypothèses localement supposées sur ces fonctions.
		Une fois ces hypothèses prouvées, la propriété de correction de la fonction d'élection idéalisée est une simple conjonction entre la propriété de correction de la politique d'ordonnancement et la propriété de raffinement. Il est plus intéressant de prouver une relation d'implémentation plutôt que de prouver directement la propriété de correction de la fonction d'élection idéalisée, car les obligations de preuve de la preuve par raffinement sont plus concrètes. La preuve par raffinement nécessitant l'établissement d'abstractions avant son écriture, elle nous propose des termes de preuves plus directs que ceux qui auraient été proposés par la preuve directe.

	Toutes les propriétés à prouver à ce niveau de raffinement sont des \emph{invariants} -- des prédicats qui doivent être vrais sur tous les états atteignables depuis l'état initial, obtenus en exécutant un nombre arbitraire de fois la fonction d'élection idéalisée. La preuve de certains de ces invariants a cependant nécessité la conception d'invariants \emph{inductifs} supplémentaires, vrais sur l'état initial, et préservés par chaque exécution subséquente de la fonction d'élection idéalisée. Ces invariants ont été conçus de telle manière à ce qu'ensembles, ils impliquent l'invariant initial à prouver.

	La dernière étape de raffinement permet de montrer que la fonction d'élection exécutable raffine la fonction d'élection idéalisée. Elle consiste à montrer que l'ensemble des états atteignables par la fonction d'élection exécutable est inclus dans l'ensemble des états atteignables par la fonction d'élection idéalisée. 
	La différence notable entre les deux fonctions d'élection est que la fonction d'élection idéalisée calcule en une grosse itération ce que la fonction d'élection exécutable calcule en plusieurs petites itérations. La fonction d'élection idéalisée n'est pas assujettie à de telles contraintes et peut de ce fait factoriser ces longues séquences de petites itérations en une seule grosse itération. Ce faisant, la fonction d'élection idéalisée nous avait épargné la preuve d'invariants supplémentaires à propos des états intermédiaires générés par la fonction d'élection exécutable, qu'il a fallu prouver dans cette étape de raffinement.

	La conclusion globale indiquant que la fonction d'élection exécutable est correcte découle de la propriété de correction de la politique d'ordonnancement et des preuves des deux raffinements successifs.\\

	Ainsi s'achève la section sur la conduite de la preuve de correction de la fonction d'élection de notre ordonnanceur. La prochaine section décrira l'implémentation de l'interface et de l'état interne de l'ordonnanceur.

	\section{Implémentation et modèle de l'état et de son interface}

	\label{sec:sched_impl}

	Cette section commence par rappeler le rôle de la monade d'état dans le monde mathématique et de la nécessité de son interface. Elle illustre ensuite le problème de \emph{dualité} des modèles et de l'objet réel modélisé avec un exemple s'étant présenté pendant l'implémentation.
	Dans une seconde partie, cette section présente notre implémentation et les modèles relatifs à l'état et aux fonctions de l'interface. Elle décrit l'implémentation de l'interface avec la monade d'état, qui permet à la fonction d'élection d'intéragir avec l'état interne du programme.
		\subsection{Dualité implémentation/modélisation}
		\label{sec:model_distortion}
			La monade d'état représente l'état du programme dans le monde fonctionnel du modèle qui en est autrement dépourvu. Les fonctions qui permettent d'interagir avec la monade d'état constituent son interface. Lors de la compilation du code Gallina vers du code C, la monade d'état disparaît, pour laisser place à l'état intrinsèque du monde impératif.
			Il existe donc deux penchants de l'interface : le modèle, intéragissant avec la monade d'état, et son implémentation, intéragissant directement avec l'état intrinsèque du programme. Cette interface est supposée correcte : on suppose que l'appel à une fonction de l'interface a les mêmes effets dans le monde mathématique et dans le monde réel. La méthode usuelle pour minimiser l'impact de cette supposition est de créer une interface constituée de fonctions dont les effets sur l'état sont les plus simples possibles, idéalement de l'ordre d'une simple instruction assembleur.
			Malheureusement, cette méthode est souvent en tension avec l'effort de preuve, qui augmente de manière exponentielle lorsque les modèles se rapprochent de plus en plus du matériel. Ainsi, chaque modèle trouve son propre point d'équilibre entre coût de la preuve et risque de distorsion entre le modèle et la réalité.

			Pour illustrer ce propos, voici un exemple relatif à l'établissement de la preuve décrite dans la section précédente. Afin de simplifier grandement la preuve, notre monade d'état contient des listes. Le modèle de l'interface repose de ce fait sur les listes fonctionnelles du langage, qui créent des copies à chaque opération et qui ne manquent jamais de mémoire. Ce type de liste \emph{idéalisé} ne peut être implémenté qu'avec des logiciels ad-hoc lourds comme, par exemple, les ramasse-miettes. Incorporer de tels logiciels dans la base de confiance est en nette contradiction avec notre méthodologie.
			Nous avons de ce fait décidé d'opter pour une interface utilisant des listes avec modification en place, de telle sorte à fournir une interface qui soit pratique à utiliser dans le code monadique, dont le modèle fonctionnel est simple du point de vue de la preuve, et qui soit facile à implémenter en C. Ce choix nous contraints néanmoins à ne jamais laisser l'utilisateur manipuler les objets de listes directement, sous peine de créer une distorsion entre la réalité et le modèle.

Pour illustrer cette distorsion, prenons par exemple la séquence d'instructions idéalisée suivante :
\begin{verbatim}
do liste_initiale <- get_liste_initiale;
do liste_modifiee <- modifie_liste liste_initiale;
do queue_liste_initiale <- get_tail liste_initiale;
\end{verbatim}

Dans le monde fonctionnel, l'interface utilise les listes \emph{idéalisées} qui retournent des \emph{copies} (éventuellement modifiées) de la liste initiale. La variable \texttt{queue\_liste\_initiale} contient effectivement une référence vers une copie de la queue de la liste initiale. Cependant, l'implémentation réelle de l'interface utilise une implémentation de liste sans copie, avec une modification des listes en place. La variable \texttt{queue\_liste\_initiale} contient une référence vers la queue de la seule liste existante, c'est à dire celle de la liste modifiée.

Afin d'éviter de s'exposer à ce problème de distortion, notre interface ne fournit donc pas d'objet de liste, mais des fonctions pour modifier les listes contenues dans l'état. Cette approche a cependant le désavantage de créer des primitives non-triviales, comme par exemple l'insertion triée dans une liste. Nous reviendrons sur ce choix d'inclusion de fonctions non-triviales dans l'interface (et donc dans la base de confiance) à la fin de ce chapitre, dans la section \ref{sec:interfaceTCB}. 
		
		\subsection{Implémentation vue comme un cas particulier de l'interface abstraite}
	\label{sec:monad}

	Cette sous-section détaille comment l'interface du code vérifié avec l'état a été conçue, décrivant à la fois les modèles et leur implémentation.
	Pour rappel, les différentes parties de l'interface et leurs interactions sont représentées dans la section \ref{sec:project_overview}, dans la figure \ref{fig:project_overview}.

	Cette section décrira d'abord l'état interne de la fonction d'élection, composé de deux parties : une partie en lecture seule appelée l'environnement, et une partie mutable. Ensuite, la section expose comment la fonction d'élection utilise les \emph{oracles} pour récupèrer les informations de l'environnement, et comment la fonction d'élection utilise l'interface pour manipuler la partie mutable de l'état.

	\subsubsection{État de l'ordonnanceur}

	Le modèle de l'état est divisé en deux parties : l'environnement, contenant les valeurs calculées par le back-end sur lesquelles la fonction d'élection n'a aucun contrôle, et la partie de l'état modifiable par la fonction d'élection.

	Le modèle de l'environnement \texttt{Env} est défini comme une fonction qui, en fonction de l'instant passé en paramètre, retourne la liste des \emph{jobs} à ajouter à la liste des \emph{jobs} prêts à être ordonnancés.
\begin{listing}[ht]
\begin{minted}{coq}
Record Job  :=
mk_Job {
    jobid : CNat ;
    arrival : CNat ;
    duration : CNat ;
    budget : CNat ;
    deadline : CNat
}.

Definition Env : Type := CNat -> list Job.
\end{minted}
\caption{Définitions des modèles de la structure \texttt{Job} et de l'environnement}
\label{code:job_env_definition}
\end{listing}

	Le modèle de la partie mutable \texttt{State} est constitué d'une structure avec deux champs. Le premier est un simple compteur de temps, gardant le compte des interruptions d'horloge ; l'autre champ est une liste d'\texttt{entries}. Les \emph{entries} sont des structures qui associent chaque \emph{job} à son budget d'exécution restant et à son échéance relative. Le modèle de cet état mutable est donc défini de la manière suivante :
\begin{listing}[ht]
\begin{minipage}[c]{0.50\linewidth}
\begin{minted}{coq}
Record Entry :=
mk_Entry {
    id : CNat ;
    cnt : CNat ;
    del : CNat
}.
\end{minted}
\end{minipage}
\begin{minipage}[c]{0.50\linewidth}
\begin{minted}{coq}
Record State :=                     
mk_State {                      
    now : CNat ;                 
    active : list Entry ;
}.

\end{minted}
\end{minipage}
\caption{Définitions des modèles des structures Entry et de la partie mutable de l'état}
\label{code:entry_state_definition}
\end{listing}

	La monade d'état, composé de l'environnement et de la partie mutable, a le type suivant :

\begin{listing}[!ht]
	\begin{minted}{coq}
Definition RT (A : Type) : Type :=
    Env -> State -> A * State.
	\end{minted}
	\caption{Définition du type de la monade d'état et d'environnement}
	\label{code:monad_type}
\end{listing}

	Dans notre modèle, chaque fonction monadique a un type de retour \texttt{RT A}, et prend donc en paramètre l'environnement et l'état mutable actuel et renvoie une valeur de type arbitraire \texttt{A} ``enrobée'' avec la nouvelle partie mutable de l'état.

	Les paragraphes suivants décrivent comment l'état est implémenté.
	L'implémentation profite du fait que la fonction d'élection crée une nouvelle structure \texttt{entry} pour chaque \emph{job} à ordonnancer. L'implémentation réserve de l'espace mémoire supplémentaire dans la structure de chaque \emph{job}, afin que la fonction d'élection puisse y écrire la structure \texttt{entry} associée.

	Notre implémentation fournit un ensemble jouet de job à ordonnancer sous la forme d'un tableau de \texttt{coq\_N} éléments (où \texttt{coq\_N} est une borne supérieure au nombre de \emph{jobs} à ordonnancer connue à la compilation). Les jobs sont discriminés par leur position au sein de ce tableau. Chaque élément de ce tableau correspond donc à un \emph{job} de l'ensemble à ordonnancer. Il contient les informations immuables qui le caractérisent (par exemple sa \emph{release date}, son échéance, son budget temporel d'exécution), ainsi qu'une portion de mémoire non initialisée qui sera utilisée par la fonction d'élection pour y écrire la structure \texttt{entry} qui lui est associée. De plus, chaque élément contient deux mots mémoire supplémentaires. Ils sont utilisés pour maintenir deux listes nécessaires au fonctionnement de l'ordonnanceur. La première est la liste des nouveaux \emph{jobs} à ordonnancer, exposée par l'environnement et maintenue par le \emph{back-end}. La seconde est la liste des \emph{jobs} prêt à être exécutés, maintenue par la fonction d'élection.

	Chaque élément est représenté en C avec le type décrit en listing \ref{code:array_elem_impl}.
	\begin{listing}[!ht]
	\begin{minted}{c}
typedef struct __internal_s__ {
    struct __internal_job__   job;
    struct __internal_entry__ entry;
    int jobset_next_job_index;
    int active_next_entry_index;
} internal_t;
	\end{minted}
	\caption{Implémentation du type des éléments du tableau contenant les \emph{jobs} à ordonnancer}
	\label{code:array_elem_impl}
	\end{listing}

	\begin{listing}[!ht]
	\begin{minted}{c}
internal_t INTERNAL_ARRAY[coq_N] = EXAMPLE_JOB_SET;
int JOBS_ARRIVING_HEAD_INDEX = -1;
int ACTIVE_ENTRIES_HEAD_INDEX = -1;
unsigned int CLOCK = 0;
coq_CBool JOB_DONE = false;
	\end{minted}
	\caption{Implémentation de l'environnement et de la partie mutable de l'état de l'ordonnanceur}
	\label{code:sched_state_impl}
	\end{listing}

	L'implémentation de l'état, donnée en listing \ref{code:sched_state_impl}, est donc constituée :
	\begin{itemize}
		\item du tableau contenant les informations sur les \emph{jobs} et l'espace supplémentaire requis par la fonction d'élection pour y stocker les structures \texttt{entry}
		\item d'une référence vers la liste des nouveaux \emph{jobs} à ordonnancer
		\item d'une référence vers la liste des \emph{jobs} prêts à être ordonnancés
		\item de la variable gardant le compte des interruptions d'horloge
		\item de la variable indiquant si le dernier \emph{job} exécuté a terminé son exécution
	\end{itemize}

	\subsubsection{Oracles}

	Il existe deux fonctions qui retournent des valeurs de l'environnement calculées par le back-end. La première, \texttt{jobs\_arriving}, récupère les nouveaux \emph{jobs} à ordonnancer ; la seconde, \texttt{job\_terminating}, retourne si le dernier \emph{job} exécuté a terminé son exécution (si un tel \emph{job} existe). Nous appelons ces fonctions les \emph{oracles} de l'ordonnanceur, car leur modèle ne décrit que les contraintes imposées sur les résultats qu'elles produisent, et non pas comment ces résultats sont fournis.

	La fonction \texttt{jobs\_arriving}, qui récupére les identifiants des nouveaux jobs à ordonnancer, a un modèle n'imposant qu'une seule contrainte à l'implémentation : dans la liste des identifiants de \emph{jobs} retournés par l'oracle, aucun identifiant ne doit être plus grand que $N$, où $N$ est une borne supérieure définie à la compilation. Cette borne est la même que la borne \texttt{coq\_N} présentée dans la section précédente sur l'implémentation de l'état. La liste renvoyée par l'oracle est cachée derrière le type opaque \texttt{JobSet}.\\
	Le modèle de cette fonction est exposé dans le listing \ref{code:jobs_arriving_model}.

	\begin{listing}[!ht]
	\begin{minted}{coq}
Definition jobs_arriving (N : nat) : RT JobSet :=
  fun env s =>
    let f :=  List.filter
      (fun j =>  Nat.ltb j N)
      (map jobid (env s.(now))) 
  in (f, s).
	\end{minted}
	\caption{Modèle de l'oracle \texttt{jobs\_arriving}}
	\label{code:jobs_arriving_model}
	\end{listing}

	Rappelons que ce modèle n'indique aucunement comment calculer une telle liste de \emph{jobs}. Ce choix est délibéré, afin de s'assurer que la fonction d'élection n'est pas en capacité de prévoir l'arrivée de nouveaux \emph{jobs}, guarantissant que n'importe quel ensemble arbitraire de \emph{jobs} puisse être utilisé avec cet ordonnanceur.

	L'implémentation de la fonction retournant la prédiction de l'oracle est particulièrement simple : elle retourne simplement la variable de l'état contenant la référence vers la liste des nouveaux \emph{jobs} à ordonnancer. Plus précisement, cette variable contient l'indice de l'élément du tableau étant en tête de cette liste. La liste a été construite par le back-end avant l'exécution de la fonction d'élection qui fait appel à l'oracle. La construction de cette liste est donc totalement déléguée à l'implémentation du back-end, sans aucune répercussion sur le monde mathématique.

	\begin{listing}[!ht]
	\begin{minted}{c}
static inline coq_JobSet
Primitives_jobs_arriving(coq_CNat n) {
     return JOBS_ARRIVING_HEAD_INDEX;
}
	\end{minted}
	\caption{Implémentation de l'oracle \texttt{jobs\_arriving}}
	\label{code:jobs_arriving_impl}
	\end{listing}

	L'autre oracle est une fonction qui retourne si le \emph{job} exécuté durant la précédente période d'exécution a terminé son exécution (si un tel \emph{job} existe). Le modèle de cet oracle est plus strict ; il force l'implémentation de l'oracle à retourner \texttt{True} si le nombre de périodes passées à exécuter le \emph{job} excède son \emph{WCET}. Son modèle est présenté en figure \ref{code:jobs_terminating_model}.

	\begin{listing}[!ht]
	\begin{minted}{coq}
Definition job_terminating : RT CBool :=
fun _ s => ((
    match head s.(active) with
    | None => false
    | Some e =>
        let j := Jobs (e.(id)) in
        Nat.leb e.(cnt) (j.(budget) - j.(duration))
    end
), s).
	\end{minted}
	\caption{Modèle de l'oracle \texttt{job\_terminating}}
	\label{code:jobs_terminating_model}
	\end{listing}

	Ce modèle décrémente le compteur \texttt{e.(cnt)} à partir d'une valeur initiale de \texttt{j.(budget)} à chaque fois que le \emph{job} est élu pour exécution. Tout comme l'implémentation de l'oracle précédent, l'implémentation de cet oracle est particulièrement simple, comme démontré en listing \ref{code:job_terminating_impl}. Elle ne fait que retourner la variable \texttt{JOB\_DONE} de l'état interne; cette variable est mise à jour par le \emph{back-end} lorsqu'un \emph{job} termine son exécution, et remis à \texttt{False} après chaque appel à la fonction d'élection.

	\begin{listing}[!ht]
	\begin{minted}{c}
static inline coq_CBool Primitives_job_terminating(void) {
    return JOB_DONE;
}
	\end{minted}
	\caption{Implémentation de l'oracle \texttt{job\_terminating}}
	\label{code:job_terminating_impl}
	\end{listing}

	La dernière portion de cette section détaillera l'interface de la fonction d'élection avec la partie mutable de l'état.

	\subsubsection{Types opaque et interface avec l'état mutable}

	Tous les types utilisés par la fonction d'élection sont des types opaques, c'est à dire que des types agnostiques de leur représentation et uniquement définis au travers des fonctions qui permettent de les manipuler, qu'on appelle leurs \emph{primitives}.

	Ces travaux comportent des types opaques pour des valeurs booléennes, pour des entiers, pour les structures en lecture seule telles que celles contenant les informations initiales sur les \emph{jobs} ou la liste des nouveaux \emph{jobs} à ordonnancer, ainsi que pour les structures contenant des données mutables telles que les structures \texttt{entry}. La plupart des primitives sur ces types sont extrêmement simples ; par exemple, le type \texttt{CBool} permettant de représenter les booléens est muni de trois primitives : \texttt{not}, \texttt{and}, et \texttt{or}. Le listing \ref{code:boolean_definition_prim} montre la modélisation de ce type avec sa primitive \texttt{or}. Son implémentation est décrite dans le listing \ref{code:boolean_definition_impl}.

	\begin{listing}[!ht]
	\begin{minted}{coq}
Definition CBool := bool.

Definition or (b1 b2 : CBool) : RT CBool :=
  ret (orb b1 b2).
	\end{minted}
	\caption{Modèle du type opaque \texttt{CBool} et de sa primitive \texttt{or}}
	\label{code:boolean_definition_prim}
	\end{listing}

	\begin{listing}[!ht]
	\begin{minted}{c}
typedef int coq_CBool;

static inline coq_CBool CBool_or(coq_CBool b1, coq_CBool b2) {
    return b1 || b2;
};
	\end{minted}
	\caption{Implémentation du type opaque \texttt{CBool} et de sa primitive \texttt{or}}
	\label{code:boolean_definition_impl}
	\end{listing}

	Les structures mutables, telles que le type \texttt{entry}, sont munies de primitives permettant de les créer, ainsi que de lire et modifier leurs champs. Dans le même esprit, les structures de données en lecture seule ne sont munies que de primitives permettant de lire les informations qu'elles renferment. Par exemple, le type opaque \texttt{JobSet}, retourné par l'oracle \texttt{jobs\_arriving} n'est muni que de trois primitives : une vérifiant si l'ensemble est vide, une autre récupérant le premier élément de l'ensemble, et une dernière retournant un plus petit \texttt{JobSet} contenant les \emph{jobs} restants. Le type opaque \texttt{Job} quand à lui ne possède que des primitives permettant de récupérer individuellement chaque champ de la structure.
	
	Enfin, il existe huit autres primitives conçues pour intéragir directement avec l'état mutable de la fonction d'élection. La fonction d'élection a notamment besoin de gérer le compteur d'interruption d'horloge ; pour cela il existe deux primitives, un \emph{getter} et un \emph{setter} pour la variable de compteur.
	Ces fonctions sont extrêmement simples à modéliser et implémenter, comme en attestent les listings \ref{code:time_counter_model} et \ref{code:time_counter_impl}.

	\begin{listing}[!ht]
	\begin{minted}{coq}
Definition get_time_counter : RT CNat :=
fun _ s => ((now s), s).

Definition set_time_counter (counter : nat) : RT unit :=
fun _ s => (tt,
    {|
        now    := counter ;
        active := (active s) ;
    |}).
	\end{minted}
	\caption{Modèle des fonctions de récupération et de modification de la variable de compteur de l'état}
	\label{code:time_counter_model}
	\end{listing}

	\begin{listing}[!ht]
	\begin{minted}{c}
static inline coq_CNat State_get_time_counter(){
    return CLOCK;
};

static inline void State_set_time_counter(coq_CNat counter){
    CLOCK = counter;
};
	\end{minted}
	\caption{Implémentation des fonctions de récupération et de modification de la variable de compteur de l'état}
	\label{code:time_counter_impl}
	\end{listing}

	La fonction d'élection doit aussi maintenir la liste des jobs prêts à être ordonnancés. Elle a à sa disposition six primitives :
	\begin{itemize}
		\item une primitive permettant de vérifier si cette liste est vide
		\item une primitive retournant le \emph{job} à la tête de la liste
		\item une primitive ajoutant un \emph{job} à la liste de manière triée selon une fonction de comparaison fournie en paramètre
		\item une primitive retirant le premier job de la liste
		\item une primitive mettant à jour la structure \texttt{entry} liée au premier \emph{job} de la liste
		\item une primitive mettant à jour la structure \texttt{entry} de tous les \emph{job} de la liste
	\end{itemize}

	\begin{listing}[!ht]
	    \centering
	%    \begin{verbatim}
	    \begin{minted}{coq}
Definition insert_new_active_entry (entry : Entry)
           (comp_func : Entry -> Entry -> CBool) : RT unit :=
  fun _ s => (tt, {|
    now := now s ;
    active := (insert_Entry_aux entry (active s) comp_func);
  |}
).

Fixpoint insert_Entry_aux (entry : Entry) (entry_list : list Entry)
         (comp_func : Entry -> Entry -> CBool) : list Entry :=
  match entry_list with
  | nil => cons entry nil
  | cons head tail =>
      match comp_func entry head with
      | true => cons entry (cons head tail)
      | false => cons head (insert_Entry_aux entry tail comp_func)
      end
  end.


	    \end{minted}
	    \hrule
	    \begin{minted}{c}


void State_insert_new_active_entry
     (coq_Entry entry, entry_cmp_func_type entry_comp_func) {
  int *entry_index_ptr = &(ACTIVE_ENTRIES_HEAD_INDEX);
  int next_index = -1;
  while (*entry_index_ptr != -1) {
    if (entry_comp_func(entry, &(INTERNAL_ARRAY[*entry_index_ptr].entry))) {
      next_index = *entry_index_ptr;
      break;
    }
    entry_index_ptr = &(INTERNAL_ARRAY[*entry_index_ptr].active_next_entry_index);
  }
  *entry_index_ptr = entry->id;
  INTERNAL_ARRAY[entry->id].active_next_entry_index = next_index;
}
	    \end{minted}
	    \caption{Comparaison entre le modèle de la primitive d'insertion triée dans la liste et son implémentation}
	    \label{code:model_impl_cmp}
	\end{listing}

		La plupart de ces primitives sont aussi simples à modéliser qu'à implémenter ; cependant deux d'entre elles dérogent à cette règle. La première est la fonction \texttt{insert\_new\_active\_entry}, qui insère une structure \texttt{entry} (contenant les valeurs manipulables d'un job par la fonction d'élection) de manière triée selon une fonction de comparaison. Son modèle est donné dans la moitié supérieure du listing \ref{code:model_impl_cmp}.

		Étant donné une structure \texttt{entry} et une fonction de comparaison, ce modèle extrait d'abord la liste du modèle de l'état de l'ordonnanceur. Le modèle passe cette liste à une fonction auxiliaire purement fonctionnelle qui retourne la liste contenant la structure \texttt{entry} à insérer. Cette fonction auxiliaire crée en réalité une nouvelle liste, en comparant chaque structure \texttt{entry} de la liste initiale avec la structure \texttt{entry} à insérer grâce à la fonction de comparaison passée en paramètre, et en ajoutant au fur et à mesure des copies des éléments à cette liste.

		Pour des raisons de performances et de minimisation de la base de confiance, l'implémentation ne crée pas de nouvelle liste à chaque nouvelle insertion, mais modifie en place la liste actuelle. L'interface choisie empêche le problème de distortion exposé dans la section précédente \ref{sec:model_distortion}, où le modèle crée des copies et où l'implémentation modifie la liste en place.

		Le code de l'implémentation, disponible dans la motié inférieure du listing \ref{code:model_impl_cmp}, est similaire à la fonction auxiliaire récursive du modèle, mais utilise une boucle et met à jour le lien entre les éléments quand la nouvelle struture \texttt{entry} est insérée, plutôt que de créer des copies des éléments de la liste en les réordonnant correctement. 

		La fonction \texttt{update\_active\_entries}, qui applique une modification arbitraire à chaque structure \texttt{entry} de la liste, a subit la même méthode de modélisation et d'implémentation. L'implémentation modifie les éléments de la liste en place plutôt que de créer une nouvelle liste avec des copies modifiés des éléments originaux.

	\label{sec:implementation}

		Nous défendons l'argument que les primitives présentées dans cette section, définissant l'interface de la fonction d'élection avec l'implémentation réelle de l'ordonnanceur et de son état interne, sont assez simples pour se convaincre que les effets de leur implémentation correspond bien au effet décrits par leur modèle. Néanmoins, les fonctions \texttt{insert\_new\_active\_entry} et \texttt{update\_active\_entries} présentées à la fin de cette section sont objectivement plus complexes que les autres primitives présentées. Cet argument est débatable sur ces deux fonctions ; aussi la section \ref{sec:interfaceTCB} présentera des arguments supplémentaires en faveur de ce choix de primitives.

	\section{Discussion sur la méthodologie suivie}

		Cette section sert de retour d'expérience sur la contribution, en prenant du recul sur le travail fourni.
		Elle commence par présenter différentes métriques concernant l'ordonnanceur, discutant des proportions de code prouvé, de lignes de preuves et de lignes d'implémentation, et du temps investi. Dans une seconde partie, cette section discutera des choix fait pour ces travaux, et plus particulièrement du choix des primitives et de l'impact de ce choix sur la confiance apportée par la preuve.

		\subsection{Métriques}

		La partie technique de cette contribution est le fruit de 9 mois de travail de deux personnes à temps plein, aidés ponctuellement par d'autres membres de nos équipes de recherche respectives. La présentation de l'ordonnanceur en conférence et la publication du papier ont eu lieu un an et demi après le premier commit sur le dépôt de code du projet.

			\subsubsection{Implémentation}
			La fonction d'élection, écrite en Gallina, compte à elle seule environ 150 lignes de code Gallina ; sans surprise, le nombre de lignes de code C produit lors de la compilation est équivalent. L'implémentation de l'état, des types opaques et des fonctions formant l'interface avec l'état compte environ 350 lignes de code C, composé majoritairement de fichiers \emph{headers} particulièrement verbeux.

			Concernant le back-end de \emph{simulation} de l'ordonnanceur, celui-ci a requis environ 100 lignes de code C. L'implémentation de la partition de Pip contenant l'ordonnanceur préemptif temps réel, sans compter la fonction d'élection ni l'implémentation de l'état et de l'interface, compte pour environ 1150 lignes de code. Sont pris en compte, les fichiers de code C et leur \emph{headers}, les fichiers de code assembleur, mais aussi les scripts d'édition de liens. En revanche, ce nombre de lignes ne contient pas les lignes de la librairie utilisateur de Pip. Sur ces 1150 lignes, environ 900 lignes sont requises par Pip pour créer les partitions de l'ordonnanceur et des \emph{jobs}, ainsi que pour mapper leur code dans leur espace d'adressage respectifs. Les 250 lignes restantes sont dédiées au fonctionnement de l'ordonnanceur en lui-même.

			\subsubsection{Modèles et preuves}
			
			Les modèles de l'état, des types opaques, et des fonctions permettant d'intéragir avec l'état représentent en tout environ 650 lignes de code.

			Pour rappel, la preuve se découpe en trois parties majeures : la définition de la politique d'ordonnancement \emph{Earliest Deadline First} et l'établissement de sa propriété de correction, l'implémentation d'une fonction d'élection idéalisée et de sa preuve de raffinement de la politique d'ordonnancement, puis d'un raffinement vers la fonction qui sera traduite en C, compilée et exécutée.

			La première partie de cette preuve concernant la correction de la politique d'ordonnancement est relativement courte. L'écriture de la politique d'ordonnancement, avec sa preuve de correction représente environ 1100 lignes de code et de preuve. Ce nombre de lignes relativement restreint est dû au haut niveau d'abstraction, profitant à l'établissement de la preuve : cela permet de se concentrer sur l'essentiel sans s'encombrer de détails d'implémentation.
		
			Comme énoncé dans la section \ref{sec:proof_insight}, le premier raffinement, allant de la politique d'ordonnancement vers la fonction d'élection idéalisée, représente la majeure partie de la preuve, et qui a necéssité de redoubler d'ingéniosité. Cela s'explique par le fossé d'abstraction entre la politique d'ordonnancement (composée de termes algébriques relativement simples), et la couche d'abstraction de la fonction d'élection idéalisée qui décrit la politique en fonction d'opération sur des structures de données, en particulier des listes chainées. Cette partie de la preuve représente environ 1600 lignes de Coq (comptabilisant aussi les lignes de code de la fonction d'élection idéalisée).

			Le second raffinement est relativement simple comparé au premier raffinement, car les fonctions d'élection idéalisées et exécutables sont à des niveaux d'abstractions relativement proches ; en particulier, elles utilisent les mêmes structures de données. À titre de comparaison, cette dernière étape de raffinement est courte, environ 500 lignes de preuve.

			Ainsi, l'établissement de la preuve sur la fonction d'élection longue de 150 lignes a nécessité au total 3250 lignes de Coq, quand le nombre total de ligne nécessaires pour implémenter l'ordonnanceur avoisinne les 1400 lignes de code. Le code prouvé représente donc environ 10\% du code total de l'ordonnanceur.


		\subsection{Choix des primitives et discussion sur la base de confiance}
		\label{sec:interfaceTCB}

		Les preuves formelles transfèrent la confiance accordées aux hypothèses vers la conclusion d'une démonstration. Lorsque ces hypothèses sont relatives à la sécurité informatique, on parle alors de base de confiance (\emph{Trusted Computing Base} ou \emph{TCB}). On peut distinguer trois catégories d'hypothèses dans les bases de confiance. La première catégorie, renfermant les hypothèses les plus fondamentales, sont les hypothèses sur la correction du matériel et des outils utilisés pour vérifier la preuve. La seconde catégorie englobe les hypothèses sur l'ensemble des logiciels non vérifiés s'exécutant avec le code prouvé. La troisième et dernière catégorie regroupe les hypothèses concernant le processus de compilation, et plus particulièrement sur la sémantique associée aux opérations utilisées par le code source (sur laquelle la preuve repose) et la sémantique utilisée par le code compilé (qui est réellement exécuté). Cette sous-section questionnera la base de confiance de l'ordonnanceur présenté dans ce chapitre, en la comparant avec celle de travaux connexes.

\subsubsection{Barrières fondamentales}

Il existe des hypothèses universelles, présentes dans la base de confiance de l'intégralité de la communauté de vérification formelle de code.

Les premières de ces hypothèses sont celles de l'outil utilisé pour vérifier la preuve formelle : dans notre cas, l'assistant de preuve Coq. Sa partie critique est circonscrite par son noyau -- la partie minimiale de l'assistant qui vérifie si une tentative de preuve est valide au sens formel. On pourrait accroitre notre confiance dans le noyau en le vérifiant formellement ; c'est ce que le projet MetaCoq \cite{metacoq} vise à atteindre, en formalisant le noyau de Coq dans Coq.

Néanmoins, il existe, dans la base de confiance commune, d'autres hypothèses plus fortes par plusieurs ordres de grandeur : la spécification du matériel exécutant le code vérifié. On peut découper cet argument en deux parties :

\begin{itemize}
	\item Tout d'abord, la (quasi?) totalité des fabriquants ne fournissent pas de spécification formelle du fonctionnement de leur matériel. Cela implique que pour raisonner sur ce matériel, il faut \emph{interpréter} la documentation informelle pour produire la spécification formelle. Ce travail est fastidieux et délicat, ce qui le rend particulièrement sujet à l'erreur. Par effet de bord, il est difficile de vérifier du logiciel qui s'appuierait directement sur des primitives fournies par le hardware. Néanmoins, de nouveaux projets visent à améliorer les choses, avec notamment la démocratisation du développement de matériel open-source et l'apparition de spécifications formelles pour certains processeurs \cite{reid2017guards} ;
	\item Par ailleurs, si on considérait qu'il existait une spécification formelle du matériel sur lequel le code s'exécuterait, il n'existe pour l'instant aucune méthode permettant de s'assurer que le matériel se comporte exactement comme la spécification le décrit.
\end{itemize}

Ainsi, en dépit de la complexité grandissante des divers composants matériels, exécuter du code sur du matériel spécifique inclus le bon fonctionnement de ce matériel dans la base de confiance, peu importe les efforts fournis.

\subsubsection{Inclusion de librairies et primitives dans la base de confiance}

Dans la section \ref{sec:implementation}, nous avons soutenu l'argument que les fonctions de l'interface utilisées par la fonction d'élection étaient assez simples pour se convaincre \emph{sans l'ombre d'un doute} que leur implémentation correspondait à leur spécification. Cependant deux fonctions de cette interface sont trop complexes pour pouvoir l'affirmer : une fonction exécutant une insertion triée dans une liste, et une fonction appliquant une modification à chaque élément d'une liste.

La raison principale derrière ce choix d'interface est que nous pensons que le coeur de notre contribution est la preuve de correction de la fonction d'élection. Nous avons estimé que l'aspect de gestion de la mémoire requise avec l'utilisation de liste dépassait ce cadre, et que cette preuve était orthogonale aux propriétés que nous avons prouvé. Étant donné leurs tailles raisonnables (respectivement 13 et 7 lignes de code C), nous avons décidé des les considérer comme du logiciel de confiance -- comme les oracles et le back-end -- et que nous n'allions pas les détailler dans le modèle.

Néanmoins, dans le cadre d'une preuve où chaque élément logiciel \emph{doit} être prouvé (par exemple dans le cadre d'une preuve reposant seulement sur des primitives du matériel), on pourrait imaginer appliquer la même méthodologie pour se débarasser de ces primitives. La première étape serait de concevoir une nouvelle interface qui décompose les deux primitives, puis d'écrire la fonction d'élection en utilisant uniquement les nouvelles primitives, puis en montrant que la nouvelle fonction d'élection raffine la fonction d'élection originale.

Par ailleurs, nous souhaitons revenir sur l'existence même de cette discussion à propos de nos choix d'interface. Nous pensons que notre méthodologie de preuve, et particulièrement le \emph{shallow embedding}, met en lumière les hypothèses introduites dans la base de confiance, et nous pousse à les discuter. Plus spécifiquement, puisque la preuve et l'implémentation sont exprimées dans le même langage, ils partagent la même définition des hypothèses de la base de confiance, ce que nous pensons particulièrement bénéfique au processus. À titre de comparaison, il nous semble peu discutable que les preuves conduites avec un \emph{deep embedding}, et donc conduites sur la représentation de l'\emph{AST} du code du langage représenté, ajoute une couche supplémentaire d'obfuscation aux hypothèses introduites dans la base de confiance.

\subsubsection{Raisonnement sur le code Gallina et confiance dans le code compilé}

Une chaîne d'outils qui propage les propriétés jusqu'au code compilé est le modèle d'excellence que chaque projet muni de preuve formelle sur du code devrait chercher à atteindre. Nous nous efforçons d'atteindre ce but. Comme mentionné dans l'état de l'art, la chaîne de compilation de Pip utilise soit Digger, soit $\partial x$ pour compiler le code Gallina \emph{shallow embedded} vers du code C classique. Aucun de ces deux outils n'est pour l'instant muni de preuve de préservation de la sémantique ; $\partial x$ a cependant été développé pour poursuivre ce but.

De nombreux travaux existent déjà à ce sujet.
Œuf \cite{oeuf} permet de compiler un grand sous-ensemble de code Gallina idiomatique vers du code C, avec la garantie que :
\begin{quote}
    \emph{des appels valides aux fonctions compilées par Œuf auront un comportement équivalent à l'implémentation originale en Gallina} -- \cite{oeuf}, Section 2.1
\end{quote}
Le programme Gallina est compilé vers la représentation \texttt{Cminor} de CompCert, et peut donc être compilé vers du code assembleur avec CompCert.
D'une manière assez similaire, le projet CertiCoq \cite{anand2017certicoq} vise à vérifier la compilation de \emph{n'importe quel} programme Gallina en \texttt{Clight}, qui peut donc être compilé à nouveau vers du code assembleur par n'importe quel compilateur C -- y compris CompCert.

Ces deux contributions de la communauté ont un point commun : elles ajoutent un ramasse-miette au code C compilé. Un ramasse-miette est un logiciel complexe permettant au développeur de ne pas se préoccuper de libérer la mémoire allouée dynamiquement lorsqu'il n'en a plus besoin. Ce logiciel fait parti de l'environnement d'exécution de Gallina, mais n'est pas nativement inclus avec le langage C ou disponible sur du matériel. L'utilisation d'un ramasse-miette a généralement un impact négatif sur les performances, et peu engendrer une surconsommation de mémoire. De plus, l'inclusion d'un ramasse-miette dans le code compilé ajoute un bout de logiciel complexe (et non-prouvé) dans la base de confiance. Le compromis apporté par l'inclusion d'un tel logiciel contraste avec les contraintes typiques des systèmes embarqués, qui s'accomodent plutôt de logiciels légers munis d'un minimum de dépendances.

D'autres travaux ont abordé le problème avec une méthodologie opposée, partant du code C pour le vérifier formellement. RefinedC \cite{refinedC}, la contribution la plus récente du projet RustBelt, utilise des annotations de code C pour créer des spécifications et automatiser la plupart des preuves de programmes. Le code C est compilé vers un langage appelé Caesium, qui est représenté sous la forme d'un \emph{deep-embedding} en Gallina, et sur lequel la preuve est vérifiée.
Le code source initial étant du C, il peut être compilé directement sans inclure de ramasse-miette. Néanmoins, tout comme notre propre méthodologie, cette méthodologie suppose que leur outil de compilation de C vers Gallina est correct et que la sémantique de Caesium est correcte.

Le projet \emph{VST} (pour \emph{Verified Software Toolchain} a suivi la même approche. Leur contribution la plus récente est VST-Floyd \cite{cao2018vst}. VST-Floyd est un écosystème de lemmes et de tactiques conçues pour faciliter l'utilisation de \emph{Verifiable C}, une logique de séparation d'ordre supérieur. \emph{Verifiable C} est prouvé correct selon la sémantique opérationnelle de \texttt{Clight}, formalisé dans Coq. Cela permet de raisonner sur n'importe quel programme \texttt{Clight}, procurant outils et méthodes pour faciliter la vérification de programmes.

\section{Synthèse des travaux et contributions de ce chapitre}

	Ce chapitre a présenté les travaux qui ont mené à une implémentation d'un ordonnanceur \emph{Earliest Deadline First} pour des séquences arbitraires de \emph{jobs}, dont la fonction d'élection a été prouvée correcte formellement. Plus spécifiquement, il a été montré que l'implémentation exécutable de la fonction d'élection respecte la politique d'ordonnancement \emph{Earliest Deadline First}. L'ordonnanceur présenté s'exécute en espace utilisateur dans une partition de mémoire sur un système s'exécutant avec Pip. Ce chapitre a présenté les étapes de la preuve de correction de la fonction d'élection, puis a présenté l'interface du code vérifié avec le code faisant partie de la TCB. Nous avons ensuite discuté de la TCB de nos travaux en la comparant avec la TCB de travaux connexes. À notre connaissance, nos travaux sont les premiers qui proposent une implémentation prouvée d'un ordonnanceur EDF pouvant ordonnancer une séquence arbitraire de \emph{jobs}.

	De plus, l'ordonnanceur implémenté utilise le service de transfert de flot d'exécution décrit dans le chapitre précédent, montrant par l'exemple que le service est suffisant pour implémenter efficacement des composants systèmes tel qu'un ordonnanceur. La preuve de correction de l'ordonnanceur a été menée selon la méthode de co-design usuelle de Pip, mais a cependant été conduite par raffinement, contrairement aux preuves d'isolation de Pip qui ont été prouvée par méthode directe. Ces travaux montrent que cette méthodologie est efficace pour mener des preuves sur d'autres propriétés que les propriétés classiques de Pip et qu'elle s'accomode d'autres méthodes de preuve, comme par exemple le raffinement.


    \chapter{Ajout incrémental de propriétés sur le code prouvé}

	Ce chapitre présente une preuve de concept affranchissant le code des services de Pip de toutes ses dépendances au modèle actuel d'isolation. Ces travaux n'ont pas été publiés, mais sont néanmoins consultables sur la branche \texttt{state\_abstraction} du projet Pip, disponible sur le dépôt accessible à l'adresse suivante : \url{https://gitlab.univ-lille.fr/2xs/pip/pipcore}. Cette branche contient une implémentation imparfaite et restreinte à la fonction \texttt{switchContextCont} des idées exposées dans ce chapitre.

	Ce chapitre aborde de manière très superficielle certains objets issus de la théorie des catégories, qui pourraient attiser votre curiosité. Je souhaite mentionner ici \cite{categorytheoryforprogrammers}, un livre (grauit !) qui aborde selon le point de vue d'un programmeur des concepts de la théorie des catégories. Cette lecture vous permettra certainement de mieux cerner -- entre autres -- le concept de foncteur ou encore de monade.

	\section{Motivations}
		\subsection{Modularisation de la méthodologie de preuve de Pip}
	La section de discussion du premier chapitre a mis en avant le fait que Pip était conçu autour des propriétés d'isolation formellement prouvées. Le modèle des fonctions de l'interface et de l'état, jusqu'à la monade intégrée au code, sont liés aux propriétés d'isolation. Cette forte proximité est une conséquence de la philosophie de conception minimaliste de Pip, qui a incité à ne définir que les éléments strictement nécessaires à l'établissement de le preuve de préservation de l'isolation. Cette approche a permis de minimiser l'effort de preuve permettant de garantir la propriété d'isolation, mais présente un désavantage majeur : le code des services de Pip n'est pas indépendant des modèles sur lesquels il repose.

	Ainsi, il n'existe qu'un modèle unique dans Pip qui ne peut évoluer que de manière itérative. Chaque évolution rend caduques les propriétés établies sur l'ancien modèle, et implique de produire une nouvelle preuve des mêmes propriétés avec le nouveau modèle. Le moindre ajout de chaque itération rendant de plus en plus difficile l'établissement la preuve à produire.
	Ceci est un frein considérable à la vérification de nouvelles propriétés sur le code de Pip, telle que la preuve fonctionnelle du service évoquée dans le second chapitre. Si les nouvelles propriétés impliquent des changements trop importants sur le modèle, l'effort de preuve à fournir deviendrait inatteignable après seulement quelques itérations.

		\subsection{Raisonner sur le lien entre le Yield et la fonction d'élection}
		% Description informelle des propriétés
	

	\section{Architecture monolithique}

		Cette section décrira de manière synthétique les composants actuels de Pip, en essayant de décrire leurs dépendances d'un point de vue logiciel. Elle commencera par donner brièvement une vue d'ensemble du projet. Puis, dans une première partie, elle décrira les dépendances du code des services sur les modèles décrits dans Pip. Elle dépliera les définitions pour mettre en lumière les dépendances qui existent entre les modèles des différents composants. Ensuite, dans une seconde partie, la section se penchera sur la méthode de preuve nécessaire à l'établissement de la preuve d'isolation. La section se concluera sur le processus de compilation du code, compilant le code des services de Gallina vers du code C.
		
		\subsection{Vue générale}

			\begin{figure}[!ht]
				\begin{tikzpicture}[>=triangle 45,font=\sffamily, every text node part/.style={align=center}, scale=1, every node/.style={transform shape}] {
	%\node[draw, pattern=south west lines, minimum width=5.6cm, minimum height = 3.6cm] at (0, -3.1) {};
	\node[draw, pattern=south west lines, minimum width=5.6cm, minimum height = 6.2cm] (model) at (0, -1.8) {};
	\node[below=0cm of model] {Modèle monolithique en Gallina};
	\node[draw, fill=white, minimum width = 5cm, minimum height =   2cm] (services) at (0,0) {Code des services};
	\node[draw, fill=white, minimum width = 5cm, minimum height = 0.7cm] (functions_model) at (0, -1.9) {Modèle des fonctions};
	\node[draw, fill=white, minimum width = 5cm, minimum height = 0.7cm] (types) at (0, -2.7) {Modèle de types};
	\node[draw, fill=white, minimum width = 5cm, minimum height = 0.7cm] (state_monad) at (0, -3.5) {Monade d'état};
	\node[draw, fill=white, minimum width = 5cm, minimum height = 0.7cm] (state_model) at (0, -4.3) {Modèle de l'état};

	\node[draw, pattern=south west lines, minimum width=5.6cm, minimum height = 2cm] (impl) at (8, -2.3) {};
	\node[below=0cm of impl] {Implémentation en C};
	\node[draw, fill=white, minimum width=5cm, minimum height=2cm] (Cservices) at (8,0) {Code des services\\(Compilé en C)};
	\node[draw, fill=white, minimum width=5cm, minimum height = 0.7cm] (functions) at (8, -1.9) {Fonctions sur l'état};
	\node[draw, fill=white, minimum width=5cm, minimum height = 0.7cm] (types) at (8, -2.7) {Types C};

	\draw[->] (services) -- (Cservices) node[midway, above] {Digger} node[midway, below] {$\partial x$};
}

\end{tikzpicture}

				\caption{Architecture actuelle de Pip et dépendances des composants}
				\label{fig:currentPipArchitecture}
			\end{figure}
			\begin{listing}[!ht]
				\coqcode{code/switchContextCont.v}
				\caption{Code du bloc de continuation \texttt{switchContextCont} du service de transfert de flot d'exécution}
				\label{code:switchContextCont}
			\end{listing}

		\subsection{Dépendance du code au modèle d'isolation}

			\subsubsection{Monade dépendante du modèle de l'état}

			Dans l'architecture actuelle de Pip, le code des services repose directement sur les définitions de la monade, en utilisant le type monadique \texttt{LLI}, et les fonctions \texttt{bind} et \texttt{ret} pour représenter la mise en séquence des instructions des services. Dans l'exemple présenté en listing \ref{code:switchContextCont}, la fonction retourne un type monadique \texttt{LLI yield\_checks}, utilise la fonction \texttt{bind} au travers du sucre syntaxique \texttt{perform [...] := [...] ;}, et indique sa valeur de retour grâce à la fonction \texttt{ret}.
			Malheureusement, le type monadique \texttt{LLI}, décrit en listing \ref{code:LLImonad}, dépend de l'état \texttt{state}, décrit en listing \ref{code:CurrentIsolationState}. \texttt{state} est le modèle de l'état conçu pour la preuve de préservation de l'isolation. Ceci est une première dépendance du code au modèle d'isolation, dont le code devra se passer pour devenir indépendants des modèles construits pour Pip.

			\begin{listing}[!ht]
				\coqcode{code/LLIMonad.v}
				\caption{Définition du type de la monade d'état \texttt{LLI} dans le modèle actuel de Pip}
				\label{code:LLImonad}
			\end{listing}

			\begin{listing}[!ht]
				\coqcode{code/CurrentIsolationState.v}
				\caption{Définition de l'état \texttt{state} dans le modèle actuel de Pip}
				\label{code:CurrentIsolationState}
			\end{listing}

			\subsubsection{Code dépendant des modèles de types}

			D'autres dépendances du code aux modèles d'isolation passent par la représentation des types. Le code dépend des types utilisés pour représenter ses propres arguments, et valeur de retour enrobée par le type monadique \texttt{LLI}, mais aussi les arguments et valeurs de retour des fonctions de l'interface. Par exemple, la fonction \texttt{switchContextCont} décrite dans le listing \ref{code:switchContextCont}, dépend du modèle des types \texttt{page}, \texttt{interruptMask}, \texttt{contextAddr} et \texttt{yield\_checks}. Les modèles de ces types sont représentés dans le listing \ref{code:CurrentTypesModel}.
Cette dépendance renforce les liens entre le modèle d'isolation de Pip et le code de ses services, et doit donc disparaître.

			\begin{listing}[!ht]
				\coqcode{code/CurrentTypesModel.v}
				\caption{Définition des types nécessaires à la fonction \texttt{switchContextCont} dans le modèle actuel de Pip}
				\label{code:CurrentTypesModel}
			\end{listing}

			\subsubsection{Code dépendant des modèles des fonctions intéragissant avec l'état}
			Enfin, la dernière dépendance du code aux modèles est par le biais des fonctions intéragissant avec l'état. Le code des services fait directement appel \emph{aux modèles} de ces fonctions. Ainsi, la fonction \texttt{switchContextCont} présentée dans le listing \ref{code:switchContextCont}, est dépendant des modèles des fonctions \texttt{setInterruptMask}, \texttt{updateMMURoot}, \texttt{updateCurPartition}, \texttt{getInterruptMaskFromCtx}, \texttt{getPageRootPartition}, \texttt{noInterruptRequest} et \texttt{loadContext}. Cette dépendance n'a pas lieu d'être, et doit être supprimée pour atteindre un code des services agnostique des modèles.

			\begin{listing}[!ht]
				\coqcode{code/CurrentFunctionsModel.v}
				\caption{Définition des fonctions de l'interface avec l'état nécessaire à la fonction \texttt{switchContextCont} dans le modèle actuel de Pip}
				\label{code:CurrentFunctionsModel}
			\end{listing}

			\subsubsection{Extraction de l'\emph{AST} dépendant des modèles}
			\label{sec:AST_extr}
			Ce dernier paragraphe est dédié au fichier source Coq extrayant l'\emph{AST} du code des services de Pip. Ce fichier, une fois évalué par Coq, produit le fichier attendu en entrée par Digger, un des outils de compilation du code Gallina \emph{shallow embedded} vers du code C utilisé dans le projet Pip. Ce fichier a pour dépendances l'ensemble des modèles d'isolation sur lesquels reposent le code. Ainsi, dans l'état actuel du projet, l'extraction de l'\emph{AST} du code des services n'est possible que si l'intégralité des modèles d'isolation peut être évalué par Pip. Cette dépendance doit être supprimée pour que le code des services puisse être compilé en C sans avoir recours aux modèles.

		\subsection{Processus de preuve sur le code dépendant du modèle}
		\label{sec:dependant_code}

			Cette sous-section sera dédiée à la structure actuelle de la preuve d'isolation sur le code des services, mettant en avant les dépendances des différents groupes de fichiers nécessaires à chaques preuves.

			\subsubsection{Définition des propriétés d'isolation et des fonctions nécessaires à la définition des propriétés}

			Les premiers fichiers nécessaires à l'établissement de la preuve sont ceux contenant les définitions nécessaires à l'expression des propriétés d'isolation sur le noyau. Ces définitions additionnelles sont totalement fictives ; elles n'ont pas vocation à être exprimées en C. Elles servent de fondation à l'expression des triplets de Hoare d'isolation à montrer par la preuve formelle. Par exemple, ces fonctions peuvent permettre de définir des ensembles nécessaires à certaines propriétés d'isolation. C'est le cas de la fonction \texttt{getAccessibleMappedPages} qui récupère les pages mappées et accessibles dans l'espace d'adressage d'une partition, nécessaire à la propriété d'isolation noyau \texttt{kernelDataIsolation}. Ces fonctions peuvent aussi être des miroirs purement fonctionnels de code monadique présent dans les services de Pip, telle que la fonction \texttt{readPhysical} permettant de lire l'adresse d'une page mémoire ; ces fonctions sont parfois requises par les définitions des fonctions précédemment mentionnées. Ces définitions servent ensuite à définir les propriétés d'isolation souhaitées. Les fonctions et propriétés définies de cette manière dépendent donc des modèles d'isolation, que ce soit le modèle de types, de l'état, ou des fonctions de l'interface.

			\subsubsection{Définition des triplets de Hoare, des lemmes intermédiaires et des scripts de preuve}

			Une fois que ces définitions établies, il est possible d'exprimer les triplets de Hoare sur le code des services. À titre d'exemple, le listing \ref{code:switchContextCont_triplet} décrit le triplet de Hoare de la fonction \texttt{switchContextCont}. Sous chaque triplet (et chaque lemme intermédiaire) se trouve un script de preuve, décrivant les règles d'inférence (ou tactiques) à appliquer successivement pour faire progresser Coq vers la conclusion. Les triplets de Hoare dépendent du code des services, des fonctions fictives utiles à la définition des propriétés, et dépendent donc à fortiori des modèles d'isolation.
			\begin{listing}[!ht]
				\coqcode{code/switchContextCont_triplet.v}
				\caption{Définition du triplet de Hoare de la fonction \texttt{switchContextCont} pour la preuve de préservation de l'isolation de Pip}
				\label{code:switchContextCont_triplet}
			\end{listing}

		
	\section{Abstraction des modèles dans le code des services de Pip}

	Cette section détaillera l'objet de la preuve de concept mise à l'honneur dans ce chapitre : la modularisation des modèles et preuves des services de Pip, ainsi que l'autonomie du code des services vis à vis des modèles. Elle commencera par donner une vue globale de la nouvelle architecture du projet, indiquant les nouvelles relations entre les différents composants du projet. Dans un second temps, elle détaillera les interfaces créées, en illustrant de manière minimale les changements apportés à la fonction \texttt{switchContextCont}. Cette section mettra en évidence les différences avec l'implémentation précédente dépendantes des modèles. Cette seconde partie décrira aussi le processus d'extraction de l'\emph{AST} du code des services. Dans une dernière partie, cette section décrira la nouvelle structure des fichiers de preuve, en illustrant les différences (plus marginales) avec l'architecture précédente.
		
		\subsection{Vue générale}

		La principale contribution de cette preuve de concept est l'ajout d'\emph{interfaces} décrivant les dépendances fondamentales du code des services aux autres composants logiciels évoqués dans la section précédente \ref{sec:dependant_code}. Le code des services de Pip repose sur cette interface, qui ne décrit que les opérations ou types à fournir au code. L'implémentation réelle (et exécutable) de cette interface est réalisée en C, et s'exécutera conjointement avec le code des services compilé par Digger ou $\partial x$. Du coté du monde de la preuve formelle, de \emph{multiples} modèles peuvent décrire cette interface et ses effets. La figure \ref{fig:new_pip_architecture} décrit l'architecture de Pip selon cette preuve de concept. La colonne du milieu représente les interfaces nouvellement créées, sur lesquelles le code des services repose. La colonne de gauche représente les modèles décrivant les interfaces, et les preuves reposant sur ces interfaces. La colonne de droite représente l'implémentation réelle de l'interface en C sur laquelle repose le code des services compilé par Digger ou $\partial x$. Les flèches continues dans cette figure indiquent des composants dérivés automatiquement d'autres composants, par exemple le code des services compilé en C est dérivé automatiquement du code des services écrit en Gallina. Les flèches discontinues traduisent une relation d'implémentation. Par exemple, le modèle des fonctions implémente l'interface des fonctions.

			\begin{figure}[!ht]
				\begin{tikzpicture}[>=triangle 45,font=\sffamily, every text node part/.style={align=center}, scale=1, every node/.style={transform shape}] {
	%\node[draw, pattern=south west lines, minimum width=5.6cm, minimum height = 3.6cm] at (0, -3.1) {};
	%\node[draw, pattern=south west lines, minimum width=5.6cm, minimum height = 6.2cm] (model) at (0, -1.8) {};
	%\node[below=0cm of model] {Modèle monolithique en Gallina};
	
	\node[draw, pattern=south west lines, minimum width=4.6cm, minimum height = 3.6cm] (models) at (0, -3.1) {};
	\node[below=0cm of models] {Interface agnostique\\des modèles};
	\node[draw, fill=white, minimum width = 4cm, minimum height = 2cm] (services) at (0, 0) {Code des services};
	\node[draw, fill=white, minimum width = 4cm, minimum height = 0.7cm] (functions_interface) at (0, -1.9) {Interface des fonctions};
	\node[draw, fill=white, minimum width = 4cm, minimum height = 0.7cm] (types_interface) at (0, -2.7) {Interface des types};
	\node[draw, fill=white, minimum width = 4cm, minimum height = 0.7cm] (state_monad) at (0, -3.5) {Monade d'état générique};
	\node[draw, fill=white, minimum width = 4cm, minimum height = 0.7cm] (state_interface) at (0, -4.3) {Interface de l'état};
	%\node[draw, fill=white] (models) at (2, -2.3) {Modèle des fonctions,\\des types et de l'état};
	%\node[draw, fill=white] (state_monad) at (-2, -2.3) {Monade\\d'état};

	\node[draw, pattern=south west lines, minimum width=4.6cm, minimum height = 3.6cm] (models) at (-5, -3.1) {};
	\node[below=0cm of models] {Modèles d'isolation};
	\node[draw, fill=white, minimum width = 4cm, minimum height = 2cm] (proofs) at (-5, 0) {Fonctions ``fictives''\\Triplets de Hoare\\Scripts de preuve};
	\node[draw, fill=white, minimum width = 4cm, minimum height = 0.7cm] (functions_model) at (-5, -1.9) {Modèles de fonctions};
	\node[draw, fill=white, minimum width = 4cm, minimum height = 0.7cm] (types_model) at (-5, -2.7) {Modèles de types};
	\node[draw, fill=white, minimum width = 4cm, minimum height = 0.7cm] (derived_monad) at (-5, -3.5) {Monade spécifique};
	\node[draw, fill=white, minimum width = 4cm, minimum height = 0.7cm] (state_model) at (-5, -4.3) {Modèle de l'état};

	\node[draw, pattern=south west lines, minimum width=4.6cm, minimum height = 2cm] (impl) at (5, -2.3) {};
	\node[below=0cm of impl] {Implémentation exécutable};
	\node[draw, fill=white, minimum width=4cm, minimum height=2cm] (Cservices) at (5,0) {Code des services\\(Compilé en C)};
	\node[draw, fill=white, minimum width=4cm, minimum height = 0.7cm] (functions) at (5, -1.9) {Fonctions sur l'état};
	\node[draw, fill=white, minimum width=4cm, minimum height = 0.7cm] (types) at (5, -2.7) {Types};

	\draw[->] (services) -- (Cservices);
	\draw[->, dashed] (functions_model) to (functions_interface) ;
	\draw[<-] (derived_monad) to (state_monad) ;
	\draw[->, dashed] (types_model) to (types_interface) ;
	\draw[->, dashed] (state_model) to (state_interface) ;
	\draw[->, dashed] (types) to (types_interface) ;
	\draw[->, dashed] (functions) to (functions_interface) ;
}

\end{tikzpicture}

				\caption{Nouvelle architecture de Pip et dépendances des composants apportées par la preuve de concept}
				\label{fig:new_pip_architecture}
			\end{figure}

			Chaque interface est décrite en Coq sous la forme d'un \emph{Module Type}. Les implémentations de \emph{Modules Type} sont décrit en Coq par des \emph{Modules}, qui doivent nécessairement donner une définition à chaque élément déclaré dans le \emph{Module Type}.

		\subsection{Définition de code générique indépendant des modèles}

			\subsubsection{Abstraction du modèle de l'état}

			Le premier composant abstrait par cette preuve de concept est la définition de l'état. Cette abstraction donne naissance au premier \emph{Module Type} de cette contribution, le \emph{Module Type} de l'état. Ce \emph{Module Type} est extrêmement simple : il ne contient qu'une unique déclaration, le type \texttt{state}. La définition du \emph{Module Type} est présentée en listing \ref{code:StateParameter}. Ainsi, chaque modèle devra fournir son propre \emph{Module} contenant la définition du type de l'état.
			\begin{listing}[!ht]
				\coqcode{code/StateModel.v}
				\caption{Définition de l'interface de l'état}
				\label{code:StateParameter}
			\end{listing}

			La définition du \emph{Module} implémentant le modèle d'isolation reprend les définitions présentées dans la section précédente, et est disponible en annexe en listing \ref{code:IsolationState}.

			\subsubsection{Définition d'une monade d'état agnostique du modèle de l'état}

			La seconde modification apportée par cette contribution est la création d'une monade agnostique du modèle de l'état. Bien que la monade d'état ait besoin de faire référence à l'état, elle n'a pas besoin de sa définition. Ainsi, il est possible de créer un \emph{Module} paramétré par le \emph{Module Type} de l'état. Le \emph{Module} ainsi créé est de ce fait un foncteur (\emph{functor} en anglais), une fonction des \emph{Modules} vers les \emph{Modules}. En particulier, le foncteur de la monade d'état prend en paramètre un \emph{Module} implémentant le \emph{Module Type} \texttt{StateModel}, et renvoie un \emph{Module} décrivant la monade d'état associée. Le \emph{Module} implémentant la monade d'état agnostique au modèle de l'état est disponible en annexe, \ref{code:StateAgnosticMonad}.

			\subsubsection{Abstraction du modèle de types utilisés par Pip}

			La seconde interface créée est celle des types utilisés par le code des services et les fonctions sur l'état. Tout comme l'interface de l'état, cette interface est créée grâce à un \emph{Module Type} dont les déclarations pourront être utilisées par le code des services et les fonctions sur l'état. De la même manière, les objets déclarés dans ce \emph{Module Type} doivent être implémenté par chaque modèle sous la forme d'un \emph{Module}. Ces déclarations doivent aussi être implémentées en C ; cette implémentation sera utilisée par le code des services traduit en C par Digger ou $\partial x$. Le code du \emph{Module Type} déclarant les définitions nécessaires à la fonction \texttt{switchContextCont} est disponible en listing \ref{code:TypesParameters}.

			\begin{listing}[!ht]
				\coqcode{code/TypesParameters.v}
				\caption{Définition de l'interface des types nécessaires à la fonction \texttt{switchContextCont}}
				\label{code:TypesParameters}
			\end{listing}


			\subsubsection{Abstraction du modèle des fonctions de l'interface avec l'état}

			\begin{listing}[!ht]
				\coqcode{code/InterfaceParameters.v}
				\caption{Définition de l'interface des fonctions utilisées par le code des services (restreinte aux définitions nécessaires à la fonction \texttt{switchContextCont})}
				\label{code:InterfaceParameters}
			\end{listing}
			La dernière interface créée est celle déclarant les fonctions intéragissant avec l'état. Cette interface a besoin de faire référence à la fois aux types utilisés par les fonctions mais aussi à la monade, puisque les fonctions décrites sont monadiques. Ainsi, cette interface est définie par un \emph{Module Type} paramétré par le \emph{Module Type} déclarant les types nécessaires aux fonctions et par le \emph{Module Type} de l'état. Ce \emph{Module Type} représente donc le type d'un foncteur allant d'un \emph{Module} décrivant les types et d'un \emph{Module} décrivant l'état vers un \emph{Module} décrivant les fonctions monadiques d'interaction avec l'état utilisant les types passés en paramètre.

			L'implémentation de ce \emph{Module Type} (restreint aux définitions nécessaires à la fonction \texttt{switchContextCont}) est disponible en listing \ref{code:InterfaceParameters}. Cette implémentation utilise le foncteur \texttt{StateAgnosticMonad} sur le \emph{Module} d'état \texttt{State} passé en paramètre pour instancier le \emph{Module} de la monade d'état \texttt{SAMM} qui lui est nécessaire. Cette instanciation permet ensuite de déclarer les prototypes de fonction utilisant le type monadique défini dans ce \emph{Module}. Les fonctions dont les prototypes ont été déclarés dans ce \emph{Module Type}, ainsi que la monade d'état utilisée devront être définis par le \emph{Module} implémentant le \emph{Module Type}.

			\subsubsection{Code agnostique des modèles}

			La création de toutes ces interfaces permet de définir le code des services affranchit de toute dépendance aux modèles. Le code des services (restreint à la seule fonction \texttt{switchContextCont} dans cette preuve de concept) est disponible en listing \ref{code:ModelAgnosticCode}. Le code est défini dans un \emph{Module} dépendant de l'implémentation de trois \emph{Modules} : un \emph{Module} décrivant l'état, un \emph{Module} décrivant les types, et un \emph{Module} décrivant les fonctions monadiques. Ainsi, le code de Pip est devenu un foncteur allant de ces \emph{Modules} vers le \emph{Module} décrivant le code \textbf{spécifique} aux \emph{Modules} dont il dépend. Le code des services décrit comme un foncteur du listing \ref{code:ModelAgnosticCode} est néanmoins identique au code basé directement sur les modèles décrit en listing \ref{code:switchContextCont}.

			\textcolor{red}{Il y a t'il quelque chose d'intéressant à développer ici ?}

			\begin{listing}[!ht]
				\coqcode{code/ModelAgnosticCode.v}
				\caption{Définition du code affranchi de toute dépendance aux modèles}
				\label{code:ModelAgnosticCode}
			\end{listing}

		\subsection{Extraction de code}

		Pour que la preuve de concept présentée dans ce chapitre soit pertinente, il faut que le code des services puisse être compilé vers du code C. Comme évoqué dans les chapitres précédents, Pip est conçu pour s'interfacer avec deux outils permettant la compilation de code Gallina \emph{shalow-embedded} vers du code C : Digger ou $\partial x$.

		Cette sous-section discute du fichier source qui, une fois evalué par Coq, produit une représentation de l'\emph{AST} du code des services nécessaire à Digger pour produire le code C. La section précédente \ref{sec:AST_extr} discutait du fait que le fichier source importait tous les modèles d'isolation, liant de ce fait le code des services à ces modèles. Cependant, dans cette preuve de concept, le code n'est plus qu'un foncteur des modèles vers une instance spécifique du code. D'une certaine manière, l'ancien code des services est une instance particulière du code paramétrique de cette preuve de concept : le cas où le code aurait été instancié avec les \emph{Modules} des modèles d'isolation.

		Malheureusement, il n'est pas possible pour Coq de produire l'\emph{AST} d'un foncteur au moment où j'écris ces lignes. Néanmoins, Coq accepte de produire l'\emph{AST} d'une fonction -- et ce même sans connaître la définition des objets sur lesquels cette fonction repose. Ainsi, il est possible de définir des \emph{Modules} ``creux'', qui se contentent de déclarer qu'ils implémentent les \emph{Modules Type} attendus par le code des services agnostique des modèles, sans préciser d'implémentation. Dès lors, il est possible d'instancier le code avec ces modèles creux, ce qui transforme le code des services en fonctions dont Coq peut extraire l'\emph{AST}. De cette manière, il est possible d'extraire l'\emph{AST} des services de Pip sous la forme proposée par cette preuve de concept sans dépendre d'une implémentation particulière des modèles.

		\subsection{Méthode de preuve sur des foncteurs}

		Cette preuve de concept ne serait pas utile s'il n'était pas possible de raisonner formellement sur le code des services. Cette brève sous-section discute de la manière de structurer les nouveaux modèles et les nouvelles preuves afin de pouvoir prouver des propriétés formelles. L'établissement de modèles et la méthode de preuve restent \emph{identiques} aux méthodes utilisées jusqu'à présent.

		Tout d'abord, les instances spécifiques des modèles peuvent (et doivent probablement) dépendre les unes des autres. Par exemple, le modèle d'isolation des fonction monadiques peut dépendre du modèle d'isolation des types ou du modèle d'isolation de l'état : cette dépendance n'affecte en rien le code des services, et n'a de répercussions que sur le monde mathématique. Autrement dit, chaque \emph{Module} implémentant les \emph{Modules Type} (et n'importe quel autre fichier lié à un modèle spécifique) est libre de dépendre de n'importe quel autre fichier. Ce libre réseau de dépendances pour établir un modèle était déjà effectif dans le projet Pip avant l'introduction de cette preuve de concept. Ainsi, la règle nécessaire et suffisante indiquant si un modèle est valide au sens de cette preuve de concept est que ce modèle doit contenir une implémentation des \emph{Modules} de type, d'état et de fonctions sur l'état.

		Par ailleurs, afin de pouvoir raisonner sur le code des services avec les modèles décrit dans les \emph{Modules}, il est nécessaire de l'instancier avec ces \emph{Modules}. Par exemple, afin de pouvoir exprimer le triplet de Hoare de la fonction \texttt{switchContextCont} sur les modèles et propriétés d'isolation, le code des services doit tout d'abord être instancié avec les modèles d'isolation. La définition du triplet de Hoare de cette fonction est illustrée en annexe \ref{code:switchContextCont_agn_triple}.

		Comme dans la méthodologie de preuve précédente, il est tout à fait possible de définir des fonctions exclusives aux modèles ; ces fonctions pouvant permettre de définir les propriétés à prouver par exemple.

		Ainsi, la seule différence notable relative à la méthodologie de preuve apportée par cette preuve de concept est la nécessité de créer les modèles au travers de \emph{Modules}.

	\section{Perspectives de recherche et discussion}
		\subsection{Établissement d'un modèle alternatif permettant de prouver les propriétés fonctionnelles du service de transfert de flôt d'exécution}
		\subsection{Lien entre la preuve de bon fonctionnement et le back end de l'ordonnanceur}
		\subsection{Discussion de la méthodologie}
			% Conduire les deux preuves indépendamment est différent de conduire deux preuves l'une après l'autre (en oubliant la première)
			% Lever les interrogations 
				% - preuve en meme temps aide à débusquer les incohérences / simplifications abusives
				% - mais trop complexe : explosion des termes -> trop couteux


    %%%%%%%%%%%%%%%%%%%%%%%%%%%%%%%%%
    %% End of adding your content. %%
    %%%%%%%%%%%%%%%%%%%%%%%%%%%%%%%%%


    % Add the following chapters not to the current ›part‹ but one level above instead.
    \makeatletter
        \def\toclevel@chapter{-1}
        \def\toclevel@section{0}
    \makeatother

    \chapter{Conclusion}
    % This is where you conclude your thesis.

	\section{Conclusion}
		%\subsection{Résumé des contributions}

	\section{Perspectives}
		\subsection{Preuve fonctionelle de Pip}
		\subsection{Preuve du backend}
		\subsection{Logique de séparation}
		\subsection{Event-driven Earliest Deadline First}

	\section{Retour d'expérience ?}
		% Conseils à mon moi de début de thèse


    % Following are the files and commands for the bibliography and the author’s publications.
    \pagestyle{plain}

    \renewcommand*{\bibfont}{\small}
    \printbibheading
    \addcontentsline{toc}{chapter}{Bibliographie}
    \printbibliography[heading = none]

    \appendix

    \part{Annexes}

\chapter{Annexes de la première contribution}

\section{Implémentation de la routine de sauvegarde du contexte et d'harmonisation de la pile}

\begin{codeenv}
	\asmcode{code/cg_yieldGlue.s}
	\caption{Implémentation de la routine de sauvegarde du contexte et d'harmonisation de la pile}
	\label{code:cg_yieldGlue}
\end{codeenv}


\section{Création du contexte générique et appel vers le code prouvé}

\begin{codeenv}
	\ccode{code/yieldGlue.c}
	\caption{Création du contexte générique et appel vers le code prouvé}
	\label{code:yieldGlue}
\end{codeenv}


\end{document}
