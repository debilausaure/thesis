\documentclass
[
    twoside,                 % The thesis is formatted like a book. That is, odd and even pages are handled differently.
    openright,               % Starts a new chapter on an odd page number (right side).
    cleardoublepage = empty, % Clear pages inserted in order to have new chapters appear on odd pages are formatted with an empty style.
    fontsize = 11 pt,        % The size of the font.
    french,                % Support for American English.
    captions = tableheading, % Places the correct amount of space when the caption of a table is above the table.
    numbers = noenddot,      % Does not use a period at the end of numbered titles, such as sections or figures.
    footheight = 35 pt,      % Defines the height of the foot. Due to the line, it needs extra height.
%    draft,                   % Only displays boxes of figures. This option is useful if compilation takes a long time.
]
{scrbook}


\input{settings/document} % Contains commands that are used for certain information that is printed.


%%%%%%%%%%%%%%%%%%%%%%%%%%%%%%%%%%%%%%
%% Please adjust your options here. %%
%%%%%%%%%%%%%%%%%%%%%%%%%%%%%%%%%%%%%%

    % This section contains commands with important data for your thesis. Please adjust them in order for the document to be printed correctly.

    % Defines the length of the amount that a printed page will be cut from EACH side (including the inner side). This option only takes effect with \printVersiontrue and \professionalPrinttrue.
    \extraBorder{3 mm}

    % Shifts the inner margin outward by the amount specified. When the book is bound, part of the page will not be seen anymore. This option compensates for this loss. It only takes effect with \printVersiontrue.
    \bindingCorrection{6 mm}

    % Use the following command if this is a master thesis.

%    \printVersiontrue      % Use this value if you want to prepare your thesis for physical printing. In this case, links will not be colored. Without \professionalPrinttrue, the content will be moved outward by the binding correction, increasing the inner margin and decreasing the outer margin.
%    \professionalPrinttrue % Use this value if you want to have extra border for cutting and are not bound to paper formats (like A4). This option will increase the page size by the extra border on every side plus the binding correction once for the width. It only takes effect in combination with \printVersiontrue.
%    \fancyTheoremsfalse  % Use this value if you want to use the classical theorem style, where the text is italic. Further, with this style, the QED symbol is colorless.
%    \boldNumberSetsfalse % Use this value if you want variables for number sets (like N or R) to appear in blackboard bold rather than bold.

    % The title of the thesis.
    \myTitle{Conception, implémentation et preuve d'un service de transfert de flot d'exécution au sein d'un noyau de système d'exploitation}

    % The author’s name.
    \myName{Florian Vanhems}

    % The author’s program.
    \myProgram{Informatique}

    % A short summary of the thesis. These information will be used for the PDF meta data.
    \mySubject{
The work described in this document is related to formal proofs on operating systems.
The first breakthrough in the domain was the SeL4 project ; demonstrating that producing a complete proof on a microkernel was achievable, albeit very costly. In order to bring the proof's cost down, the CertikOS project showcased a more layered and modular approach, leveraging refinements.
The Pip kernel team tackled the problem from the opposite side, focusing on minimalism, using a shallow embedding methodology and getting rid of refinement altogether. This thesis' contributions are more specifically tied to the Pip kernel.
Previous work on the Pip protokernel focused on providing an isolation proof to Pip's services manipulating the system's memory. Yet, another critical aspect of the kernel -- handling the execution flow transfer from a partition to another -- remained to be designed.
The first contribution of this thesis outlines the design of a single service able to handle all possible control flow transfers in a system ; namely interrupts, faults and explicit control flow transfers. The design focuses on minimalism and code factorization in order to reduce the overall proof effort. An implementation of the service is provided for the Pip kernel. We believe the idea behind the service is general enough to be implemented in other kernels and other architectures.
The second contribution outlined in this thesis is the first formally proven correct userland implementation of an Earliest Deadline First scheduler for arbitrary jobs. The formal proof guarantees that the scheduler's election function respects the Earliest Deadline First scheduling policy, and is guaranteed to be optimal on mono-processor systems. This proof was partly conducted using Pip's usual methodology, leveraging a shallow embedding of the scheduler's code in Coq and a state monad. Nonetheless, while the Pip kernel properties were proven directly, the presented scheduler proofs include three refinement levels ; from the scheduling policy to the actual implementation. Furthermore, the scheduler uses the previously described service in order to pass the control flow to partitions and regain the execution flow through interrupts, showcasing its usability and versatility.
The last contribution presented in this thesis is a proof of concept severing Pip's isolation model from its code. This isolation model independance allows to build alternative models designed to reason on new properties while limiting the proof effort. As such, this contribution opens new research perspectives that were previously too costly to consider. Nonetheless, this proof of concept does not bring the same level of confidence on the composition of properties about the code formally proven on different models.
}

    % Some keywords of the thesis. These information will be used for the PDF meta data. Please use | as a separator and try to avoid commas.
    \myKeywords{Ph.D. thesis | manuscrit de thèse | computer science | informatique | context switch | commutation de contexte | scheduler | ordonnanceur | scheduling | ordonnancement | formal proof | preuve formelle | Coq | proof assistant | assistant de preuve | Earliest Deadline First | models | modèles | operating system | système d'exploitation | kernel | noyau | verification | vérification}

%%%%%%%%%%%%%%%%%%%%%%%%%%%%%%%%%%%%%%
%% End of options to adjust. %%%%%%%%%
%%%%%%%%%%%%%%%%%%%%%%%%%%%%%%%%%%%%%%


\input{settings/thesis-format}                        % Contains commands that define the general format and layout of the thesis.
\input{settings/bibliography-format}       % Contains commands for the layout of the bibliography.
% This file contains most of the packages used for this document. If you want to add a package, do it below the appropriate ribbon.
% Some packages are already included in other files in the ›settings‹ folder if they were already necessary. Thus, make sure to go through these files too if you want to know whether a certain package is already included.
%
% This file contains the following parts:
%   • Typography
%   • Math
%   • Fonts
%   • Graphics
%   • Tables
%   • Enumerations
%   • Algorithms
%   • Spaces and Special Characters
%   • Miscellaneous
%   • Additional Packages
%   • Hyperlinks

%%%%%%%%%%
%% Math %%
%%%%%%%%%%

% The following packages are the standard packages used in order to typeset math. They contain a lot of useful commands.
\usepackage{amsmath}
\usepackage{amssymb}
\usepackage{amsthm}
\usepackage{thmtools}
\usepackage{mathtools}
\usepackage{thm-restate}
\usepackage{dsfont}        % Yields far better blackboard-bold letters than \mathbb. Use \mathds in order to write such letters.
\usepackage{braceMnSymbol} % Adjusts overbraces and underbraces such that longer versions are put together seamlessly.

%%%%%%%%%%%
%% Fonts %%
%%%%%%%%%%%

\usepackage
[
    mono=false, % Disables the mono/typewriter font.
]
{libertinus-otf} % The main font used in this thesis.

\usepackage{fontspec}  % Package required to load custom fonts
\setmonofont{FiraCode} % Use Fira Code as default mono font
[
	Extension = .ttf,
	Path = fonts/Fira-Code/,
	UprightFont = *-Regular,
	BoldFont = *-Bold,
	ItalicFont = *-Light,
	Scale = 0.85
]

\usepackage{url} % Responsible for URL formatting.
\usepackage{bm}  % Allows to use sensible bold letters in math mode. This package has to go after the font packages. Otherwise it does not work correctly!

%%%%%%%%%%%%%%
%% Graphics %%
%%%%%%%%%%%%%%

\usepackage{graphicx} % The standard package for including graphics into your document.
\usepackage
[
    subrefformat = simple, % Formats the label of the \subref command without parentheses.
    labelformat = simple,  % Formats the mark of a subfigure without parentheses.
]
{subcaption}         % Enables it to have subfigures inside of a single figure.
\usepackage{wrapfig} % Allows to put figures next to text.

% Changing the \columnsep adds some space next to a warpfigure.
\columnsep = \mymargininnersep
% The reference label of a subfigure is redefined to have a non-breaking space and parentheses. (Thus, the subfigures show parentheses although the package options removed parentheses; otherwise, two pairs of brackets would be seen.)
\renewcommand*{\thesubfigure}{~(\alph{subfigure})}

% tikz has already been included in the file "settings/thesis-format.tex"

%%%%%%%%%%%%
%% Tables %%
%%%%%%%%%%%%

\usepackage{array}     % Improves the way that tables can be formatted.
\usepackage{booktabs}  % Adds lines (called ›rules‹) that can be used in tables and improves spacing.
\usepackage{longtable} % Allows to make tables that span multiple pages.
\usepackage{pdflscape} % Allows to change a page into landscape. This is handy if a table is very wide.


%%%%%%%%%%%%%%%%%%
%% Enumerations %%
%%%%%%%%%%%%%%%%%%

\usepackage{enumitem} % Adds tons of useful features to enumeration environments.


%%%%%%%%%%%%%%%%
%% Algorithms %%
%%%%%%%%%%%%%%%%

\usepackage[outputdir=build, newfloat=true]{minted} % WARNING : Requires shell escape
\newmintedfile[coqcode]{coq}{
	fontfamily=tt, % select the correct font family
	linenos=false,
	numberblanklines=true,
	numbersep=5pt,
	gobble=0,
	frame=lines,
	framerule=0.4pt,
	framesep=2mm,
	funcnamehighlighting=true,
	tabsize=4,
	obeytabs=false,
	mathescape=true
	samepage=false, %with this setting you can force the list to appear on the same page
	showspaces=false,
	showtabs =false,
	texcl=false,
}

\newmintedfile[ccode]{c}{
	fontfamily=tt, % select the correct font family
	linenos=false,
	numberblanklines=true,
	numbersep=5pt,
	gobble=0,
	frame=lines,
	framerule=0.4pt,
	framesep=2mm,
	funcnamehighlighting=true,
	tabsize=4,
	obeytabs=false,
	mathescape=true
	samepage=false, %with this setting you can force the list to appear on the same page
	showspaces=false,
	showtabs =false,
	texcl=false,
}

\newmintedfile[asmcode]{nasm}{
	fontfamily=tt, % select the correct font family
	linenos=false,
	numberblanklines=true,
	numbersep=5pt,
	gobble=0,
	frame=lines,
	framerule=0.4pt,
	framesep=2mm,
	funcnamehighlighting=true,
	tabsize=4,
	obeytabs=false,
	mathescape=true
	samepage=false, %with this setting you can force the list to appear on the same page
	showspaces=false,
	showtabs =false,
	texcl=false,
}

%%%%%%%%%%%%%%%%%%%%%%%%%%%%%%%%%%%
%% Spaces and Special Characters %%
%%%%%%%%%%%%%%%%%%%%%%%%%%%%%%%%%%%

\usepackage{xspace}   % Adds the functionality that a space after a command will be shown as a space in the output.
\usepackage
[
    shortcuts, % Allows to use short symbols for non-breaking hyphens and dashes instead of lengthy commands.
]
{extdash}             % Adds non-breaking hyphens and dashes.
\usepackage{setspace} % Allows to easily chnage the spacing inside of the document.

%%%%%%%%%%%%%%%%
%% Typography %%
%%%%%%%%%%%%%%%%

\usepackage
[
    babel = true, % Enables language-specific tuning.
]
{microtype}           % Uses the text space more efficiently.
\usepackage{csquotes} % Uses the correct quotes according to the current language.

%%%%%%%%%%%%%%%%%%%
%% Miscellaneous %%
%%%%%%%%%%%%%%%%%%%

\usepackage{xparse}    % Is used in order to define reasonable commands.
\usepackage{footnote}  % Allows it to extend the environments footnotes can be used in. It is said that this package is in conflict with ›hyperref‹. I did not note any troubles. However, if something is fishy, it is probably best to not use this package.
\usepackage{afterpage} % Adds the \afterpage command, which specifies that the provided argument shall be processed after the current page is finished.
\usepackage
[
    textsize = scriptsize, % Determines the text size of the TODO note.
]
{todonotes}            % Adds TODO notes to the document. These are small text areas inside of the margin of a page.

%%%%%%%%%%%%%%%%%%%%%%%%%%%%%%%%%%%%%%%%%%%%%%%%%%%%%%%%%%%%%%%%%
%% If you want to add new packages, add them below this ribbon %%
%%%%%%%%%%%%%%%%%%%%%%%%%%%%%%%%%%%%%%%%%%%%%%%%%%%%%%%%%%%%%%%%%


%%%%%%%%%%%%%%%%
%% Hyperlinks %%
%%%%%%%%%%%%%%%%

\usepackage
[
    bookmarks = true,                 % Generates boodmarks for the PDF.
    bookmarksopen = false,            % The bookmarks are closed by default.
    bookmarksnumbered = true,         % The bookmarks use the numbers of the corresponding headline.
    pdfstartpage = 1,                 % The first page seen when opening the PDF.
    pdftitle = {{\printTitle}},       % The PDF’s title in the meta data.
    pdfauthor = {{\printAuthor}},     % The PDF’s author name in the meta data.
    pdfsubject = {{\printSubject}},   % The PDF’s subject in the meta data.
    pdfkeywords = {{\printKeywords}}, % The PDF’s keywords in the meta data.
    breaklinks = true,                % Allows it to break links.
    \ifprintVersion
        hidelinks,                    % In the printed version, links are not highlighted, as they are not clickable.
    \else
    colorlinks = true,            % The text of hyperlinks is colored instead of having a colored box around it.
    allcolors = stroke1,          % Every hyperlink uses the same color. If you want to change specific colors, use the commands below.
    %        linkcolor = stroke1,          % The color of an in-document hyperlink.
    %        citecolor = stroke1,          % The color of a citation.
    %        filecolor = stroke1,          % The color of a file link.
    %        pagecolor = stroke1,          % The color of a reference to a page.
    %        urlcolor = stroke1,           % The color of a weblink.
    \fi
]
{hyperref} % The standard package that is used for creating hyperlinks inside of a document.

\usepackage
[
    %    capitalise, % Capitalizes the words in front of the labels. This can also be done by simply using \Cref instead of \cref. In order to have a greater variety, this option is not used.
    noabbrev,   % The words in front of the labels are not abbreviated.
    nameinlink, % Extends the link of a reference to the word in front of it.
]
{cleveref} % This package must be included after ›hyperref‹. It creates clever references that know what they refer to.
     % Contains the packages that this template provides.
\input{settings/commands}      % Contains newly defined commands useful for mathematics.

% This is the thesis. The front matter as well as the references should not be changed. The main matter can be edited freely.
\begin{document}


    \frontmatter
    % This file contains the layout of the title page.

% As taken from the MADIS doctoral school page : 
% https://lilliad.univ-lille.fr/doctorant/conseils-redaction-page-garde

%La page de garde (ou page de titre) de votre thèse doit comporter au moins les éléments suivants : 
%    le nom de l’université et son logo : Université de Lille ;
%    le nom de votre école doctorale ;
%    le nom de votre laboratoire ;
%    le titre de la thèse ;
%    la mention «Thèse préparée et soutenue publiquement par Votre Nom le XX/XX/20XX, pour obtenir le grade de Docteur en Votre discipline de thèse» (ou toute autre formulation équivalente) ;
%    la liste des membres du jury, avec leur fonction et leur affiliation.

%Votre discipline ou votre spécialité doit être indiquée telle qu’elle a été saisie lors de l’enregistrement du jury de thèse auprès du Service des affaires doctorales - Bureau des soutenances de l’Université. 

%Si vous avez rédigé votre thèse en anglais, n’oubliez pas de faire figurer les titres en anglais et en français sur votre page de garde. Tous deux sont nécessaires.


% This page uses a different geometry, as the content will be centered (not including the binding correction).
\ifprintVersion
    \ifprofessionalPrint
        \newgeometry
        {
            textwidth = 134 mm,
            textheight = 220 mm,
            top = 38 mm + \extraborderlength,
            inner = 38 mm + \mybindingcorrection + \extraborderlength,
        }
    \else
        \newgeometry
        {
            textwidth = 134 mm,
            textheight = 220 mm,
            top = 38 mm,
            inner = 38 mm + \mybindingcorrection,
        }
    \fi
\else
    \newgeometry
    {
        textwidth = 134 mm,
        textheight = 220 mm,
        top = 38 mm,
        inner = 38 mm,
    }
\fi

% The format of the title page.
\begin{titlepage}
    \sffamily
    \begin{center}
	{
        	\def\svgwidth{20 em}
		\input{images/ULille_black.pdf_tex}\\
		{\tiny{CRIStAL - UMR 9189}\hfil{ED MADIS - 631}}
	}
        \vfil
	{
		{
			{\Huge{\textsc{\textbf{Thèse}}}}
		}\\[1 em]
		{
			{présentée et soutenue publiquement le}\\
			{\textbf{T.B.D.}}
		}\\[1 em]
		{
			{pour l'obtention du grade de}\\
			{\LARGE{\textsc{\textbf{Docteur de l'Université de Lille}}}}\\
			{\large\textit{spécialité \printProgram}}
		}\\[1 em]
		{
			{par}\\[0.3 em]
			{\Large\textbf{\printAuthor}}
		}
	}
        \vfil
        {\LARGE
            \rule[1 ex]{\textwidth}{1.5 pt}
            \onehalfspacing\printTitleBold\\[1 ex]
            \rule[-1 ex]{\textwidth}{1.5 pt}
        }
    \end{center}
    
    \vfil
	{\small \centering \underline{Composition du jury :}}
    \begin{table}[h]
	\small
        \sffamily 
        {%\def\arraystretch{1.2}
	    \begin{tabular}{
		>{\raggedright\arraybackslash}p{0.32\textwidth}
		>{\bfseries\raggedright\arraybackslash}p{0.262\textwidth}% 0.738
		>{\itshape\raggedleft\arraybackslash}p{0.32\textwidth}
	    }
		June Andronick		& Examinatrice		& ~~~~~~~~~Professeur associée, UNSW Sydney\\
		Emmanuel Baccelli	& Rapporteur		& ~~Professeur des universités, Freie Universität Berlin\\
		Gilles Grimaud		& Directeur de thèse	& ~~Professeur des universités, Université de Lille\\
		Julia Lawall		& Rapportrice		& ~~~~~Directrice de recherche, Inria Paris\\
            \end{tabular}
        }
    \end{table}
\end{titlepage}

\restoregeometry


    \pagestyle{plain}

    \addchap{Abstract}
    % This file should contain the abstract.

Les travaux présentés dans ce document de thèse sont liés à la vérification formelle de propriétés sur des composants de systèmes d'exploitation. Les premiers travaux piliers de ce domaine sont ceux du projet seL4 ; démontrant que la vérification de propriétés formelles sur un micro noyau est réalisable, malgré un coût élevé. Pour réduire le coût de la preuve, le projet CertikOS a proposé une méthode de preuve plus étagée et plus modulaire, en tirant à l'extrême la méthode de preuve par raffinement. L'équipe du noyau Pip a pris le contrepied de ces travaux, en utilisant une méthodologie reposant sur un \emph{shallow embedding} et en prouvant les propriétés désirées directement plutôt qu'en utilisant la méthode par raffinement.

Les travaux présentés dans cette thèse sont plus spécifiquement liés au noyau Pip. Les travaux précédents sur le noyau Pip ont porté sur une preuve de préservation de l'isolation des services fournis par Pip manipulant la mémoire. Cependant, un aspect critique du noyau devait encore être conçu : le transfert de flot d'exécution d'une partition de mémoire à une autre.

La première contribution de cette thèse présente un nouveau service de Pip conçu pour supporter tous les transferts de flots d'exécution possibles au sein d'un système -- les interruptions, les fautes, et les appels explicites. Ce service gère de manière unifiée ces transferts de flot d'exécution afin de réduire au minimum l'effort de preuve. Une implémentation est proposée pour le noyau Pip.

La seconde contribution de cette thèse est la première implémentation au monde d'un ordonnanceur \emph{Earliest Deadline First} pour jobs arbitraires muni d'une preuve formelle de sa correction. La preuve garantit que la fonction d'élection respecte la politique \emph{EDF}, garantissant l'optimalité du planning sur les machines mono-processeur. La preuve a été conduite en partie en suivant la méthodologie habituelle de Pip, utilisant un \emph{shallow embedding} et une monade d'état. Elle a cependant été réalisée par raffinement. De plus, l'ordonnanceur se sert du service de transfert de flot d'exécution ; montrant la polyvalence et l'utilisabilité du service.

La dernière contribution présentée dans cette thèse est une preuve de concept libérant le code des services de Pip de ses liens avec le modèle d'isolation. Cette indépendance permet de créer des modèles alternatifs, permettant de raisonner sur le code à propos de nouvelles propriété tout en limitant l'effort de preuve. Cette contribution ouvre de nouvelles perspectives de recherche liées à la réduction du coup de raisonnement sur des propriétés additionnelles sur Pip. Cette preuve de concept n'apporte cependant pas que des avantages : en particulier sur la confiance accordée à la conjonction de propriétés formellement prouvées sur des modèles différents.


    % Temporary switch of language for the abstract

    \selectlanguage{english}
    \addchap{Abstract}
    % This file should contain the English abstract.
The work described in this document is related to formal proofs on operating systems.

The first breakthrough in the domain was the SeL4 project ; demonstrating that producing a complete proof on a microkernel was achievable, albeit very costly. 
In order to bring the proof's cost down, the CertikOS project showcased a more layered and modular approach, leveraging \emph{refinements}.
The Pip kernel team tackled the problem from the opposite side, by using a \emph{shallow embedding} methodology and getting rid of refinement altogether. This thesis' contributions are more specifically tied to the Pip kernel.

Previous work on the Pip protokernel focused on providing an isolation proof to Pip's services manipulating the system's memory. Yet, another critical aspect of the kernel -- handling the execution flow transfer from a partition to another -- remained to be designed.

The first contribution of this thesis outlines the design of a single service able to handle all possible control flow transfers in a system ; namely interrupts, faults and explicit control flow transfers. The design focuses on minimalism and code factorization in order to reduce the overall proof effort. An implementation of the service is provided for the Pip kernel. We believe the idea behind the service is general enough to be implemented in other kernels and other architectures.

The second contribution outlined in this thesis is the first formally proven correct userland implementation of an Earliest Deadline First scheduler for arbitrary jobs. The formal proof guarantees that the scheduler's election function respects the Earliest Deadline First scheduling policy, and is guaranteed to be optimal on mono-processor systems. This proof was partly conducted using Pip's usual methodology, leveraging a shallow embedding of the scheduler's code in Coq and a state monad. Nonetheless, while the Pip kernel properties were proven directly, the presented scheduler proofs include three refinement levels ; from the scheduling policy to the actual implementation. Furthermore, the scheduler uses the previously described service in order to pass the control flow to partitions and regain the execution flow through interrupts, showcasing its usability and versatility.

The last contribution presented in this thesis is a proof of concept severing Pip's isolation model from its code. This isolation model independance allows to build alternative models designed to reason on new properties while limiting the proof effort. As such, this contribution opens new research perspectives that were previously too costly to consider. Nonetheless, this proof of concept does not bring the same level of confidence on the composition of properties about the code formally proven on different models.


    % Switch back to French for the rest of the document

    \selectlanguage{french}
    % Unfortunately we have to redefine the parts commands because the language switch has redefined our redefined commands    
    \renewcommand*{\partformat}{}
% This command calls \partformat (#2) and displays the name of the part (#3).
\renewcommand*{\partlineswithprefixformat}[3]%
{%
    #2
    \thispagestyle{empty}
    \setlength{\mytmpa}{0.618\mypaperwidth}%
    \setlength{\mytmpb}{0.382\mypaperheight}%
    \ifprintVersion
        \ifprofessionalPrint
            \setlength{\mytmpa}{0.618\mypaperwidth + \mybindingcorrection + \extraborderlength}%
            \setlength{\mytmpb}{0.382\mypaperheight + \extraborderlength}%
        \fi
    \fi
    \begin{tikzpicture}[overlay, remember picture]%
        \node [inner sep = 0, outer sep = 0, anchor = north] at (current page.north west)%
        {%
            \begin{tikzpicture}[overlay, remember picture]%
            \draw[color = stroke1, line width = 0.7 mm] (\mytmpa, 0) -- (\mytmpa, -\mytmpb);%
            \end{tikzpicture}%
        };%
        \node (align) [align = right, below = \mytmpb - 2 ex, inner sep = 0, outer sep = 0, anchor = north west] at (current page.north west)%
        {%
            %\hspace{\mytmpa}\hspace{0.5 em}\partname\ \thepart\\[1 ex]
            \hspace{\mytmpa}\hspace{0.5 em}\partname\\[1 ex]
            \color{stroke1}#3%
        };%
    \end{tikzpicture}%
}


    \addchap{Remerciements}
    % Here you can write whom you want to thank.

Ni se nourrir, ni se loger n'est gratuit.
Je crois qu'avec ça j'ai tout dit.
Ce monde est cruel
Ce monde est cruel.
Ça y est, j'ai tout dit.

Pas manger, ça fait mourir, et je suis habitué au chauffage.
Tes besoins vitaux sont payants : t'as compris la prise d'otage.
Depuis tout petit dans la merde, tu sais qu'il faudra mailler.
Au moins un peu pour le loyer, au moins un peu pour grailler.
Depuis tout petit dans la merde, on t'apprend à travailler ;
personne ne va te ravitailler à l'œil,
personne ne va s'apitoyer, pas de bol.

Ce monde est cruel.
Ce monde est cruel.
Je peux développer encore, je le fais sans aucun effort.
Pour travailler (donc pour manger), on te prend à trois ans
-- on te lâche à vingt-cinq (tes meilleures années).
Si tu pars avant, tu démarres en bas de la pyramide
et tu fermes ta gueule. Tu fais les pires des tâches,
tu gravis les étages au ralenti. Tu tapines en stage,
t'es sous-payé et on t'oblige à sourire --
car c'est une chance (merci !) déjà d'être là
avec tes vieux diplômes. Tiens, parlons des diplômes.

Personne n'est sûr, mais fais-le quand même pour la sécurité.
D'ailleurs, toute ta vie, pense à sécuriser :
même si tu amasses -- ne dépense pas,
on ne sait pas ce qui peut arriver.
Tu peux mourir, c'est vrai. Mais, si c'est pas le cas,
tu peux souffrir du manque puis être interdit par ta banque
et ça, ça fait peur. Les banques ça fait peur.
Des banques privées s'enrichissent, et des pays s'endettent.
De tout petits groupes très riches face au reste du monde,
face au bétail, face à la masse de salariés sans tête.

N'oublie jamais qui gagne quoi lorsque tu taffes.
Si ça te fâche et que tu ne veux plus,
n'oublie jamais : tu ne manges plus.
Ça ressemble à un choix...
Si c'est pas de l'esclavagisme,
c'est quand même pas vraiment humaniste.

[...]

Ce monde est cruel.
Ce monde est cruel.
Et j'ai tellement de chance à côté des autres,
je trouve ça tellement cruel.

Hein ? Comment ça ? Dieu donnerait de la chance, du talent,
à certains mais pas à d'autres ? Ça me rend parano.
Je ne sais plus si je me suis entraîné, si, tout ça, je le mérite ?
Si l'univers était avec moi ou si ça fait dix ans que je me bats...

[...]

En vrai, je ne sais pas comment ça se passe.
En vrai, je ne sais pas qui maintient le cap.
Si ça vient de moi ou si ça vient des astres.

Ce monde est cruel.
Ce monde est cruel.
Faut changer les choses, si ce monde est cruel,
c'est sûr qu'il y en a d'autres.
Je remercie les anges, je remercie les autres,
je remercie les miens, remerciez les vôtres.
Ce monde est cruel, mais je vous remercie quand même.

Merci pour tout


    \setuptoc{toc}{totoc}
    \tableofcontents

    \pagestyle{headings}
    \mainmatter

    %%%%%%%%%%%%%%%%%%%%%%%%%%%%%%%%%%%%%%%%%%%%%%%%%
    %% Please add the content of your thesis here. %%
    %%%%%%%%%%%%%%%%%%%%%%%%%%%%%%%%%%%%%%%%%%%%%%%%%

    \part{Corps du document}
    
    \chapter{Introduction}
    % Here you introduce your topic to the reader.

\section{Contexte}

\subsection{Technologique}

\subsection{Humain}

Cette thèse a été menée à l'Université de Lille, en collaboration avec le \emph{Centre de Recherche en Informatique, Signal et Automatique de Lille} (communement abrégé en laboratoire CRIStAL). Cette thèse a été financée par une dotation de l'Université de Lille.

Cette thèse a été dirigée par Gilles Grimaud, directeur de l'équipe <<~\emph{eXtra Small, eXtra Safe}~>> (abrégé 2XS) du CRIStAL. L'équipe se spécialise dans la conception de logiciels et matériels apportant sécurité, fiabilité et efficacité aux systèmes embarqués fortement contraints. Les travaux menés dans l'équipe portent sur la conception d'un noyau de système d'exploitation munis de preuves formelles de propriétés d'isolation de la mémoire, sur les moyens d'attaque physique sur du logiciel (Bluetooth, LoRa, analyse de la consommation, ...), sur la détection de malware et obfuscation d'applications Android, mais aussi sur des objets mathématiques plus théoriques comme par exemple les fonctions corécursives et leur représentation dans un assistant de preuve.

\emph{2XS} a des relations privilégiées avec d'autres équipes du laboratoire, notamment celles faisant partie du même groupe thématique <<~\emph{Systèmes embarqués adaptables et sécurisés}~>>. Cette thèse a notamment tiré profit d'une forte proximité avec l'équipe \emph{SyCoMoRES}, dont les travaux portent sur la conception et l’analyse des systèmes embarqués temps réel, basé sur l’analyse symbolique de composants paramétriques. La seconde contribution de cette thèse est le fruit de cette collaboration.

Par ailleurs, l'équipe \emph{2XS} est hébergée à l'\emph{Institut de Recherche sur les Composants logiciels et matériels pour l’Information et la Communication Avancée} (abrégé IRCICA). L'IRCICA est un établissement conçu pour favoriser la recherche interdisciplinaire, ce qui a notamment permis à l'équipe de saisir de nombreuses opportunités de collaboration avec l'\emph{Institut d'Électronique, de Microélectronique et de Nanotechnologies} (abrégé laboratoire IEMN), et plus particulièrement avec le groupe de recherche \emph{CSAM} notamment sur les travaux relatifs à l'attaque de logiciel au travers de moyens physiques.

Les travaux présentés dans cette thèse sont liés au noyau de système d'exploitation nommé Pip développé dans l'équipe \emph{2XS}.

\subsection{Pip}

Pip est un noyau de système d'exploitation \emph{minimal} dont le seul but est de garantir l'isolation d'applications s'exécutant sur le système. Pour ce faire, Pip est muni de preuves formelles que ses services préservent les propriétés d'isolation lors de leur exécution. Pip utilise la mémoire virtuelle comme moyen de garantir ces propriétés.

Le projet Pip a démarré avec trois thèses fondatrices :
\begin{itemize}
	\item La thèse de Narjes Jomaa, soutenue en décembre 2018, a porté sur l'aspect formel du noyau. Narjes a développé une méthodologie permettant de raisonner sur le code des services de Pip, ainsi qu'une méthodologie de co-design du code des services avec les preuves formelles afin d'alléger l'effort de preuve global. Narjes est à l'origine des preuves de préservation de l'isolation fournies par Pip ;
	\item La thèse de Quentin Bergougnoux, soutenue en juin 2019, a porté sur l'implémentation du noyau sur l'architecture Intel x86, en particulier sur le code des services actuellement présents dans le noyau. Ses travaux ont aussi porté sur des preuves de concept explorant les possibilités de portage de Pip sur un environnement multicœur ;
	\item La thèse de Mahiedinne Yaker, soutenue en décembre 2019, a porté sur l'implémentation de Pip sur une plateforme embarquée basée sur l'architecture Intel, offrant des perspectives de travail sur les systèmes embarqués. Ces travaux ont aussi portés sur des réflexions autour de la conception de systèmes où les entités y demeurant ne se font pas mutuellement confiance.
\end{itemize}

De ces travaux fondateurs ont émergé de nouvelles opportunités de recherche, dont certains se sont transformés en sujets de thèse. Trois nouvelles thèses ont été pourvues, portant sur des sujets étendants les travaux fondateurs :
\begin{itemize}
	\item La thèse de Nicolas Dejon, soutenue en décembre 2022, qui porte sur l'application des propriétés d'isolation de Pip aux systèmes dépourvus de mémoire virtuelle, mais pouvant restreindre l'accès à certaines portions de mémoire grâce à une \emph{MPU}. Ces caractéristiques sont courantes sur des systèmes beaucoup plus modestes, et se prêtent particulièrement bien à de l'\emph{IoT} ;
	\item Les travaux initiaux de Sofia Santiago Fernandez qui portent sur la preuve de préservation de la sémantique du code des services lors de la compilation du code Gallina \emph{shallow-embedded} vers du code C ;
	\item Mes propres travaux de thèse, présentés dans ce document, portant sur la formalisation du transfert de flôt d'exécution au sein du noyau et de travaux préliminaires relatifs à l'ajout de nouvelles propriétés non relatives à l'isolation.
\end{itemize}

Les doctorants n'ont pas été les seules personnes recrutées pour participer au développement de Pip : c'est par exemple le cas de Damien Amara, recruté en tant qu'ingénieur de recherche. Damien a contribué de manière significative à l'implémentation de Pip sur l'architecture Armv7, ainsi qu'à la version de Pip pour les systèmes munis d'une \emph{MPU}. Pip a aussi été au cœur de nombreuses collaborations industrielles par exemple dans le cadre de projets européens, notamment avec Orange.

\section{Objet}

\subsection{Transfert de flot d'exécution}

\subsection{Ordonnanceur}

\subsection{Preuve}

\section{Présentation du document}

\subsection{Plan}

\subsection{Axes de lecture}



    \chapter{État de l'art (20-30 pages)}

	Ce chapitre a pour intention de définir et préciser les différentes notions nécessaires à la lecture des travaux de thèses, ainsi que de définir le contexte scientifique du travail. Il portera, dans une première section, sur les détails des différents transferts de flot d'exécution dans les systèmes modernes, ainsi que les changements d'états inhérents à ces transferts de flot d'exécution. Cette section abordera ensuite les problèmes de sécurité liés au transfert de flot d'exécution, ainsi que les techniques de mitigation de ces problèmes. Cette section terminera sur les problématiques temps réel touchant au transfert de flot d'exécution.
	La seconde section de ce chapitre fera un état des lieux de la preuve de programme. Elle commencera par discuter de ce qu'est une preuve et de leur vérificaton automatique, ainsi que des stratégies de conduite de preuve. Cette section continuera sur la preuve de programme, en particulier comment raisonner sur un programme impératif. Elle abordera aussi les notions de représentation du langage. Enfin, elle terminera sur les exemples de systèmes vérifiés formellement.

	\section{Transfert de flôt d'exécution}

		Cette section va détailler les différents transferts de flot d'exécution mis à disposition dans les cpus modernes.
		Dans cette section, nous détaillerons les transferts de flot d'exécution qui impliquent une reconfiguration explicite de l'état de la machine ayant un impact sur les droits d'accès aux ressources. Les appels d'une fonction d'un programme vers une autre fonction ne seront pas considérés dans cette section, même s'ils pourraient être considérés comme un transfert de flôt d'exécution.


		\subsection{Hardware}

			\subsubsection{Appels explicites}
			Les transferts les plus courants sont les transferts de flot d'exécution explicites, c'est-à-dire dont la cible est explicitement fournie lors de l’appel, ou clairement établie dans la documentation.

Par exemple, dans Linux, un processus peut demander l’ouverture d’un fichier avec l’appel système open(). Cet appel transfère le flot d’exécution d’un processus non privilégié vers le noyau Linux disposant du plus haut niveau de privilèges. Les fonctions appelables par des transferts explicites servent d’interface entre des logiciels disposant de droits distincts.



Ce type de transfert de flôt d'exécution, d'apparence assez anodine, est pourtant l'objet d'attaques multiples, dont le but est de faire dévier l'exécution (de préférence en mode privilégié) vers du code choisi par l'attaquant. Pour y arriver, un attaquant doit exploiter une vulnérabilité dans une portion de code, qui lui donnera le contrôle d'une zone de mémoire d'intérêt (la pile, le tas, ou même le code). Une fois qu'il contrôle cette zone mémoire, il lui suffit d'écrire un \emph{shellcode}, et d'exploiter une vulnérabilité dans du code privilégié pour que l'exécution du \texttt{return} de la fonction compromise saute dans le shellcode. L'attaquant gagne à ce moment le contrôle de la machine.

Commence alors un jeu du chat et de la souris pour essayer de mitiger l'impact de ces vulnérabilités. Pour compliquer la vie de l'attaquant, et qu'il lui soit plus difficile d'exécuter son shellcode, de nombreuses stratégies ont été entreprises par les fabriquant de matériels ainsi que par les développeurs de systèmes d'exploitation. 

\paragraph{Canaries}
Une première statégie est l'ajout de \emph{canary} qui visent à détecter les corruptions mémoires. Les canaries sont des valeurs écrites dans la pile ou le tas et qui sont générées aléatoirement à chaque exécution. Lors de la sortie de la frame protégée par le canary, le code vérifie que la valeur du canary correspond bien à celle qui avait été écrite initialement ; si ce n'est pas le cas, c'est qu'une corruption mémoire a eu lieu et une faute est levée.

Une des techniques permettant de vaincre les canaries est de lire la valeur initiale du canary avant de corrompre la mémoire. En effet, la canary \textbf{reste la même pour l'intégralité de l'exécution}. Une fois cette valeur récupérée, il suffit de corrompre la mémoire en réécrivant cette valeur au bon endroit pour échapper à la détection. De plus, si l'exploitation de la vulnérabilité permet de corrompre la mémoire de manière fine, il suffit d'éviter d'écrire sur la canary.

\paragraph{Droits fins pour les zones mémoires}
Une autre stratégie a été de définir des droits fins concernant l'accès aux différentes zones mémoires de l'espace d'adressage des processus. Le mécanisme de mémoire virtuelle permet de définir des droits d'accès propres à chaque page mémoire configurée (lecture, écriture, éxecution, accessible en mode non priviligié). Par exemple, les pages mémoire contenant du code sont typiquement configurées pour des accès en lecture et exécution, alors que les pages contenant des données (pour la pile, le tas, les sections de données d'un binaire) sont configurées pour des accès en lecture/écriture.

Cette stratégie de défense empêche un attaquant d'exploiter une vulnérabilité pour écrire un shellcode code dans la mémoire si on considère que chaque page de mémoire est soit exécutable, soit accessible en écriture. Cependant, il existe des cas d'usage légitimes qui violent cette contrainte, par exemple lors de compilation à la volée (ou JIT, pour Just-In-Time). Fatalement, de tels logiciels sont devenus la cible privilégiée des attaquants, on pourra par exemple citer Webkit.% https://github.com/saelo/cve-2018-4233
Heureusement, il est peu probable que de tels logiciels aient besoin de s'exécuter en mode privilégié. 

Pour affaiblir ce vecteur d'attaque, cette stratégie de défense est renforcée par des mécanismes de sécurité supplémentaires tels que le \emph{Supervisor Mode Access Prevention} (SMAP) et le \emph{Supervisor Mode Execution Prevention} (SMEP). SMAP permet au processeur de lever une faute lorsque qu'il exécute du code privilégié et qu'il essaie d'accéder (en lecture ou en écriture) à des données présentes dans l'espace utilisateur. SMEP permet en complément de lever une faute lorsque le processeur essaie d'exécuter du code dans l'espace utilisateur alors qu'il se trouve dans un mode d'exécution privilégié.

Ces mécanismes permettent d'isoler le code privilégié de potentiels shellcodes écrit en espace utilisateur. Ainsi, pour compromettre intégralement un système, l'attaquant doit à présent exploiter une vulnérabilité dans le code privilégié, ayant à sa disposition des pages mémoire soit accessibles en écriture soit exécutables et qui, de surcrois, ne font pas partie de l'espace utilisateur.
Nait alors une nouvelle technique d'exploitation de vulnérabilité. 

\paragraph{Return Oriented Programming}
Le ROP (pour \emph{Return Oriented Programming}) consiste à attaquer du code vulnérable en n'utilisant que le code déjà accessible dans l'environnement d'origine, mais en exécutant des portions arbitraires de celui-ci. L'attaque consiste à repérer des \emph{gadgets} : de brèves portions de code ayant un effet spécifique sur la mémoire ou les registres, suivi d'une instruction \texttt{return}. Pour l'attaquant, il suffit de dévier le flot d'exécution sur l'un de ces gadgets et de manipuler la mémoire, de manière à ce que l'exécution du gadget entraine l'exécution du suivant. L'attaquant parvient au final à exécuter son shellcode constitué d'une succession de gadgets, contournant les mécanismes de sécurité mentionnés dans le paragraphe précédent.\\

Plusieurs contre-mesures ont émergé pour rendre plus difficile le ROP.

\paragraph{Address Space Layout Randomization}
L'ASLR (pour \emph{Address Space Layout Randomization}) rend imprédictible l'adresse des différentes zones de mémoire au sein d'un espace d'adressage virtuel. Les adresses du binaire, de la pile, du tas, des librairies, du noyau, etc. sont rendus aléatoires à chaque nouvelle exécution. L'ASLR est un de ce fait un frein considérable au développement d'un shellcode en ROP, puisqu'il est impossible de prédire où se situeront les gadgets lors de la prochaine exécution.

L'ASLR n'est cependant pas parfait. Les adresses des zones mémoire restantes peuvent être révélées par des pointeurs dans les zones mémoires controlées par l'attaquant, %https://google.github.io/security-research/pocs/linux/cve-2021-22555/writeup.html#bypassing-kaslrsmep and %https://google.github.io/security-research/pocs/linux/bleedingtooth/writeup.html#leaking-the-memory-layout
ou grâce à des attaques micro-architecturales % Prefetch Side-Channel Attacks: Bypassing SMAP and Kernel ASLR. In: CCS’16 (2016) && Practical Timing Side Channel Attacks against Kernel Space ASLR. In: S&P’13 (2013)

\paragraph{Vérification de l'intégrité du flôt d'exécution}

Une autre approche permettant de réduire la marge de manoeuvre de l'attaquant et de vérifier que le flot d'exécution est conforme à celui attendu. À chaque appel et à chaque retour de fonction, le processeur vérifie si la cible du saut est valide. Plusieurs implémentations existent, notamment des implémentations métarielles au sein des processeurs, mais aussi certaines implémentation logicielles notamment provenant de compilateurs. Windows, macOS, Android, iOS utilisent déjà un mécanisme de vérification du flot d'exécution.
% https://community.arm.com/arm-community-blogs/b/architectures-and-processors-blog/posts/armv8-1-m-pointer-authentication-and-branch-target-identification-extension
% https://www.intel.com/content/dam/develop/external/us/en/documents/catc17-introduction-intel-cet-844137.pdf
% https://clang.llvm.org/docs/ControlFlowIntegrity.html


\paragraph{eXecute Only Memory}
Le XOM (pour \emph{eXecute Only Memory}), est une fonctionnalité de certain processeurs permettant de déclencher une faute lorsque qu'un accès en lecture est fait sur les pages mémoires configurées comme étant exécutables. Avant cette fonctionnalité, aucune distinction n'était faite entre le processus de récupération des instructions par le processeur et la lecture de données par l'utilisateur. Cette fonctionnalité rend considérablement pour difficile la recherche de gadgets, puisqu'il est impossible pour l'attaquant de lire le code qu'il souhaite compromettre directement sur la cible.
On pourrait cependant argumenter que cette fonctionnalité relève de la sécurité par l'obscurité, et qu'elle n'est pas réellement efficace.

\paragraph{\blockquote{Mieux vaut prévenir que guérir}}

Ces contre-mesures, sans cesse contournées par de nouvelles méthodes d'attaque, supposent qu'il existera toujours des vulnérabilités dans le logiciel comme dans le matériel et tentent donc de limiter au maximum leur impact sur les systèmes affectés. Une toute autre classe de mesures essaie de régler le problème en s'attaquant à l'existence même des vulnérabilités, plutôt que d'essayer minimiser leurs conséquences.

On pourrait citer les méthodes d’analyse statique, les méthodes d’exécution symbolique, de fuzzing, et plus particulièrement le langage Rust conçu pour éradiquer ces vulnérabilités par conception. Par ailleurs, des travaux ont été entamés pour prouver formellement les fonctionnalités de Rust.


			\subsubsection{Fautes}

Les différentes formes de fautes logicielles constituent aussi une forme de transfert de flot d'exécution avec élévation de privilèges. Les logiciels sont susceptibles de déclencher des fautes logicielles de différentes façons, par exemple :
\begin{itemize}
  \item décodage impossible de la prochaine instruction ;
  \item demande d'exécution d'une instruction impossible (division par zéro...) ; 
  \item demande d'accès à une adresse mémoire protégée, résultat de l'activité d'une MMU ;  
  \item demande d'exécution d'une instruction privilégiée en mode non-privilégié.
\end{itemize}
Dans ces différentes situations il s'agit de transferts implicites depuis le logiciel en faute vers une fonction d'un logiciel en charge du traitement de cette faute. Les différentes fonctions de gestion des fautes sont généralement définies par des éléments de configuration du matériel, et, le plus souvent, par l'intermédiaire d'une table (ou vecteur) dont le nom change d'une architecture de microprocesseur à l'autre. Ce vecteur précise généralement le niveau d'élévation de privilèges associé à l'exécution de la fonction de traitement de la faute. Sur les architectures Intel cette table est appelée \emph{IDT} (pour \emph{Interrupt Descriptor Table}).

\subsubsection{Interruptions matérielles}
Les interruptions matérielles sont des transferts non explicites à priori non contrôlés par le code non privilégié. Elles sont déclenchées par le matériel, signalant un événement important à traiter, tel que l'arrivée d'un paquet réseau par exemple. Les fonctions de traitement des interruptions matérielles ainsi que leur niveau de privilèges sont aussi définis dans l'\emph{IDT}.

Puisque les fautes et interruptions déclenchent un changement de privilèges, elles sont un vecteur d'attaque supplémentaire d'intérêt pour un attaquant cherchant à s'octroyer de nouveaux droits. En effet, les mêmes types de failles peuvent résider dans les routines de gestion de ces portions de logiciel. De plus, les interruptions et fautes brisent le flot d'exécution et modifient potentiellement l'environnement d'exécution du code interrompu. Cela les rends d'autant plus susceptibles de contenir des vulnérabilités, qui tombent alors dans la catégorie des vulnérabilités de \emph{concurrence critique}.

Même en ayant pleinement conscience des différentes interactions et dépendances entre les différents composants d'un système, les vulnérabilités de concurrence critique sont \textbf{notoirement difficiles à cerner}, principalement à cause d'un phénomène d'explosion combinatoire. Il peut s'avérer difficile de détecter une telle vulnérabilité par les tests, puisqu'ils sont souvent effectués dans des environnement très controlés où les mêmes conditions d'exécution sont artificiellement répétées, occultant d'autres fils d'exécution possibles. Malgré cela, si le développeur parvient à exhiber un fil d'exécution contenant un comportement anormal, il peut alors être délicat de reproduire le fil d'exécution ayant conduit à ce comportement. En effet, le fil d'exécution peut etre le résultat de nombreuses interactions - parfois non-déterministes - du programme avec son environnement. De plus, attacher un debuggueur tel que \texttt{gdb} peut modifier subtilement ces interactions, de manière telle qu'il soit impossible d'exhiber à nouveau le comportement anormal : on parle alors d'Heisenbug. Pour illustrer la difficulté à cerner cette catégorie de bugs, on pourrait citer un problème d'incohérence de cache dans le noyau de système d'exploitation de la Nintendo Switch après une interruption matérielle ayant mené à un changement de coeur. Les effets de ce bug avaient été observés dès la sortie de la console ; il n'a cependant été trouvé et corrigé qu'à la sortie du firmware 14.0.0 de la console, soit plus de 5 ans après les premiers rapports. % https://gist.githubusercontent.com/plutooo/2aadbd4a718e269df474079dd2e584fb/raw/7b3af77b5202366c8934c88ef251f1e905967040/gistfile1.txt
On pourrait aussi citer une vulnérabilité exploitée dans la pile IPV6 du noyau FreeBSD de la console Playstation 4 de Sony, profitant d'une situation de concurrence critique pour déclecher un Use-After-Free et compromettant le système d'exploitation. Cette faille présente sur tous les firmwares de la console depuis son lancement en 2013 a été découverte en 2018 puis divulguée et patchée en 2020, soit 7 ans après sa mise sur le marché.

Certains débuggers (notamment \texttt{rr}~\cite{mozRR}) ont implémenté une fonctionnalité "\emph{record and replay}" (enregistre et rejoue), permettant de capturer une trace du programme inspecté, puis de rejouer dynamiquement cette même trace à la demande. Cette fonctionnalité résoud le problème de la reproductibilité des comportement anormaux des programmes, et permet de surcrois de revenir à un état précédent de l'exécution lors d'une session de débuggage, ce qui est impossible avec les débuggers classiques. Certains émulateurs tels que Xen ou Qemu proposent des fonctionnalités "record and replay" sur les machines virtualisées. Malheureusement, les fonctionnalités "record and replay" pour des programmes sur plusieurs coeurs sont actuellement extrêmement lents, et profiteraient grandement d'implémentation matérielles si elles venaient à exister \cite{mozRR}.

De nombreux travaux ont été menés afin de détecter les situations de concurrence critique, par exemple par analyse statique~\cite{racerX}. D'autres travaux ont développés des méthodes plus particulières permettant de découvrir des situations de concurrence critique liées aux interruptions matérielles~\cite{sdracer}. Parallèlement, dans le monde de la preuve formelle, on pourrait citer les travaux ayant abouti à la logique de séparation~\cite{separationlogic}.

		\subsection{Software}

			\subsubsection{Changement de droits}

Sur l'architecture Intel x86, les différents transferts de flot d'exécution peuvent s'opérer par le biais de différentes instructions et événements matériels. Néanmoins, lorsqu'un changement de droits est requis lors d'un transfert, les différents chemins sont régis par un ensemble de mécanismes de contrôle relativement homogènes.

\paragraph{Changement de droits sous x86}
\label{ring}

Les privilèges attribués au code s'exécutant actuellement sur la machine sont ceux du \emph{segment} chargé dans le registre \texttt{CS} (pour \emph{code segment}). Les segments sont définis dans la \emph{GDT} (pour \emph{Global Descriptor Table}) à l'initialisation de la machine. La \emph{GDT} est une table globale spécifiée par Intel, dont l'adresse est accessible par un registre dédié. Dans cette table par exemple, Linux se contente de définir deux segments de code. Un premier segment associé au niveau de privilèges maximum du microprocesseur, nommé par Intel \emph{ring 0}, utilisé pour le code responsable du système ; et un second segment non-privilégié, associé au niveau de privilèges \emph{ring 3}, pour le reste du code. 

Contrairement au code s'exécutant avec le segment privilégié, le code s'exécutant sans privilège ne peut pas changer de segment à sa guise. Des mécanismes de contrôle du processeur déclenchent une faute si du code non-privilégié essaie de modifier son segment de code. Pour y parvenir, il est possible d'utiliser les \emph{gates}, qui sont des tremplins définis dans les tables globales du système (comme l'\emph{IDT} ou la \emph{GDT}), permettant au code non privilégié d'appeler une fonction prédéfinie qui s'exécute avec d'autres droits.

Lorsqu'un changement de segment déclenche un changement de niveau de privilèges, le processeur change de pile. Ce changement de pile permet d'éviter aux routines privilégiées les échecs dus à un manque de place sur la pile, ainsi qu'à les prémunir d'éventuelles interférences avec les procédures non privilégiées~\cite{intel_stack_switch}. 
Une pile doit être définie par niveau de privilèges (\emph{ring}) utilisé par le système ; leurs adresses doivent être renseignées dans une structure appelée \emph{TSS} (pour \emph{Task State Segment}). Cette structure est initialisée conjointement avec la \emph{GDT} qui contient son descripteur. Un registre dédié indique au processeur la position de ce descripteur dans la \emph{GDT}.

\paragraph{Appels systèmes sur x86}

Afin d'obtenir un transfert de flot d'exécution avec élévation de privilèges, le logiciel appelant non-privilégié peut exécuter des instructions dédiés aux différentes \emph{gates} des tables globales. Tout d'abord, l'instruction \texttt{int} permet d'appeler les \emph{gates} situées dans l'\emph{IDT}. Ces \emph{gates} sont soit des \emph{interrupt gates}, des \emph{trap gates} ou des \emph{task gates}~\cite{intel_idt_gates}. \texttt{int} s'accompagne d'un argument correspondant à l'index de la \emph{gate} ciblée dans l'\emph{IDT}. Le code ainsi appelé sera exécuté avec le niveau de privilèges spécifié par le segment de code indiqué dans la gate (et chargé dans le registre \texttt{CS}).

L'instruction \texttt{callf} permet d'utiliser les \emph{gates} situées dans la \emph{GDT}. Ces gates sont soit des \emph{call gates} ou des \emph{task gates}. Ces gates permettent de copier un nombre fixé d'arguments depuis la pile de l'appelant dans la pile du code privilégié à l'appel de l'instruction \texttt{callf}. Le nombre d'arguments à copier est renseigné dans la \emph{gate} ciblée par l'instruction. Là aussi, l'élevation de privilèges est spécifiée par le segment de code indiqué dans la gate et chargé dans \texttt{CS} lors de l'appel.

La troisième manière de déclencher une élévation de privilèges est l'instruction \texttt{sysenter}. Cette 
instruction ne sollicite aucune \emph{gate} : à la place, elle utilise les \emph{MSR} (pour \emph{model-specific registers}), qui sont des registres de contrôle du processeur. Ces \emph{MSR} sont manipulables grâce aux instructions \texttt{wrmsr} et \texttt{rdmsr}, qui sont des instructions privilégiées et qui permettent d'écrire et de lire dans ces registres respectivement. Un appel à \texttt{sysenter} utilise les MSR \texttt{0x174}, \texttt{0x175} et \texttt{0x176} pour charger \texttt{CS} \texttt{EIP} \texttt{SS} \texttt{ESP}. Le système doit donc avoir initialisé ces registres avant l'utilisation de \texttt{sysenter}. De plus, \texttt{sysenter} ne sauvegarde pas l'adresse de retour ni l'adresse de la pile lors d'un appel, qui doivent être placés dans les registres \texttt{ECX} et \texttt{EDX} au moment de l'appel à \texttt{sysexit} pour retourner dans le code appelant.

\paragraph{Interruptions et fautes sur x86}

Les interruptions liées au matériel sur l'architecture x86 étaient autrefois gérées par un coprocesseur (le PIC 8259 \emph{pour Programmable Interrupt Controller} ou plus récemment, l'APIC pour \emph{Advanced Programmable Interrupt Controller}). Ce coprocesseur est maintenant intégré au processeur, mais nous continuerons de parler de coprocesseur pour honorer l'histoire. Ce coprocesseur utilise les \emph{gates} situées dans l'IDT de la même manière que l'instruction \texttt{int}. Il est possible de configurer ce coprocesseur
pour qu'il utilise une certaine plage de niveaux d'interruption, ou pour qu'il masque temporairement la venue de nouvelles interruptions. Les fautes utilisent elles aussi l'\emph{IDT}, et utilisent les trente-deux premières \emph{gates} de la table, en fonction de la faute à déclencher. Les fautes et interruptions déclenchées par le processeur ou le coprocesseur ont toujours le droit d'utiliser les \emph{gates}, peu importe le niveau de privilèges du code s'exécutant au moment de l'interruption.

\paragraph{Fonctionnement de la \emph{MMU} sur l'architecture Intel 32 bits}

Sur l'architecture Intel x86, mais aussi sur toutes les autres architectures supportant une \emph{MMU} (pour \emph{Memory Management Unit}), il est possible d'associer des droits d'accès spécifiques à chaque pages de mémoire configurée dans l'espace d'adressage virtuel.

Le concept de traduction d'adresse virtuelle vers l'adresse réelle est le suivant. Les bits de poids forts de l'adresse virtuelle servent à traverser les tables de la MMU (sur l'architecture Intel x86, le \emph{Page Directory} et les \emph{Page Tables}). Les bits de poids faible correspondent à l'emplacement de l'adresse désirée dans la page réelle obtenue après traduction (souvent appelé \emph{offset}).

En particulier, sur Intel x86 et en mode de pagination 32 bits pour des pages de 4 Kio, les espaces d'adressage sont configurés par une structure de données arborescente de pages de 4Kio. Cette structure de données a deux étages : la racine appelée PD pour \emph{Page Directory}, et les feuilles appelées PT (pour \emph{Page Tables}). Le développeur renseigne l'adresse du Page Directory à utiliser dans le registre \texttt{CR3} du processeur ; cette adresse est alignée sur 4Kio, les 12 bits de poids faibles (11-0) sont ignorés ou sont réservés pour un autre usage.

Plus précisement, le \emph{Page Directory} est constitué de 1024 entrées de 32 bits appelées les \emph{PDE} (pour \emph{Page Table Entries}). Les 20 bits de poids fort (31-12) de ces entrées déterminent l'adresse de la \emph{Page Table} à utiliser, alignée sur 4Kio. Les \emph{Page Tables} sont aussi constituée de 1024 entrées de 32 bits appelées \emph{PTE} (pour \emph{Page Table Entries}). De la même manière, les 20 bits de poids forts (31-12) déterminent l'adresse de la page de mémoire réelle, alignée sur 4Kio. (Il est aussi possible de configurer des pages de 4Mio plutot que des pages de 4Kio en modifiants certains bits de controle des \emph{PDE}.)

Lors de la traduction d'une adresse virtuelle, les 10 bits de poids forts de l'adresse virtuelle (31-22) déterminent le numéro de \emph{PDE} à utiliser, les bits (21-12) déterminent le numéro de \emph{PTE}, et les 12 bits de poids faible restants (11-0) déterminent l'\emph{offset} de l'adresse cible dans la page réelle.~\cite{intel_32bits_paging}

\paragraph{Contrôle d'accès par la MMU sur Intel x86}
Dans le mode de pagination 32 bits d'Intel, les droits associés à chaque page sont présents dans les \emph{PTE}, dans les 12 bits de poids faible. Le bit 1 (\texttt{R/W}) permet d'empêcher les accès en écriture sur la page. Le bit 2 (\texttt{U/S}) permet d'empêcher n'importe quel accès utilisateur à la page - en lecture ou en écriture. Le niveau de privilège de l'accès dépend du \emph{CPL} (pour \emph{Current Privilege Level}) de l'instruction courante, souvent déterminée par le segment de code actuel.

Cependant, d'autres fonctionnalités de contrôle d'accès globaux existent.
Le bit \emph{SMAP} (pour \emph{Supervisor Mode Access Protection} présent dans le registre CR4 permet d'empêcher du code privilégié d'accéder aux pages mémoire annoncées comme étant des pages mémoire utilisateur (bit \texttt{U/S}).

Les architectures x86 (et x64) présentent dans le détail, une grande diversité de modalités de transferts de flot d'exécution, dont cette section a fait une synthèse incomplète. Cette pluralité de modalités de transfert de flot de contrôle est une source de failles de sécurité importante, tant les éléments de configurations sont nombreux et sujet à des paramétrages contradictoires. Le proto-noyau Pip a proposé une méthodologie nouvelle pour produire la preuve de bonne configuration des MMU, garantissant la propriété d'isolation des logiciels. La section suivante présente cette stratégie de conduite de preuve, puis la section suivante présente notre contribution pour adapter cette stratégie à la preuve de sécurité lors des transferts
de flot d'exécution avec élévation de privilèges.

				Espace d'adressage, niveau de privilèges

			\subsubsection{Capture de l'état d'exécution}

Un transfert de flot d'exécution implique l'existence de deux entités distinctes présentes à l'intérieur du système. Nous allons ici développer les morceaux de logiciels nécessaires permettant de reprendre l'exécution d'une entité.


Nous avons vu dans la section précédente que le transfert de flôt d'exécution peut survenir à l'insu total de l'entité interrompue. De ce fait, l'entité est peut être en plein travail, et ne s'est pas nécessairement préparée à ce transfert qui requisitionnerait instantanément une partie des ressources disponibles. C'est pourquoi il incombe au système d'exploitation de faire le nécessaire pour que le transfert soit transparent pour l'entité interrompue. L'entité doit pouvoir reprendre son travail comme si le transfert n'avait jamais eu lieu : le système d'exploitation doit garantir son \emph{contexte d'exécution}. 

\paragraph{Le contexte d'exécution}

Le contexte d'exécution doit alors contenir tout ce qui est susceptible d'être modifié ultérieurement par l'autre entité ou lors du transfert de flot d'exécution. Au minimum, le pointeur d'instruction est préservé. Lors d'un changement de droits sous x86, le processeur opère un changement de pile. 


De quoi est constitué ce contexte d'exécution ? À priori cela concerne au moins certains registres du processeur, la mémoire accessible par l'entité, 

Se pose alors la question de ce qu'est l'état d'un programme. Sur un système donné et un programme donné, quel est l'ensemble des informations qu'il est nécessaire de conserver pour reproduire à l'identique le comportement original du programme ?




		\subsection{Failles de sécurités associées}
			%https://google.github.io/security-research/pocs/linux/bleedingtooth/writeup.html#achieving-rip-control
			%https://google.github.io/security-research/pocs/linux/cve-2021-22555/writeup.html
			%https://github.com/Bonfee/CVE-2022-0995
			CVE historiques ? :D

			%https://pointer-authentication.github.io/

		\subsection{Ordonnancement}

			\subsubsection{Partage équitable du CPU}

			\subsubsection{Respect des contraintes de temps}

	\section{Preuve de code}

		\subsection{Vérification automatique d'une preuve}

			\subsubsection{Qu'est ce qu'une preuve ?}

				\paragraph{Axiomes}
				\paragraph{Hypothèses}
				\paragraph{Raisonnement}

			\subsubsection{Exemple de Coq}

			\subsubsection{Stratégie de conduite / vérification de preuve}

				\paragraph{Preuve directe}
				\paragraph{Preuve par raffinement}

		\subsection{Preuve de programme}

			\subsubsection{Raisonner sur un programme impératif}

				\paragraph{preconditions}

				\paragraph{règles de transition} (sémantique opérationnelle)

				\paragraph{postconditions} (Hoare, logique de séparation)

			\subsubsection{Langage}

				\paragraph{Représentation du programme} (deep/shallow)

				\paragraph{Monade}

		\subsection{Illustration système}

			\subsubsection{SeL4}
			\subsubsection{CertikOS}
			\subsubsection{Pip}


    \chapter{Service de transfert de flot d'exécution avec preuve d'isolation}

% Réecrire le modele de writeContext qui devrait écrire dans le modèle si la page donnée est une page noyau

	\section{Motivations}
		\subsection{Failles de sécurité}
		\subsection{Changement d'espace d'adressage opération privilégiée}
		\subsection{Arguments de co-design (minimaliste, générique)}
			

	\section{Description du service}
		\subsection{Définition}
		% protoype (paramètres)
		% definitions des structures de données
		\subsection{Illustration}
			\subsubsection{Appels explicites}
			\subsubsection{Fautes}
			\subsubsection{Interruptions}

		\subsection{Décomposition des opérations et généralisation}

	\section{Preuve d'isolation}
		\subsection{Définition de l'interface/monade}
			% choix des types (générique en fonction des architectures - contextes)
			% limite de la preuve (écritures atomiques / conceptuelles)
		\subsection{Rappel? des propriétes d'isolation}
		
		\subsection{Déroulement de la preuve}
			\subsubsection{Validation des paramètres}
			\subsubsection{Modification de l'état}
			\subsubsection{Transfert de flot d'exécution}

	\section{Retour d'expérience}
	% Remarques pragmatiques sur cette contribution
		\subsection{Métriques}
		\subsection{Prise de recul sur la nature de la preuve}


    \chapter{Politique d'ordonnancement prouvée}

	\section{Motivations}
	% Intro : ce qu'on va étudier : utliser notre méthode de dévelopement de logiciel pour prouver un algorithme de selection des cibles du transfert de flot d'exécution
	% Préemption n'est pas la politique
	% Autre type de preuve : garantir le respect des échéances / pourquoi est ce vraiment critique ?
	% Différence algorithme / imlplémentation de l'algorithme

		\subsection{Objet de preuve}
		% Description informelle des propriétés
	

	\section{Description structurelle}

		\subsection{Définition prototype et oracle}

		\subsection{Place de l'ordonnancement dans Pip}
			\subsubsection{Ordonnancement dans une partition}
			\subsubsection{Positionnement par rapport à un système d'exploitation grand public}
				% Ne pas avoir à prouver les propriétés de sécurité demandées par du code noyau
				% ordonnanceur et méthodologie sont agnostiques du système d'exploitation

		\subsection{Décomposition des éléments de l'ordonnanceur}
			\subsubsection{Vue générale}
			\subsubsection{Fonction d'élection}
			\subsubsection{État}
			\subsubsection{Interface avec l'état (monade et oracles)}
			\subsubsection{Back-end}

		
	\section{Conduite de la preuve}
		% section 5 de l'article

	\section{Mise en oeuvre / Implémentation}
		\subsection{Dualité implémentation/modélisation}
			% Méthode générale
		
		\subsection{Implémentation vue comme un cas particulier de l'interface abstraite}
			%section 4 du papier

	\section{Discussion sur la méthodologie suivie}
		\subsection{Métriques}
		\subsection{Choix des primitives}


    \chapter{Ajout incrémental de propriétés sur le code prouvé}


	Dans ce chapitre, nous présentons une preuve de concept d'abstraction des
	\section{Motivations}
		\subsection{Modularisation de la méthodologie de preuve de Pip}
	La section de discussion du premier chapitre a mis en avant le fait que Pip était conçu autour des propriétés d'isolation formellement prouvées. Le modèle des fonctions de l'interface et de l'état, jusqu'à la monade intégrée au code, sont liés aux propriétés d'isolation. Cette forte proximité est une conséquence de la philosophie de conception minimaliste de Pip, qui nous a poussé à ne définir que les éléments strictement nécessaires à l'établissement de le preuve de préservation de l'isolation. Cette approche nous a permis de minimiser l'effort de preuve permettant de garantir la propriété d'isolation, mais présente un désavantage majeur : le code des services de Pip n'est pas indépendant des modèles sur lesquels il repose.

	Ainsi, il n'existe qu'un modèle unique dans Pip qui ne peut évoluer que de manière itérative. Chaque évolution rend caduques les propriétés établies sur l'ancien modèle, et implique de produire une nouvelle preuve des mêmes propriétés avec le nouveau modèle. Le moindre ajout de chaque itération rendant de plus en plus difficile l'établissement la preuve à produire.
	Ceci est un frein considérable à la vérification de nouvelles propriétés sur le code de Pip, telle que la preuve fonctionnelle du service évoquée dans le second chapitre. Si les nouvelles propriétés impliquent des changements trop importants sur le modèle, l'effort de preuve à fournir deviendrait inatteignable après seulement quelques itérations.

		\subsection{Raisonner sur le lien entre le Yield et la fonction d'élection}
		% Description informelle des propriétés
	

	\section{Architecture monolithique}

		Cette section décrira de manière synthétique les composants actuels de Pip, en essayant de décrire leurs dépendances d'un point de vue logiciel. Elle commencera par donner brièvement une vue d'ensemble du projet. Puis, dans une première partie, elle décrira les dépendances du code des services sur les modèles décrits dans Pip. Elle dépliera les définitions pour mettre en lumière les dépendances qui existent entre les modèles des différents composants. Ensuite, dans une seconde partie, la section se penchera sur la méthode de preuve nécessaire à l'établissement de la preuve d'isolation. La section se concluera sur le processus de compilation du code, compilant le code des services de Gallina vers du code C.
		
		\subsection{Vue générale}

			\begin{figure}[!ht]
				\begin{tikzpicture}[>=triangle 45,font=\sffamily, every text node part/.style={align=center}, scale=1, every node/.style={transform shape}] {
	\node[draw, fill=white, minimum width=5cm, minimum height=2cm] (services) at (0,0) {Code des services};
	\node[draw, fill=white, minimum width=5cm, minimum height=2cm] (Cservices) at (8,0) {Code des services\\(Compilé en C)};
	\draw[->] (services) -- (Cservices) node[midway, above] {Digger} node[midway, below] {$\partial x$};
	\node[draw, fill=white, minimum width=2.5cm, minimum height=2cm] (functions_model) at (0,-2.5) {Fonctions\\sur l'état};
	\node[draw, fill=white, minimum width=2.5cm, minimum height=2cm] (state_model) at (-1.25, -5) {État};
	\node[draw, fill=white, minimum width=2.5cm, minimum height=2cm] (types_model) at ( 1.25, -5) {Types};
	\node[draw, fill=white, minimum width=2.5cm, minimum height=2cm] (state_monad) at ( 1.25, -7.5) {Monade d'état};

	\node[draw, pattern=south west lines, minimum width=5.6cm, minimum height = 2cm] at (8, -2.1) {};
	\node[draw, fill=white, minimum width=5cm, minimum height = 0.7cm] (functions) at (8, -1.7) {Fonctions sur l'état};
	\node[draw, fill=white, minimum width=5cm, minimum height = 0.7cm] (types) at (8, -2.5) {Types C};
}

\end{tikzpicture}

				\caption{Architecture actuelle de Pip et dépendances des composants}
				\label{fig:currentPipArchitecture}
			\end{figure}
			\begin{listing}[!ht]
				\coqcode{code/switchContextCont.v}
				\caption{Code du bloc de continuation \texttt{switchContextCont} du service de transfert de flot d'exécution}
				\label{code:switchContextCont}
			\end{listing}

		\subsection{Dépendance du code au modèle d'isolation}

			\subsubsection{Monade dépendante du modèle de l'état}

			Dans l'architecture actuelle de Pip, le code des services repose directement sur les définitions de la monade, en utilisant le type monadique \texttt{LLI}, et les fonctions \texttt{bind} et \texttt{ret} pour représenter la mise en séquence des instructions des services. Dans l'exemple présenté en listing \ref{code:switchContextCont}, la fonction retourne un type monadique \texttt{LLI yield\_checks}, utilise la fonction \texttt{bind} au travers du sucre syntaxique \texttt{perform [...] := [...] ;}, et indique sa valeur de retour grâce à la fonction \texttt{ret}.
			Malheureusement, le type monadique \texttt{LLI}, décrit en listing \ref{code:LLImonad}, dépend de l'état \texttt{state}, décrit en listing \ref{code:CurrentIsolationState}. \texttt{state} est le modèle de l'état conçu pour la preuve de préservation de l'isolation. Ceci est une première dépendance du code au modèle d'isolationi, dont le code devra se passer pour devenir indépendants des modèles construits pour Pip.

			\begin{listing}[!ht]
				\coqcode{code/LLIMonad.v}
				\caption{Définition du type de la monade d'état \texttt{LLI} dans le modèle actuel de Pip}
				\label{code:LLImonad}
			\end{listing}

			\begin{listing}[!ht]
				\coqcode{code/CurrentIsolationState.v}
				\caption{Définition de l'état \texttt{state} dans le modèle actuel de Pip}
				\label{code:CurrentIsolationState}
			\end{listing}

			\subsubsection{Code dépendant des modèles de types}

			D'autres dépendances du code aux modèles d'isolation passent par la représentation des types. Le code dépend des types utilisés pour représenter ses propres arguments, et valeur de retour enrobée par le type monadique \texttt{LLI}, mais aussi les arguments et valeurs de retour des fonctions de l'interface. Par exemple, la fonction \texttt{switchContextCont} décrite dans le listing \ref{code:switchContextCont}, dépend du modèle des types \texttt{page}, \texttt{interruptMask}, \texttt{contextAddr} et \texttt{yield\_checks}. Les modèles de ces types sont représentés dans le listing \ref{code:CurrentTypesModel}.
Cette dépendance renforce les liens entre le modèle d'isolation de Pip et le code de ses services, et doit donc disparaître.

			\begin{listing}[!ht]
				\coqcode{code/CurrentTypesModel.v}
				\caption{Définition des types nécessaires à la fonction \texttt{switchContextCont} dans le modèle actuel de Pip}
				\label{code:CurrentTypesModel}
			\end{listing}

			\subsubsection{Code dépendant des modèles des fonctions intéragissant avec l'état}
			Enfin, la dernière dépendance du code aux modèles est par le biais des fonctions intéragissant avec l'état. Le code des services fait directement appel \emph{aux modèles} de ces fonctions. Ainsi, la fonction \texttt{switchContextCont} présentée dans le listing \ref{code:switchContextCont}, est dépendant des modèles des fonctions \texttt{setInterruptMask}, \texttt{updateMMURoot}, \texttt{updateCurPartition}, \texttt{getInterruptMaskFromCtx}, \texttt{getPageRootPartition}, \texttt{noInterruptRequest} et \texttt{loadContext}. Cette dépendance n'a pas lieu d'être, et doit être supprimée pour atteindre un code des services agnostique des modèles.

			\begin{listing}[!ht]
				\coqcode{code/CurrentFunctionsModel.v}
				\caption{Définition des fonctions de l'interface avec l'état nécessaire à la fonction \texttt{switchContextCont} dans le modèle actuel de Pip}
				\label{code:CurrentFunctionsModel}
			\end{listing}

			\subsubsection{Extraction de l'\emph{AST} dépendant des modèles}
			Ce dernier paragraphe est dédié au fichier source Coq extrayant l'\emph{AST} du code des services de Pip. Ce fichier, une fois évalué par Coq, produit le fichier attendu en entrée par Digger, un des outils de compilation du code Gallina \emph{shallow embedded} vers du code C utilisé dans le projet Pip. Ce fichier a pour dépendances l'ensemble des modèles d'isolation sur lesquels reposent le code. Ainsi, dans l'état actuel du projet, l'extraction de l'\emph{AST} du code des services n'est possible que si l'intégralité des modèles d'isolation peut être évalué par Pip. Cette dépendance doit être supprimée pour que le code des services puisse être compilé en C sans avoir recours aux modèles.

		\subsection{Processus de preuve sur le code dépendant du modèle}
		\label{sec:dependant_code}

			Cette sous-section sera dédiée à la structure actuelle de la preuve d'isolation sur le code des services, mettant en avant les dépendances des différents groupes de fichiers nécessaires à chaques preuves.

			\subsubsection{Définition des propriétés d'isolation et des fonctions nécessaires à la définition des propriétés}

			Les premiers fichiers nécessaires à l'établissement de la preuve sont ceux contenant les définitions nécessaires à l'expression des propriétés d'isolation sur le noyau. Ces définitions additionnelles sont totalement fictives ; elles n'ont pas vocation à être exprimées en C. Elles servent de fondation à l'expression des triplets de Hoare d'isolation à montrer par la preuve formelle. Par exemple, ces fonctions peuvent permettre de définir des ensembles nécessaires à certaines propriétés d'isolation. C'est le cas de la fonction \texttt{getAccessibleMappedPages} qui récupère les pages mappées et accessibles dans l'espace d'adressage d'une partition, nécessaire à la propriété d'isolation noyau \texttt{kernelDataIsolation}. Ces fonctions peuvent aussi être des miroirs purement fonctionnels de code monadique présent dans les services de Pip, telle que la fonction \texttt{readPhysical} permettant de lire l'adresse d'une page mémoire ; ces fonctions sont parfois requises par les définitions des fonctions précédemment mentionnées. Ces définitions servent ensuite à définir les propriétés d'isolation souhaitées. Les fonctions et propriétés définies de cette manière dépendent donc des modèles d'isolation, que ce soit le modèle de types, de l'état, ou des fonctions de l'interface.

			\subsubsection{Définition des triplets de Hoare, des lemmes intermédiaires et des scripts de preuve}

			Une fois que ces définitions établies, il est possible d'exprimer les triplets de Hoare sur le code des services. À titre d'exemple, le listing \ref{code:switchContextCont_triplet} décrit le triplet de Hoare de la fonction \texttt{switchContextCont}. Sous chaque triplet (et chaque lemme intermédiaire) se trouve un script de preuve, décrivant les règles d'inférence (ou tactiques) à appliquer successivement pour faire progresser Coq vers la conclusion. Les triplets de Hoare dépendent du code des services, des fonctions fictives utiles à la définition des propriétés, et dépendent donc à fortiori des modèles d'isolation.
			\begin{listing}[!ht]
				\coqcode{code/switchContextCont_triplet.v}
				\caption{Définition du triplet de Hoare de la fonction \texttt{switchContextCont} pour la preuve de préservation de l'isolation de Pip}
				\label{code:switchContextCont_triplet}
			\end{listing}

		
	\section{Abstraction des modèles dans le code prouvé}

	Cette section détaillera l'objet de la preuve de concept mise à l'honneur dans ce chapitre : la modularisation des modèles et preuves des services de Pip, ainsi que l'autonomie du code des services vis à vis des modèles. Elle commencera par donner une vue globale de la nouvelle architecture du projet, indiquant les nouvelles relations entre les différents composants du projet. Dans un second temps, elle détaillera les interfaces créées, en illustrant de manière minimale les changements apportés à la fonction \texttt{switchContextCont}. Cette section mettra en évidence les différences avec l'implémentation précédente dépendantes des modèles. Cette seconde partie décrira aussi le processus d'extraction de l'\emph{AST} du code des services. Dans une dernière partie, cette section décrira la nouvelle structure des fichiers de preuve, en illustrant les différences (plus marginales) avec l'architecture précédente.

		\subsection{Définition de code générique indépendant des modèles}

			\subsubsection{Vue générale}

		La principale contribution de cette preuve de concept est l'ajout d'\emph{interfaces} décrivant les dépendances fondamentales du code des services aux autres composants logiciels évoqués dans la section précédente \ref{sec:dependant_code}. Le code des services de Pip repose sur cette interface, qui ne décrit que les opérations ou types à fournir au code. L'implémentation réelle (et exécutable) de cette interface est réalisée en C, et s'exécutera conjointement avec le code des services compilé par Digger ou $\partial x$. Du coté du monde de la preuve formelle, de \emph{multiples} modèles peuvent décrire cette interface et ses effets. La figure \ref{fig:new_pip_architecture} décrit l'architecture de Pip selon cette preuve de concept. La colonne du milieu représente les interfaces nouvellement créées, sur lesquelles le code des services repose. La colonne de gauche représente les modèles décrivant les interfaces, et les preuves reposant sur ces interfaces. La colonne de droite représente l'implémentation réelle de l'interface en C sur laquelle repose le code des services compilé par Digger ou $\partial x$.

			\begin{figure}[!ht]
				\begin{tikzpicture}[>=triangle 45,font=\sffamily, every text node part/.style={align=center}, scale=1, every node/.style={transform shape}] {
	%\node[draw, pattern=south west lines, minimum width=5.6cm, minimum height = 3.6cm] at (0, -3.1) {};
	%\node[draw, pattern=south west lines, minimum width=5.6cm, minimum height = 6.2cm] (model) at (0, -1.8) {};
	%\node[below=0cm of model] {Modèle monolithique en Gallina};
	
	\node[draw, pattern=south west lines, minimum width=4.6cm, minimum height = 3.6cm] (models) at (0, -3.1) {};
	\node[below=0cm of models] {Interface agnostique\\des modèles};
	\node[draw, fill=white, minimum width = 4cm, minimum height = 2cm] (services) at (0, 0) {Code des services};
	\node[draw, fill=white, minimum width = 4cm, minimum height = 0.7cm] (functions_interface) at (0, -1.9) {Interface des fonctions};
	\node[draw, fill=white, minimum width = 4cm, minimum height = 0.7cm] (types_interface) at (0, -2.7) {Interface des types};
	\node[draw, fill=white, minimum width = 4cm, minimum height = 0.7cm] (state_monad) at (0, -3.5) {Monade d'état générique};
	\node[draw, fill=white, minimum width = 4cm, minimum height = 0.7cm] (state_interface) at (0, -4.3) {Interface de l'état};
	%\node[draw, fill=white] (models) at (2, -2.3) {Modèle des fonctions,\\des types et de l'état};
	%\node[draw, fill=white] (state_monad) at (-2, -2.3) {Monade\\d'état};

	\node[draw, pattern=south west lines, minimum width=4.6cm, minimum height = 3.6cm] (models) at (-5, -3.1) {};
	\node[below=0cm of models] {Modèles d'isolation};
	\node[draw, fill=white, minimum width = 4cm, minimum height = 2cm] (proofs) at (-5, 0) {Fonctions ``fictives''\\Triplets de Hoare\\Scripts de preuve};
	\node[draw, fill=white, minimum width = 4cm, minimum height = 0.7cm] (functions_model) at (-5, -1.9) {Modèles de fonctions};
	\node[draw, fill=white, minimum width = 4cm, minimum height = 0.7cm] (types_model) at (-5, -2.7) {Modèles de types};
	\node[draw, fill=white, minimum width = 4cm, minimum height = 0.7cm] (derived_monad) at (-5, -3.5) {Monade spécifique};
	\node[draw, fill=white, minimum width = 4cm, minimum height = 0.7cm] (state_model) at (-5, -4.3) {Modèle de l'état};

	\node[draw, pattern=south west lines, minimum width=4.6cm, minimum height = 2cm] (impl) at (5, -2.3) {};
	\node[below=0cm of impl] {Implémentation exécutable};
	\node[draw, fill=white, minimum width=4cm, minimum height=2cm] (Cservices) at (5,0) {Code des services\\(Compilé en C)};
	\node[draw, fill=white, minimum width=4cm, minimum height = 0.7cm] (functions) at (5, -1.9) {Fonctions sur l'état};
	\node[draw, fill=white, minimum width=4cm, minimum height = 0.7cm] (types) at (5, -2.7) {Types};

	\draw[->] (services) -- (Cservices);
	\draw[->, dashed] (functions_model) to (functions_interface) ;
	\draw[<-] (derived_monad) to (state_monad) ;
	\draw[->, dashed] (types_model) to (types_interface) ;
	\draw[->, dashed] (state_model) to (state_interface) ;
	\draw[->, dashed] (types) to (types_interface) ;
	\draw[->, dashed] (functions) to (functions_interface) ;
}

\end{tikzpicture}

				\caption{Nouvelle architecture de Pip et dépendances des composants selon la preuve de concept}
				\label{fig:new_pip_architecture}
			\end{figure}

			\subsubsection{Abstraction du modèle de types utilisés par Pip}

			\begin{listing}[!ht]
				\coqcode{code/TypesParameters.v}
				\caption{Définition de l'interface des types nécessaires à la fonction \texttt{switchContextCont}}
				\label{code:TypesParameter}
			\end{listing}

			\subsubsection{Abstraction du modèle de l'état}

			\begin{listing}[!ht]
				\coqcode{code/StateModel.v}
				\caption{Définition de l'interface de l'état}
				\label{code:StateParameter}
			\end{listing}

			\subsubsection{Définition d'une monade d'état agnostique du modèle de l'état}

			\begin{listing}[!ht]
				\coqcode{code/StateAgnosticMonad.v}
				\caption{Définition de la monade d'état}
				\label{code:StateAgnosticMonad}
			\end{listing}

			\subsubsection{Abstraction du modèle des fonctions de l'interface avec l'état}

			\begin{listing}[!ht]
				\coqcode{code/InterfaceParameters.v}
				\caption{Définition de l'interface des la monade d'état}
				\label{code:InterfaceParameters}
			\end{listing}

			\subsubsection{Code agnostique des modèles}

			\begin{listing}[!ht]
				\coqcode{code/ModelAgnosticCode.v}
				\caption{Définition du code affranchi de toute dépendance aux modèles}
				\label{code:ModelAgnosticCode}
			\end{listing}

		\subsection{Définition du modèle d'isolation s'interfaçant avec le code générique}
			\subsubsection{Définition du modèle des types du modèle de l'isolation}

			\subsubsection{Définition du modèle de l'état du modèle d'isolation}

			\subsubsection{Instanciation de la monade d'état relative au modèle d'isolation}

			\subsubsection{Définition du modèle des fonctions de l'interface avec l'état}

		\subsection{Méthode de preuve sur le code générique}
		% Instantiation de la monade pour raisonner sur les propriétés à prouver

		\subsection{Extraction de code}

	%\section{Illustration sur la fonction Yield de Pip}
	%	% Preuve fonctionnelle de Yield
	%	\subsection{Définition des propriétés de bon fonctionnement du Yield}
	%	\subsection{Modélisation de l'état nécessaire au bon fonctionnement de Yield}
	%	\subsection{Sketch of proof / Déroulé de la preuve}

	\section{Perspectives de recherche et discussion}
		\subsection{Établissement d'un modèle alternatif permettant de prouver les propriétés fonctionnelles du service de transfert de flôt d'exécution}
		\subsection{Lien entre la preuve de bon fonctionnement et le back end de l'ordonnanceur}
		\subsection{Discussion de la méthodologie}
			% Conduire les deux preuves indépendamment est différent de conduire deux preuves l'une après l'autre (en oubliant la première)
			% Lever les interrogations 
				% - preuve en meme temps aide à débusquer les incohérences / simplifications abusives
				% - mais trop complexe : explosion des termes -> trop couteux


    %%%%%%%%%%%%%%%%%%%%%%%%%%%%%%%%%
    %% End of adding your content. %%
    %%%%%%%%%%%%%%%%%%%%%%%%%%%%%%%%%


    % Add the following chapters not to the current ›part‹ but one level above instead.
    \makeatletter
        \def\toclevel@chapter{-1}
        \def\toclevel@section{0}
    \makeatother

    \chapter{Conclusion}
    % This is where you conclude your thesis.

	\section{Conclusion}
		%\subsection{Résumé des contributions}

	\section{Perspectives}
		\subsection{Preuve fonctionelle de Pip}
		\subsection{Preuve du backend}
		\subsection{Logique de séparation}
		\subsection{Event-driven Earliest Deadline First}

	\section{Retour d'expérience ?}
		% Conseils à mon moi de début de thèse


    % Following are the files and commands for the bibliography and the author’s publications.
    \pagestyle{plain}

    \renewcommand*{\bibfont}{\small}
    \printbibheading
    \addcontentsline{toc}{chapter}{Bibliographie}
    \printbibliography[heading = none]

    \appendix

    \part{Annexes}

\chapter{Annexes de la première contribution}

\section{Implémentation de la routine de sauvegarde du contexte et d'harmonisation de la pile}

\begin{codeenv}
	\asmcode{code/cg_yieldGlue.s}
	\caption{Implémentation de la routine de sauvegarde du contexte et d'harmonisation de la pile}
	\label{code:cg_yieldGlue}
\end{codeenv}


\section{Création du contexte générique et appel vers le code prouvé}

\begin{codeenv}
	\ccode{code/yieldGlue.c}
	\caption{Création du contexte générique et appel vers le code prouvé}
	\label{code:yieldGlue}
\end{codeenv}


\end{document}
