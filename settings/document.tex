% This file contains all sorts of commands that are used in order to specify certain options for the document.

\newif\ifprintVersion   % Defines a binary variable that signals whether the document is prepared for physical or digital print.
\newif\ifprofessionalPrint % Defines a binary variable that signals whether the print will be done by a professional printing service that requests extra margin for page cutting and is not bound to paper formats like A4.
\newif\iffancyTheorems  % Defines a binary variable that signals whether theorems are formatted in the classical style or in a new format that better suits the overall flavor of this thesis.
\newif\ifboldNumberSets % Defines a binary variable that signals whether the variables for number sets (like N or R) should be in bold. If not, they are in blackboard bold instead.

% Set all variables to their default values.
\printVersionfalse
\professionalPrintfalse
\fancyTheoremstrue
\boldNumberSetstrue

%%%%%%%%%%%%%%%%%%%%%%%%
% The following commands define certain strings that provide important information for the document.

% The title of the thesis.
\newcommand*{\printTitle}{}
\newcommand*{\myTitle}[1]{\renewcommand*{\printTitle}{#1}}
\newcommand*{\printTitleBold}{\textbf{\printTitle}}

% The author’s name.
\newcommand*{\printAuthor}{}
\newcommand*{\myName}[1]{\renewcommand*{\printAuthor}{#1}}

% The name of the author’s program.
\newcommand*{\printProgram}{}
\newcommand*{\myProgram}[1]{\renewcommand*{\printProgram}{#1}}

% A short description of the topic of the thesis. This string will be used for the PDF metadata.
\newcommand*{\printSubject}{}
\newcommand*{\mySubject}[1]{\renewcommand*{\printSubject}{#1}}

% A short description of the topic of the thesis. This string will be used for the PDF metadata.
\newcommand*{\printKeywords}{}
\newcommand*{\myKeywords}[1]{\renewcommand*{\printKeywords}{#1}}

% Defines the extra length added to each side for the print version.
\newlength{\extraborderlength}
\newcommand*{\extraBorder}[1]{\setlength{\extraborderlength}{#1}}

% Defines the length of the binding correction. (The class ›scrbook‹ has a binding correction but it does not work due to all the other packages that are loaded.)
\newlength{\mybindingcorrection}
\newcommand*{\bindingCorrection}[1]{\setlength{\mybindingcorrection}{#1}}
