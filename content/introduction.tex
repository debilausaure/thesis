% Here you introduce your topic to the reader.

\section{Contexte}

\subsection{Technologique}

\subsection{Humain}

Université de Lille

Ou placer l'IRCICA ?

CRIStAL
	SEAS
		Émeraude
			
			Vlad Rusu, Giuseppe Lipari
		2XS
			Attaques du logiciel par des moyens physiques
			Hautes garanties sur les systèmes par le biais de preuves de programme

Transition sur Pip

\subsection{Pip}

Noyau de système d'exploitation minimal utilisant une Memory Management Unit, muni de preuves formelles de sécurité.

3 thèses déjà produites sur le projet (Narjes, Quentin, Mahiedinne), un post doc embauché,
Thèse sur la partie théorique formelle du noyau, thèse sur les aspects systèmes, intégration de Pip à des systèmes embarqués

3 nouvelles thèses démarrées : 
- application des propriétés de sécurité de Pip à un système muni d'une MPU
- preuve de conservation de la sémantique lors du passage du code Coq au code C
- formalisation du transfert du flot d'exécution

\section{Objet}

\subsection{Transfert de flot d'exécution}

\subsection{Ordonnanceur}

\subsection{Preuve}

\section{Présentation du document}

\subsection{Plan}

\subsection{Axes de lecture}

\coqcode{code/state.v}

\begin{figure}
	\centering
	\begin{tikzpicture}
\node[minimum width=3cm, minimum height=3cm] (vidt) at (0,0) {};
\node[above=0.1cm of vidt] {Partition VIDT};
\node[above left=0.1cm of vidt] {No.};
%\draw[dashed] (3.5, -1.5) -- (3.5, 1.5);
%\draw[dashed] (6.5, -1.5) -- (6.5, 1.5);
\node[draw, semithick, minimum width=3cm, minimum height=0.5cm] (ctx_ptr1) at (0,1.30) {context pointer};
\node[left=0.2cm of ctx_ptr1] {\texttt{0}};
\node[draw, semithick, minimum width=3cm, minimum height=0.5cm] (ctx_ptr2) at (0,0.73) {context pointer};
\node[left=0.2cm of ctx_ptr2] {\texttt{1}};
\node[minimum width=3cm, minimum height=0.5cm] (dots) at (0,-0.2) {\vdots};
\node[left=0.2cm of dots] {\vdots};
\node[draw, minimum width=3cm, minimum height=1cm, pattern=south west lines] (ctx) at (5, 0.5) {};
\node[draw, semithick, minimum width=3cm, minimum height=0.5cm] (ctx_ptr3) at (0,-1.36) {context pointer};
\node[left=0.1cm of ctx_ptr3] {\texttt{255}};
\node[below=0.02cm of ctx] {Context};
%\node[above=0.6cm of ctx] {Partition memory};
\node[draw, color=black!25, pattern color=black!25, minimum width=3cm, minimum height=1cm, pattern=south west lines] (ctx2) at (5, 2) {};
\node[draw, color=black!25, pattern color=black!25, minimum width=3cm, minimum height=1cm, pattern=south west lines] (ctx3) at (5, -1.5) {};
%%%%%%%%%%%%%%%%%%%%%%%%%%%%%%%%%%%%%%%%%%%%%%%%%%%%%%%%%%%%%%%
\draw[dashed] (ctx_ptr2.south west) -- (ctx_ptr3.north west);
\draw[dashed] (ctx_ptr2.south east) -- (ctx_ptr3.north east);
\draw[->, dotted] (ctx_ptr1.east) -- (ctx2.west);
\draw[->] (ctx_ptr2.east) -- (ctx.west);
\draw[->, dotted] (ctx_ptr3.east) -- (ctx3.west);
\end{tikzpicture}

	\caption{The structure of a VIDT}
	\label{fig:vidt}
\end{figure}

