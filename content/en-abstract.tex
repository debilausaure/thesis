% This file should contain the English abstract.
The work described in this document is related to formal proofs on operating systems.

The first breakthrough in the domain was the SeL4 project ; demonstrating that producing a complete proof on a microkernel was achievable, albeit very costly. 
In order to bring the proof's cost down, the CertikOS project showcased a more layered and modular approach, leveraging \emph{refinements}.
The Pip kernel team tackled the problem from the other side, by using a \emph{shallow embedding} methodology and getting rid of refinement altogether. This thesis' contributions are more specifically tied to the Pip kernel.

Previous work on the Pip protokernel focused on providing an isolation proof to Pip's services manipulating the system's memory. Yet, another critical aspect of the kernel -- handling the execution flow transfer from a partition to another -- remained to be designed.

The first contribution of this thesis outlines the design of a single service able to handle all possible control flow transfers in a system ; namely interrupts, faults and explicit control flow transfers. The design focuses on minimalism and code factorization in order to reduce the overall proof effort. An implementation of the service is provided for the Pip kernel. We believe the idea behind the service is general enough to be implemented in other kernels and other architectures.

The second contribution outlined in this thesis is the first formally proven correct userland implementation of an Earliest Deadline First scheduler for arbitrary jobs. The formal proof guarantees that the election function of the scheduler respects the earliest deadline first scheduling policy, and is guaranteed to be optimal on mono-processor systems. This proof was partly conducted using Pip's usual methodology, leveraging a shallow embedding of the scheduler's code in Coq and a state monad. Nonetheless, while the Pip kernel properties were proven directly, the presented scheduler proofs include three refinement levels ; from the scheduling policy to the actual implementation. Furthermore, the scheduler uses the previously described service in order to pass the control flow to partitions and regain the execution flow through interrupts, showcasing its usability and versatility.

The last contribution presented in this thesis is a proof of concept severing Pip's isolation model from its code. This isolation model independance allows to build alternative models designed to reason on new properties while limiting the proof effort. As such, this contribution opens new research perspectives that were previously too costly to consider. Nonetheless, this proof of concept does not bring the same level of confidence on the composition of properties about the code formally proven on different models.
