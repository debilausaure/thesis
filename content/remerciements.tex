Ce manuscrit est pour moi la synthèse des mes années d'apprentissage et de recherche. Il n'existe cependant qu'au travers du soutien que j'ai reçu durant ces années, et je souhaite -- avant que vous ne commenciez à lire ce document -- rendre un hommage tout particulier aux personnes sans qui ce document n'aurait pas abouti.

Je voudrais commencer par remercier mes collègues, et en premier lieu mon directeur de thèse, Gilles Grimaud. Gilles, merci pour tout. 

% Here you can write whom you want to thank.
{\color{white}
Ni se nourrir, ni se loger n'est gratuit.
Je crois qu'avec ça j'ai tout dit.
Ce monde est cruel
Ce monde est cruel.
Ça y est, j'ai tout dit.

Pas manger, ça fait mourir, et je suis habitué au chauffage.
Tes besoins vitaux sont payants : t'as compris la prise d'otage.
Depuis tout petit dans la merde, tu sais qu'il faudra mailler.
Au moins un peu pour le loyer, au moins un peu pour grailler.
Depuis tout petit dans la merde, on t'apprend à travailler ;
personne ne va te ravitailler à l'œil,
personne ne va s'apitoyer, pas de bol.

Ce monde est cruel.
Ce monde est cruel.
Je peux développer encore, je le fais sans aucun effort.
Pour travailler (donc pour manger), on te prend à trois ans
-- on te lâche à vingt-cinq (tes meilleures années).
Si tu pars avant, tu démarres en bas de la pyramide
et tu fermes ta gueule. Tu fais les pires des tâches,
tu gravis les étages au ralenti. Tu tapines en stage,
t'es sous-payé et on t'oblige à sourire --
car c'est une chance (merci !) déjà d'être là
avec tes vieux diplômes. Tiens, parlons des diplômes.

Personne n'est sûr, mais fais-le quand même pour la sécurité.
D'ailleurs, toute ta vie, pense à sécuriser :
même si tu amasses -- ne dépense pas,
on ne sait pas ce qui peut arriver.
Tu peux mourir, c'est vrai. Mais, si c'est pas le cas,
tu peux souffrir du manque puis être interdit par ta banque
et ça, ça fait peur. Les banques ça fait peur.
Des banques privées s'enrichissent, et des pays s'endettent.
De tout petits groupes très riches face au reste du monde,
face au bétail, face à la masse de salariés sans tête.

N'oublie jamais qui gagne quoi lorsque tu taffes.
Si ça te fâche et que tu ne veux plus,
n'oublie jamais : tu ne manges plus.
Ça ressemble à un choix...
Si c'est pas de l'esclavagisme,
c'est quand même pas vraiment humaniste.

[...]

Ce monde est cruel.
Ce monde est cruel.
Et j'ai tellement de chance à côté des autres,
je trouve ça tellement cruel.

Hein ? Comment ça ? Dieu donnerait de la chance, du talent,
à certains mais pas à d'autres ? Ça me rend parano.
Je ne sais plus si je me suis entraîné, si, tout ça, je le mérite ?
Si l'univers était avec moi ou si ça fait dix ans que je me bats...

[...]

En vrai, je ne sais pas comment ça se passe.
En vrai, je ne sais pas qui maintient le cap.
Si ça vient de moi ou si ça vient des astres.

Ce monde est cruel.
Ce monde est cruel.
Faut changer les choses, si ce monde est cruel,
c'est sûr qu'il y en a d'autres.
Je remercie les anges, je remercie les autres,
je remercie les miens, remerciez les vôtres.
Ce monde est cruel, mais je vous remercie quand même.

Merci pour tout
}
