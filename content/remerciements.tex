\vfil

Ce manuscrit est pour moi la synthèse des mes années d'apprentissage et de recherche. Il n'existe cependant qu'au travers du soutien que j'ai reçu durant ces années, et je souhaite -- avant que vous ne commenciez à lire ce document -- rendre un hommage tout particulier aux personnes sans qui ce document n'aurait pas abouti.\\

Je voudrais commencer par remercier mes collègues. À toutes les personnes avec qui j'ai pu travailler, merci pour votre sérieux, votre patience et votre bienveillance. Merci pour ces moments passés à travailler avec vous ; mais aussi pour ces moments de vie commune. Si j'ai pu aller jusqu'au bout de cette thèse, c'est aussi parce qu'il était agréable de venir travailler, même lorsque la fatigue, la lassitude et la morosité auraient dû venir à bout de ma motivation. Par ailleurs, j'ai tant progressé grâce à vous. Gilles, David, Samuel, Vlad, Étienne, Narjes, Damien, Nicolas, Thomas, Alexandre, Pierre, Olivier, Michael, Clément, Giuseppe, merci pour votre aide. Merci de m'avoir expliqué vos travaux. Merci d'avoir partagé votre expérience lors de ces longues discussions. Merci de m'avoir inculqué le doute scientifique. Je me sens privilégié d'avoir pu travailler et d'avoir pu apprendre avec vous.\\

Je voudrais à présent remercier ma famille, qui m'a apporté un soutien inconditionnel lors de mes études. Si la période de la pandémie a été difficile à vivre, elle aurait été sans nul doute un calvaire sans vous. Merci pour votre écoute. Merci pour vos encouragements. Je voudrais remercier plus particulièrement mes parents et mes grands parents, qui ont toujours fait de mon bien être et de ma réussite leur priorité. Si j'ai pu prétendre à cette thèse et la mener à bien, c'est aussi grâce à vous. Merci du fond du cœur.\\

Je voudrais ensuite remercier mes amis. Merci pour tous ces moments d'allégresse passés en votre compagnie. J'ai conscience que ces dernières années d'étude et de travail n'auraient pas été les mêmes sans vous. Merci d'avoir été présents pour moi.\\

Enfin je voudrais remercier ma compagne, Juliette, qui m'a accompagnée au quotidien durant cette période. Merci de m'avoir incité à aller de l'avant. Merci de m'avoir insufflé l'envie de continuer lorsque la motivation me faisait défaut. Merci d'avoir été conciliante, au détriment parfois de tes propres envies, pour que je puisse arriver là où je suis aujourd'hui. Si j'écris ces lignes aujourd'hui, à quelques semaines de ma soutenance, c'est aussi grâce à toi. J'espère pouvoir te rendre à mon tour l'inspiration que tu m'as apportée.

\vfil
