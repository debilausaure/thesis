\chapter{Méthode d'ajout de nouvelles propriétés sur le code des services}

	Ce chapitre présente une preuve de concept affranchissant le code des services de Pip de toutes ses dépendances au modèle actuel d'isolation. Ces travaux n'ont pas été publiés, mais sont néanmoins consultables sur la branche \texttt{state\_abstraction} du projet Pip, disponible sur le dépôt accessible à l'adresse suivante : \url{https://gitlab.univ-lille.fr/2xs/pip/pipcore}. Cette branche contient une implémentation imparfaite des idées exposées dans ce chapitre, et restreinte à la fonction \texttt{switchContextCont}.

	Par ailleurs, ce chapitre aborde de manière très superficielle certains objets issus de la théorie des catégories, qui pourraient attiser votre curiosité. Je souhaite mentionner ici \cite{categorytheoryforprogrammers}, un livre (gratuit !) qui aborde selon le point de vue d'un programmeur des concepts de la théorie des catégories. Cette lecture vous permettra certainement de mieux cerner -- entre autres -- le concept de foncteur ou encore de monade.

	\section{Motivations}

	La section de discussion du premier chapitre a mis en avant le fait que Pip était conçu autour des propriétés d'isolation formellement prouvées. Le modèle des fonctions de l'interface et de l'état, jusqu'à la monade intégrée au code, sont liés aux propriétés d'isolation. Cette forte proximité est une conséquence de la philosophie de conception minimaliste de Pip, qui a incité à ne définir que les éléments strictement nécessaires à l'établissement de le preuve de préservation de l'isolation. Cette approche a permis de minimiser l'effort de preuve permettant de garantir la propriété d'isolation, mais présente un désavantage majeur : le code des services de Pip n'est pas indépendant des modèles sur lesquels il repose.

	Ainsi, il n'existe qu'un modèle unique dans Pip qui ne peut évoluer que de manière itérative. Chaque évolution rend caduques les propriétés établies sur l'ancien modèle, et implique de produire une nouvelle preuve des mêmes propriétés avec le nouveau modèle. Le moindre ajout de chaque itération rendant de plus en plus difficile l'établissement la preuve à produire.

	Ceci est un frein considérable à la vérification de nouvelles propriétés sur le code de Pip, telle que la preuve fonctionnelle du service évoquée dans le second chapitre. Si les nouvelles propriétés impliquent des changements trop importants sur le modèle, l'effort de preuve à fournir deviendrait inatteignable après seulement quelques itérations.

De plus, il n'est pas possible d'utiliser la composition verticale des propriétés telle que proposée dans le chapitre précédent. Ici, il n'est pas question d'apporter des garanties formelles à un nouvel élément logiciel ; on souhaite apporter de nouvelles garanties formelles à du logiciel existant. On pourrait par exemple reprendre le constat du premier chapitre sur la pertinence de la preuve de préservation de l'isolation sur le code du service de transfert de flôt d'exécution (\ref{sec:proof_limits}). On souhaiterait par exemple pouvoir prouver des propriétés relatives à la correction du service en plus des propriétés d'isolation.

	Ce constat a motivé la preuve de concept développée dans ce chapitre, qui propose une solution au problème de la croissance exponentielle de l'effort de preuve.

	\section{Architecture monolithique}

		Cette section décrira de manière synthétique les composants actuels de Pip, en essayant de décrire leurs dépendances d'un point de vue logiciel. Elle commencera par donner brièvement une vue d'ensemble du projet. Puis, dans une première partie, elle décrira les dépendances du code des services sur les modèles décrits dans Pip. Elle dépliera les définitions pour mettre en lumière les dépendances qui existent entre les modèles des différents composants. Ensuite, dans une seconde partie, la section se penchera sur la méthode de preuve nécessaire à l'établissement de la preuve d'isolation. La section se concluera sur le processus de compilation du code, compilant le code des services de Gallina vers du code C.
		
		\subsection{Vue générale}

			\begin{figure}[!ht]
				\begin{tikzpicture}[>=triangle 45,font=\sffamily, every text node part/.style={align=center}, scale=1, every node/.style={transform shape}] {
	%\node[draw, pattern=south west lines, minimum width=5.6cm, minimum height = 3.6cm] at (0, -3.1) {};
	\node[draw, pattern=south west lines, minimum width=5.6cm, minimum height = 6.2cm] (model) at (0, -1.8) {};
	\node[below=0cm of model] {Modèle monolithique en Gallina};
	\node[draw, fill=white, minimum width = 5cm, minimum height =   2cm] (services) at (0,0) {Code des services};
	\node[draw, fill=white, minimum width = 5cm, minimum height = 0.7cm] (functions_model) at (0, -1.9) {Modèle des fonctions};
	\node[draw, fill=white, minimum width = 5cm, minimum height = 0.7cm] (types) at (0, -2.7) {Modèle de types};
	\node[draw, fill=white, minimum width = 5cm, minimum height = 0.7cm] (state_monad) at (0, -3.5) {Monade d'état};
	\node[draw, fill=white, minimum width = 5cm, minimum height = 0.7cm] (state_model) at (0, -4.3) {Modèle de l'état};

	\node[draw, pattern=south west lines, minimum width=5.6cm, minimum height = 2cm] (impl) at (8, -2.3) {};
	\node[below=0cm of impl] {Implémentation en C};
	\node[draw, fill=white, minimum width=5cm, minimum height=2cm] (Cservices) at (8,0) {Code des services\\(Compilé en C)};
	\node[draw, fill=white, minimum width=5cm, minimum height = 0.7cm] (functions) at (8, -1.9) {Fonctions sur l'état};
	\node[draw, fill=white, minimum width=5cm, minimum height = 0.7cm] (types) at (8, -2.7) {Types C};

	\draw[->] (services) -- (Cservices) node[midway, above] {Digger} node[midway, below] {$\partial x$};
}

\end{tikzpicture}

				\caption{Architecture actuelle de Pip et dépendances des composants}
				\label{fig:currentPipArchitecture}
			\end{figure}

			L'architecture actuelle de Pip est décrite dans la figure \ref{fig:currentPipArchitecture}. Les modèles nécessaires à la spécification des propriétés d'isolation sont placés dans la partie gauche de la figure. Ces modèles sont inter-dépendants : le modèle de l'état dépend du modèle des types, la monade d'état dépend du modèle de l'état, et le modèle des fonctions de l'interface dépend de la monade d'état, et du modèle des types. De plus, le code des services est dépendant de tout ces modèles. Par ailleurs, les fonctions ``fictives'' permettant de décrire les propriétés d'isolation, les triplets de Hoare et propriétés d'isolation en elles-même sont dépendants du code des services et des modèles précédents.

			La partie exécutable du noyau est représentée dans la partie droite de la figure \ref{fig:currentPipArchitecture}. Le code des services écrit en Gallina est compilé en code C grâce à Digger ou $\partial x$. Le code des services compilé en C repose sur les implémentations exécutables des fonctions intéragissant avec l'état et les types réels. Le modèle de l'état et la monade d'état disparaissent pour laisser place à l'état et la séquentialité intrinsèque du code exécutable.

			Le listing \ref{code:switchContextCont} représente la portion du code des services prise pour exemple dans ce chapitre. Les prochaines sections vont commenter les dépendances de cette portion de code avec les modèles d'isolation, puis montrer comment elle peut être affranchie de ces dépendances graĉe à cette contribution.
			\begin{listing}[!ht]
				\coqcode{code/switchContextCont.v}
				\caption{Code du bloc de continuation \texttt{switchContextCont} du service de transfert de flot d'exécution}
				\label{code:switchContextCont}
			\end{listing}

		\subsection{Dépendance du code au modèle d'isolation}

			\subsubsection{Monade dépendante du modèle de l'état}

			Dans l'architecture actuelle de Pip, le code des services repose directement sur les définitions de la monade, en utilisant le type monadique \texttt{LLI}, et les fonctions \texttt{bind} et \texttt{ret} pour représenter la mise en séquence des instructions des services. Dans l'exemple présenté en listing \ref{code:switchContextCont}, la fonction retourne un type monadique \texttt{LLI yield\_checks}, utilise la fonction \texttt{bind} au travers du sucre syntaxique \texttt{perform [...] := [...] in}, et indique sa valeur de retour grâce à la fonction \texttt{ret}.
			Malheureusement, le type monadique \texttt{LLI}, décrit en listing \ref{code:LLImonad}, dépend de l'état \texttt{state}, décrit en listing \ref{code:CurrentIsolationState}. \texttt{state} est le modèle de l'état conçu pour la preuve de préservation de l'isolation. Ceci est une première dépendance du code au modèle d'isolation, dont le code devra se passer pour devenir indépendants des modèles construits pour Pip.

			\begin{listing}[!ht]
				\coqcode{code/LLIMonad.v}
				\caption{Définition du type de la monade d'état \texttt{LLI} dans le modèle actuel de Pip}
				\label{code:LLImonad}
			\end{listing}

			\begin{listing}[!ht]
				\coqcode{code/CurrentIsolationState.v}
				\caption{Définition de l'état \texttt{state} dans le modèle actuel de Pip}
				\label{code:CurrentIsolationState}
			\end{listing}

			\subsubsection{Code dépendant des modèles de types}

			D'autres dépendances du code aux modèles d'isolation passent par la représentation des types. Le code dépend des types utilisés pour représenter ses propres arguments, et valeur de retour enrobée par le type monadique \texttt{LLI}, mais aussi les arguments et valeurs de retour des fonctions de l'interface. Par exemple, la fonction \texttt{switchContextCont} décrite dans le listing \ref{code:switchContextCont}, dépend du modèle des types \texttt{page}, \texttt{interruptMask}, \texttt{contextAddr} et \texttt{yield\_checks}. Les modèles de ces types sont représentés dans le listing \ref{code:CurrentTypesModel}.
Cette dépendance renforce les liens entre le modèle d'isolation de Pip et le code de ses services, et doit donc disparaître.

			\begin{listing}[!ht]
				\coqcode{code/CurrentTypesModel.v}
				\caption{Définition des types nécessaires à la fonction \texttt{switchContextCont} dans le modèle actuel de Pip}
				\label{code:CurrentTypesModel}
			\end{listing}

			\subsubsection{Code dépendant des modèles des fonctions intéragissant avec l'état}
			Enfin, la dernière dépendance du code aux modèles est par le biais des fonctions intéragissant avec l'état. Le code des services fait directement appel \emph{aux modèles} de ces fonctions. Ainsi, la fonction \texttt{switchContextCont} présentée dans le listing \ref{code:switchContextCont}, est dépendant des modèles des fonctions \texttt{setInterruptMask}, \texttt{updateMMURoot}, \texttt{updateCurPartition}, \texttt{getInterruptMaskFromCtx}, \texttt{getPageRootPartition}, \texttt{noInterruptRequest} et \texttt{loadContext}. Cette dépendance n'a pas lieu d'être, et doit être supprimée pour atteindre un code des services agnostique des modèles.

			\begin{listing}[!ht]
				\coqcode{code/CurrentFunctionsModel.v}
				\caption{Définition des fonctions de l'interface avec l'état nécessaire à la fonction \texttt{switchContextCont} dans le modèle actuel de Pip}
				\label{code:CurrentFunctionsModel}
			\end{listing}

			\subsubsection{Extraction de l'\emph{AST} dépendant des modèles}
			\label{sec:AST_extr}
			Ce dernier paragraphe est dédié au fichier source Coq extrayant l'\emph{AST} du code des services de Pip. Ce fichier, une fois évalué par Coq, produit le fichier attendu en entrée par Digger, un des outils de compilation du code Gallina \emph{shallow embedded} vers du code C utilisé dans le projet Pip. Ce fichier a pour dépendances l'ensemble des modèles d'isolation sur lesquels reposent le code. Ainsi, dans l'état actuel du projet, l'extraction de l'\emph{AST} du code des services n'est possible que si l'intégralité des modèles d'isolation peut être évalué par Pip. Coq ne permet pas d'extraire un fragment d'AST si chaque fonction et chaque type utilisés par ce fragment ne sont pas implémentés ou axiomatisés. Cette dépendance doit donc être supprimée pour que le code des services puisse être compilé en C sans avoir recours aux modèles.

		\subsection{Processus de preuve sur le code dépendant du modèle}
		\label{sec:dependant_code}

			Cette sous-section est dédiée à la structure actuelle de la preuve d'isolation sur le code des services, mettant en avant les dépendances des différents groupes de fichiers nécessaires à chaques preuves.

			\subsubsection{Définition des propriétés d'isolation et des fonctions ``fictives''}

			Les premiers fichiers nécessaires à l'établissement de la preuve sont ceux contenant les fonctions ``fictives'', nécessaires à l'expression des propriétés d'isolation sur le noyau. Pour rappel, les fonctions ``fictives'' ont vocation à spécifier ce qui doit être prouvé et non à être exécutées en tant que tel. Elles servent de fondation à l'expression des triplets de Hoare qu'on souhaite prouver pour garantir de manière formelle les propriétés souhaitées sur le code, comme les propriétés d'isolation. Ces fonctions peuvent permettre entre autres de définir des ensembles nécessaires à certaines propriétés d'isolation. C'est le cas de la fonction \texttt{getAccessibleMappedPages} qui récupère les pages mappées et accessibles dans l'espace d'adressage d'une partition, nécessaire à la propriété d'isolation noyau \texttt{kernelDataIsolation}. Ces fonctions peuvent aussi être des miroirs purement fonctionnels de code monadique présent dans les services de Pip, telle que la fonction \texttt{readPhysical} permettant de lire l'adresse d'une page mémoire. 

			Une fois les définitions de ces fonctions établies, il est possible de définir les propriétés qui justifient leur existence, c'est à dire ici les propriétés d'isolation. Les fonctions ``fictives'' et propriétés définies de cette manière dépendent -- \textbf{à juste titre} -- des modèles d'isolation, que ce soit le modèle de types, de l'état, ou des fonctions intéragissant avec l'état. En effet, ces dépendances sont nécessaires à l'établissement d'une preuve formelle ; d'une certaine manière, ces modèles projettent les dépendances de l'objet étudié (le code des services de Pip) vers le monde mathématique et permettent de les représenter. 

			\subsubsection{Définition des triplets de Hoare, des lemmes intermédiaires et des scripts de preuve}

			Une fois que ces définitions établies, il est possible d'exprimer les triplets de Hoare sur le code des services. À titre d'exemple, le listing \ref{code:switchContextCont_triplet} décrit le triplet de Hoare de la fonction \texttt{switchContextCont}. Sous chaque triplet (et chaque lemme intermédiaire) se trouve un script de preuve, décrivant les règles d'inférence (ou tactiques) à appliquer successivement pour faire progresser Coq vers la conclusion. Les triplets de Hoare dépendent du code des services, des fonctions fictives utiles à la définition des propriétés, et dépendent donc à fortiori des modèles d'isolation, encore une fois, à juste titre.
			\begin{listing}[!ht]
				\coqcode{code/switchContextCont_triplet.v}
				\caption{Définition du triplet de Hoare de la fonction \texttt{switchContextCont} pour la preuve de préservation de l'isolation de Pip}
				\label{code:switchContextCont_triplet}
			\end{listing}

		
	\section{Abstraction des modèles dans le code des services de Pip}

	\textcolor{red}{FUTUR !}

	Cette section détaillera l'objet de la preuve de concept mise à l'honneur dans ce chapitre : la modularisation des modèles et preuves des services de Pip, ainsi que l'autonomie du code des services vis à vis des modèles. Elle commencera par donner une vue globale de la nouvelle architecture du projet, indiquant les nouvelles relations entre les différents composants du projet. Dans un second temps, elle détaillera les interfaces créées, en illustrant les changements apportés à Pip du point de vue de la fonction \texttt{switchContextCont}, utilisée comme exemple dans la section précédente. Cette section mettra en évidence les différences avec l'implémentation précédente dépendantes des modèles. Cette seconde partie décrira aussi le processus d'extraction de l'\emph{AST} du code des services. Dans une dernière partie, cette section décrira la nouvelle structure des fichiers de preuve, en illustrant les différences (plus conceptuelles) avec l'architecture précédente.
		
		\subsection{Vue générale}

		La principale contribution de cette preuve de concept est l'ajout d'\emph{interfaces} décrivant les dépendances fondamentales du code des services aux autres composants logiciels évoqués dans la section précédente \ref{sec:dependant_code}. Le code des services de Pip repose sur cette interface, qui ne décrit que les opérations ou types à fournir au code. L'implémentation réelle (et exécutable) de cette interface est réalisée en C, et s'exécutera conjointement avec le code des services compilé par Digger ou $\partial x$. Du coté du monde de la preuve formelle, de \emph{multiples} modèles peuvent décrire cette interface et ses effets. La figure \ref{fig:new_pip_architecture} décrit l'architecture de Pip selon cette preuve de concept. La colonne du milieu représente les interfaces nouvellement créées, sur lesquelles le code des services repose. La colonne de gauche représente les modèles décrivant les interfaces, et les preuves reposant sur ces interfaces. La colonne de droite représente l'implémentation réelle de l'interface en C sur laquelle repose le code des services compilé par Digger ou $\partial x$. Les flèches continues dans cette figure indiquent des composants dérivés automatiquement d'autres composants, par exemple le code des services compilé en C est dérivé automatiquement du code des services écrit en Gallina. Les flèches discontinues traduisent une relation d'implémentation. Par exemple, le modèle des fonctions implémente l'interface des fonctions.

			\begin{figure}[!ht]
				\begin{tikzpicture}[>=triangle 45,font=\sffamily, every text node part/.style={align=center}, scale=1, every node/.style={transform shape}] {
	%\node[draw, pattern=south west lines, minimum width=5.6cm, minimum height = 3.6cm] at (0, -3.1) {};
	%\node[draw, pattern=south west lines, minimum width=5.6cm, minimum height = 6.2cm] (model) at (0, -1.8) {};
	%\node[below=0cm of model] {Modèle monolithique en Gallina};
	
	\node[draw, pattern=south west lines, minimum width=4.6cm, minimum height = 3.6cm] (models) at (0, -3.1) {};
	\node[below=0cm of models] {Interface agnostique\\des modèles};
	\node[draw, fill=white, minimum width = 4cm, minimum height = 2cm] (services) at (0, 0) {Code des services};
	\node[draw, fill=white, minimum width = 4cm, minimum height = 0.7cm] (functions_interface) at (0, -1.9) {Interface des fonctions};
	\node[draw, fill=white, minimum width = 4cm, minimum height = 0.7cm] (types_interface) at (0, -2.7) {Interface des types};
	\node[draw, fill=white, minimum width = 4cm, minimum height = 0.7cm] (state_monad) at (0, -3.5) {Monade d'état générique};
	\node[draw, fill=white, minimum width = 4cm, minimum height = 0.7cm] (state_interface) at (0, -4.3) {Interface de l'état};
	%\node[draw, fill=white] (models) at (2, -2.3) {Modèle des fonctions,\\des types et de l'état};
	%\node[draw, fill=white] (state_monad) at (-2, -2.3) {Monade\\d'état};

	\node[draw, pattern=south west lines, minimum width=4.6cm, minimum height = 3.6cm] (models) at (-5, -3.1) {};
	\node[below=0cm of models] {Modèles d'isolation};
	\node[draw, fill=white, minimum width = 4cm, minimum height = 2cm] (proofs) at (-5, 0) {Fonctions ``fictives''\\Triplets de Hoare\\Scripts de preuve};
	\node[draw, fill=white, minimum width = 4cm, minimum height = 0.7cm] (functions_model) at (-5, -1.9) {Modèles de fonctions};
	\node[draw, fill=white, minimum width = 4cm, minimum height = 0.7cm] (types_model) at (-5, -2.7) {Modèles de types};
	\node[draw, fill=white, minimum width = 4cm, minimum height = 0.7cm] (derived_monad) at (-5, -3.5) {Monade spécifique};
	\node[draw, fill=white, minimum width = 4cm, minimum height = 0.7cm] (state_model) at (-5, -4.3) {Modèle de l'état};

	\node[draw, pattern=south west lines, minimum width=4.6cm, minimum height = 2cm] (impl) at (5, -2.3) {};
	\node[below=0cm of impl] {Implémentation exécutable};
	\node[draw, fill=white, minimum width=4cm, minimum height=2cm] (Cservices) at (5,0) {Code des services\\(Compilé en C)};
	\node[draw, fill=white, minimum width=4cm, minimum height = 0.7cm] (functions) at (5, -1.9) {Fonctions sur l'état};
	\node[draw, fill=white, minimum width=4cm, minimum height = 0.7cm] (types) at (5, -2.7) {Types};

	\draw[->] (services) -- (Cservices);
	\draw[->, dashed] (functions_model) to (functions_interface) ;
	\draw[<-] (derived_monad) to (state_monad) ;
	\draw[->, dashed] (types_model) to (types_interface) ;
	\draw[->, dashed] (state_model) to (state_interface) ;
	\draw[->, dashed] (types) to (types_interface) ;
	\draw[->, dashed] (functions) to (functions_interface) ;
}

\end{tikzpicture}

				\caption{Nouvelle architecture de Pip et dépendances des composants apportées par la preuve de concept}
				\label{fig:new_pip_architecture}
			\end{figure}

			Chaque interface est décrite en Coq sous la forme d'un \emph{Module Type}. Les implémentations de \emph{Modules Type} sont décrit en Coq par des \emph{Modules}, qui doivent nécessairement donner une définition à chaque élément déclaré dans le \emph{Module Type}.

		\subsection{Définition de code générique indépendant des modèles}

			\subsubsection{Abstraction du modèle de l'état}

			Le premier composant abstrait par cette preuve de concept est la définition de l'état. Cette abstraction donne naissance au premier \emph{Module Type} de cette contribution, le \emph{Module Type} de l'état. Ce \emph{Module Type} est extrêmement simple : il ne contient qu'une unique déclaration, le type \texttt{state}. La définition du \emph{Module Type} est présentée en listing \ref{code:StateParameter}. Ainsi, chaque modèle devra fournir son propre \emph{Module} contenant la définition du type de l'état.
			\begin{listing}[!ht]
				\coqcode{code/StateModel.v}
				\caption{Définition de l'interface de l'état}
				\label{code:StateParameter}
			\end{listing}

			La définition du \emph{Module} implémentant le modèle d'isolation reprend les définitions présentées dans la section précédente, et est disponible en annexe en listing \ref{code:IsolationState}.

			\subsubsection{Définition d'une monade d'état agnostique du modèle de l'état}

			La seconde modification apportée par cette contribution est la création d'une monade agnostique du modèle de l'état. Bien que la monade d'état ait besoin de faire référence à l'état, elle n'a pas besoin de sa définition. Ainsi, il est possible de créer un \emph{Module} paramétré par le \emph{Module Type} de l'état. Le \emph{Module} ainsi créé est de ce fait un foncteur (\emph{functor} en anglais), une fonction des \emph{Modules} vers les \emph{Modules}. En particulier, le foncteur de la monade d'état prend en paramètre un \emph{Module} implémentant le \emph{Module Type} \texttt{StateModel}, et renvoie un \emph{Module} décrivant la monade d'état associée. Le \emph{Module} implémentant la monade d'état agnostique au modèle de l'état est disponible en annexe, \ref{code:StateAgnosticMonad}.

			\subsubsection{Abstraction du modèle de types utilisés par Pip}

			La seconde interface créée est celle des types utilisés par le code des services et les fonctions sur l'état. Tout comme l'interface de l'état, cette interface est créée grâce à un \emph{Module Type} dont les déclarations pourront être utilisées par le code des services et les fonctions sur l'état. De la même manière, les objets déclarés dans ce \emph{Module Type} doivent être implémenté par chaque modèle sous la forme d'un \emph{Module}. Ces déclarations doivent aussi être implémentées en C ; cette implémentation sera utilisée par le code des services traduit en C par Digger ou $\partial x$. Le code du \emph{Module Type} déclarant les définitions nécessaires à la fonction \texttt{switchContextCont} est disponible en listing \ref{code:TypesParameters}.

			\begin{listing}[!ht]
				\coqcode{code/TypesParameters.v}
				\caption{Définition de l'interface des types nécessaires à la fonction \texttt{switchContextCont}}
				\label{code:TypesParameters}
			\end{listing}


			\subsubsection{Abstraction du modèle des fonctions de l'interface avec l'état}

			\begin{listing}[!ht]
				\coqcode{code/InterfaceParameters.v}
				\caption{Définition de l'interface des fonctions utilisées par le code des services (restreinte aux définitions nécessaires à la fonction \texttt{switchContextCont})}
				\label{code:InterfaceParameters}
			\end{listing}
			La dernière interface créée est celle déclarant les fonctions intéragissant avec l'état. Cette interface a besoin de faire référence à la fois aux types utilisés par les fonctions mais aussi à la monade, puisque les fonctions décrites sont monadiques. Ainsi, cette interface est définie par un \emph{Module Type} paramétré par le \emph{Module Type} déclarant les types nécessaires aux fonctions et par le \emph{Module Type} de l'état. Ce \emph{Module Type} représente donc le type d'un foncteur allant d'un \emph{Module} décrivant les types et d'un \emph{Module} décrivant l'état vers un \emph{Module} décrivant les fonctions monadiques d'interaction avec l'état utilisant les types passés en paramètre.

			L'implémentation de ce \emph{Module Type} (restreint aux définitions nécessaires à la fonction \texttt{switchContextCont}) est disponible en listing \ref{code:InterfaceParameters}. Cette implémentation utilise le foncteur \texttt{StateAgnosticMonad} sur le \emph{Module} d'état \texttt{State} passé en paramètre pour instancier le \emph{Module} de la monade d'état \texttt{SAMM} qui lui est nécessaire. Cette instanciation permet ensuite de déclarer les prototypes de fonction utilisant le type monadique défini dans ce \emph{Module}. Les fonctions dont les prototypes ont été déclarés dans ce \emph{Module Type}, ainsi que la monade d'état utilisée devront être définis par le \emph{Module} implémentant le \emph{Module Type}.

			\subsubsection{Code agnostique des modèles}

			La création de toutes ces interfaces permet de définir le code des services affranchit de toute dépendance aux modèles. Le code des services (restreint à la seule fonction \texttt{switchContextCont} dans cette preuve de concept) est disponible en listing \ref{code:ModelAgnosticCode}. Le code est défini dans un \emph{Module} dépendant de l'implémentation de trois \emph{Modules} : un \emph{Module} décrivant l'état, un \emph{Module} décrivant les types, et un \emph{Module} décrivant les fonctions monadiques. Ainsi, le code de Pip est devenu un foncteur allant de ces \emph{Modules} vers le \emph{Module} décrivant le code \textbf{spécifique} aux \emph{Modules} dont il dépend.

Deux différences minimes se dégagent de cette nouvelle manière de représenter le code. La première est la déclaration des modèles paramètres du \emph{Module} du code des services, mis en évidence dans le listing \ref{code:code_parameters}. 
	\begin{listing}[!ht]
	\begin{minted}{coq}
Module ModelAgnosticCode (Export Types : TypesModel)
                         (Export State : StateModel)
                         (Export Interface : InterfaceModel Types State).
	\end{minted}
	\caption{Déclarations des \emph{Modules} paramètres du \emph{Module} du code des services}
	\label{code:code_parameters}
	\end{listing}

La seconde différence de cette nouvelle représentation du code est l'utilisation de la monade agnostique de l'état \texttt{SAM}, à la place de la monade dépendante du modèle de l'état d'isolation \texttt{LLI}, comme décrit dans le listing \ref{code:instanciated_monad}.

	\begin{listing}[!ht]
	\begin{minted}{coq}
Definition switchContextCont (* [...] *) : SAM yield_checks :=
	\end{minted}
	\caption{Monade instanciée \texttt{SAM}, type de retour de la fonction \texttt{switchContextCont} dans la preuve de concept}
	\label{code:instanciated_monad}
	\end{listing}

	Une représentation complète de la fonction \texttt{switchContextCont} est disponible en listing \ref{code:ModelAgnosticCode}. Seules ces deux différences permettent de distinguer le code des services écrit comme un foncteur dans ce listing du code dépendant directement des modèles décrit en listing \ref{code:switchContextCont}.


	\begin{listing}[!ht]
		\coqcode{code/ModelAgnosticCode.v}
		\caption{Définition du code affranchi de toute dépendance aux modèles}
		\label{code:ModelAgnosticCode}
	\end{listing}

		\subsection{Extraction de l'\emph{AST} du code des services}

		Pour que la preuve de concept présentée dans ce chapitre soit pertinente, il faut que le code des services puisse être compilé vers du code C. Comme évoqué dans les chapitres précédents, Pip est conçu pour s'interfacer avec deux outils permettant la compilation de code Gallina \emph{shallow-embedded} vers du code C : Digger ou $\partial x$.

		Cette sous-section discute du fichier source qui, une fois evalué par Coq, produit une représentation de l'\emph{AST} du code des services nécessaire à Digger pour produire le code C. La section précédente \ref{sec:AST_extr} discutait du fait que le fichier source importait tous les modèles d'isolation, liant de ce fait le code des services à ces modèles. Cependant, dans cette preuve de concept, le code n'est plus qu'un foncteur des modèles vers une instance spécifique du code. D'une certaine manière, l'ancien code des services est une instance particulière du code paramétrique de cette preuve de concept : le cas où le code aurait été instancié avec les \emph{Modules} des modèles d'isolation.

		Malheureusement, il n'est pas possible pour Coq de produire l'\emph{AST} d'un foncteur au moment où j'écris ces lignes. Néanmoins, Coq accepte de produire l'\emph{AST} d'une fonction -- et ce même sans connaître la définition des objets sur lesquels cette fonction repose. Ainsi, il est possible de définir des \emph{Modules} ``creux'', qui se contentent de déclarer qu'ils implémentent les \emph{Modules Type} attendus par le code des services agnostique des modèles, sans préciser d'implémentation. Dès lors, il est possible d'instancier le code avec ces modèles creux, ce qui transforme le code des services en fonctions dont Coq peut extraire l'\emph{AST}. De cette manière, il est possible d'extraire l'\emph{AST} des services de Pip sous la forme proposée par cette preuve de concept sans dépendre d'une implémentation particulière des modèles.

	Les listings \ref{code:LLICodeExtraction} et \ref{code:HollowCodeExtraction} présentent la même portion de code extraits avec l'architecture actuelle de Pip ou dans la preuve de concept en instanciant le code de Pip avec les modèles creux. La ligne de code concernée est la suivante : \texttt{updateCurPartition targetPartDesc ;;}, provenant de la fonction \texttt{switchContextCont} décrit en listing \ref{code:switchContextCont}. Dans le listing \ref{code:LLICodeExtraction}, la fonction \texttt{bind} utilisée provient du fichier \texttt{Hardware} où est définie la monade reposant sur le modèle de l'état d'isolation. La fonction \texttt{updateCurPartition} provient du fichier \texttt{MAL}, qui définit une partie des modèles des fonctions intéragissant avec l'état selon le modèle d'isolation.

	Dans le listing \ref{code:HollowCodeExtraction}, la fonction \texttt{bind} provient de la monade générique \texttt{SAMM} instanciée avec le modèle creux de l'état. La fonction \texttt{updateCurPartition} provient de la l'implémentation creuse de l'interface des fonctions intéragissant avec l'état.

	\begin{listing}[!ht]
	\begin{minted}{coq}
    (* [...] *)
Hardware.bind (MAL.updateCurPartition targetPartDesc) (\_ ->
    (* [...] *)
	\end{minted}
	\caption{Fragment de l'\emph{AST} du code de Pip actuel représenté sous forme de code Haskell}
	\label{code:LLICodeExtraction}
	\end{listing}

	\begin{listing}[!ht]
	\begin{minted}{coq}
    (* [...] *)
HollowModel._HollowInterface__SAMM__bind
  (HollowModel._HollowInterface__updateCurPartition targetPartDesc)
    (\_ ->
    (* [...] *)
	\end{minted}
	\caption{Fragment de l'\emph{AST} du code de Pip instancié sur les modèles creux représenté sous forme de code Haskell}
	\label{code:HollowCodeExtraction}
	\end{listing}

		\subsection{Méthode de preuve sur des foncteurs}

		Cette preuve de concept ne serait pas utile s'il n'était pas possible de raisonner formellement sur le code des services. Cette brève sous-section discute de la manière de structurer les nouveaux modèles et les nouvelles preuves afin de pouvoir prouver des propriétés formelles. L'établissement de modèles et la méthode de preuve restent \emph{identiques} aux méthodes utilisées jusqu'à présent.

		Tout d'abord, il est important de souligner que les \emph{Modules} implémentant les \emph{Modules Types} sont libres des choix de leur implémentation : ils peuvent dépendre de n'importe quel modèle et décrire ce que bon leur semble. Pour rappel, ces \emph{Modules} sont une projection de l'implémentation exécutable de l'interface dans le monde des mathématiques. La seule règle que doivent respecter ces \emph{Modules} est qu'ils doivent définir l'ensemble des éléments déclarés dans le \emph{Module Type} qu'ils implémentent.

		Dans cette preuve de concept, le code des services a été séparé des modèles. Afin de pouvoir raisonner formellement sur le code des services, il est désormais nécessaire de le spécialiser avec des modèles. Plus précisemment, il est nécessaire d'instancier un \emph{Module} contenant le code des services spécialisé avec les \emph{Modules} décrivant les modèles sur lesquels on souhaite raisonner. Par exemple, afin de pouvoir exprimer le triplet de Hoare de la fonction \texttt{switchContextCont} sur les modèles et propriétés d'isolation, le code des services doit tout d'abord être instancié avec les modèles d'isolation. La définition du triplet de Hoare de cette fonction est illustrée en annexe \ref{code:switchContextCont_agn_triple}.

		Ce triplet de Hoare, lié cette preuve de concept, est en tout point similaire à la définition actuelle du triplet de Hoare décrit en listing \ref{code:switchContextCont_triplet} et pour cause : cette définition fait appel \emph{aux mêmes objets} que la définition actuelle du triplet. Pour y parvenir, la preuve de concept \textbf{instancie} le code des services (et donc de la fonction \texttt{switchContextCont}) avec les modèles d'isolation. Le code de la fonction passe alors de l'état de foncteur à l'état de fonction, tel qu'exprimé actuellement dans les preuves de préservation de l'isolation de Pip.

		La ligne d'intérêt dans l'annexe \ref{code:switchContextCont_agn_triple}, créant une instance du code des services spécialisés avec les modèles d'isolation, est isolée dans le listing \ref{code:ModuleInstanciation}. Cette ligne crée un nouveau \emph{Module} en appliquant le foncteur du code des services aux \emph{Modules} de type, de l'état et des fonctions d'isolation.
	\begin{listing}[!ht]
	\begin{minted}{coq}
Module IsolationSpecificCode :=
  ModelAgnosticCode IsolationTypes IsolationState IsolationInterface.
	\end{minted}
	\caption{Instanciation du code des services sur les modèles d'isolation}
	\label{code:ModuleInstanciation}
	\end{listing}

		Ainsi, la seule différence notable relative à la méthodologie de preuve apportée par cette preuve de concept est la nécessité de créer les modèles au travers de \emph{Modules}.

	\section{Perspectives de recherche et discussion}
		Cette preuve de concept, bien qu'incomplète, montre qu'il est possible d'adapter le code des services de Pip et le modèle d'isolation actuel afin que le code des services devienne indépendant des modèles d'isolation. Cette indépendance permettrait à Pip d'intégrer de nouveaux modèles conçus pour raisonner sur d'autres aspects du code que ceux abordés par le modèle actuel. Cette section commencera par évoquer des perspectives de recherche à court terme pouvant émerger de cette preuve de concept, puis discutera des limites apportées par les preuves de propriétés sur des modèles parallèles.

		\subsection{Perspectives de recherche liées au développement d'un modèle alternatif}
		Comme évoqué dans les motivations de cette contribution en début de chapitre, l'établissement d'un modèle du code dédié à une preuve fonctionnelle des services est une des principales perspectives de cette contribution. On pourrait s'intéresser particulièrement aux propriétés de correction du service de transfert de flot d'exécution, pour lequel les propriétés d'isolation se sont révélées être peu intéressantes.

		De plus, l'établissement de propriétés de bon fonctionnement sur le service de tranfert de flot d'exécution permettrait d'étudier à quel point il est possible de réutiliser les propriétés prouvées formellement sur la base de confiance (\emph{TCB}) afin d'établir de nouvelles propriétés sur un nouvel objet. 
		\subsubsection{Lien entre la preuve de bon fonctionnement et le back end de l'ordonnanceur}
	
		\textcolor{red}{Rappel composition vetticale}

		Cette perspective de recherche pourrait se concrétiser sur l'établissement d'une preuve de correction du \emph{back-end} de l'ordonnanceur présenté dans le chapitre précédent reposant sur la preuve de correction du service de transfert de flot d'exécution. Cette preuve se servirait des propriétés sur la sauvegarde et la restauration de contextes du service pour transférer le flot d'exécution aux \emph{jobs} élus par la fonction d'élection, et montrerait que les \emph{jobs} élus par la fonction d'élection sont effectivement exécutés après leur élection.

		Cette perspective de recherche mettrait en évidence une composition verticale de propriétés formellement prouvées, dont le lien entre les morceaux de logiciels munis de propriétés formelles est direct : c'est à dire sans l'intermédiaire de code inclus dans la base de confiance. 

		\subsection{Discussion sur la coexistence de preuve sur des modèles distincts}

		La preuve de concept présentée dans ce chapitre permet de faire coexister plusieurs modèles décrivant le code des services dans le but de conduire des preuves formelles sur ce code. Une question se pose alors : quelle est la relation de deux preuves formelles conduites sur des modèles différents et que disent-elles du code qui s'exécutera finalement sur la machine ?

		La réponse est simple : les preuves formelles menées sur des modèles différents \emph{ne sont aucunement reliées}, et ne \emph{garantissent rien} sur le code exécutable. Au risque d'enfoncer une porte déjà ouverte dans les chapitres précédents, les modèles décrivant un objet du monde réel ne sont pas intrinsèquement équivalents à l'objet modélisé. Chaque preuve formelle propose une vérité absolument indéniable dans un monde féérique décrit par des modèles mathématiques ; transposer ces modèles au monde réel transfère inévitablement aux conclusions de la preuve formelle l'incertitude que les modèles décrivent correctement l'objet étudié. Ce fait n'enlève rien à l'utilité des modèles qui permettent d'étudier l'objet décrit. Il est cependant important de garder à l'esprit que les modèles ne renferment pas une vérité universelle ; vérité universelle dont l'existence même est une question philosophique.

		Cet argument s'applique aussi aux modèles entre eux : des modèles différents décrivent des univers mathématiques différents. Ainsi, prétendre que les propriétés prouvées sur un modèle le sont aussi sur un autre introduit le même type d'incertitude. C'est pourquoi cette méthodologie n'est pas strictement équivalente à la méthodogie de preuve de propriétés n'impliquant qu'un unique modèle. Dans le cas d'un unique modèle, les propriétés doivent être prouvées \emph{conjointement} sur le modèle unique. Une seule incertitude s'applique donc dans ce cas : il faut que le modèle unique décrive correctement l'objet étudié. Dans le cas de modèles parallèles, les propriétés sont prouvées indépendamment les unes des autres sur des modèles différents. Dans ce cas, il existe autant d'incertitude concernant la correction des modèles qu'il existe de modèles. Comme les raisonnements sur des modèles disjoints sont eux ausi disjoints, si les modèles (hypothèses) de chacun de ces raisonnements sont contradictoires les uns avec les autres, les raisonnements qui s'appuient sur chacuns d'eux n'ont aucune chance d'exhiber ces contradictions.

		\subsection{Preuve générique applicable à différents modèles}

		Un des rêves initiaux de cette contribution était de proposer une preuve générique, pouvant s'appliquer à tous les modèles. Cette preuve aurait été conçue comme une ``preuve à trous'', dont les morceaux manquants difficiles, et spécifiques à chaque modèles auraient été la partie manuelle de la preuve ; les autres aspects de la preuve ayant été couverts par la preuve générique.

		L'architecture aurait été la suivante : chaque fonction de l'interface aurait été munie d'un invariant paramétrique, à définir et prouver comme des fonctions d'un \emph{Module}. Ces invariants, disponibles pour chacunes des opérations du code, auraient été appliqués successivement par la preuve pour arriver à ses conclusions finales. J'espérais initialement que cette preuve générique allègerait le processus de preuve en faisant disparaître l'aspect répétitif de la preuve de code actuel, où la vaste majorité du raisonnement consiste à récupérer les nouvelles propriétés apportées par l'exécution des fonctions de l'interface, puis à les propager comme des propriétés supplémentaires disponibles lors de l'exécution des instructions suivantes.

		Cette preuve générique est cependant beaucoup moins intéressante qu'initialement espéré. Tout d'abord, cette preuve générique ne simplifie aucunement le processus de preuve de propriétés actuelle. En effet, toute la difficulté de la preuve formelle réside dans la recherche et la preuve d'invariants, ce que cette preuve générique ne nous épargne pas. D'un point de vue subjectif, faire disparaître l'application successive des invariants sur les instructions retire une grosse portion du sentiment de progression dans l'établissement de la preuve formelle. De plus, il est parfois nécessaire d'exécuter plusieurs instructions avant de pouvoir arriver à un invariant simple, notamment dans le cas de modification de structures. Une preuve générique, s'appliquant donc à tous les modèles, s'accomoderait difficilement de ces écarts spécifiques à chaque preuve. Cette perspective de recherche semble donc finalement peu prometteuse.

		\section{Synthèse des travaux du chapitre et des contributions}

		Ce chapitre a présenté une preuve de concept séparant les modèles d'isolation de Pip du code des services. Cette séparation permet de définir des modèles alternatifs permettant de raisonner sur d'autres propriétés que les propriétés d'isolation classiques de Pip. Cette séparation a été rendue possible par la définition de \emph{Modules} rendant explicite l'interface du code des services avec le reste de la TCB (telle que les types, l'état, et les primitives d'interaction avec l'état). Ce chapitre a aussi montré que cette séparation ne remet pas en question les fonctionnalités traditionnelles de Pip, telle que l'extraction de l'AST du code permettant de compiler Pip vers du C ou la preuve formelle de propriétés sur le code.

		En outre, cette contribution permet d'ouvrir les perspectives de recherche sur Pip, notamment celles concernant la preuve de propriétés alternatives sur le code des services. Le premier chapitre de contribution sur le service de transfert de flot d'exécution avait mentionné que les propriétés d'isolation n'étaient pas particulièrement pertinentes sur ce service. Ces travaux proposent une solution permettant d'établir des propriétés alternatives pertinentes, comme la preuve de bon fonctionnement du service.
